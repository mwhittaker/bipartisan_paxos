\section{Background}

\subsection{Consensus and Paxos}
Assume we have a number of clients, each with a value that they'd like to
propose. Consensus is the challenge of agreeing a single one of these proposed
values. A consensus protocol is a protocol that implements consensus. Clients
propose commands by sending them to the protocol. The protocol eventually
chooses a single one of the proposed values and returns it to the clients.

Paxos~\cite{lamport2001paxos, lamport2001paxos, van2015paxos, lampson2001abcd,
mazieres2007paxos} is one of the oldest and most well studied consensus
protocols. Today, Paxos is practically synonymous with consensus. We'll see
later that BPaxos uses Paxos to implement consensus, so it's important to be
familiar with \emph{what} Paxos is. Fortunately though, BPaxos treats Paxos
like a black box, so we don't have to concern ourselves with \emph{how} Paxos
works.

\subsection{State Machine Replication and MultiPaxos}
Whereas consensus involves agreeing on a \emph{single} value, state machine
replication is all about agreeing on a \emph{sequence} of values called a
\defword{log}. A state machine replication protocol involves some number of
replicas of a state machine, with each state machine beginning in the same
initial state. Clients propose commands to the replication protocol, and the
protocol orders the commands into an agreed upon log that grows over time.
Replicas execute entries in the log in increasing sequential order. By
beginning in the same initial state and executing the same commands in the same
order, all the replicas are guaranteed to remain in sync.

\begin{itemize}
  \item what is state machine replication
  \item what is multipaxos
  \item consensus in each instance
  \item cartoon of log growing over time and commands getting executed
  \item again, details unimportant, but note that it is single leader (cartoon).
\end{itemize}

{\newlength{\logentryinnersep}
\setlength{\logentryinnersep}{2pt}
\newlength{\logentrylinewidth}
\setlength{\logentrylinewidth}{1pt}
\newlength{\logentrywidth}
\setlength{\logentrywidth}{\widthof{\scriptsize$a \gets 1$}+2\logentryinnersep}
\newcommand{\logindexcolor}{flatred}
\newcommand{\cmdi}{$a \gets 0$}
\newcommand{\cmdii}{$b \gets 0$}
\newcommand{\cmdiii}{$a \gets b$}

\tikzstyle{logentry}=[draw,
                      font=\scriptsize,
                      inner sep=\logentryinnersep,
                      line width=\logentrylinewidth,
                      minimum height=\logentrywidth,
                      minimum width=\logentrywidth]
\tikzstyle{executed}=[fill=gray, opacity=0.2, draw opacity=1, text opacity=1]
\tikzstyle{logindex}=[\logindexcolor]

\newcommand{\rightof}[1]{-\logentrylinewidth of #1}

\newcommand{\multipaxoslog}[6]{%
  \node[logentry, label={[logindex]90:0}, #2] (0) {#1};
  \node[logentry, label={[logindex]90:1}, right=\rightof{0}, #4] (1) {#3};
  \node[logentry, label={[logindex]90:2}, right=\rightof{1}, #6] (2) {#5};
}

\begin{figure}[ht]
  \begin{subfigure}[t]{0.45\columnwidth}
    \centering
    \begin{tikzpicture}
      \multipaxoslog{\cmdi}{}%
                    {}{}%
                    {}{}
    \end{tikzpicture}
    \caption{%
      \cmdi{} is chosen in entry $\textcolor{\logindexcolor}{0}$.
    }
    \figlabel{ExampleMultiPaxosExecutionA}
  \end{subfigure}\hspace{0.1\columnwidth}%
  \begin{subfigure}[t]{0.45\columnwidth}
    \centering
    \begin{tikzpicture}
      \multipaxoslog{\cmdi}{executed}%
                    {}{}%
                    {}{}
    \end{tikzpicture}
    \caption{%
      \cmdi{} is executed.
    }
    \figlabel{ExampleMultiPaxosExecutionB}
  \end{subfigure}

  \vspace{2pt}\textcolor{flatgray}{\rule{\columnwidth}{0.4pt}}

  \begin{subfigure}[t]{0.45\columnwidth}
    \centering
    \begin{tikzpicture}
      \multipaxoslog{\cmdi}{executed}%
                    {}{}%
                    {\cmdiii}{}
    \end{tikzpicture}
    \caption{%
      \cmdiii{} is chosen in entry $\textcolor{\logindexcolor}{2}$.
    }
    \figlabel{ExampleMultiPaxosExecutionC}
  \end{subfigure}\hspace{0.1\columnwidth}%
  \begin{subfigure}[t]{0.45\columnwidth}
    \centering
    \begin{tikzpicture}
      \multipaxoslog{\cmdi}{executed}%
                    {}{}%
                    {\cmdiii}{}
    \end{tikzpicture}
    \caption{%
      Nothing is executed.
    }
    \figlabel{ExampleMultiPaxosExecutionD}
  \end{subfigure}

  \vspace{2pt}\textcolor{flatgray}{\rule{\columnwidth}{0.4pt}}

  \begin{subfigure}[t]{0.45\columnwidth}
    \centering
    \begin{tikzpicture}
      \multipaxoslog{\cmdi}{executed}%
                    {\cmdii}{}%
                    {\cmdiii}{}
    \end{tikzpicture}
    \caption{%
      \cmdii{} is chosen in entry $\textcolor{\logindexcolor}{1}$.
    }
    \figlabel{ExampleMultiPaxosExecutionE}
  \end{subfigure}\hspace{0.1\columnwidth}%
  \begin{subfigure}[t]{0.45\columnwidth}
    \centering
    \begin{tikzpicture}
      \multipaxoslog{\cmdi}{executed}%
                    {\cmdii}{executed}%
                    {\cmdiii}{executed}
    \end{tikzpicture}
    \caption{%
      \cmdii{}, \cmdiii{} are executed.
    }
    \figlabel{ExampleMultiPaxosExecutionF}
  \end{subfigure}

  \caption{%
    An example of a MultiPaxos replica executing commands over time, as they
    are chosen
  }
  \figlabel{ExampleMultiPaxosExecution}
\end{figure}
}

\subsection{Generalized Consensus}
