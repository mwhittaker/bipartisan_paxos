\section{Background}

\subsection{Paxos}
Assume we have a number of clients, each with a value that they would like to
propose. \defword{Consensus} is the challenge of agreeing on a single one of
these proposed values. A consensus protocol is a protocol that implements
consensus. Clients propose commands by sending them to the protocol. The
protocol eventually chooses a single one of the proposed values and returns it
to the clients.

Paxos~\cite{lamport1998part, lamport2001paxos} is one of the oldest and most
well studied consensus protocols. We will see later that BPaxos uses Paxos to
implement consensus, so it is important to be familiar with \emph{what} Paxos
is. Fortunately though, BPaxos treats Paxos like a black box, so we do not have
to concern ourselves with \emph{how} Paxos works.

\subsection{MultiPaxos}
Whereas consensus protocols like Paxos agree on a \emph{single} value,
\defword{state machine replication} protocols like MultiPaxos agree on a
\emph{sequence} of values called a log.  A state machine replication protocol
involves some number of replicas of a state machine, with each state machine
beginning in the same initial state. Clients propose commands to the
replication protocol, and the protocol orders the commands into an agreed upon
log that grows over time. Replicas execute entries in the log in prefix order.
By beginning in the same initial state and executing the same commands in the
same order, all the replicas are guaranteed to remain in sync.

MultiPaxos~\cite{van2015paxos} is one of the earliest and most popular state
machine replication protocols. MultiPaxos uses one instance of Paxos for every
log entry, agreeing on the log entries one by one. For example, it runs one
instance of Paxos to agree on the command chosen in log entry 0, one instance
for log entry 1, and so on. Over time, more and more commands are chosen, and
the log of chosen commands grows and grows. MultiPaxos replicas execute
commands as they are chosen, taking care not to execute the commands out of
order.

For example, consider the example execution of a MultiPaxos replica depicted in
\figref{ExampleMultiPaxosExecution}. The replica implements a key-value store
with keys $a$ and $b$. First, the command $a \gets 0$ (i.e.\ set $a$ to $0$) is
chosen in log entry $0$ (\figref{ExampleMultiPaxosExecutionA}), and the replica
executes the command (\figref{ExampleMultiPaxosExecutionB}). Then, the command
$a \gets b$ is chosen in log entry $2$ (\figref{ExampleMultiPaxosExecutionC}).
The replica cannot yet execute the command, because it must first execute the
command in log entry $1$, which has not yet been chosen
(\figref{ExampleMultiPaxosExecutionD}).  Finally, $b \gets 0$ is chosen in log
entry $1$ (\figref{ExampleMultiPaxosExecutionE}), and the replica can execute
the commands in both log entries $1$ and $2$. Note that the replica executes
the log in prefix order, waiting to execute a command if previous commands have
not yet been chosen.

{\newlength{\logentryinnersep}
\setlength{\logentryinnersep}{2pt}
\newlength{\logentrylinewidth}
\setlength{\logentrylinewidth}{1pt}
\newlength{\logentrywidth}
\setlength{\logentrywidth}{\widthof{\scriptsize$a \gets 1$}+2\logentryinnersep}
\newcommand{\logindexcolor}{flatred}
\newcommand{\cmdi}{$a \gets 0$}
\newcommand{\cmdii}{$b \gets 0$}
\newcommand{\cmdiii}{$a \gets b$}

\tikzstyle{logentry}=[draw,
                      font=\scriptsize,
                      inner sep=\logentryinnersep,
                      line width=\logentrylinewidth,
                      minimum height=\logentrywidth,
                      minimum width=\logentrywidth]
\tikzstyle{executed}=[fill=gray, opacity=0.2, draw opacity=1, text opacity=1]
\tikzstyle{logindex}=[\logindexcolor]

\newcommand{\rightof}[1]{-\logentrylinewidth of #1}

\newcommand{\multipaxoslog}[6]{%
  \node[logentry, label={[logindex]90:0}, #2] (0) {#1};
  \node[logentry, label={[logindex]90:1}, right=\rightof{0}, #4] (1) {#3};
  \node[logentry, label={[logindex]90:2}, right=\rightof{1}, #6] (2) {#5};
}

\begin{figure}[ht]
  \begin{subfigure}[t]{0.45\columnwidth}
    \centering
    \begin{tikzpicture}
      \multipaxoslog{\cmdi}{}%
                    {}{}%
                    {}{}
    \end{tikzpicture}
    \caption{%
      \cmdi{} is chosen in entry $\textcolor{\logindexcolor}{0}$.
    }
    \figlabel{ExampleMultiPaxosExecutionA}
  \end{subfigure}\hspace{0.1\columnwidth}%
  \begin{subfigure}[t]{0.45\columnwidth}
    \centering
    \begin{tikzpicture}
      \multipaxoslog{\cmdi}{executed}%
                    {}{}%
                    {}{}
    \end{tikzpicture}
    \caption{%
      \cmdi{} is executed.
    }
    \figlabel{ExampleMultiPaxosExecutionB}
  \end{subfigure}

  \vspace{2pt}\textcolor{flatgray}{\rule{\columnwidth}{0.4pt}}

  \begin{subfigure}[t]{0.45\columnwidth}
    \centering
    \begin{tikzpicture}
      \multipaxoslog{\cmdi}{executed}%
                    {}{}%
                    {\cmdiii}{}
    \end{tikzpicture}
    \caption{%
      \cmdiii{} is chosen in entry $\textcolor{\logindexcolor}{2}$.
    }
    \figlabel{ExampleMultiPaxosExecutionC}
  \end{subfigure}\hspace{0.1\columnwidth}%
  \begin{subfigure}[t]{0.45\columnwidth}
    \centering
    \begin{tikzpicture}
      \multipaxoslog{\cmdi}{executed}%
                    {}{}%
                    {\cmdiii}{}
    \end{tikzpicture}
    \caption{%
      Nothing is executed.
    }
    \figlabel{ExampleMultiPaxosExecutionD}
  \end{subfigure}

  \vspace{2pt}\textcolor{flatgray}{\rule{\columnwidth}{0.4pt}}

  \begin{subfigure}[t]{0.45\columnwidth}
    \centering
    \begin{tikzpicture}
      \multipaxoslog{\cmdi}{executed}%
                    {\cmdii}{}%
                    {\cmdiii}{}
    \end{tikzpicture}
    \caption{%
      \cmdii{} is chosen in entry $\textcolor{\logindexcolor}{1}$.
    }
    \figlabel{ExampleMultiPaxosExecutionE}
  \end{subfigure}\hspace{0.1\columnwidth}%
  \begin{subfigure}[t]{0.45\columnwidth}
    \centering
    \begin{tikzpicture}
      \multipaxoslog{\cmdi}{executed}%
                    {\cmdii}{executed}%
                    {\cmdiii}{executed}
    \end{tikzpicture}
    \caption{%
      \cmdii{}, \cmdiii{} are executed.
    }
    \figlabel{ExampleMultiPaxosExecutionF}
  \end{subfigure}

  \caption{%
    An example of a MultiPaxos replica executing commands over time, as they
    are chosen
  }
  \figlabel{ExampleMultiPaxosExecution}
\end{figure}
}

MultiPaxos is implemented with a set of nodes called proposers and a set of
nodes called acceptors. For this paper, we do not need to worry about the
details of how MultiPaxos works, but let us focus briefly on its communication
pattern. One of the proposers is designated a leader. Clients send all state
machine commands to this single leader. When the leader receives a command $x$,
it selects a log entry in which to place $x$ and then performs one round trip
of communication with the acceptors to get $x$ chosen in the log entry. Then,
it executes the command---once all commands in earlier log entries have been
chosen and executed---and returns to the client.  This communication pattern is
illustrated in \figref{MultiPaxosCommunication}.

{% #1: starting point for bottom middle
% #2: width
% #3: height
\newcommand{\crown}[3]{{
  \newcommand{\start}{#1}
  \newcommand{\crownwidth}{#2}
  \newcommand{\crownheight}{#3}
  \definecolor{crownfg}{HTML}{F1C40F}
  \definecolor{crownbg}{HTML}{c9ab03}
  \path[draw=crownbg!50, fill=crownbg] \start{}
          -- ++(\crownwidth/2, \crownheight/2)
          -- ++(-\crownwidth/4, \crownheight/2)
          -- ++(-\crownwidth/4, -\crownheight/2)
          -- ++(-\crownwidth/4, \crownheight/2)
          -- ++(-\crownwidth/4, -\crownheight/2)
          -- ++(\crownwidth/2, -\crownheight/2);
  \path[draw=crownfg!50, fill=crownfg] \start{}
          -- ++(\crownwidth/2, 0)
          -- ++(0, \crownheight)
          -- ++(-\crownwidth/4, -\crownheight/2)
          -- ++(-\crownwidth/4, \crownheight/2)
          -- ++(-\crownwidth/4, -\crownheight/2)
          -- ++(-\crownwidth/4, \crownheight/2)
          -- ++(0, -\crownheight)
          -- ++(\crownwidth/2, 0);
}}

\newcommand{\halffill}[2]{{
  \newcommand{\thenode}{#1}
  \newcommand{\thecolor}{#2}
  \path[fill=\thecolor]
    ($(\thenode.west)!0.25!(\thenode.south west)$)
    -- ($(\thenode.east)!0.25!(\thenode.south east)$)
    -- (\thenode.south east)
    -- (\thenode.south west)
    -- cycle;
}}

\centering
\tikzstyle{proc}=[draw, circle, thick]
\tikzstyle{proclabel}=[inner sep=0pt]
\tikzstyle{fakeproclabel}=[inner sep=0pt, white]
\tikzstyle{comm}=[-latex, thick]
\tikzstyle{commnum}=[fill=white, inner sep=0pt]
\newcommand{\proposercolor}{flatred!50}
\newcommand{\acceptorcolor}{flatblue!50}

\begin{figure}[ht]
  \centering
  \begin{tikzpicture}[scale=0.75]
    \node[proc] (c) at (0, 2) {$c$};
    \node[proc, fill=\proposercolor] (p0) at (2, 2) {$p_0$};
    \node[proc, fill=\proposercolor] (p1) at (4, 2) {$p_1$};
    \node[proc, fill=\acceptorcolor] (a0) at (0, 0) {$a_0$};
    \node[proc, fill=\acceptorcolor] (a1) at (2, 0) {$a_1$};
    \node[proc, fill=\acceptorcolor] (a2) at (4, 0) {$a_2$};

    \crown{(p0.north)++(0, -0.15)}{1}{0.5}

    \node[proclabel, anchor=east] at (-0.5, 2) {Client};
    \node[fakeproclabel, anchor=west] (ps) at (5, 2) {Proposers};
    \halffill{ps}{\proposercolor}
    \node[proclabel, anchor=west] (ps) at (5, 2) {Proposers};
    \node[fakeproclabel, anchor=west] (as) at (5, 0) {Acceptors};
    \halffill{as}{\acceptorcolor}
    \node[proclabel, anchor=west] (as) at (5, 0) {Acceptors};

    \draw[comm, bend left=10] (c) to node[commnum, midway]{1} (p0);
    \draw[comm, bend left=10] (p0) to node[commnum, midway]{2} (a0);
    \draw[comm, bend left=10] (p0) to node[commnum, midway]{2} (a1);
    \draw[comm, bend left=10] (p0) to node[commnum, midway]{2} (a2);
    \draw[comm, bend left=10] (a0) to node[commnum, midway]{3} (p0);
    \draw[comm, bend left=10] (a1) to node[commnum, midway]{3} (p0);
    \draw[comm, bend left=10] (a2) to node[commnum, midway]{3} (p0);
    \draw[comm, bend left=10] (p0) to node[commnum, midway]{4} (c);
  \end{tikzpicture}
  \caption{%
    MultiPaxos communication pattern. The leader is adorned with a crown.
  }
  \figlabel{MultiPaxosCommunication}
\end{figure}
}

\subsection{Multileader and Generalized Consensus}
MultiPaxos has a number of inefficiencies. Here, we focus on two well-known
ones. First, MultiPaxos' throughput is bottlenecked by the leader. As shown in
\figref{MultiPaxosCommunication}, \emph{every} command goes through the leader.
Thus, MultiPaxos can run only as fast as the leader can. Protocols like
Mencius~\cite{mao2008mencius}, EPaxos~\cite{moraru2013there}, and
Caesar~\cite{arun2017speeding} bypass the single leader bottleneck by having
multiple leaders that can process requests in parallel. We call these protocols
\defword{multileader} protocols.

Second, MultiPaxos requires that replicas execute \emph{all} commands in the
same order. That is, MultiPaxos establishes a \emph{total order} of commands.
This is overkill. If two commands commute, they can be executed by replicas in
either order. For example, key-value store replicas executing the log in
\figref{ExampleMultiPaxosExecution} could execute commands $a \gets 0$ and $b
\gets 0$ in either order since the two commands commute. More formally, we say
two commands \defword{conflict} if executing them in opposite orders yields
either different outputs or a different final state. State machine replication
protocols that only require \emph{conflicting} commands to be executed in the
same order are said to implement generalized
consensus~\cite{lamport2005generalized}. Colloquially, we say such a protocol
is \defword{generalized}. Generalized protocols establish a \emph{partial
order} of commands (as opposed to a total order) in which only conflicting
commands have to be ordered.

As a MultiPaxos leader receives commands from clients, it places them in
increasing log entries. The first command is placed in log entry 0, the second
in log entry 1, and so on. In this way, the leader acts as a sequencer,
sequencing commands into a single total order. Multileader protocols however, by
virtue of having multiple leaders, do not have a single designated node that
processes every command. This makes it challenging to establish a single total
order. As a result, most multileader protocols are also generalized. With
multiple concurrently executing leaders, it is easier to establish a partial
order than it is to establish a total order. Moreover, generalization allows
leaders processing non-conflicting commands to operate completely independently
from one another. While it is possible for a multileader protocol to establish
a total order (e.g., Mencius~\cite{mao2008mencius}), such protocols run only as
fast as the slowest replica (which lowers throughput), and involve all-to-all
communication among the leaders (which also lowers throughput).
