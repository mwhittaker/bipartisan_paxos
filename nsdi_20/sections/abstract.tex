\begin{abstract}
  There is no shortage of state machine replication protocols. From Generalized
  Paxos to EPaxos, a huge number replication protocols have been proposed that
  achieve high throughput \emph{and} low latency. The protocols with the
  highest throughput and lowest latency often lump together different logical
  nodes (e.g., a proposer, an acceptor, and a replica) into a single
  ``super-node'' that does it all. This co-location enables a number of clever
  optimizations that lead to high performance.
  %
  In this paper, we take a different approach. We present Bipartisan Paxos
  (BPaxos), a \emph{modular} state machine replication protocol. Instead of
  coupling nodes together, BPaxos divides them into a small number of
  independent modules. This decoupling is not a cure-all; it does slightly
  increase latency. But, it offers two major benefits that more than
  compensate.

  First, conventional wisdom says that state machine replication protocols
  don't scale. BPaxos revises this wisdom, noting that while some components of
  a protocol don't scale, others do! By modularizing these components, BPaxos
  is able to scale up the components that benefit from scaling whithout
  affecting the other components. This leads to higher throughput without the need
  for complex protocol optimizations.
  %
  Second, modularity makes BPaxos easy to understand. Every module can be
  understand and proven correct in isolation and can be pieced together in a
  straightforward way to understand the protocol as a whole.
\end{abstract}
