\section{Bipartisan Paxos}
BPaxos is a modular state machine replication protocol that is both multileader
and generalized. Throughout the paper, we make the standard assumptions that
the network is asynchronous, that state machines are deterministic, and that
machines can fail by crashing but cannot act maliciously. We also assume that
at most $f$ machines can fail for some integer-valued parameter $f$.

\subsection{Goodbye Logs, Hello Graphs}
MultiPaxos is \emph{not} generalized. It totally orders all commands by
sequencing them into a \emph{log}. BPaxos is generalized, so it ditches the log
and instead partially orders commands into a \emph{directed graph}, like the
ones shown in \figref{ExampleBPaxosExecution}.

BPaxos graphs are completely analogous to MultiPaxos logs. Every MultiPaxos log
entry corresponds to a \defword{vertex} in a BPaxos graph. Every MultiPaxos log
entry holds a command; so does every vertex. Every log entry is uniquely
identified by it's index (e.g., \textcolor{flatred}{$0$}); every vertex is
uniquely identified by a \defword{vertex id} (e.g.,
\textcolor{flatred}{$v_0$}). The one difference between graphs and logs are the
edges. Every BPaxos vertex $v$ has edges to some set of other vertices. These
edges are called the \defword{dependencies} of $v$. Note that we view a
vertex's dependencies as belonging to the vertex, so when we refer to a vertex,
we are also referring to its dependencies. The similarities between MultiPaxos
logs and BPaxos graphs are summarized in \tabref{MultiPaxosVsBPaxos}.

\begin{table}[ht]
  \centering
  \caption{A comparison of MultiPaxos log entries and BPaxos vertices.}
  \tablabel{MultiPaxosVsBPaxos}
  \begin{tabular}{r|l}
    \textbf{BPaxos} & \textbf{MultiPaxos} \\\hline
    graph           & log \\
    vertex          & log entry \\
    vertex id       & index \\
    command         & command \\
    dependencies    & - \\
  \end{tabular}
\end{table}

MultiPaxos grows its \emph{log} over time by repeatedly reaching consensus on
one \emph{log entry} at a time. BPaxos grows its \emph{graph} over time by
repeatedly reaching consensus on one \emph{vertex} (and its dependencies) at a
time. MultiPaxos replicas execute logs in prefix order, making sure not to
execute a command until after executing \emph{all previous commands}. BPaxos
replicas execute graphs in prefix order (i.e. reverse topological order),
making sure not to execute a command until after executing \emph{its
dependencies}.

An example of how BPaxos graphs grow over time and how a BPaxos replica
executes these graphs in shown in \figref{ExampleBPaxosExecution}. As you read
through the figure, note the similarities with
\figref{ExampleMultiPaxosExecution}.
%
First, the command $a \gets 0$ is chosen in vertex $v_0$ with no dependencies
(\figref{ExampleBPaxosExecutionA}).
%
Because the vertex has no dependencies, the replica executes $a \gets 0$
immediately (\figref{ExampleBPaxosExecutionB}).
%
Next, the command $a \gets b$ is chosen in vertex $v_2$ with dependencies on
vertices $v_0$ and $v_1$ (\figref{ExampleBPaxosExecutionC}).
%
$v_2$ depends on $v_1$, but a command has not yet been chosen in $v_1$, so the
replica does \emph{not} yet execute $a \gets b$
(\figref{ExampleBPaxosExecutionD}).
%
Finally, the command $b \gets 0$ is chosen in vertex $v_1$ with no
dependencies (\figref{ExampleBPaxosExecutionE}).
%
Because $b \gets 0$ has no dependencies, the replica executes it immediately.
Moreover, all of $v_2$'s dependencies have been executed, so the replica now
executes $a \gets b$ (\figref{ExampleBPaxosExecutionF}).

{\newlength{\vertexinnersep}
\setlength{\vertexinnersep}{2pt}
\newlength{\vertexlinewidth}
\setlength{\vertexlinewidth}{1pt}
\newlength{\vertexwidth}
\setlength{\vertexwidth}{\widthof{$a \gets 1$}+2\vertexinnersep}
\newcommand{\graphindexcolor}{flatred}
\newcommand{\cmdi}{$a \gets 0$}
\newcommand{\cmdii}{$b \gets 0$}
\newcommand{\cmdiii}{$a \gets b$}
\newcommand{\xscale}{0.8}
\newcommand{\yscale}{0.75}

\tikzstyle{vertex}=[draw,
                      inner sep=\vertexinnersep,
                      line width=\vertexlinewidth,
                      minimum height=\vertexwidth,
                      minimum width=\vertexwidth]

\tikzstyle{executed}=[fill=gray, opacity=0.2, draw opacity=1, text opacity=1]

\tikzstyle{dep}=[-latex, thick]

\tikzstyle{graphindex}=[\graphindexcolor]

\begin{figure}
  \begin{subfigure}[t]{0.45\columnwidth}
    \centering
    \begin{tikzpicture}[xscale=\xscale, yscale=\yscale]
      \node[vertex, label={[graphindex]90:$v_0$}] (0) at (0, 2) {\cmdi{}};
    \end{tikzpicture}
    \caption{%
      \cmdi{} is chosen in entry $\textcolor{\graphindexcolor}{v_0}$.
    }
    \figlabel{ExampleBPaxosExecutionA}
  \end{subfigure}\hspace{0.1\columnwidth}%
  \begin{subfigure}[t]{0.45\columnwidth}
    \centering
    \begin{tikzpicture}[xscale=\xscale, yscale=\yscale]
      \node[vertex, executed, label={[graphindex]90:$v_0$}] (0) at (0, 2)
        {\cmdi{}};
    \end{tikzpicture}
    \caption{%
      \cmdi{} is executed.
    }
    \figlabel{ExampleBPaxosExecutionB}
  \end{subfigure}

  \vspace{2pt}\textcolor{flatgray}{\rule{\columnwidth}{0.4pt}}

  \begin{subfigure}[t]{0.45\columnwidth}
    \centering
    \begin{tikzpicture}[xscale=\xscale, yscale=\yscale]
      \node[vertex, executed, label={[graphindex]90:$v_0$}] (0) at (0, 2)
        {\cmdi{}};
      \node[vertex, label={[graphindex]90:$v_2$}] (2) at (2, 1) {\cmdiii{}};
      \node[vertex, label={[graphindex]90:$v_1$}] (1) at (0, 0) {};
      \draw[dep] (2) to (0);
      \draw[dep] (2) to (1);
    \end{tikzpicture}
    \caption{%
      \cmdiii{} is chosen in entry $\textcolor{\graphindexcolor}{v_2}$.
    }
    \figlabel{ExampleBPaxosExecutionC}
  \end{subfigure}\hspace{0.1\columnwidth}%
  \begin{subfigure}[t]{0.45\columnwidth}
    \centering
    \begin{tikzpicture}[xscale=\xscale, yscale=\yscale]
      \node[vertex, executed, label={[graphindex]90:$v_0$}] (0) at (0, 2)
        {\cmdi{}};
      \node[vertex, label={[graphindex]90:$v_2$}] (2) at (2, 1) {\cmdiii{}};
      \node[vertex, label={[graphindex]90:$v_1$}] (1) at (0, 0) {};
      \draw[dep] (2) to (0);
      \draw[dep] (2) to (1);
    \end{tikzpicture}
    \caption{%
      Nothing is executed.
    }
    \figlabel{ExampleBPaxosExecutionD}
  \end{subfigure}

  \vspace{2pt}\textcolor{flatgray}{\rule{\columnwidth}{0.4pt}}

  \begin{subfigure}[t]{0.45\columnwidth}
    \centering
    \begin{tikzpicture}[xscale=\xscale, yscale=\yscale]
      \node[vertex, executed, label={[graphindex]90:$v_0$}] (0) at (0, 2)
        {\cmdi{}};
      \node[vertex, label={[graphindex]90:$v_2$}] (2) at (2, 1) {\cmdiii{}};
      \node[vertex, label={[graphindex]90:$v_1$}] (1) at (0, 0) {\cmdii{}};
      \draw[dep] (2) to (0);
      \draw[dep] (2) to (1);
    \end{tikzpicture}
    \caption{%
      \cmdii{} is chosen in entry $\textcolor{\graphindexcolor}{v_1}$.
    }
    \figlabel{ExampleBPaxosExecutionE}
  \end{subfigure}\hspace{0.1\columnwidth}%
  \begin{subfigure}[t]{0.45\columnwidth}
    \centering
    \begin{tikzpicture}[xscale=\xscale, yscale=\yscale]
      \node[vertex, executed, label={[graphindex]90:$v_0$}] (0) at (0, 2)
        {\cmdi{}};
      \node[vertex, executed, label={[graphindex]90:$v_2$}] (2) at (2, 1)
        {\cmdiii{}};
      \node[vertex, executed, label={[graphindex]90:$v_1$}] (1) at (0, 0)
        {\cmdii{}};
      \draw[dep] (2) to (0);
      \draw[dep] (2) to (1);
    \end{tikzpicture}
    \caption{%
      \cmdii{}, \cmdiii{} are executed.
    }
    \figlabel{ExampleBPaxosExecutionF}
  \end{subfigure}

  \caption{%
    An example of a BPaxos replica executing commands over time, as they are
    chosen.
  }
  \figlabel{ExampleBPaxosExecution}
\end{figure}
}

Before we discuss the mechanisms that BPaxos uses to construct these graphs,
note the following three graph properties.

\paragraph{Vertices are chosen once and for all}
BPaxos reaches consensus on every vertex, so once a vertex has been chosen, it
will never change. It's command won't change, it won't lose dependencies,
and it won't get new dependencies.

\paragraph{Cycles can happen, but aren't a problem}
We'll see in a moment that BPaxos graphs can sometimes be cyclic. These cycles
are a nuisance, but easily handled. Instead of executing graphs in reverse
topological order one \emph{command} at a time, replicas instead execute graphs
in reverse topological order one \emph{strongly connected component} at a time.
The commands within a strongly connected component are executed in an arbitrary
yet deterministic order (e.g., in vertex id order). This is illustrated in
\figref{ExampleBPaxosCycleExecution}.

{\newlength{\cyclevertexinnersep}
\setlength{\cyclevertexinnersep}{4pt}
\newlength{\cyclevertexlinewidth}
\setlength{\cyclevertexlinewidth}{1pt}
\newlength{\cyclevertexwidth}
\setlength{\cyclevertexwidth}{\widthof{$x$}+2\cyclevertexinnersep}
\newcommand{\graphindexcolor}{flatred}
\newcommand{\cmdi}{$a \gets 0$}
\newcommand{\cmdii}{$b \gets 0$}
\newcommand{\cmdiii}{$a \gets b$}
\newcommand{\xscale}{0.5}
\newcommand{\yscale}{0.5}

\tikzstyle{vertex}=[draw,
                    inner sep=\cyclevertexinnersep,
                    line width=\cyclevertexlinewidth,
                    minimum height=\cyclevertexwidth,
                    minimum width=\cyclevertexwidth]

\tikzstyle{executed}=[fill=gray, opacity=0.2, draw opacity=1, text opacity=1]

\tikzstyle{dep}=[-latex, thick]

\tikzstyle{graphindex}=[\graphindexcolor]

\begin{figure}
  \centering

  \begin{subfigure}[t]{0.3\columnwidth}
    \centering
    \begin{tikzpicture}[xscale=\xscale, yscale=\yscale]
      \node[vertex, executed, label={[graphindex]90:$v_x$}] (x) at (0, 1) {$x$};
      \node[vertex, label={[graphindex]90:$v_y$}] (y) at (2, 2) {$y$};
      \node[vertex, label={[graphindex]-90:$v_z$}] (z) at (2, 0) {};
      \draw[dep] (y) to (x);
      \draw[dep, bend left] (y) to (z);
    \end{tikzpicture}
    \caption{}
    \figlabel{ExampleBPaxosCycleExecutionA}
  \end{subfigure}\hspace{0.04\columnwidth}%
  \begin{subfigure}[t]{0.3\columnwidth}
    \centering
    \begin{tikzpicture}[xscale=\xscale, yscale=\yscale]
      \node[vertex, executed, label={[graphindex]90:$v_x$}] (x) at (0, 1) {$x$};
      \node[vertex, label={[graphindex]90:$v_y$}] (y) at (2, 2) {$y$};
      \node[vertex, label={[graphindex]-90:$v_z$}] (z) at (2, 0) {$z$};
      \draw[dep] (y) to (x);
      \draw[dep, bend left] (y) to (z);
      \draw[dep, bend left] (z) to (y);
    \end{tikzpicture}
    \caption{}
    \figlabel{ExampleBPaxosCycleExecutionB}
  \end{subfigure}\hspace{0.04\columnwidth}%
  \begin{subfigure}[t]{0.3\columnwidth}
    \centering
    \begin{tikzpicture}[xscale=\xscale, yscale=\yscale]
      \node[vertex, executed, label={[graphindex]90:$v_x$}] (x) at (0, 1) {$x$};
      \node[vertex, executed, label={[graphindex]90:$v_y$}] (y) at (2, 2) {$y$};
      \node[vertex, executed, label={[graphindex]-90:$v_z$}] (z) at (2, 0) {$z$};
      \draw[dep] (y) to (x);
      \draw[dep, bend left] (y) to (z);
      \draw[dep, bend left] (z) to (y);
    \end{tikzpicture}
    \caption{}
    \figlabel{ExampleBPaxosCycleExecutionB}
  \end{subfigure}

  \caption{%c
    An example of a BPaxos replica executing a cyclic graph.
    (a) $y$ cannot be exeucted until $v_z$ is chosen.
    (b) $v_z$ is chosen. $v_y$ and $v_z$ form a strongly connected component.
    (c) $y$ and $z$ are executed in an arbitrary yet deterministic order; $y$
        then $z$ or $z$ then $y$.%
  }
  \figlabel{ExampleBPaxosCycleExecution}
\end{figure}
}

\paragraph{Conflicting commands depend on each other}
Because BPaxos is generalized, only conflicting commands have to be ordered
with respect to each other. BPaxos ensures this by maintaining the following
invariant:
\begin{invariant}[\defword{dependency invariant}]
  If two conflicting commands $x$ and $y$ are chosen in vertices $v_x$ and
  $v_y$, then either $v_x$ depends on $v_y$ or $v_y$ depends on $v_x$ or both.
  That is, there is at least one edge between vertices $v_x$ and $v_y$.
\end{invariant}
Because every conflicting pair of commands has an edge between them, every
replica is guaranteed to execute the commands in the same order.
Non-conflicting commands do not need an edge between them and can be executed
in either order.

\subsection{Protocol Overview}

\subsection{Dependency Service}
describe implementation
description of why $2f+1$
proof of correctness
maybe describe as set instead of graph?

\subsection{Proposers and Consensus Service}
mention you can use paxos with proposers being paxos proposers and this service
implemented as paxos acceptors


\subsection{Recovery}

add pseudocode (including dep service) jk
add protocol diagram
