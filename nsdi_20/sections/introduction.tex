\section{Introduction}
State machine replication protocols like MultiPaxos~\cite{lamport1998part,
lamport2001paxos} and Raft~\cite{ongaro2014search} allow a state machine to be
executed in unison across a number of machines, despite the possibility of
faults. Today, state machine replication is pervasive. Nearly every strongly
consistent distributed system is implemented with some form of state machine
replication~\cite{corbett2013spanner, thomson2012calvin, hunt2010zookeeper,
burrows2006chubby, baker2011megastore, cockroach2019website, cosmos2019website,
tidb2019website, yugabyte2019website}.

MultiPaxos is one of the oldest and one of the most widely used state machine
replication protocols. Despite its popularity, MultiPaxos doesn't have optimal
throughput or optimal latency. In MultiPaxos, \emph{every} command is sent to a
single elected leader. This leader becomes a bottleneck, limiting the
throughput of the protocol. Moreover, when a client sends a command to the
leader, it must wait at least two round trips before receiving a response. This
is twice as long as the theoretical minimum of one round
trip~\cite{lamport2006lower}.

An enormous number of state machine replication protocols have been proposed to
address MultiPaxos' suboptimal performance. These protocols use sophisticated
techniques that increase MultiPaxos' throughput, decrease its latency,
or both. For example, techniques like
  deploying multiple leaders~\cite{mao2008mencius, moraru2013there,
  arun2017speeding},
  %
  using flexible quorum sizes~\cite{howard2016flexible, nawab2018dpaxos}, and
  %
  separating the control path from the data path~\cite{biely2012s}
increase MultiPaxos' throughput. Techniques like
  bypassing the leader~\cite{lamport2006fast, ports2015designing, li2016just}
  and
  %
  speculatively executing commands~\cite{ports2015designing, li2016just,
  park2019exploiting}
decrease MultiPaxos' latency. Techniques like
  exploiting commutativity~\cite{lamport2005generalized, moraru2013there,
  arun2017speeding, park2019exploiting}
do both.

Many of these sophisticated protocols try to \emph{simultaneously} increase
throughput and decrease latency, using a combination of the techniques
described in the previous paragraph. For example, NoPaxos ``outperforms both
latency- and throughput-optimized protocols on their respective
metrics''~\cite{li2016just}, whereas EPaxos achieves ``optimal commit latency
in the wide-area'' while ``achieving high throughput''~\cite{moraru2013there}.

Trying to increase throughput \emph{and} decrease latency is a complex
endeavor. Protocols that aim to improve \emph{both} are forced to implement
multiple of the techniques mentioned above in a single protocol. In isolation,
these techniques are challenging to implement. When superimposed, they become
even harder. The techniques have to be sewn together in subtle and intricate
ways. These protocols become increasingly complex, with different components
tightly integrated together. Eventually, it becomes difficult to understand any
single piece of a protocol without first having a strong grasp on
the protocol as a whole. Paradoxically, newcomers must first understand the
protocol before they can begin to understand it!

In this paper, we take a different approach. Instead of chasing both throughput
\emph{and} latency, \textbf{we trade off a bit of latency for modularity}. We
present Bipartisan Paxos (BPaxos), a state machine replication protocol that
sacrifices optimal latency for a modular design. BPaxos is composed of a number
of independent modules. Each module can be understood in isolation and composed
together in a straightforward way to form the protocol as a whole. BPaxos'
modular design leads to simplicity and (surprisingly) higher throughput.

% Unfortunately, these advanced techniques don't come for free. They
% significantly complicate the protocols. MultiPaxos is already notoriously
% difficult to understand~\cite{van2015paxos, ongaro2014search}, and these
% sophisticated protocols make MultiPaxos seem like a piece of cake. In fact,
% these protocols can be so complex that bugs in the protocols can go
% undiscovered for years. Generalized Paxos~\cite{lamport2005generalized} and
% Zyzyva~\cite{kotla2007zyzzyva} had bugs that went undiscovered for seven and
% ten years respectively~\cite{sutra2011fast, abraham2017revisiting}. In writing
% this paper, we discovered bugs ourselves in EPaxos~\cite{moraru2013there} and
% DPaxos~\cite{nawab2018dpaxos} six years and one year after their publications
% respectively.
% %
% \TODO[mwhittaker]{Add bugs to appendix and reference appendix.}
%
% Worse yet, the vast majority of these protocols are incomplete. Many omit
% features, like garbage collection, that are necessary to implement the
% protocols in practice. These omitted features would add even more complexity to
% the already complex protocols.
%
% In this paper, we present a new state machine replication protocol called
% Bipartisan Paxos, or BPaxos for short. BPaxos is simpler, has higher
% throughput, and is more complete than state of the art replication protocols.

\paragraph{Simplicity}
It's hard to quantify the ``complexity'' of a protocol, how hard it is for
someone to understand. But, where there's smoke there's fire, and if we look at
protocols closely enough, we start to see smoke. Generalized
Paxos~\cite{lamport2005generalized} was published in 2005. Seven years later,
someone found a bug in one of its assumptions~\cite{sutra2011fast}.
Zyzyva~\cite{kotla2007zyzzyva}, a Byzantine replication protocol, was published
in 2007. Ten years later, the authors published a paper noting that the
protocol is actually unsafe~\cite{abraham2017revisiting}. In writing this
paper, we discovered bugs ourselves in two other protocols,
EPaxos~\cite{moraru2013there} and DPaxos~\cite{nawab2018dpaxos}, which we
confirmed with the protocols' authors. These long undiscovered bugs suggest
that protocols chasing high throughput and low latency are often forced to
sacrifice simplicity.
\TODO[mwhittaker]{Add bugs to appendix and reference.}

BPaxos' modular design makes it easier to understand. Each module can be
understood and proven correct in isolation, allowing newcomers to understand
the protocol piece by piece.
% Say something about how some modules, like consensus, implement well known
% abstractions and how we can take existing protocols and plug them in. E.g.,
% we use Paxos to implement consensus. Other protocols like EPaxos and Caesar
% implement their own consensus and have to prove everything is correct.

% Most of the sophisticated state machine replication protocols try to
% simultaneously achieve high throughput \emph{and} low latency. The key insight
% that enables BPaxos' simplicity is that trading off a bit of latency can
% significantly reduce complexity. BPaxos exploits this insight to achieve
% simplicity in three ways.
% \TODO[mwhittaker]{Quantify what we mean by ``a bit of latency''.}
%
% First, \textbf{BPaxos is modular.} Many protocols couple and co-locate
% components together to decrease latency. BPaxos takes the opposite approach and
% instead decouples the protocol into a number of independent modules. Each
% module can be understood in isolation, making the protocol understandable piece
% by piece.
%
% Second, \textbf{BPaxos leverages existing protocols.} Because BPaxos is
% composed of a number of independent modules, BPaxos is able to use existing,
% well known protocols to implement some modules. This way, BPaxos avoids
% reimplementing the wheel. For example, BPaxos uses Paxos to implement a
% consensus module.
%
% Third, \textbf{BPaxos avoids fast paths.} Most sophisticated state machine
% replication protocols, starting with Fast Paxos~\cite{lamport2006fast}, have a
% fast path and a slow path. The protocols optimistically attempt to take the
% fast path, but are sometimes forced to revert to the slow path.  Fast paths
% decrease latency in the best case but complicate parts of the protocols (e.g.,
% the recovery procedure). BPaxos does not use fast paths, again sacrificing
% latency for simplicity.

\paragraph{High Throughput}
We initially modularized BPaxos to trade off latency for simplicity.
Surprisingly, we found that modularizing the protocol also led to high
throughput. Many existing protocols pack a handful of logical processes onto a
single physical process. For example, a single process may play the role of a
Paxos proposer, acceptor, and state machine replica. This co-location leads to
lower latency. Messages sent between logical nodes do not have to traverse the
network if the two nodes are physically co-located.

BPaxos does \emph{not} couple logical processes together. Instead, we found
that by decoupling the protocol, we are able to significantly increase the
protocol's throughput using a simple trick: scaling. We perform a bottleneck
analysis on the protocol's components. Once we identify the bottleneck
component, we simply scale up the component until it is no longer the
bottleneck.

For example, if our protocol consisted of proposers, acceptors, and state
machine replicas and if we concluded that the proposers were the bottleneck, we
would simply add more proposers. This straightforward scaling is not so easy to
do in tightly coupled protocols. If every node is a proposer, an acceptor, and
a state machine replica, for example, then adding more proposers also
introduces more acceptors and more replicas. Adding more of a certain component
can actually slow down a protocol instead of speeding it up. For example, more
acceptors lead to larger quorums which lead to slower protocols.

\paragraph{Completeness}
Every node in a state machine replication protocol stores some state. For
example, MultiPaxos acceptors store a collection of votes and round numbers,
and MultiPaxos replicas store a log of state machine commands. This state grows
over time and must be garbage collected in order to avoid exhausting a
protocol's physical memory.

Garbage collection algorithms exist for protocols like MultiPaxos and Raft, but
many sophisticated state machine replication protocols are introduced without
detailing a garbage collection algorithm~\cite{%
  moraru2013there, % EPaxos
  arun2017speeding, % Caesar
  zhang2018building, % TAPIR
  mu2016consolidating % Janus
}. This leaves practitioners and protocol implementors responsible for piecing
together how to implement garbage collection correctly. Unfortunately, these
garbage collection algorithms are subtle and hard to get right. In this paper,
we detail a complete garbage collection algorithm for BPaxos. These details are
essential to implement BPaxos completely, and we believe they are also useful
to understand how to implement garbage collection in other sophisticated state
machine replication protocols.

In summary, we present the following contributions:
\begin{itemize}
  \item
    We introduce Bipartisan Paxos: a modular state machine replication protocol
    that trades off a bit of latency for a significant increase in simplicity.
  \item
    We describe how BPaxos' modularity leads to a straightforward form of
    scaling that yields high throughput without added complexity.
  \item
    We detail a garbage collection algorithm for BPaxos, a protocol that also
    sheds light on how to perform garbage collection in similar protocols.
\end{itemize}

% \begin{itemize}
%   \item consensus is important and pervasive
%   \item multipaxos is the most commonly used everywhere
%   \item multipaxos doesn't have the best throughput or latency. single master
%     means throughput bottleneck. two round trip from client not optimal either.
%   \item as a result, a ton of protocols attempting to improve on multipaxos
%     using a ton of techniques, (fast quorums, generalized, speculative exec,
%     multimaster, flexible quorum sizes, etc)
%   \item
%     multipaxos is complicated and these fancier protocols are even more complicated
%   \item Generalized, Paxos, Zyzyva, EPaxos, DPaxos all buggy
%   \item complex protocols are discouraging to implement in practice and easy to get wrong
%   \item moreover, these protocols are incomplete, leaving out essential features like gc
%   \item
%   \item in this paper, we present a new consensus protocol that is simpler than
%     state of the art, has higher throughput than state of the art, and is more
%     complete than state of the art.
%   \item
%   \item novelty 1: simplicity
%   \item --- insight 1: trade latency for simplicity
%   \item --- Paxos has a bad rep for being complicated
%   \item --- Fancier variants are even more complicated
%   \item --- modularity, re-use, no fast paths
%   \item --- emphasize not simpler than multpaxos but simpler than alternatives
%   \item novelty 2: high throughput
%   \item --- insight 2: decouple protocols leads to a huge number of benefits
%   \item --- decoupling and modularity reduce latency but improve throughput
%   \item --- decoupling helps identify throughput bottlenecks and scale the protocol
%   \item --- We can get state of the art throughput without adding complexity
%   \item novelty 3: garbage collection
%   \item --- gc is necessary. without it, processes run out of memory
%   \item --- mp has gc
%   \item --- all other protocols do not discuss it
%   \item --- gc is considerably more complex in these scenarios
%   \item --- we are the first to present gc
%   \item --- gc simplified by modular architecture
% \end{itemize}
%
%
%
%
%
%
%
%
% \begin{itemize}
%   \item Introduction
%
%   \item Background
%     \begin{itemize}
%       \item MultiPaxos
%     \end{itemize}
%
%   \item Bipartisan Paxos
%     \begin{itemize}
%       \item Main idea overview
%       \item Protocol overview with figure
%       \item Dependency service
%       \item Consensus Service
%       \item Replicas
%       \item BPaxos Nodes
%       \item Example
%       \item Full pseudo-code
%       \item Recovery
%     \end{itemize}
%
%   \item Scaling
%     \begin{itemize}
%       \item Other protocols tightly coupled
%       \item Our protocol decoupled, already gives boost
%       \item Bottleneck analysis
%       \item Scale to eliminate bottleneck
%     \end{itemize}
%
%   \item Garbage Collection
%     \begin{itemize}
%       \item Leaders
%       \item Acceptors and Proposers
%       \item Dependency service, new dep service properties
%       \item Replica
%     \end{itemize}
%
%   \item Practical Considerations
%     \begin{itemize}
%       \item Client table
%       \item Dependency compaction
%       \item
%     \end{itemize}
%
%   \item Evaluation
%     \begin{itemize}
%       \item
%         latency throughput of protocol in lan vs epaxos and paxos (for various
%         params)
%       \item
%         same graph as above without decoupling and without scaling to show
%         effects
%       \item
%         latency throughput of protocols in lan with batching
%       \item
%         graphs to show garbage collection effects on throughput (makes it
%         jittery), memory usage over time with projection on how long before
%         running out.
%       \item
%         latency throughput of protocol in WAN to show that it's actually not
%         great
%     \end{itemize}
%
%   \item Related work
%     \begin{itemize}
%       \item EPaxos
%       \item Caesar
%       \item Mencius
%       \item PODC announcement
%       \item SPaxos (decoupling)
%       \item Multithreaded paxos (essentially decoupling)
%       \item all other papers on gc
%     \end{itemize}
%
%   \item Appendix
%     \begin{itemize}
%       \item Execute conflicting commands in same order equivalent to gen paxos
%       \item Bugs in other protocols.
%       \item TLA+ specification of protocol.
%     \end{itemize}
% \end{itemize}
