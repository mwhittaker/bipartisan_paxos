\section{Practical Considerations}\seclabel{PracticalConsiderations}

\subsection{Ensuring Exactly Once Semantics}
If a client proposes a command to a state machine replication protocol but
does not hear back quickly enough, it resends the command to the protocol to
make sure that the command eventually gets executed. Thus, a replication
protocol might \emph{receive} a command more than once, but it has to guarantee
that it never \emph{executes} the command more than once. Executing a command
more than once would violate exactly once semantics.

Non-generalized protocols like Paxos~\cite{van2015paxos}, Viewstamped
Replication~\cite{liskov2012viewstamped}, and Raft~\cite{ongaro2014search} all
employ the following technique to avoid executing a command more than once.
First, before a client proposes a command to a replication protocol, it
annotates the command with a monotonically increasing integer-valued id.
Moreover, clients only send one command at a time, waiting to receive a
response from one command before sending another. Second, every replica
maintains a \defword{client table}, like the one illustrated below. A client
table has one entry per client. The entry for a client records the largest id
of any command that the replica has executed for that client, along with the
result of executing the command with that id. A replica only executes commands
for a client if it has a larger id than the one recorded in the client table.
If it receives a command with the same id as the one in the client table, it
replies with the recorded output instead of executing the command a second
time.

\begin{center}
  \begin{tabular}{|c|c|c|}
    \hline
    \textbf{Client} & \textbf{Id} & \textbf{Output} \\\hline
    $10.31.14.41$   & $2$         & ``foo'' \\
    $10.54.13.123$  & $1$         & ``bar'' \\\hline
  \end{tabular}
\end{center}

Naively applying this same trick to BPaxos (or any generalized protocol) is
unsafe. For example, imagine a client issues command $x$ with id $1$. The
command gets chosen and is executed by replica 1. Then, the client issues
non-conflicting command $y$ with id $2$. The command gets chosen and is
executed by replica 2. Because $y$ has a larger id than $x$, replica 2 will
never execute $x$.

To fix this bug, a replica must record the ids of \emph{all} commands that it
has executed for a client, along with the output corresponding to the largest
of these ids. Replicas only execute commands they have not previously executed,
and relay the cached output if they receive a command with the corresponding
id.

% Note that other people don't do this.

\subsection{Dependency Compaction}
Upon receiving a command $x$ in vertex $v_x$, a dependency service node returns
the set of all previously received vertices with commands that conflict with
$x$. Over time, as the dependency service receives more and more commands,
these dependency sets get bigger and bigger. As the dependency sets get bigger,
BPaxos' throughput decreases because more time is spent sending these large
dependency sets, and less time is spent doing useful work.

To combat this, a BPaxos dependency service node has to compact dependencies in
some way. Recall that BPaxos leader $i$ creates vertex ids $(i, 0), (i, 1), (i,
2)$, and so on. Thus, vertex ids across all the leaders form a two-dimensional
array with one column for every leader index and one row for every
monotonically increasing id.

{\newlength{\depcompactioninnersep}
\setlength{\depcompactioninnersep}{4pt}
\newlength{\depcompactionlinewidth}
\setlength{\depcompactionlinewidth}{1pt}
\newlength{\depcompactionwidth}
\setlength{\depcompactionwidth}{\widthof{$X$}+2\depcompactioninnersep}
\newcommand{\logindexcolor}{flatred}

\tikzstyle{entry}=[draw,
                   inner sep=\depcompactioninnersep,
                   line width=\depcompactionlinewidth,
                   minimum height=\depcompactionwidth,
                   minimum width=\depcompactionwidth]
\tikzstyle{index}=[color=\logindexcolor,
                   inner sep=\depcompactioninnersep,
                   line width=\depcompactionlinewidth,
                   minimum height=\depcompactionwidth,
                   minimum width=\depcompactionwidth]
\tikzstyle{deps}=[fill=flatblue!20]
\tikzstyle{dep}=[thick, -latex]

\begin{figure}[ht]
  \centering
  \begin{tikzpicture}
    \node[entry, deps] (00) at (0, 0) {$b$};
    \node[entry, deps, right=-\depcompactionlinewidth of 00] (10) {$d$};
    \node[entry, deps, right=-\depcompactionlinewidth of 10] (20) {};
    \node[entry, deps, above=-\depcompactionlinewidth of 00] (01) {$a$};
    \node[entry, deps, right=-\depcompactionlinewidth of 01] (11) {};
    \node[entry, deps, right=-\depcompactionlinewidth of 11] (21) {$e$};
    \node[entry, above=-\depcompactionlinewidth of 01] (02) {};
    \node[entry, deps, right=-\depcompactionlinewidth of 02] (12) {$c$};
    \node[entry, right=-\depcompactionlinewidth of 12] (22) {};
    \node[entry, above=-\depcompactionlinewidth of 02] (03) {};
    \node[entry, right=-\depcompactionlinewidth of 03] (13) {};
    \node[entry, right=-\depcompactionlinewidth of 13] (23) {};

    \node[index, below=-\depcompactionlinewidth of 00] (c0) {$0$};
    \node[index, below=-\depcompactionlinewidth of 10] (c1) {$1$};
    \node[index, below=-\depcompactionlinewidth of 20] (c2) {$2$};
    \node[index, left=-\depcompactionlinewidth of 00] (r0) {$0$};
    \node[index, left=-\depcompactionlinewidth of 01] (r1) {$1$};
    \node[index, left=-\depcompactionlinewidth of 02] (r2) {$2$};
    \node[index, left=-\depcompactionlinewidth of 03] (r3) {$3$};

    \node at ($(c1) + (0, -\depcompactionwidth)$) {leader index};
    \node at ($(r1)!0.5!(r2) + (-\depcompactionwidth, 0)$) {id};
  \end{tikzpicture}
  \caption{An example of dependency compaction}
  \figlabel{DependencyCompaction}
\end{figure}
}

For example, consider a dependency service node that has received commands $a$,
$b$, $c$, $d$, and $e$ in vertices $(0, 1)$, $(0, 0)$, $(1, 2)$, $(1, 0)$, and
$(2, 1)$ as shown in \figref{DependencyCompaction}. \emph{Without dependency
compaction}, if the dependency service node receives a command that conflicts
with commands $a$, $b$, $c$, $d$, and $e$, it would return the vertex ids of
these five commands. In our example, the dependency service node returns only
five dependencies, but in a real deployment, the node could return hundreds of
thousands of dependencies.

\emph{With dependency compaction} on the other hand, the dependency service
node instead artificially adds more dependencies. In particular, for every
leader $i$, it computes the largest id $j$ for which a dependency $(i, j)$
exists. Then, it adds $\setst{(i, k)}{k \leq j}$ to the dependencies. In other
words, it finds the largest dependency in each column and then adds all of the
vertex ids below it as dependencies. In \figref{DependencyCompaction}, the
inflated set of dependencies is highlighted in blue. Even though more
dependencies have been added, the set of inflated dependencies can be
represented more compactly, with a single integer for every leader (i.e., the
id of the largest command for that leader). Thus, every BPaxos dependency set
can be succinctly represented with $N$ integers (for $N$ leaders).

% \subsection{Garbage Collection}
% BPaxos nodes---replicas, proposers, acceptors, etc.---all maintain state. For
% example, dependency service nodes maintain a set of previously received
% commands, and replicas maintain a graph of chosen vertices. Over time, this
% state grows. In a real deployment of BPaxos, this state must be garbage
% collected or else the physical nodes on which the BPaxos components are
% deployed will run out of memory, and the protocol will grind to a halt.
%
% We unfortunately don't have space to explain BPaxos garbage collection in full
% detail, but we highlight the most important bits. BPaxos proposers and
% acceptors are garbage collected as in regular Paxos (after all, they are just
% regular Paxos proposers and acceptors). In particular, once a proposer or
% acceptor knows that a vertex id has been chosen by at least $f+1$ out of $2f+1$
% replicas, it can garbage collect information about the vertex.
%
% A BPaxos leader can garbage collect information about a vertex id as soon it
% has computed the vertex's dependencies and sent them to a proposer.
%
% A dependency service node can garbage collect any prefix of its two-dimensional
% array of vertices, like the one shown in \figref{DependencyCompaction}. Here, a
% prefix is a set of vertex ids such that if $(i, j)$ is in the set, then every
% vertex $(i, k)$ with $k \leq j$ is in the set. After a dependency service node
% garbage collects a prefix, it includes the prefix in all subsequent
% dependencies. This is very similar to dependency compaction.
%
% A replica can garbage collect the executed prefix of its graph, replacing it
% with a snapshot of the state machine and a copy of the client table. However, a
% replica cannot garbage collect a graph prefix at any time. If two replicas
% garbage collect their graph prefixes, then one of the prefixes must be a prefix
% of the other\footnote{Note that in protocols like MultiPaxos, this is also
% true, but given any two log prefixes, one is automatically a prefix of the
% other.}. Thus, replicas must agree on an increasing sequence of graph prefixes
% and can garbage collect only after executing one of these prefixes. To do so,
% BPaxos replicas use a mechanism borrowed from Generalized
% Paxos~\cite{lamport2005generalized}. They periodically issue special
% ``snapshot'' commands to the protocol that conflict with every other command.
% Replicas take snapshots whenever they encounter a snapshot command.
