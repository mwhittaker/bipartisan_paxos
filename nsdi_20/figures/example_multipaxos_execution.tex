\newlength{\logentryinnersep}
\setlength{\logentryinnersep}{2pt}
\newlength{\logentrylinewidth}
\setlength{\logentrylinewidth}{1pt}
\newlength{\logentrywidth}
\setlength{\logentrywidth}{\widthof{$a \gets 1$}+2\logentryinnersep}
\newcommand{\logindexcolor}{flatred}
\newcommand{\cmdi}{$a \gets 0$}
\newcommand{\cmdii}{$b \gets 0$}
\newcommand{\cmdiii}{$a \gets b$}

\tikzstyle{logentry}=[draw,
                      inner sep=\logentryinnersep,
                      line width=\logentrylinewidth,
                      minimum height=\logentrywidth,
                      minimum width=\logentrywidth]
\tikzstyle{executed}=[fill=gray, opacity=0.2, draw opacity=1, text opacity=1]

\tikzstyle{logindex}=[\logindexcolor]

\newcommand{\rightof}[1]{-\logentrylinewidth of #1}

\newcommand{\multipaxoslog}[6]{%
  \node[logentry, label={[logindex]90:0}, #2] (0) {#1};
  \node[logentry, label={[logindex]90:1}, right=\rightof{0}, #4] (1) {#3};
  \node[logentry, label={[logindex]90:2}, right=\rightof{1}, #6] (2) {#5};
}

\begin{figure}[ht]
  \begin{subfigure}[t]{0.45\columnwidth}
    \centering
    \begin{tikzpicture}
      \multipaxoslog{\cmdi}{}%
                    {}{}%
                    {}{}
    \end{tikzpicture}
    \caption{%
      \cmdi{} is chosen in entry $\textcolor{\logindexcolor}{0}$.
    }
    \figlabel{ExampleMultiPaxosExecutionA}
  \end{subfigure}\hspace{0.1\columnwidth}%
  \begin{subfigure}[t]{0.45\columnwidth}
    \centering
    \begin{tikzpicture}
      \multipaxoslog{\cmdi}{executed}%
                    {}{}%
                    {}{}
    \end{tikzpicture}
    \caption{%
      \cmdi{} is executed.
    }
    \figlabel{ExampleMultiPaxosExecutionB}
  \end{subfigure}

  \vspace{2pt}\textcolor{flatgray}{\rule{\columnwidth}{0.4pt}}

  \begin{subfigure}[t]{0.45\columnwidth}
    \centering
    \begin{tikzpicture}
      \multipaxoslog{\cmdi}{executed}%
                    {}{}%
                    {\cmdiii}{}
    \end{tikzpicture}
    \caption{%
      \cmdiii{} is chosen in entry $\textcolor{\logindexcolor}{2}$.
    }
    \figlabel{ExampleMultiPaxosExecutionC}
  \end{subfigure}\hspace{0.1\columnwidth}%
  \begin{subfigure}[t]{0.45\columnwidth}
    \centering
    \begin{tikzpicture}
      \multipaxoslog{\cmdi}{executed}%
                    {}{}%
                    {\cmdiii}{}
    \end{tikzpicture}
    \caption{%
      Nothing is executed.
    }
    \figlabel{ExampleMultiPaxosExecutionD}
  \end{subfigure}

  \vspace{2pt}\textcolor{flatgray}{\rule{\columnwidth}{0.4pt}}

  \begin{subfigure}[t]{0.45\columnwidth}
    \centering
    \begin{tikzpicture}
      \multipaxoslog{\cmdi}{executed}%
                    {\cmdii}{}%
                    {\cmdiii}{}
    \end{tikzpicture}
    \caption{%
      \cmdii{} is chosen in entry $\textcolor{\logindexcolor}{1}$.
    }
    \figlabel{ExampleMultiPaxosExecutionE}
  \end{subfigure}\hspace{0.1\columnwidth}%
  \begin{subfigure}[t]{0.45\columnwidth}
    \centering
    \begin{tikzpicture}
      \multipaxoslog{\cmdi}{executed}%
                    {\cmdii}{executed}%
                    {\cmdiii}{executed}
    \end{tikzpicture}
    \caption{%
      \cmdii{}, \cmdiii{} are executed.
    }
    \figlabel{ExampleMultiPaxosExecutionF}
  \end{subfigure}

  \caption{%
    An example of a MultiPaxos replica executing commands over time, as they
    are chosen
  }
  \figlabel{ExampleMultiPaxosExecution}
\end{figure}
