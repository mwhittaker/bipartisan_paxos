\newlength{\cyclevertexinnersep}
\setlength{\cyclevertexinnersep}{4pt}
\newlength{\cyclevertexlinewidth}
\setlength{\cyclevertexlinewidth}{1pt}
\newlength{\cyclevertexwidth}
\setlength{\cyclevertexwidth}{\widthof{$x$}+2\cyclevertexinnersep}
\newcommand{\graphindexcolor}{flatred}
\newcommand{\cmdi}{$a \gets 0$}
\newcommand{\cmdii}{$b \gets 0$}
\newcommand{\cmdiii}{$a \gets b$}
\newcommand{\xscale}{0.5}
\newcommand{\yscale}{0.5}

\tikzstyle{vertex}=[draw,
                    inner sep=\cyclevertexinnersep,
                    line width=\cyclevertexlinewidth,
                    minimum height=\cyclevertexwidth,
                    minimum width=\cyclevertexwidth]

\tikzstyle{executed}=[fill=gray, opacity=0.2, draw opacity=1, text opacity=1]

\tikzstyle{dep}=[-latex, thick]

\tikzstyle{graphindex}=[\graphindexcolor]

\begin{figure}
  \centering

  \begin{subfigure}[t]{0.3\columnwidth}
    \centering
    \begin{tikzpicture}[xscale=\xscale, yscale=\yscale]
      \node[vertex, executed, label={[graphindex]90:$v_x$}] (x) at (0, 1) {$x$};
      \node[vertex, label={[graphindex]90:$v_y$}] (y) at (2, 2) {$y$};
      \node[vertex, label={[graphindex]-90:$v_z$}] (z) at (2, 0) {};
      \draw[dep] (y) to (x);
      \draw[dep, bend left] (y) to (z);
    \end{tikzpicture}
    \caption{}
    \figlabel{ExampleBPaxosCycleExecutionA}
  \end{subfigure}\hspace{0.04\columnwidth}%
  \begin{subfigure}[t]{0.3\columnwidth}
    \centering
    \begin{tikzpicture}[xscale=\xscale, yscale=\yscale]
      \node[vertex, executed, label={[graphindex]90:$v_x$}] (x) at (0, 1) {$x$};
      \node[vertex, label={[graphindex]90:$v_y$}] (y) at (2, 2) {$y$};
      \node[vertex, label={[graphindex]-90:$v_z$}] (z) at (2, 0) {$z$};
      \draw[dep] (y) to (x);
      \draw[dep, bend left] (y) to (z);
      \draw[dep, bend left] (z) to (y);
    \end{tikzpicture}
    \caption{}
    \figlabel{ExampleBPaxosCycleExecutionB}
  \end{subfigure}\hspace{0.04\columnwidth}%
  \begin{subfigure}[t]{0.3\columnwidth}
    \centering
    \begin{tikzpicture}[xscale=\xscale, yscale=\yscale]
      \node[vertex, executed, label={[graphindex]90:$v_x$}] (x) at (0, 1) {$x$};
      \node[vertex, executed, label={[graphindex]90:$v_y$}] (y) at (2, 2) {$y$};
      \node[vertex, executed, label={[graphindex]-90:$v_z$}] (z) at (2, 0) {$z$};
      \draw[dep] (y) to (x);
      \draw[dep, bend left] (y) to (z);
      \draw[dep, bend left] (z) to (y);
    \end{tikzpicture}
    \caption{}
    \figlabel{ExampleBPaxosCycleExecutionB}
  \end{subfigure}

  \caption{%c
    An example of a BPaxos replica executing a cyclic graph.
    (a) $y$ cannot be exeucted until $v_z$ is chosen.
    (b) $v_z$ is chosen. $v_y$ and $v_z$ form a strongly connected component.
    (c) $y$ and $z$ are executed in an arbitrary yet deterministic order; $y$
        then $z$ or $z$ then $y$.%
  }
  \figlabel{ExampleBPaxosCycleExecution}
\end{figure}
