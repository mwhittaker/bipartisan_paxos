\newlength{\vertexinnersep}
\setlength{\vertexinnersep}{2pt}
\newlength{\vertexlinewidth}
\setlength{\vertexlinewidth}{1pt}
\newlength{\vertexwidth}
\setlength{\vertexwidth}{\widthof{\scriptsize $a \gets 1$}+2\vertexinnersep}
\newcommand{\graphindexcolor}{flatred}
\newcommand{\cmdi}{$a \gets 0$}
\newcommand{\cmdii}{$b \gets 0$}
\newcommand{\cmdiii}{$a \gets b$}
\newcommand{\xscale}{0.75}
\newcommand{\yscale}{0.75}

\tikzstyle{vertex}=[draw,
                    font=\scriptsize,
                    inner sep=\vertexinnersep,
                    line width=\vertexlinewidth,
                    minimum height=\vertexwidth,
                    minimum width=\vertexwidth]

\tikzstyle{executed}=[fill=gray, opacity=0.2, draw opacity=1, text opacity=1]

\tikzstyle{dep}=[-latex, thick]

\tikzstyle{graphindex}=[\graphindexcolor]

\begin{figure}
  \begin{subfigure}[t]{0.45\columnwidth}
    \centering
    \begin{tikzpicture}[xscale=\xscale, yscale=\yscale]
      \node[vertex, label={[graphindex]90:$v_0$}] (0) at (0, 2) {\cmdi{}};
    \end{tikzpicture}
    \caption{%
      \cmdi{} is chosen in entry $\textcolor{\graphindexcolor}{v_0}$.
    }
    \figlabel{ExampleBPaxosExecutionA}
  \end{subfigure}\hspace{0.1\columnwidth}%
  \begin{subfigure}[t]{0.45\columnwidth}
    \centering
    \begin{tikzpicture}[xscale=\xscale, yscale=\yscale]
      \node[vertex, executed, label={[graphindex]90:$v_0$}] (0) at (0, 2)
        {\cmdi{}};
    \end{tikzpicture}
    \caption{%
      \cmdi{} is executed.
    }
    \figlabel{ExampleBPaxosExecutionB}
  \end{subfigure}

  \vspace{2pt}\textcolor{flatgray}{\rule{\columnwidth}{0.4pt}}

  \begin{subfigure}[t]{0.45\columnwidth}
    \centering
    \begin{tikzpicture}[xscale=\xscale, yscale=\yscale]
      \node[vertex, executed, label={[graphindex]90:$v_0$}] (0) at (0, 2)
        {\cmdi{}};
      \node[vertex, label={[graphindex]90:$v_2$}] (2) at (2, 1) {\cmdiii{}};
      \node[vertex, label={[graphindex]90:$v_1$}] (1) at (0, 0) {};
      \draw[dep] (2) to (0);
      \draw[dep] (2) to (1);
    \end{tikzpicture}
    \caption{%
      \cmdiii{} is chosen in entry $\textcolor{\graphindexcolor}{v_2}$.
    }
    \figlabel{ExampleBPaxosExecutionC}
  \end{subfigure}\hspace{0.1\columnwidth}%
  \begin{subfigure}[t]{0.45\columnwidth}
    \centering
    \begin{tikzpicture}[xscale=\xscale, yscale=\yscale]
      \node[vertex, executed, label={[graphindex]90:$v_0$}] (0) at (0, 2)
        {\cmdi{}};
      \node[vertex, label={[graphindex]90:$v_2$}] (2) at (2, 1) {\cmdiii{}};
      \node[vertex, label={[graphindex]90:$v_1$}] (1) at (0, 0) {};
      \draw[dep] (2) to (0);
      \draw[dep] (2) to (1);
    \end{tikzpicture}
    \caption{%
      Nothing is executed.
    }
    \figlabel{ExampleBPaxosExecutionD}
  \end{subfigure}

  \vspace{2pt}\textcolor{flatgray}{\rule{\columnwidth}{0.4pt}}

  \begin{subfigure}[t]{0.45\columnwidth}
    \centering
    \begin{tikzpicture}[xscale=\xscale, yscale=\yscale]
      \node[vertex, executed, label={[graphindex]90:$v_0$}] (0) at (0, 2)
        {\cmdi{}};
      \node[vertex, label={[graphindex]90:$v_2$}] (2) at (2, 1) {\cmdiii{}};
      \node[vertex, label={[graphindex]90:$v_1$}] (1) at (0, 0) {\cmdii{}};
      \draw[dep] (2) to (0);
      \draw[dep] (2) to (1);
    \end{tikzpicture}
    \caption{%
      \cmdii{} is chosen in entry $\textcolor{\graphindexcolor}{v_1}$.
    }
    \figlabel{ExampleBPaxosExecutionE}
  \end{subfigure}\hspace{0.1\columnwidth}%
  \begin{subfigure}[t]{0.45\columnwidth}
    \centering
    \begin{tikzpicture}[xscale=\xscale, yscale=\yscale]
      \node[vertex, executed, label={[graphindex]90:$v_0$}] (0) at (0, 2)
        {\cmdi{}};
      \node[vertex, executed, label={[graphindex]90:$v_2$}] (2) at (2, 1)
        {\cmdiii{}};
      \node[vertex, executed, label={[graphindex]90:$v_1$}] (1) at (0, 0)
        {\cmdii{}};
      \draw[dep] (2) to (0);
      \draw[dep] (2) to (1);
    \end{tikzpicture}
    \caption{%
      \cmdii{}, \cmdiii{} are executed.
    }
    \figlabel{ExampleBPaxosExecutionF}
  \end{subfigure}

  \caption{%
    An example of a BPaxos replica executing commands over time, as they are
    chosen.
  }
  \figlabel{ExampleBPaxosExecution}
\end{figure}
