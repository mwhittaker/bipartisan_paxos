\documentclass{mwhittaker}
\title{EPaxos Bug}

\usepackage{ifthen}
\usepackage{pervasives}
\usepackage{tikz}
\usetikzlibrary{calc}
\usetikzlibrary{positioning}
\usetikzlibrary{shapes.misc}


% \instance draws a boxed EPaxos instance where
%
%   - #4 is the node id,
%   - #3 is the instance name,
%   - #4 is the command name,
%   - #5 is the sequence number,
%   - #6 is the status of the command,
%   - #7 is the coordinate of the instance, and
%   - #1 is either top or bottom to draw the instance name on the top or bottom
%     of the instance box.
%
% For example, `\instance[top]{Q.1}{x}{10}{pre-accepted}{1, 2}`.
\newcommand{\instance}[7][top]{{%
  % Validate status.
  \ifthenelse{\NOT \(
    \equal{#6}{} \OR
    \equal{#6}{pre-accepted} \OR
    \equal{#6}{accepted} \OR
    \equal{#6}{committed}
  \)}{%
    \errmessage{Instance status must be pre-accepted, accepted, or committed.}
  }{}

  % Choose box color and abbreviated status based on status.
  \newcommand{\instancecolor}{black}
  \newcommand{\abbreviatedstatus}{}
  \ifthenelse{\equal{#6}{pre-accepted}}{%
    \renewcommand{\instancecolor}{flatblue}
    \renewcommand{\abbreviatedstatus}{pre}
  }{\ifthenelse{\equal{#6}{accepted}}{%
    \renewcommand{\instancecolor}{flatred}
    \renewcommand{\abbreviatedstatus}{acc}
  }{\ifthenelse{\equal{#6}{committed}}{%
    \renewcommand{\instancecolor}{flatgreen}
    \renewcommand{\abbreviatedstatus}{com}
  }{%
    % Nothing.
  }}}

  % Choose label.
  \ifthenelse{\equal{#1}{top}}{%
    \newcommand{\instancelabel}{90}
  }{\ifthenelse{\equal{#1}{bottom}}{%
    \newcommand{\instancelabel}{270}
  }{%
    \errmessage{The optional instance argument must be top or bottom, not #1.}
  }}

  % Draw box.
  \node[
    draw=\instancecolor,
    line width=1pt,
    label={\instancelabel:$#3$},
    minimum height=1.5cm,
    minimum width=1.5cm,
    align=center
  ] (#2) at (#7) {\abbreviatedstatus{}\\$#4$\\$#5$};%
}}

\tikzstyle{epaxosnode}=[draw, line width=1pt, shape=circle]

% Node x-coordinates.
\newcommand{\Qx}{0}
\newcommand{\Rx}{2}
\newcommand{\Sx}{4}
\newcommand{\Tx}{6}
\newcommand{\Ux}{8}

\begin{document}
\maketitle

In this document, we outline a bug in EPaxos~\cite{moraru2013there,
moraru2013proof}. We consider an EPaxos deployment with $n = 5$ EPaxos nodes
(i.e., $f = 2$) and show that EPaxos does not satisfy Theorem $6$ in
\cite{moraru2013proof}. That is, we'll show an execution of the EPaxos
deployment in which two different tuples are committed in the same EPaxos
instance.

Consider the five EPaxos nodes $Q, R, S, T, U$ with EPaxos instances $Q.1$ and
$U.1$. Initially, $Q$ receives a $Request(\gamma)$ message from a client. It
marks the tuple $(\gamma, seq_\gamma, deps_\gamma) = (\gamma, 1, \emptyset)$ as
pre-committed and sends the tuple to a fast quorum of other acceptors. This is
illustrated in \figref{Empty}.

\begin{figure}[h]
  \centering

  \begin{tikzpicture}
    \node[epaxosnode] at (\Qx, 4) {$Q$};
    \instance[top]{QQ1}{Q.1}{\gamma}{1}{pre-accepted}{\Qx, 2}
    \instance[bottom]{QU1}{U.1}{}{}{}{\Qx, 0}

    \node[epaxosnode] at (\Rx, 4) {$R$};
    \instance[top]{RQ1}{Q.1}{}{}{}{\Rx, 2}
    \instance[bottom]{RU1}{U.1}{}{}{}{\Rx, 0}

    \node[epaxosnode] at (\Sx, 4) {$S$};
    \instance[top]{SQ1}{Q.1}{}{}{}{\Sx, 2}
    \instance[bottom]{SU1}{U.1}{}{}{}{\Sx, 0}

    \node[epaxosnode] at (\Tx, 4) {$T$};
    \instance[top]{TQ1}{Q.1}{}{}{}{\Tx, 2}
    \instance[bottom]{TU1}{U.1}{}{}{}{\Tx, 0}

    \node[epaxosnode] at (\Ux, 4) {$U$};
    \instance[top]{UQ1}{Q.1}{}{}{}{\Ux, 2}
    \instance[bottom]{UU1}{U.1}{}{}{}{\Ux, 0}
  \end{tikzpicture}

  \caption{}\figlabel{Empty}
\end{figure}

Nodes $R, S, T$ receive $Q$'s $PreAccept$ request and similarly mark the tuple
as pre-accepted. Then, they respond with a $PreAcceptOk$. When $Q$ receives
these responses, it runs the commit phase, committing $\gamma$ and sending
$Commit$ messages to the other replicas. This is illustrated in
\figref{GammaCommitted}.

\begin{figure}[h]
  \centering

  \begin{tikzpicture}
    \node[epaxosnode] at (\Qx, 4) {$Q$};
    \instance[top]{QQ1}{Q.1}{\gamma}{1}{committed}{\Qx, 2}
    \instance[bottom]{QU1}{U.1}{}{}{}{\Qx, 0}

    \node[epaxosnode] at (\Rx, 4) {$R$};
    \instance[top]{RQ1}{Q.1}{\gamma}{1}{pre-accepted}{\Rx, 2}
    \instance[bottom]{RU1}{U.1}{}{}{}{\Rx, 0}

    \node[epaxosnode] at (\Sx, 4) {$S$};
    \instance[top]{SQ1}{Q.1}{\gamma}{1}{pre-accepted}{\Sx, 2}
    \instance[bottom]{SU1}{U.1}{}{}{}{\Sx, 0}

    \node[epaxosnode] at (\Tx, 4) {$T$};
    \instance[top]{TQ1}{Q.1}{\gamma}{1}{pre-accepted}{\Tx, 2}
    \instance[bottom]{TU1}{U.1}{}{}{}{\Tx, 0}

    \node[epaxosnode] at (\Ux, 4) {$U$};
    \instance[top]{UQ1}{Q.1}{}{}{}{\Ux, 2}
    \instance[bottom]{UU1}{U.1}{}{}{}{\Ux, 0}
  \end{tikzpicture}

  \caption{}\figlabel{GammaCommitted}
\end{figure}

Next, $U$ receives a $Request(\delta)$ message from a client where $\gamma$ and
$\delta$ conflict. It marks the tuple $(\delta, seq_\delta, deps_\delta) =
(\delta, 1, \emptyset)$ as pre-committed and sends the tuple to a fast quorum
of other acceptors. $Q$ and $R$ receive the $PreAccept$ message and compute
$(\delta, seq_\delta, deps_\delta) = (\delta, 2, \set{Q.1})$. They mark this
tuple as pre-accepted and respond to $U$. This is illustrated in
\figref{DeltaPreAccepted}.

\begin{figure}[h]
  \centering

  \begin{tikzpicture}
    \node[epaxosnode] at (\Qx, 4) {$Q$};
    \instance[top]{QQ1}{Q.1}{\gamma}{1}{committed}{\Qx, 2}
    \instance[bottom]{QU1}{U.1}{\delta}{2}{pre-accepted}{\Qx, 0}
    \draw[->, thick] (QU1) to (QQ1);

    \node[epaxosnode] at (\Rx, 4) {$R$};
    \instance[top]{RQ1}{Q.1}{\gamma}{1}{pre-accepted}{\Rx, 2}
    \instance[bottom]{RU1}{U.1}{\delta}{2}{pre-accepted}{\Rx, 0}
    \draw[->, thick] (RU1) to (RQ1);

    \node[epaxosnode] at (\Sx, 4) {$S$};
    \instance[top]{SQ1}{Q.1}{\gamma}{1}{pre-accepted}{\Sx, 2}
    \instance[bottom]{SU1}{U.1}{}{}{}{\Sx, 0}

    \node[epaxosnode] at (\Tx, 4) {$T$};
    \instance[top]{TQ1}{Q.1}{\gamma}{1}{pre-accepted}{\Tx, 2}
    \instance[bottom]{TU1}{U.1}{}{}{}{\Tx, 0}

    \node[epaxosnode] at (\Ux, 4) {$U$};
    \instance[top]{UQ1}{Q.1}{}{}{}{\Ux, 2}
    \instance[bottom]{UU1}{U.1}{\delta}{1}{pre-accepted}{\Ux, 0}
  \end{tikzpicture}

  \caption{}\figlabel{DeltaPreAccepted}
\end{figure}

When $U$ receives these replies, it cannot take the fast path. Instead, it
computes $(\delta, seq_\delta, deps_\delta) = (\delta, 2, \set{Q.1})$ and
begins the Paxos-Accept phase. It contacts $Q$ and $R$ and gets $\delta$
chosen. This is shown in \figref{DeltaCommitted}.

\begin{figure}[h]
  \centering

  \begin{tikzpicture}
    \node[epaxosnode] at (\Qx, 4) {$Q$};
    \instance[top]{QQ1}{Q.1}{\gamma}{1}{committed}{\Qx, 2}
    \instance[bottom]{QU1}{U.1}{\delta}{2}{accepted}{\Qx, 0}
    \draw[->, thick] (QU1) to (QQ1);

    \node[epaxosnode] at (\Rx, 4) {$R$};
    \instance[top]{RQ1}{Q.1}{\gamma}{1}{pre-accepted}{\Rx, 2}
    \instance[bottom]{RU1}{U.1}{\delta}{2}{accepted}{\Rx, 0}
    \draw[->, thick] (RU1) to (RQ1);

    \node[epaxosnode] at (\Sx, 4) {$S$};
    \instance[top]{SQ1}{Q.1}{\gamma}{1}{pre-accepted}{\Sx, 2}
    \instance[bottom]{SU1}{U.1}{}{}{}{\Sx, 0}

    \node[epaxosnode] at (\Tx, 4) {$T$};
    \instance[top]{TQ1}{Q.1}{\gamma}{1}{pre-accepted}{\Tx, 2}
    \instance[bottom]{TU1}{U.1}{}{}{}{\Tx, 0}

    \node[epaxosnode] at (\Ux, 4) {$U$};
    \instance[top]{UQ1}{Q.1}{}{}{}{\Ux, 2}
    \instance[bottom]{UU1}{U.1}{\delta}{2}{committed}{\Ux, 0}
    \draw[->, thick] (UU1) to (UQ1);
  \end{tikzpicture}

  \caption{}\figlabel{DeltaCommitted}
\end{figure}

At this point, nodes $Q$ and $R$ crash. Node $S$ eventually begins the recovery
protocol to get the value $\gamma$ chosen. The recovery protocol is described
on page 14 of \cite{moraru2013proof}. We assume we are running EPaxos with
FP-deps-committed and not Accept-Deps (as described on page 13 of
\cite{moraru2013proof}).

In step 1, $S$ sends $Prepare$ messages to all other replicas, and replicas $T$
and $U$ reply. The criterion of step 2 is \emph{not} met because both $S$ and
$T$ have information about $Q.1$. The criterion of step 3 is \emph{not} met
because none of $S$, $T$, and $U$ have committed $\gamma$. The criterion of
step 4 is \emph{not} met because no command has accepted $\gamma$. The
criterion of step 5 is met because $S$ and $T$ have pre-accepted $(\gamma, 1,
\emptyset{})$. In step 6, $S$ sends $TentativePreAccept$ messages to $U$, the
replica that hasn't pre-accepted $\gamma$. $U$ has recorded $\delta$

\bibliographystyle{plain}
\bibliography{references}
\end{document}
