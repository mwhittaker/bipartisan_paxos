\section{The Bipartisan Paxos Protocols}\seclabel{BipartisanPaxos}
The BPaxos protocols are state machine replication protocols. We assume an
asynchronous network model, deterministic execution, and fail-stop
failures, i.e., nodes can fail by crashing but cannot act maliciously. Throughout the paper, we
assume at most $f$ processes can fail. Every BPaxos protocol involves a set of
$f + 1$ deterministic state machine replicas that all start in the same initial
state. A set of clients repeatedly propose commands, drawn from a
set $\Cmd$, to be executed by the state machine replicas.

\begin{wrapfigure}{r}{0.29\textwidth}
  \centering

  \begin{subfigure}[b]{0.27\textwidth}
    \centering
    \tikzstyle{entry}=[draw, align=center, minimum width=0.32in, minimum height=0.32in, inner sep=2pt] \tikzstyle{vertexlabel}=[label distance=-2pt, flatred]
    \tikzstyle{arrow}=[line width=0.75pt, -latex]
    \begin{tikzpicture}[xscale=1, yscale=1.5]
      \node[entry, label={[vertexlabel]90:$1$}] (x1y2)
        {\texttt{x=1}\\[-3pt]\texttt{y=2}};
      \node[entry, label={[vertexlabel]90:$2$}, right=-0.25pt of x1y2] (x2)
        {\texttt{x=2}};
      \node[entry, label={[vertexlabel]90:$3$}, right=-0.25pt of x2] (y1)
        {\texttt{y=1}};
      \node[entry, label={[vertexlabel]90:$4$}, right=-0.25pt of y1] (x3y4)
        {\texttt{x=3}\\[-3pt]\texttt{y=4}};
      \node[entry, label={[vertexlabel]90:$5$}, right=-0.25pt of x3y4] (x22)
        {\texttt{x=2}};
    \end{tikzpicture}
    \caption{}\figlabel{LinearSMR}
  \end{subfigure}

  \begin{subfigure}[b]{0.27\textwidth}
    \centering
    \tikzstyle{vertex}=[draw, thick, align=center, minimum width=0.3in,
    minimum height=0.2in, inner sep=2pt]
    \tikzstyle{vertexlabel}=[label distance=-2pt, flatred]
    \tikzstyle{arrow}=[line width=0.75pt, -latex]
    \begin{tikzpicture}[xscale=1.2, yscale=1.5]
      \node[vertex, draw=flatblue, fill=flatblue!10,
            label={[vertexlabel]-90:$I_1$}] (x1y2) at (0, 0)
        {\texttt{x=1}\\[-3pt]\texttt{y=2}};
      \node[vertex, draw=flatgreen, fill=flatgreen!10,
            label={[vertexlabel]-90:$I_2$}] (x2) at (1, 0.5)
        {\texttt{x=2}};
      \node[vertex, draw=flatorange, fill=flatorange!10,
            label={[vertexlabel]-90:$I_3$}] (y1) at (1, -0.5)
        {\texttt{y=1}};
      \node[vertex, draw=flatpurple, fill=flatpurple!10,
            label={[vertexlabel]-90:$I_4$}] (x3y4) at (2, 0)
        {\texttt{x=3}\\[-3pt]\texttt{y=4}};
      \node[vertex, draw=flatpurple, fill=flatpurple!10,
            label={[vertexlabel]-90:$I_5$}] (x22) at (3, 0)
        {\texttt{x=2}};

      \draw[arrow] (x2) to (x1y2);
      \draw[arrow] (y1) to (x1y2);
      \draw[arrow] (x3y4) to (x1y2);
      \draw[arrow] (x3y4) to (x2);
      \draw[arrow] (x3y4) to (y1);
      \draw[arrow] (x22) to (x3y4);
      \draw[arrow, bend right=60] (x22) to (x1y2);
      \draw[arrow, bend right=60] (x3y4) to (x22);
    \end{tikzpicture}
    \caption{}\figlabel{BPaxosSMR}
  \end{subfigure}

  \caption{}
\end{wrapfigure}


Traditional state machine replication protocols~\cite{liskov2012viewstamped,
lamport1998part} reach consensus on a totally ordered log of state machine
commands, as illustrated in \figref{LinearSMR}. State machine replicas then
execute commands in log order. Because every replica starts in the same initial
state and executes deterministic commands in exactly the same order,
they all remain in sync.
%
While simple, agreeing on a totally ordered sequence of state machine commands
can be overly prescriptive~\cite{lamport2005generalized, moraru2013there}. If
two commands \emph{do} conflict (e.g., \texttt{x=1} and \texttt{x=2}), then
they \emph{do} need to be executed by every state machine replica in the same
order.  But, if two commands do \emph{not} conflict (e.g., \texttt{x=2} and
\texttt{y=1}), then they do \emph{not} need to be totally ordered. State
machine replicas can execute them in either order.

The BPaxos protocols leverage this observation and order commands only if
they conflict. To do so, the BPaxos protocols ditch the totally ordered log and
instead agree on a directed graph of commands such that every pair of
conflicting commands have an edge between them. We call these graphs
\defword{partial BPaxos graphs}. An example partial BPaxos graph is illustrated
in \figref{BPaxosSMR}. State machine replicas execute commands in these graphs
in the reverse topological order one strongly connected component at a time,
executing commands within a strongly component in an arbitrary but
deterministic order. Executing commands in this way, state machine replicas are
guaranteed to remain in sync. As an example, to execute the commands in
\figref{BPaxosSMR}, replicas first execute \texttt{x=1;y=2}, then \texttt{x=2}
and \texttt{y=1} in either order, and then \texttt{x=3;y=4} and \texttt{x=2} in
some deterministic order (e.g., in order of increasing hash).

Traditional state machine replication protocols assign every command a unique
integer log entry number. These are drawn in red in \figref{LinearSMR}. The
BPaxos protocols follow suit and also assign each command a unique identifier,
called an \defword{instance}. For example, the command \texttt{y=1} in
\figref{BPaxosSMR} is in instance $I_3$. Every instance $I$ in a partial BPaxos
graph has a number of outbound edges to other instances called the
\defword{dependencies} of $I$, denoted $\deps{I}$. For example, in
\figref{BPaxosSMR}, $\deps{I_5} = \set{I_1, I_4}$.
%
The BPaxos protocols construct partial BPaxos graphs one instance at a time,
reaching consensus on the command $x$ in the instance as well as the instance's
dependencies $\deps{I}$. That is, for every instance $I$, the BPaxos protocols
agree on a tuple $(x, \deps{I})$.
%
The correctness of the BPaxos protocols hinges on the following two key
invariants.

\begin{invariant}\invlabel{ConsensusInvariant}
  The BPaxos protocols successfully implement consensus for every instance $I$.
  That is, at most one value $(x, \deps{I})$ is chosen in instance $I$
  (consistency), and if the value $(x, \deps{I})$ is chosen, then it was
  previously proposed (nontriviality).
\end{invariant}%
%
\begin{invariant}\invlabel{ConflictInvariant}
  If $(x, \deps{I_x})$ is chosen in instance $I_x$ and $(y, \deps{I_y})$ is
  chosen in instance $I_y$, and if $x$ and $y$ conflict, then either $I_x \in
  \deps{I_y}$ or $I_y \in \deps{I_x}$ or both.
\end{invariant}

There are three BPaxos protocols: Simple BPaxos, Unanimous BPaxos, and Majority
Commit BPaxos, summarized in \tabref{BPaxosSummary}. The three protocols differ
in various ways (e.g., quorum sizes, commit latencies), but all
three follow the structure described above. They all reach consensus on a
partial BPaxos graph one instance at a time; they all execute commands in a
partial BPaxos graph in reverse topological order; and they all maintain \invref{ConsensusInvariant} and
\invref{ConflictInvariant}. In the rest of the paper, we introduce the
three protocols one by one and prove that each maintains the two key
invariants. For a more formal overview of the BPaxos protocols and a discussion
on why the two key invariants suffice for correctness, refer to
\appendixref{FormalBPaxosOverview}.

\begin{table}[ht]
  \caption{A summary of the BPaxos protocols}\tablabel{BPaxosSummary}%
  \begin{tabular}{llll}
    \toprule
    Protocol                                               & Classic Quorum Size & Fast Quorum Size   & Commit Latency \\\midrule
    Simple BPaxos (\secref{SimpleBPaxos})                  & $f + 1$             & N/A                & 4 message delays \\
    Unanimous BPaxos (\secref{UnanimousBPaxos})            & $f + 1$             & $2f + 1$           & 2 message delays \\
    Majority Commit BPaxos (\secref{MajorityCommitBPaxos}) & $f + 1$             & $\SuperQuorumSize$ & 2 message delays \\
    \bottomrule
  \end{tabular}
\end{table}
