\section{Fast Paxos}
\begin{figure}[ht]
  \begin{minipage}[t]{0.48\textwidth}
    \begin{algorithm}[H]
      \caption{Fast Paxos Phase 2a}%
      \algolabel{FastPaxos}
      \begin{algorithmic}[1]
        \State $M \gets$ phase 1b messages from a quorum $\Quorum$
        \State $k \gets$ the largest vote round in $M$
        \State $V \gets$ the vote values in $M$ for round $k$
        \If{$k = -1$}\linelabel{FastPaxosCase1}
          \State send any proposed value \linelabel{FastPaxosCase1Code}
        \ElsIf{$V = \set{v}$}\linelabel{FastPaxosCase2}
          \State send $v$ \linelabel{FastPaxosCase2Code}
        \ElsIf{$\exists v \in V.\ O4(v)$}\linelabel{FastPaxosCase3}
          \State send $v$ \linelabel{FastPaxosCase3Code}
        \Else{}\linelabel{FastPaxosCase4}
          \State send any proposed value \linelabel{FastPaxosCase4Code}
        \EndIf{}
      \end{algorithmic}
    \end{algorithm}
  \end{minipage}%
  \hspace{0.04\textwidth}%
  \begin{minipage}[t]{0.48\textwidth}
    \begin{algorithm}[H]
      \caption{Fast Paxos Phase 2a Tweak}%
      \algolabel{FastPaxosTweak}
      \begin{algorithmic}[1]
        \State $M \gets$ phase 1b messages from a quorum $\Quorum$
        \State $k \gets$ the largest vote round in $M$
        \State $V \gets$ the vote values in $M$ for round $k$
        \If{$k = -1$}\linelabel{FastPaxosTweakCase1}
          \State send 2a with any proposed value \linelabel{FastPaxosTweakCase1Code}
        \ElsIf{$k \neq 0$}\linelabel{FastPaxosTweakCase2}
          \State send 2a with unique $v \in V$ \linelabel{FastPaxosTweakCase2Code}
        \ElsIf{$\exists v \in V$ maybe chosen in round $0$}\linelabel{FastPaxosTweakCase3}
          \State send 2a with $v$ \linelabel{FastPaxosTweakCase3Code}
        \Else{}\linelabel{FastPaxosTweakCase4}
          \State send 2a with any proposed value \linelabel{FastPaxosTweakCase4Code}
        \EndIf{}
      \end{algorithmic}
    \end{algorithm}
  \end{minipage}
\end{figure}

Fast Paxos~\cite{lamport2006fast} is a two-phase consensus algorithm structured
around a set of clients, leaders, acceptors, and learners. For a full
description of Fast Paxos, we defer the reader to \cite{lamport2006fast}, but
we highlight the salient bits of Fast Paxos here. Fast Paxos proceeds in a
series of integer-valued rounds with $0$ being the smallest round and $-1$
being a null round. Every round is classified either as a fast round or a
classic round. In phase 2a of the algorithm, a leader has to choose a value to
send to the acceptors. The logic for choosing this value is shown in
\algoref{FastPaxos} where $O4(v)$ is true if there exists a fast quorum
$\FastQuorum$ of acceptors such that every acceptor in $\FastQuorum \cap
\Quorum$ voted for $v$ in round $k$. The sizes of fast and classic quorums
ensure that at most one value $v \in V$ satisfies $O4(v)$. For example, Fast
Paxos is commonly deployed with $n = 2f + 1$ acceptors, classic quorums of size
$f + 1$, and fast quorums of size $\SuperQuorumSize$. In this case, a value $v
\in V$ satisfies $O4(v)$ only if a set $\mathcal{A} \subseteq \Quorum$ of
$\QuorumMajoritySize$ or more acceptors voted for $v$ in round $k$. Because
$\mathcal{A}$ constitutes a majority of $\Quorum$, at most one $v \in V$ can
satisfy $O4(v)$.

If we assume that round $0$ is a fast round and every other round is a classic
round, we can modify the standard phase 2a algorithm shown in
\algoref{FastPaxos} to the variant shown in \algoref{FastPaxosTweak}. The
process of determining whether a value $v$ in \lineref{FastPaxosTweakCase3} of
\algoref{FastPaxosTweak} may have been chosen in round $0$ is left
intentionally abstract. The correctness proof of this alternative phase 2a is a
straightforward simplification of the proof given in~\cite{lamport2006fast}.
