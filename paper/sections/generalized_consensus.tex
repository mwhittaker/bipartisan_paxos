\section{Generalized Consensus}\seclabel{GeneralizedConsensus}
Generalized consensus~\cite{lamport1998part, sutra2011fast} involves a set of
processes known as \defword{learners} attempting to reach consensus on a
growing value. Though generalized consensus is defined in terms of an abstract
data structure known as a command-structure set, we restrict our attention to
generalized consensus on conflict graphs (see \secref{ConflictGraphs}). More
formally, given a set $\Cmd$ of commands and conflict relation $\conflict$, we
consider a set $l_1, l_2, \ldots, l_n$ of learners where each learner $l_i$
manages a conflict graph $C_i$. Over time, a set of client processes propose
commands, and learners add the proposed commands to their conflict graphs such
that the following four conditions are maintained.
\begin{itemize}
  \item \defword{Nontriviality:}
    The vertices of every conflict graph are labelled only with proposed
    commands.
  \item \defword{Stability:}
    Every conflict graph $C_i$ at time $t$ is a suffix of $C_i$ at any time after
    $t$.
  \item \defword{Consistency:}
    For every pair of conflict graphs $C_i$ and $C_j$, there exists a conflict
    graph $C$ such that $C_i$ and $C_j$ are both suffixes of $C$.
  \item \defword{Liveness:}
    If a command is proposed, then eventually every conflict graph contains it.
\end{itemize}
