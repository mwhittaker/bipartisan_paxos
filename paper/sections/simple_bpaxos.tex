\section{Simple BPaxos}\seclabel{SimpleBPaxos}
In this section, we introduce \defword{Simple BPaxos}, a BPaxos protocol that
is designed to be as simple as possible.
%
Simple BPaxos consists of three logical components: a set of Simple BPaxos
nodes, a dependency service, and a consensus service. The dependency service
helps provide \invref{ConflictInvariant}, the consensus service helps provide
\invref{ConsensusInvariant}, and the Simple BPaxos nodes glue the two together.
We explain each of these three components in turn.

\subsection{Simple BPaxos Nodes}
We assume a fixed set $b_1, \ldots, b_{f+1}$ of $f + 1$ Simple BPaxos nodes.
Each Simple BPaxos node $b_i$ is a state machine replica and is responsible for
learning a partial instance graph and processing commands as described in
\secref{BipartisanPaxos}.
%
Clients sends state machine commands to BPaxos nodes to be executed by the
replicated state machine. When a BPaxos node $b_i$ receives a command $x$, it
selects a globally unique instance $I$ for the command and sends the tuple $(I,
x)$ to the dependency service. The dependency service replies with a tuple $(I,
x, \deps{I})$ where $\deps{I}$, the dependencies of $I$, is a set of instances.
%
$b_i$ then proposes the value $(x, \deps{I})$ to the consensus service in
instance $I$, and the consensus service replies with some chosen value $(x',
\deps{I}')$ (which is equal to $(x, \deps{I})$ in the failure-free case). At
this point, the command $x'$ with dependencies $\deps{I}'$ is chosen in instance
$I$ and is added to $b_i$'s partial instance graph. $b_i$ also informs the
other Simple BPaxos nodes that the value $(x', \deps{I}')$ has been chosen in
instance $I$.
%
After $b_i$ executes command $x$, it responds to the client with the
corresponding state machine response.

\subsection{Dependency Service}
Upon receiving a tuple $(I, x)$ from a Simple BPaxos node, the dependency
service replies with a tuple $(I, x, \deps{I})$ with the following guarantee.

\begin{invariant}\invlabel{DependencyService}
If two conflicting commands $x$ and $y$ in instances $I_x$ and $I_y$ yield
responses $(I_x, x, \deps{I_x})$ and $(I_y, y, \deps{x})$ from the dependency
service, then either $I_x \in \deps{I_y}$ or $I_y \in \deps{I_x}$ (or both).
\end{invariant}

There are a couple things to note about the dependency service.
%
First, the dependency service has a precondition that at most one command can
be sent to the dependency service in any given instance. That is, if the
dependency service receives tuples $(I, x)$ and $(I, y)$, then $x = y$.
%
Second, the dependency service may process a tuple $(I, x)$ more than once,
yielding different responses each time. For example, Simple BPaxos node $b_i$
may send $(I, x)$ to the dependency service and get a response $(I, x,
\set{I_1, I_2})$. Later, $b_j$ might send $(I, x)$ to the dependency service
and get a different response of $(I, x, \set{I_2, I_3})$. Note that even though
the dependency service may produce different responses for the same request,
the dependency service maintains \invref{DependencyService} for every possible
pair of dependency service responses.

\newcommand{\out}[1]{\text{out}(#1)}
We now describe how to implement the dependency service. We employ $2f + 1$
dependency service nodes $d_{1}, \ldots, d_{2f + 1}$. When a Simple BPaxos node
$b_j$ sends the tuple $(I, x)$ to the dependency service, it sends the tuple to
all $2f + 1$ of the dependency service nodes. Every dependency service node
$d_i$ maintains a conflict graph $C_i$ with instances as vertices. When $d_i$
receives the tuple $(I, x)$ from a Simple BPaxos node, it performs the
following actions.
%
First, if $C_i$ does not already contain vertex $I$, then $d_i$ inserts vertex
$I$ into $C_i$ with label $x$ and with edges to every other instance $I'$ in
$C_i$ that is labelled with a command that conflicts with $x$.
%
Second, $d_i$ returns the tuple $(I, x, \out{I})$ where $\out{I}$ is the set of
instances in $C_i$ with an inbound edge from $I$.

When a Simple BPaxos node $b_j$ receives replies $(I, x, \deps{I}_{i_1}),
\ldots, (I, x, \deps{I}_{i_{f+1}})$ from a quorum $\Quorum$ of $f + 1$
dependency service nodes $d_{i_1}, \ldots, d_{i_{f+1}}$, it takes $(I, x,
\deps{I}_{i_1} \cup \ldots \cup \deps{x}_{i_{f+1}})$ to be the response from
the dependency service. That is, $b_j$ computes $\deps{I}$ by taking the union
of dependencies from a majority of the dependency service nodes.

To understand why this dependency service implementation maintains
\invref{DependencyService}, consider conflicting commands $x$ and $y$ in
instances $I_x$ and $I_y$. Assume $x$'s reply $(I_x, x, \deps{I_x})$ was formed
from a quorum $\Quorum_x$ and $y$'s reply $(I_y, y, \deps{y})$ was formed from
a quorum $\Quorum_y$. Any two quorums intersect, so $\Quorum_x \cap \Quorum_y$
is nonempty. Let $d_i$ be a dependency service node in this intersection. $d_i$
either received $(I_x, x)$ or $(I_y, y)$ first. If it received $I_x$ first,
then $I_y$ has an edge to $I_x$ in $C_i$, so $I_x \in \deps{I_y}$.
Symmetrically, if it received $I_y$ first, then $I_x$ has an edge to $I_y$ in
$C_i$, so $I_y \in \deps{I_x}$.

\subsection{Consensus Service}
We assume a consensus service that implements consensus for every instance $I$.
A Simple BPaxos node can propose to the consensus service that some value $v$
be chosen in some instance $I$. The consensus service replies with the value
that has been chosen in instance $I$, which may or may not be $v$. The
consensus service guarantees that for every instance $I$, at most one value is
ever chosen in $I$ (consistency) and the chosen value was previously proposed
(nontriviality). The consensus service can be implemented with any consensus
protocol (e.g., Paxos~\cite{lamport1998part, lamport2001paxos}, Fast
Paxos~\cite{lamport2006fast}, Flexible Paxos~\cite{howard2016flexible}).

\subsection{Recovery}
Note that it's possible that a command $x$ chosen in instance $I$ depends on an
unchosen instance $I'$. If instance $I'$ remains forever unchosen, then the
command $x$ will never be executed. To avoid this liveness violation, if any
Simple BPaxos node $b_i$ notices that instance $I'$ has been unchosen for some
time, $b_i$ can propose to the consensus service that the command $\noop$ be
chosen in instance $I'$ with no dependencies. $\noop$ is a distinguished
command that does not affect the state machine and does not conflict with any
other command.
%
Alternatively, $b_i$ can contact the dependency service and check if any
dependency service node has recorded a command $y$ in instance $I'$. If such a
command exists, $b_i$ can send the tuple $(I', y)$ to the dependency service,
and propose $y$ with the resulting dependencies to the consensus service. If no
such $y$ exists, $b_i$ can propose a $\noop$.

\subsection{Safety}
Reiterating \secref{BipartisanPaxos}, to prove the safety of Simple BPaxos, it
suffices to prove that Simple BPaxos maintains \invref{ConsensusInvariant} and
\invref{ConflictInvariant}.
%
Simple BPaxos maintains \invref{ConsensusInvariant} trivially by leveraging the
consensus service. Simple BPaxos maintains \invref{ConflictInvariant} by
maintaining the following invariant.

\begin{invariant}\invlabel{SimpleBpaxosInvariant}
  For every instance $I$, a value $(x, \deps{I})$ is chosen in instance $I$
  only if $(I, x, \deps{I})$ is a response from the dependency service or if
  $(x, \deps{I}) = (\noop, \emptyset)$.
\end{invariant}

\invref{ConflictInvariant} follows immediately from \invref{DependencyService}
and \invref{SimpleBpaxosInvariant}.  Simple BPaxos maintains
\invref{SimpleBpaxosInvariant} because a Simple BPaxos node $b_i$ only proposes
a value $(x, \deps{I})$ in instance $I$ if it received a reply $(I, x,
\deps{I})$ from the ordering service or if $(x, \deps{I}) = (\noop,
\emptyset)$. Note that proposing $\noop{}$s does not affect
\invref{ConflictInvariant} because $\noop$s do not conflict with any other
command, so \invref{ConflictInvariant} holds vacuously.

\subsection{Reducing Commit Latency}
For ease of exposition, we stated that clients forward state machine commands
to Simple BPaxos nodes and that Simple BPaxos nodes are responsible for
proposing values to the dependency service and to the consensus service. In
reality, there is nothing preventing clients from taking a more active role in
the protocol. A client can behave exactly like a BPaxos node, assigning
instances to commands and proposing values to the dependency service and
consensus service. Note that when clients take this more active role, they must
be sure to inform Simple BPaxos nodes when a value is chosen.  Alternatively,
the consensus service can inform the Simple BPaxos nodes directly.
%
By allowing clients to take a more active role, the commit latency of the
protocol is reduced. To keep things as simple as possible, in the rest of the
paper, we will continue to describe the BPaxos protocols with passive clients,
but keep in mind that clients can always take a more active role to reduce
commit latency.

\subsection{Discussion}
Simple BPaxos exemplifies the simplicity of the BPaxos protocols. Simple BPaxos
is composed of three subcomponents---the Simple BPaxos nodes, the dependency
service, and the consensus service---that can all be understood independently.
Moreover, by leveraging an existing consensus protocol to implement the
consensus service, Simple BPaxos is able to avoid a lot of unnecessary
complexity.
%
Simple BPaxos also achieves high throughput by being completely leaderless.
Every instance can be processed independently of every other instance, without
the need for a single arbiter to resolve conflicts.
%
Simple BPaxos does \emph{not} achieve optimal commit latency, even with active
clients. The BPaxos protocols presented in the rest of the paper do.
