\section{An Aside: Fast Paxos}\appendixlabel{FastPaxos}
\section{Fast Paxos}\seclabel{FastPaxos}
{\newcommand{\nullbot}{\textsf{null}}

\begin{algorithm}[ht]
  \caption{Fast Paxos Proposer}%
  \algolabel{FastPaxosProposer}
  \begin{algorithmic}[1]
    \GlobalState a value $v$, initially \nullbot{}
    \GlobalState a round $i$, initially $-1$

    \Upon{%
      receiving \msg{Phase2B}{0, v'} from $f + \maj{f+1}$

      \hspace{0.06in}acceptors as the proposer of round $0$ with $i=0$
    }\linelabel{Proposer1}
      \If{every value of $v'$ is the same}
        \State $v'$ is chosen, notify the learners \linelabel{Proposer2}
      \Else{} \linelabel{CoordinatedRecovery1}
        \State $i \gets 1$
        \State proceed to \lineref{computek} viewing every \msg{Phase2B}{0, v'}

        \hspace{0.13in}as a \msg{Phase1B}{1, 0, v'}
      \EndIf \linelabel{CoordinatedRecovery2}
    \EndUpon

    \Upon{recovery}\linelabel{ProposerRecovery1}
      \State $i \gets$ next largest round owned by this proposer
      % Addresses Reviewer 3.
      %
      % > I am unsure why at line 21 of Algorithm 1, the propose is sending to at
      % > least $f+1$ acceptors instead of all acceptors. As $f$ of the acceptors
      % > may crash, so sending to all acceptors in important.
      \State send \msg{Phase1A}{i} to \markrevisions{the acceptors}
    \EndUpon\linelabel{ProposerRecovery2}

    \Upon{receiving \msg{Phase1B}{i, vr, vv} from $f+1$ acceptors}
      \State $k \gets$ the largest $vr$ in any \msg{Phase1B}{i, vr, vv}\linelabel{computek}
      \If{$k = -1$}\linelabel{kn1}
        \State $v \gets$ an arbitrary value
      \ElsIf{$k > 0$}\linelabel{kg0}
        \State $v \gets$ the corresponding $vv$ in round $k$
      \ElsIf{$k = 0$}\linelabel{ke0}
      \If{there are $\maj{f + 1}$ \msg{Phase1B}{i, 0, v'}

          \hspace{0.22in} messages for some value $v'$}\linelabel{majority}
          \State $v \gets$ $v'$
        \Else{}\linelabel{nomajority}
          \State $v \gets$ an arbitrary value
        \EndIf
      \EndIf
      % Addresses Reviewer 3.
      %
      % > I am unsure why at line 21 of Algorithm 1, the propose is sending to at
      % > least $f+1$ acceptors instead of all acceptors. As $f$ of the acceptors
      % > may crash, so sending to all acceptors in important.
      \State send \msg{Phase2A}{i, v} to \markrevisions{the acceptors}
    \EndUpon

    \Upon{receiving \msg{Phase2B}{i} from $f+1$ acceptors}
      \State $v$ is chosen, notify the learners
    \EndUpon
  \end{algorithmic}
\end{algorithm}
}
{\newcommand{\nullbot}{\textsf{null}}

\begin{algorithm}[ht]
  \caption{Fast Paxos Acceptor}%
  \algolabel{FastPaxosAcceptor}
  \begin{algorithmic}[1]
    \GlobalState the largest seen round $r$, initially $-1$
    \GlobalState the largest round $vr$ voted in, initially $-1$
    \GlobalState the value $vv$ voted for in round $vr$, initially \nullbot{}

    \Upon{receiving command $x$ from client} \linelabel{Acceptor1}
      \If{$r = -1$}
        \State $r, vr, vv \gets 0, 0, x$
        \State send \msg{Phase2b}{0, x} to proposer of round $0$
      \EndIf
    \EndUpon{} \linelabel{Acceptor2}

    \Upon{receiving \msg{Phase1A}{i} from $p$ with $i > r$}\linelabel{AcceptorPhase11}
      \State $r \gets i$
      \State send \msg{Phase1B}{i, vr, vv} to $p$
    \EndUpon \linelabel{AcceptorPhase12}

    \Upon{receiving \msg{Phase2A}{i, x} from $p$ with $i \geq r$}
      \State $r, vr, vv \gets i, i, x$
      \State send \msg{Phase2B}{i} to $p$
    \EndUpon
  \end{algorithmic}
\end{algorithm}
}
{\section{Fast Paxos}\seclabel{FastPaxos}
{\newcommand{\nullbot}{\textsf{null}}

\begin{algorithm}[ht]
  \caption{Fast Paxos Proposer}%
  \algolabel{FastPaxosProposer}
  \begin{algorithmic}[1]
    \GlobalState a value $v$, initially \nullbot{}
    \GlobalState a round $i$, initially $-1$

    \Upon{%
      receiving \msg{Phase2B}{0, v'} from $f + \maj{f+1}$

      \hspace{0.06in}acceptors as the proposer of round $0$ with $i=0$
    }\linelabel{Proposer1}
      \If{every value of $v'$ is the same}
        \State $v'$ is chosen, notify the learners \linelabel{Proposer2}
      \Else{} \linelabel{CoordinatedRecovery1}
        \State $i \gets 1$
        \State proceed to \lineref{computek} viewing every \msg{Phase2B}{0, v'}

        \hspace{0.13in}as a \msg{Phase1B}{1, 0, v'}
      \EndIf \linelabel{CoordinatedRecovery2}
    \EndUpon

    \Upon{recovery}\linelabel{ProposerRecovery1}
      \State $i \gets$ next largest round owned by this proposer
      % Addresses Reviewer 3.
      %
      % > I am unsure why at line 21 of Algorithm 1, the propose is sending to at
      % > least $f+1$ acceptors instead of all acceptors. As $f$ of the acceptors
      % > may crash, so sending to all acceptors in important.
      \State send \msg{Phase1A}{i} to \markrevisions{the acceptors}
    \EndUpon\linelabel{ProposerRecovery2}

    \Upon{receiving \msg{Phase1B}{i, vr, vv} from $f+1$ acceptors}
      \State $k \gets$ the largest $vr$ in any \msg{Phase1B}{i, vr, vv}\linelabel{computek}
      \If{$k = -1$}\linelabel{kn1}
        \State $v \gets$ an arbitrary value
      \ElsIf{$k > 0$}\linelabel{kg0}
        \State $v \gets$ the corresponding $vv$ in round $k$
      \ElsIf{$k = 0$}\linelabel{ke0}
      \If{there are $\maj{f + 1}$ \msg{Phase1B}{i, 0, v'}

          \hspace{0.22in} messages for some value $v'$}\linelabel{majority}
          \State $v \gets$ $v'$
        \Else{}\linelabel{nomajority}
          \State $v \gets$ an arbitrary value
        \EndIf
      \EndIf
      % Addresses Reviewer 3.
      %
      % > I am unsure why at line 21 of Algorithm 1, the propose is sending to at
      % > least $f+1$ acceptors instead of all acceptors. As $f$ of the acceptors
      % > may crash, so sending to all acceptors in important.
      \State send \msg{Phase2A}{i, v} to \markrevisions{the acceptors}
    \EndUpon

    \Upon{receiving \msg{Phase2B}{i} from $f+1$ acceptors}
      \State $v$ is chosen, notify the learners
    \EndUpon
  \end{algorithmic}
\end{algorithm}
}
{\newcommand{\nullbot}{\textsf{null}}

\begin{algorithm}[ht]
  \caption{Fast Paxos Acceptor}%
  \algolabel{FastPaxosAcceptor}
  \begin{algorithmic}[1]
    \GlobalState the largest seen round $r$, initially $-1$
    \GlobalState the largest round $vr$ voted in, initially $-1$
    \GlobalState the value $vv$ voted for in round $vr$, initially \nullbot{}

    \Upon{receiving command $x$ from client} \linelabel{Acceptor1}
      \If{$r = -1$}
        \State $r, vr, vv \gets 0, 0, x$
        \State send \msg{Phase2b}{0, x} to proposer of round $0$
      \EndIf
    \EndUpon{} \linelabel{Acceptor2}

    \Upon{receiving \msg{Phase1A}{i} from $p$ with $i > r$}\linelabel{AcceptorPhase11}
      \State $r \gets i$
      \State send \msg{Phase1B}{i, vr, vv} to $p$
    \EndUpon \linelabel{AcceptorPhase12}

    \Upon{receiving \msg{Phase2A}{i, x} from $p$ with $i \geq r$}
      \State $r, vr, vv \gets i, i, x$
      \State send \msg{Phase2B}{i} to $p$
    \EndUpon
  \end{algorithmic}
\end{algorithm}
}
{\section{Fast Paxos}\seclabel{FastPaxos}
{\input{figures/fast_paxos_proposer.tex}}
{\input{figures/fast_paxos_acceptor.tex}}
{\input{figures/fast_paxos.tex}}

Simple \BPaxos{} is designed to be easy to understand, but as shown in
\figref{SimpleBPaxosExample}, it takes seven network delays (in the best case)
between when a client proposes a command $x$ and when a client receives the
result of executing $x$. Call this duration of time the \defword{commit time}.
Generalized multi-leader protocols like EPaxos, Caesar, and Atlas all achieve a
commit time of only four network delays. They do so by leveraging Fast
Paxos~\cite{lamport2006fast}.

Fast Paxos is a Paxos variant that allows clients to propose values directly to
the acceptors without having to initially contact a proposer. Fast Paxos is an
optimistic protocol. If all of the acceptors happen to receive the same command
from the clients, then Fast Paxos has a commit time of only three network
delays. This is called the fast path. However, if multiple clients concurrently
propose different commands, and not all of the acceptors receive the same
command, then the protocol reverts to a slow path and introduces two additional
network delays to the commit time.  In this section, we review a slightly
simplified version of Fast Paxos.

Like Paxos, a Fast Paxos deployment consists of some number of clients, $f+1$
proposers, and $2f+1$ acceptors. We also include a set of $f+1$ learners. These
nodes are notified of the value chosen by Fast Paxos. A Fast Paxos deployment
is illustrated in \figref{FastPaxos}. Proposer and acceptor pseudocode
are given in \algoref{FastPaxosProposer} and \algoref{FastPaxosAcceptor}.

Like Paxos, Fast Paxos is divided into a number of integer valued rounds.  The
key difference is that round 0 of Fast Paxos is a special ``fast round.'' A
client can propose a value directly to an acceptor in round 0 without having to
contact a proposer first. The normal case execution of Fast Paxos is
illustrated in \figref{FastPaxos1}. The execution proceeds as follows:

\begin{itemize}
  \item \textbf{(1)}
    When a client wants to propose a value $v$, it sends $v$ to all of the
    acceptors.

  \item \textbf{(2)}
    When an acceptor receives a value $v$ from a client, the acceptor ignores
    $v$ if it has already received a message in some round $i \geq 0$.
    Otherwise, it votes for $v$ by updating its state and sending a
    \msg{Phase2B}{0, v} message to the proposer that leads round $0$. This is
    shown in \algoref{FastPaxosAcceptor} \lineref{Acceptor1} --
    \lineref{Acceptor2}.

  \item \textbf{(3)}
    Let $\maj{n}$ be a majority of $n$. $\maj{n} = \frac{n}{2} + 1$ if $n$ is
    even, and $\maj{n} = \ceil{n/2}$ if $n$ is odd. If the proposer that leads
    round $0$ receives \msg{Phase2B}{0, $v'$} messages from $f + \maj{f+1}$
    acceptors for the same value $v'$, then $v'$ is chosen, and the proposer
    notifies the learners. This is shown in \algoref{FastPaxosProposer}
    \lineref{Proposer1} -- \lineref{Proposer2}. We consider what happens when
    not every value is the same momentarily.
\end{itemize}

Note that in Paxos, a value is chosen when $f+1$ acceptors vote for it in some
round $i$. In round $0$ of Fast Paxos, a value is chosen when $f + \maj{f+1}$
acceptors vote for it. The larger number of required votes is needed to ensure
the safety of recovery, which we now describe.
%
Let $p$ be the proposer leading round $0$. Recovery is the process by which a
proposer other than $p$ gets a value chosen. For example, if $p$ fails, some
other proposer must take over and get a value chosen. Recovery is illustrated
in \figref{FastPaxos2}.

\begin{itemize}
  \item \textbf{(1) and (2)}
    A recovering proposer performs Phase 1 of Paxos with at least $f+1$
    acceptors in some round $i > 0$. This is shown in
    \algoref{FastPaxosProposer} \lineref{ProposerRecovery1} --
    \lineref{ProposerRecovery2} and \algoref{FastPaxosAcceptor}
    \lineref{AcceptorPhase11} -- \lineref{AcceptorPhase12}.

  \item \textbf{(3) and (4)}
    The recovering proposer receives \msg{Phase1B}{i, vr, vv} messages from
    $f+1$ acceptors. Call this quorum of acceptors $A$. The proposer computes
    $k$ as the largest received $vr$ (\lineref{computek}). This is the largest
    round in which any acceptor in $A$ has voted. If $k = -1$ (\lineref{kn1}),
    then none of the acceptors have voted in any round less than $i$, so the
    proposer is free to propose an arbitrary value. This is the same as in
    Paxos. If $k > 0$ (\lineref{kg0}), then the proposer must propose the value
    $vv$ proposed in round $k$. Again, this is the same as in Paxos. $vv$ may
    have been chosen in round $k$, so the proposer is forced to propose it as
    well. If $k = 0$ (\lineref{ke0}), then there are two cases to consider.

    First, if $\maj{f+1}$ of the acceptors in $A$ have all voted for some value
    $v'$ in round $0$, then it's possible that $v'$ was chosen in round $0$
    (\lineref{majority}). Specifically, if all $f$ of the acceptors not in $A$
    voted for $v'$ in round $0$, then along with the $\maj{f+1}$ of acceptors in
    $A$ who also voted for $v'$ in round $0$, there is a quorum of $f +
    \maj{f+1}$ acceptors who voted for $v'$ in round $0$. In this case, the
    proposer must propose $v'$ as well since it might have been chosen. Second,
    if there does not exist $\maj{f+1}$ votes for any value $v'$, then the
    proposer concludes that no value was chosen or every will be chosen in
    round $0$, so it is free to propose an arbitrary value
    (\lineref{nomajority}).

    Note that a value must receive at least $f + \maj{f+1}$ votes in round $0$
    to be chosen. If this number were any smaller, it would be possible for a
    recovering proposer to find two distinct values $v'$ and $v''$ that
    \emph{both} could have been chosen in round $0$. In this case, the proposer
    cannot make progress. It cannot propose $v'$ because $v''$ might have been
    chosen, and it cannot propose $v''$ because $v'$ might have been chosen

    Once the recovering proposer determines which value to propose, it gets the
    value chosen with the acceptors using the normal Phase 2 of Paxos.

  \item \textbf{(5)}
    The proposer notifies the learners of the chosen value.
\end{itemize}

Finally, we consider what happens when the proposer of round $0$ receives $f +
\maj{f+1}$ \msgfont{Phase1B} messages from the acceptors, but without all of
them containing the same value $v'$. Naively, the proposer could simply
perform a recovery, executing both phases of Paxos is some round $i > 0$.
However, if we assign rounds to proposers in such a way that the proposer of
round $0$ is also the proposer of round $1$, then we can take advantage of an
optimization called \defword{coordinated recovery}. This is illustrated in
\figref{FastPaxos3} and proceeds as follows:

\begin{itemize}
  \item \textbf{(1)}
    Multiple clients send distinct commands directly to the acceptors.

  \item \textbf{(2)}
    The acceptors receive and vote for the commands and send \msgfont{Phase2B}
    messages to the leader of round $0$. However, not every acceptor receives
    the same value first, so not all the acceptors vote for the same value.

  \item \textbf{(3) and (4)}
    The proposer receives \msgfont{Phase2B} messages from $f+\maj{f+1}$
    acceptors, but the acceptors have not all voted for the same value. At this
    point, the proposer could naively perform a recovery in round $1$ by
    executing Phase 1 and then Phase 2 of Paxos. But, executing Phase 1 in
    round $1$ is redundant, since the \msgfont{Phase2B} messages in round $0$
    contain exactly the same information as the \msgfont{Phase1B} messages in
    round $1$. Specifically, the proposer can view every \msg{Phase2B}{0, v'}
    message as a proxy for a \msg{Phase1B}{1, 0, v'} message. Thus, the proposer
    instead jumps immediately to Phase 2 in round $1$ to get a value chosen
    (\lineref{CoordinatedRecovery1} -- \lineref{CoordinatedRecovery2}).

  \item \textbf{(5)}
    Finally, the proposer notifies the learners of the chosen value.
\end{itemize}
}

Simple \BPaxos{} is designed to be easy to understand, but as shown in
\figref{SimpleBPaxosExample}, it takes seven network delays (in the best case)
between when a client proposes a command $x$ and when a client receives the
result of executing $x$. Call this duration of time the \defword{commit time}.
Generalized multi-leader protocols like EPaxos, Caesar, and Atlas all achieve a
commit time of only four network delays. They do so by leveraging Fast
Paxos~\cite{lamport2006fast}.

Fast Paxos is a Paxos variant that allows clients to propose values directly to
the acceptors without having to initially contact a proposer. Fast Paxos is an
optimistic protocol. If all of the acceptors happen to receive the same command
from the clients, then Fast Paxos has a commit time of only three network
delays. This is called the fast path. However, if multiple clients concurrently
propose different commands, and not all of the acceptors receive the same
command, then the protocol reverts to a slow path and introduces two additional
network delays to the commit time.  In this section, we review a slightly
simplified version of Fast Paxos.

Like Paxos, a Fast Paxos deployment consists of some number of clients, $f+1$
proposers, and $2f+1$ acceptors. We also include a set of $f+1$ learners. These
nodes are notified of the value chosen by Fast Paxos. A Fast Paxos deployment
is illustrated in \figref{FastPaxos}. Proposer and acceptor pseudocode
are given in \algoref{FastPaxosProposer} and \algoref{FastPaxosAcceptor}.

Like Paxos, Fast Paxos is divided into a number of integer valued rounds.  The
key difference is that round 0 of Fast Paxos is a special ``fast round.'' A
client can propose a value directly to an acceptor in round 0 without having to
contact a proposer first. The normal case execution of Fast Paxos is
illustrated in \figref{FastPaxos1}. The execution proceeds as follows:

\begin{itemize}
  \item \textbf{(1)}
    When a client wants to propose a value $v$, it sends $v$ to all of the
    acceptors.

  \item \textbf{(2)}
    When an acceptor receives a value $v$ from a client, the acceptor ignores
    $v$ if it has already received a message in some round $i \geq 0$.
    Otherwise, it votes for $v$ by updating its state and sending a
    \msg{Phase2B}{0, v} message to the proposer that leads round $0$. This is
    shown in \algoref{FastPaxosAcceptor} \lineref{Acceptor1} --
    \lineref{Acceptor2}.

  \item \textbf{(3)}
    Let $\maj{n}$ be a majority of $n$. $\maj{n} = \frac{n}{2} + 1$ if $n$ is
    even, and $\maj{n} = \ceil{n/2}$ if $n$ is odd. If the proposer that leads
    round $0$ receives \msg{Phase2B}{0, $v'$} messages from $f + \maj{f+1}$
    acceptors for the same value $v'$, then $v'$ is chosen, and the proposer
    notifies the learners. This is shown in \algoref{FastPaxosProposer}
    \lineref{Proposer1} -- \lineref{Proposer2}. We consider what happens when
    not every value is the same momentarily.
\end{itemize}

Note that in Paxos, a value is chosen when $f+1$ acceptors vote for it in some
round $i$. In round $0$ of Fast Paxos, a value is chosen when $f + \maj{f+1}$
acceptors vote for it. The larger number of required votes is needed to ensure
the safety of recovery, which we now describe.
%
Let $p$ be the proposer leading round $0$. Recovery is the process by which a
proposer other than $p$ gets a value chosen. For example, if $p$ fails, some
other proposer must take over and get a value chosen. Recovery is illustrated
in \figref{FastPaxos2}.

\begin{itemize}
  \item \textbf{(1) and (2)}
    A recovering proposer performs Phase 1 of Paxos with at least $f+1$
    acceptors in some round $i > 0$. This is shown in
    \algoref{FastPaxosProposer} \lineref{ProposerRecovery1} --
    \lineref{ProposerRecovery2} and \algoref{FastPaxosAcceptor}
    \lineref{AcceptorPhase11} -- \lineref{AcceptorPhase12}.

  \item \textbf{(3) and (4)}
    The recovering proposer receives \msg{Phase1B}{i, vr, vv} messages from
    $f+1$ acceptors. Call this quorum of acceptors $A$. The proposer computes
    $k$ as the largest received $vr$ (\lineref{computek}). This is the largest
    round in which any acceptor in $A$ has voted. If $k = -1$ (\lineref{kn1}),
    then none of the acceptors have voted in any round less than $i$, so the
    proposer is free to propose an arbitrary value. This is the same as in
    Paxos. If $k > 0$ (\lineref{kg0}), then the proposer must propose the value
    $vv$ proposed in round $k$. Again, this is the same as in Paxos. $vv$ may
    have been chosen in round $k$, so the proposer is forced to propose it as
    well. If $k = 0$ (\lineref{ke0}), then there are two cases to consider.

    First, if $\maj{f+1}$ of the acceptors in $A$ have all voted for some value
    $v'$ in round $0$, then it's possible that $v'$ was chosen in round $0$
    (\lineref{majority}). Specifically, if all $f$ of the acceptors not in $A$
    voted for $v'$ in round $0$, then along with the $\maj{f+1}$ of acceptors in
    $A$ who also voted for $v'$ in round $0$, there is a quorum of $f +
    \maj{f+1}$ acceptors who voted for $v'$ in round $0$. In this case, the
    proposer must propose $v'$ as well since it might have been chosen. Second,
    if there does not exist $\maj{f+1}$ votes for any value $v'$, then the
    proposer concludes that no value was chosen or every will be chosen in
    round $0$, so it is free to propose an arbitrary value
    (\lineref{nomajority}).

    Note that a value must receive at least $f + \maj{f+1}$ votes in round $0$
    to be chosen. If this number were any smaller, it would be possible for a
    recovering proposer to find two distinct values $v'$ and $v''$ that
    \emph{both} could have been chosen in round $0$. In this case, the proposer
    cannot make progress. It cannot propose $v'$ because $v''$ might have been
    chosen, and it cannot propose $v''$ because $v'$ might have been chosen

    Once the recovering proposer determines which value to propose, it gets the
    value chosen with the acceptors using the normal Phase 2 of Paxos.

  \item \textbf{(5)}
    The proposer notifies the learners of the chosen value.
\end{itemize}

Finally, we consider what happens when the proposer of round $0$ receives $f +
\maj{f+1}$ \msgfont{Phase1B} messages from the acceptors, but without all of
them containing the same value $v'$. Naively, the proposer could simply
perform a recovery, executing both phases of Paxos is some round $i > 0$.
However, if we assign rounds to proposers in such a way that the proposer of
round $0$ is also the proposer of round $1$, then we can take advantage of an
optimization called \defword{coordinated recovery}. This is illustrated in
\figref{FastPaxos3} and proceeds as follows:

\begin{itemize}
  \item \textbf{(1)}
    Multiple clients send distinct commands directly to the acceptors.

  \item \textbf{(2)}
    The acceptors receive and vote for the commands and send \msgfont{Phase2B}
    messages to the leader of round $0$. However, not every acceptor receives
    the same value first, so not all the acceptors vote for the same value.

  \item \textbf{(3) and (4)}
    The proposer receives \msgfont{Phase2B} messages from $f+\maj{f+1}$
    acceptors, but the acceptors have not all voted for the same value. At this
    point, the proposer could naively perform a recovery in round $1$ by
    executing Phase 1 and then Phase 2 of Paxos. But, executing Phase 1 in
    round $1$ is redundant, since the \msgfont{Phase2B} messages in round $0$
    contain exactly the same information as the \msgfont{Phase1B} messages in
    round $1$. Specifically, the proposer can view every \msg{Phase2B}{0, v'}
    message as a proxy for a \msg{Phase1B}{1, 0, v'} message. Thus, the proposer
    instead jumps immediately to Phase 2 in round $1$ to get a value chosen
    (\lineref{CoordinatedRecovery1} -- \lineref{CoordinatedRecovery2}).

  \item \textbf{(5)}
    Finally, the proposer notifies the learners of the chosen value.
\end{itemize}
}

Simple \BPaxos{} is designed to be easy to understand, but as shown in
\figref{SimpleBPaxosExample}, it takes seven network delays (in the best case)
between when a client proposes a command $x$ and when a client receives the
result of executing $x$. Call this duration of time the \defword{commit time}.
Generalized multi-leader protocols like EPaxos, Caesar, and Atlas all achieve a
commit time of only four network delays. They do so by leveraging Fast
Paxos~\cite{lamport2006fast}.

Fast Paxos is a Paxos variant that allows clients to propose values directly to
the acceptors without having to initially contact a proposer. Fast Paxos is an
optimistic protocol. If all of the acceptors happen to receive the same command
from the clients, then Fast Paxos has a commit time of only three network
delays. This is called the fast path. However, if multiple clients concurrently
propose different commands, and not all of the acceptors receive the same
command, then the protocol reverts to a slow path and introduces two additional
network delays to the commit time.  In this section, we review a slightly
simplified version of Fast Paxos.

Like Paxos, a Fast Paxos deployment consists of some number of clients, $f+1$
proposers, and $2f+1$ acceptors. We also include a set of $f+1$ learners. These
nodes are notified of the value chosen by Fast Paxos. A Fast Paxos deployment
is illustrated in \figref{FastPaxos}. Proposer and acceptor pseudocode
are given in \algoref{FastPaxosProposer} and \algoref{FastPaxosAcceptor}.

Like Paxos, Fast Paxos is divided into a number of integer valued rounds.  The
key difference is that round 0 of Fast Paxos is a special ``fast round.'' A
client can propose a value directly to an acceptor in round 0 without having to
contact a proposer first. The normal case execution of Fast Paxos is
illustrated in \figref{FastPaxos1}. The execution proceeds as follows:

\begin{itemize}
  \item \textbf{(1)}
    When a client wants to propose a value $v$, it sends $v$ to all of the
    acceptors.

  \item \textbf{(2)}
    When an acceptor receives a value $v$ from a client, the acceptor ignores
    $v$ if it has already received a message in some round $i \geq 0$.
    Otherwise, it votes for $v$ by updating its state and sending a
    \msg{Phase2B}{0, v} message to the proposer that leads round $0$. This is
    shown in \algoref{FastPaxosAcceptor} \lineref{Acceptor1} --
    \lineref{Acceptor2}.

  \item \textbf{(3)}
    Let $\maj{n}$ be a majority of $n$. $\maj{n} = \frac{n}{2} + 1$ if $n$ is
    even, and $\maj{n} = \ceil{n/2}$ if $n$ is odd. If the proposer that leads
    round $0$ receives \msg{Phase2B}{0, $v'$} messages from $f + \maj{f+1}$
    acceptors for the same value $v'$, then $v'$ is chosen, and the proposer
    notifies the learners. This is shown in \algoref{FastPaxosProposer}
    \lineref{Proposer1} -- \lineref{Proposer2}. We consider what happens when
    not every value is the same momentarily.
\end{itemize}

Note that in Paxos, a value is chosen when $f+1$ acceptors vote for it in some
round $i$. In round $0$ of Fast Paxos, a value is chosen when $f + \maj{f+1}$
acceptors vote for it. The larger number of required votes is needed to ensure
the safety of recovery, which we now describe.
%
Let $p$ be the proposer leading round $0$. Recovery is the process by which a
proposer other than $p$ gets a value chosen. For example, if $p$ fails, some
other proposer must take over and get a value chosen. Recovery is illustrated
in \figref{FastPaxos2}.

\begin{itemize}
  \item \textbf{(1) and (2)}
    A recovering proposer performs Phase 1 of Paxos with at least $f+1$
    acceptors in some round $i > 0$. This is shown in
    \algoref{FastPaxosProposer} \lineref{ProposerRecovery1} --
    \lineref{ProposerRecovery2} and \algoref{FastPaxosAcceptor}
    \lineref{AcceptorPhase11} -- \lineref{AcceptorPhase12}.

  \item \textbf{(3) and (4)}
    The recovering proposer receives \msg{Phase1B}{i, vr, vv} messages from
    $f+1$ acceptors. Call this quorum of acceptors $A$. The proposer computes
    $k$ as the largest received $vr$ (\lineref{computek}). This is the largest
    round in which any acceptor in $A$ has voted. If $k = -1$ (\lineref{kn1}),
    then none of the acceptors have voted in any round less than $i$, so the
    proposer is free to propose an arbitrary value. This is the same as in
    Paxos. If $k > 0$ (\lineref{kg0}), then the proposer must propose the value
    $vv$ proposed in round $k$. Again, this is the same as in Paxos. $vv$ may
    have been chosen in round $k$, so the proposer is forced to propose it as
    well. If $k = 0$ (\lineref{ke0}), then there are two cases to consider.

    First, if $\maj{f+1}$ of the acceptors in $A$ have all voted for some value
    $v'$ in round $0$, then it's possible that $v'$ was chosen in round $0$
    (\lineref{majority}). Specifically, if all $f$ of the acceptors not in $A$
    voted for $v'$ in round $0$, then along with the $\maj{f+1}$ of acceptors in
    $A$ who also voted for $v'$ in round $0$, there is a quorum of $f +
    \maj{f+1}$ acceptors who voted for $v'$ in round $0$. In this case, the
    proposer must propose $v'$ as well since it might have been chosen. Second,
    if there does not exist $\maj{f+1}$ votes for any value $v'$, then the
    proposer concludes that no value was chosen or every will be chosen in
    round $0$, so it is free to propose an arbitrary value
    (\lineref{nomajority}).

    Note that a value must receive at least $f + \maj{f+1}$ votes in round $0$
    to be chosen. If this number were any smaller, it would be possible for a
    recovering proposer to find two distinct values $v'$ and $v''$ that
    \emph{both} could have been chosen in round $0$. In this case, the proposer
    cannot make progress. It cannot propose $v'$ because $v''$ might have been
    chosen, and it cannot propose $v''$ because $v'$ might have been chosen

    Once the recovering proposer determines which value to propose, it gets the
    value chosen with the acceptors using the normal Phase 2 of Paxos.

  \item \textbf{(5)}
    The proposer notifies the learners of the chosen value.
\end{itemize}

Finally, we consider what happens when the proposer of round $0$ receives $f +
\maj{f+1}$ \msgfont{Phase1B} messages from the acceptors, but without all of
them containing the same value $v'$. Naively, the proposer could simply
perform a recovery, executing both phases of Paxos is some round $i > 0$.
However, if we assign rounds to proposers in such a way that the proposer of
round $0$ is also the proposer of round $1$, then we can take advantage of an
optimization called \defword{coordinated recovery}. This is illustrated in
\figref{FastPaxos3} and proceeds as follows:

\begin{itemize}
  \item \textbf{(1)}
    Multiple clients send distinct commands directly to the acceptors.

  \item \textbf{(2)}
    The acceptors receive and vote for the commands and send \msgfont{Phase2B}
    messages to the leader of round $0$. However, not every acceptor receives
    the same value first, so not all the acceptors vote for the same value.

  \item \textbf{(3) and (4)}
    The proposer receives \msgfont{Phase2B} messages from $f+\maj{f+1}$
    acceptors, but the acceptors have not all voted for the same value. At this
    point, the proposer could naively perform a recovery in round $1$ by
    executing Phase 1 and then Phase 2 of Paxos. But, executing Phase 1 in
    round $1$ is redundant, since the \msgfont{Phase2B} messages in round $0$
    contain exactly the same information as the \msgfont{Phase1B} messages in
    round $1$. Specifically, the proposer can view every \msg{Phase2B}{0, v'}
    message as a proxy for a \msg{Phase1B}{1, 0, v'} message. Thus, the proposer
    instead jumps immediately to Phase 2 in round $1$ to get a value chosen
    (\lineref{CoordinatedRecovery1} -- \lineref{CoordinatedRecovery2}).

  \item \textbf{(5)}
    Finally, the proposer notifies the learners of the chosen value.
\end{itemize}


The remaining BPaxos protocols all leverage Fast Paxos~\cite{lamport2006fast}.
We assume a familiarity with Fast Paxos, but pause briefly to highlight the
salient bits of Fast Paxos here. For a more in-depth discussion of Fast Paxos,
refer to \appendixref{FastPaxos}.

Fast Paxos proceeds in a series of integer-valued rounds with $0$ being the
smallest round and $-1$ being a null round. Every round is classified either as
a fast round or a classic round. In phase 2a of the algorithm, a leader has to
choose a value to send to the acceptors. The logic for choosing this value is
shown in \algoref{FastPaxos} where $O4(v)$ is true if there exists a fast
quorum $\FastQuorum$ of acceptors such that every acceptor in $\FastQuorum \cap
\Quorum$ voted for $v$ in round $k$.
%
The key invariant of Fast Paxos is that if a value $v$ was \emph{maybe} chosen
in a round less than $i$, then a leader must propose $v$ in round $i$. This is
the case in \lineref{FastPaxosCase2Code} and \lineref{FastPaxosCase3Code} in
\algoref{FastPaxos}. A leader can propose an arbitrary value in round $i$ only
if it has concluded that no value was chosen in any round less than $i$, as is
the case in \lineref{FastPaxosCase1Code} and \lineref{FastPaxosCase4Code}.

If we assume that round $0$ is a fast round and every other round is a classic
round, we can modify the standard phase 2a algorithm shown in
\algoref{FastPaxos} to the variant shown in \algoref{FastPaxosTweak}. As with
\algoref{FastPaxos}, a leader is sometimes forced to propose a value $v$ if it
was maybe previously chosen (i.e.\ \lineref{FastPaxosTweakCase2Code} and
\lineref{FastPaxosTweakCase3Code} of \algoref{FastPaxosTweak}). The process of
determining whether a value $v$ in \lineref{FastPaxosTweakCase3} of
\algoref{FastPaxosTweak} may have been chosen in round $0$ is left
intentionally abstract. The correctness proof of this alternative phase 2a is
given in \appendixref{FastPaxos}.
