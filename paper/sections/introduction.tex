\section{Introduction}
Consensus---the problem of a set of distributed processes reaching agreement on
a single value---is a fundamental problem in distributed systems that is both
well-studied in academia and widely implemented in industry.
Paxos~\cite{lamport1998part}, one of the earliest asynchronous consensus
protocols, was developed roughly 30 years ago and has since become the de-facto
standard in industry~\cite{burrows2006chubby, chandra2007paxos,
baker2011megastore, corbett2013spanner}. Since that time, Paxos and Multi-Paxos
(the state machine replication protocol built on Paxos) have been improved
along three core dimensions: latency, throughput, and simplicity.

First, the latency of Paxos---i.e.\ the minimum number of message delays
between when a value is proposed by a client and when it is chosen by the
protocol---is higher than necessary~\cite{lamport2006lower}. Fast
Paxos~\cite{lamport2006fast} improves Paxos' latency to its theoretical minimum
by allowing clients to propose commands directly to acceptors.
%
Second, the throughput of Multi-Paxos is bottlenecked by the throughput of a
single leader. Because all clients send messages directly to a leader,
Multi-Paxos can only process requests as fast as the leader can. Fast Paxos
partially resolves this problem by allowing clients to bypass the leader, but
this leads to high conflict rates, lowering the throughput of the protocol.
Generalized Paxos~\cite{lamport2005generalized} and GPaxos~\cite{sutra2011fast}
reduce the number of conflicts by taking advantage of the commutativity of
state machine commands, but still rely on a single arbiter to resolve conflicts
when they arise. EPaxos~\cite{moraru2013there, moraru2013proof} and
Caesar~\cite{arun2017speeding} are both fully leaderless and improve on
Generalized Paxos by not relying on a single process either during normal
processing or conflict resolution.
%
Third, Paxos and Multi-Paxos have developed a reputation for being overly
complicated, leading to a number of publications attempting to clarify the
protocols~\cite{lamport2001paxos, lampson2001abcd, mazieres2007paxos,
van2015paxos} and a number of consensus protocols touted as simpler
alternatives~\cite{ongaro2014search, rystsov2018caspaxos}.

Despite the large body of work focused on improving Paxos, no consensus
protocol has claimed the trifecta of low latency, high throughput, and
simplicity. Existing protocols typically sacrifice one of these features for
the other two. In this paper, we present Bipartisan Paxos (or BPaxos, for
short): a family of asynchronous consensus protocols that accomplish all three.
The BPaxos protocols can commit a command in two message delays (the
theoretical minimum) and are fully leaderless. They do not depend on a
distinguished leader for normal processing or conflict resolution. We also
employ three techniques to make the BPaxos protocols as simple as possible.
%
First, the protocols are modular. Every protocol is composed of small
subcomponents, each of which can be understood individually.
%
Second, we re-use existing algorithms to implement these subcomponents whenever
possible, reducing the cognitive burden of understanding a new protocol that is
written entirely from scratch.
%
Third, the three BPaxos protocols---Simple BPaxos, Unanimous BPaxos, and
Majority Commit BPaxos---are all incremental refinements of one another. We
begin with a very simple protocol, Simple BPaxos, and then slowly increase the
complexity. This allows us to decouple the nuances of the protocols,
understanding each in isolation.
