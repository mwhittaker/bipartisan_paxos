\section{Introduction}

% "Multi-leader"
% "Generalized"
% "Clients can drive it"

Consensus---the task of a set of distributed processes reaching agreement on a
single value---is a fundamental problem in distributed systems that is both
well-studied in academia and widely implemented in industry.
Paxos~\cite{lamport1998part}, one of the earliest asynchronous consensus
protocols, was developed roughly 30 years ago and has since become the de-facto
standard in industry~\cite{burrows2006chubby, chandra2007paxos,
baker2011megastore, corbett2013spanner}. Since their inception, Paxos and
Multi-Paxos (the state state machine replication protocol built on Paxos) have
been improved along three core dimensions: latency, throughput, and simplicity.

First, the latency of Paxos---i.e.\ the minimum number of network delays
between when a value is proposed by a client and when it is chosen by the
protocol---is higher than necessary~\cite{lamport2006lower}. Fast
Paxos~\cite{lamport2006fast} improves Paxos' latency to its theoretical minimum
by allowing clients to propose commands directly to acceptors.
%
Second, the throughput of Multi-Paxos is bottlenecked by the throughput of a
single leader. Because all clients send messages directly to a leader,
Multi-Paxos can only process requests as fast as the leader can. Fast Paxos
partially resolves this problem by allowing clients to bypass the leader, but
this leads to high conflict rates, lowering the throughput of the protocol.
Generalized Paxos~\cite{lamport2005generalized} and GPaxos~\cite{sutra2011fast}
reduce the number of conflicts by taking advantage of the commutativity of
state machine commands, but still rely on a single arbiter to resolve conflicts
when they arise. EPaxos~\cite{moraru2013there, moraru2013proof} and
Caesar~\cite{arun2017speeding} are both fully leaderless and do not rely on a
single process either during normal processing or conflict resolution.
%
Third, Paxos and Multi-Paxos have developed a reputation for being overly
complicated, leading to a number of publications attempting to clarify the
protocols~\cite{lamport2001paxos, lampson2001abcd, mazieres2007paxos,
van2015paxos} and a number of consensus protocols touted as simpler
alternatives~\cite{ongaro2014search, rystsov2018caspaxos}.

% Introduction
%  Async consensus is theoretically interesting but also practical importance.
%  One of the earliest consensus protocols and the most common in industry is
%    multipaxos.
%  MultiPaxos is bad because it has a single bottlenecked leader.
%  Algorithms were invented to overcome this: Mencius, EPaxos.
%  Motivate why
% BPaxos
%  Introduce

\begin{itemize}
  \item
    Asynchronous consensus is theoretically interesting and also practically
    very important.

  \item
    One of the earliest asynchronous consensus algorithms, MultiPaxos, is also
    the most commonly used one in industry. Cite a bunch of papers using Paxos.

  \item
    There are some problems with Paxos in practice, and there's been a lot of
    research on fixing them.

  \item
    High commit delay. Addressed by Fast Paxos.

  \item
    Unnecessary contention. Addressed by Generalized Paxos, GPaxos.

  \item
    Low throughput because of centralization. Addressed by Mencius, EPaxos,
    Caesar.

  \item
    Complexity and lack of modularity. More complex protocols are not built up
    of simpler protocols like how MultiPaxos is composed of Paxos. Partially
    addressed by Raft.

  \item
    We present a family of asynchronous consensus algorithms that aims to
    accomplish all of these things. It is modular, builds up complex protocol
    from simple ones, spans different trade-offs, clarifies existing research.
\end{itemize}

A table showing the following metrics for the various BPaxos variants and maybe
EPaxos.
\begin{itemize}
  \item Quorum sizes for $n$
  \item Quorum sizes for $3$
  \item Fast Quorum sizes for $n$
  \item Fast Quorum sizes for $3$
  \item Best case commit delay (from a client)
  \item What else?
\end{itemize}
