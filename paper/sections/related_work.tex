\section{Related Work}

\paragraph{Egalitarian Paxos}
Egalitarian Paxos, or EPaxos, is a leaderless consensus algorithm that aims to
achieve optimal commit latency and high throughput~\cite{moraru2013there,
moraru2013proof}. Many of the core ideas behind BPaxos were taken directly from
EPaxos. For example, EPaxos nodes construct directed graphs of commands and execute commands in reverse topological order one
strongly connected component at a time. BPaxos borrows this execution model
directly, formalizing it using notions of generalized consensus, dependency
graphs, eligible suffixes, and condensations.  EPaxos also maintains
\invref{ConsensusInvariant} and \invref{ConflictInvariant}, the two invariants
core to the BPaxos protocols.

However, EPaxos and the BPaxos protocols differ in the following ways.
%
First, EPaxos requires at least three message delays to commit a command,
whereas Unanimous BPaxos and Majority Commit BPaxos only require two (the
theoretical minimum).
%
Second, the BPaxos protocols considers a value $v$ chosen on the fast path
if there exists a fast quorum of acceptors that voted for $v$ in round $0$.
EPaxos, on the other hand, considers a value $v$ chosen on the fast path in
instance $I$ only if the leader of instance $I$ receives a fast quorum of votes
for $v$ in round $0$.  That is, in EPaxos, an instance's leader has the
ultimate authority on whether a value is chosen on the fast path.
%
Third, EPaxos has smaller fast quorums of size $f + \floor{\frac{f+1}{2}}$
compared to Majority Commit BPaxos' fast quorums of size $\SuperQuorumSize$.
%
Fourth, Majority Commit BPaxos considers a value $v = (x, \deps{I})$ chosen on
the fast path only if there exists some fast quorum $\FastQuorum$ of acceptors
that voted for $v$ in round $0$ and if for every $I' \in \deps{I}$, there
exists a quorum $\Quorum' \subseteq \FastQuorum$ of acceptors that know $I'$ is
chosen. EPaxos has a similar condition, but only requires that one acceptor in
$\FastQuorum$ know that $I'$ is chosen, rather than a quorum.

In summary, we can think of EPaxos as one protocol in the BPaxos family.
Compared to the BPaxos protocols presented in this paper, EPaxos has smaller
quorum sizes and more flexible conditions under which a value can be chosen on
the fast path. But, it has suboptimal commit latency and is less modular.

\paragraph{A Family of Leaderless Generalized Consensus Algorithms}
In~\cite{losa2016brief}, Losa et al.\ briefly announce a family of leaderless
generalized consensus algorithms. Losa et al.\ propose a generic generalized
consensus algorithm that is structured as the composition of a generic
dependency-set algorithm and a generic map-agreement algorithm. The invariants
of the dependency-set algorithm are essentially identical to
\invref{ConflictInvariant}, and the invariants of the map-agreement algorithm
are essentially identical to \invref{ConsensusInvariant}. Moreover, the example
implementations of these two algorithms given in~\cite{losa2016brief} form an
algorithm very similar to Simple BPaxos. Our paper builds on this body of work
by introducing Incorrect BPaxos, Unanimous BPaxos, Deadlock BPaxos, and
Majority Commit BPaxos.

\paragraph{Caesar}
Caesar~\cite{arun2017speeding} is a consensus algorithm that is very similar to
EPaxos and Majority Commit BPaxos. Caesar distinguishes itself by allowing a
command to be chosen on the fast path even if there does \emph{not} exist a
fast quorum of nodes that agree on the command's dependencies. Instead, Caesar
assigns timestamps to commands and only requires that a fast quorum of nodes
agree on a command's timestamp (as opposed to its dependencies). Doing so,
Caesar implements generalized consensus without maintaining
\invref{ConsensusInvariant}. Caesar achieves optimal commit latency and is
completely leaderless, but it is complicated by the fact that it does not
maintain \invref{ConsensusInvariant} and is not modular. As an interesting
avenue of future work, we would like to better understand the relationship
between Caesar and the BPaxos protocols.

\paragraph{Generalized Paxos and Variants}
Generalized Paxos~\cite{lamport2005generalized} exploits the commutativity of
state machine commands to reduce the number of conflicts that arise in
non-generalized consensus algorithms (e.g., Fast Paxos). Generalized Paxos has
optimal commit latency and is leaderless during normal processing, but when a
collision occurs (i.e., when acceptors disagree on the ordering of conflicting
commands), a distinguished leader is responsible for arbitrating collision
recovery. This leader becomes a bottleneck during recovery.
%
GPaxos~\cite{sutra2011fast} is a variant of Generalized Paxos that resolves
collisions with lower latency. Generalized Paxos requires at least four message
delays to recover a collision. GPaxos' two-step recovery requires at least two,
but still relies on a centralized leader. GPaxos' one-step recovery requires
only one message delay and distributes recovery among a set of acceptors.
However, GPaxos still increments the round number during recovery which affects
all commands, not just those that collided. The Bipartisan Paxos protocols
completely decouple the recovery of independent commands.

\paragraph{High Throughput Paxos Variants}
S-Paxos~\cite{biely2012s} is a Paxos variant that decouples control flow from
data flow. While S-Paxos still relies on a single leader for command ordering,
command dissemination is distributed across all nodes in the protocol. However,
S-Paxos does not achieve optimal commit latency and is not generalized.
%
Mencius~\cite{mao2008mencius} is a Multi-Paxos variant in which replicated log
entries are assigned round-robin to a set of leaders. This is in contrast to
Multi-Paxos where a single leader owns all of the log entries. However, Mencius
is not generalized. A leader may have to wait to hear from every other leader
before executing a command even if the command does not conflict with any other
command.

\paragraph{Other Paxos Variants}
Flexible Paxos~\cite{howard2016flexible} is a Paxos variant that takes
advantage of the fact that phase 1 and phase 2 quorums must intersect, but
phase 1 quorums do not have to intersect with other phase 1 quorums and phase 2
quorums do not have to intersect with other phase 2 quorums.
%
Speculative Paxos~\cite{ports2015designing} is a Paxos variant in which
replicas speculatively execute state machine commands before they are known to
be chosen. If commands are mostly ordered, then speculative execution can
improve performance.
%
We believe that flexible quorum sizes and speculative execution could both be
added to the BPaxos protocols to further improve their performance.
