\section{Related Work}

\paragraph{Egalitarian Paxos}
Egalitarian Paxos, or EPaxos, is a leaderless consensus algorithm that aims to
achieve optimal commit latencies and high throughput~\cite{moraru2013there,
moraru2013proof}. Many of the core ideas behind BPaxos were taken directly from
EPaxos. For example, EPaxos nodes construct directed graphs of instances
labelled with commands, executing commands in reverse topological order one
strongly connected component at a time. BPaxos borrows this execution model
directly, formalizing it using notions of generalized consensus, dependency
graphs, eligible suffixes, and condensations. Moreover, EPaxos also maintains
\invref{ConsensusInvariant} and \invref{ConflictInvariant}, relying heavily on
both for correctness. Finally, Unanimous BPaxos is very similar to the basic
EPaxos protocol presented in~\cite{moraru2013proof}, while Majority Commit
BPaxos is very similar to the full EPaxos protocol.

However, BPaxos and EPaxos differ in the following ways.
%
First, EPaxos does not achieve optimal commit latency. In the best case, EPaxos
commits a command after three message delays instead of the optimal two. EPaxos
has suboptimal commit latency because a client must forward its command to an
EPaxos node, which then acts as the command's leader. An EPaxos client cannot
initiate the EPaxos protocol itself.
%
Second, BPaxos considers a value $v = (x, \deps{I})$ chosen on the fast path
only if there exists a fast quorum of acceptors that voted for $v$ in round
$0$. EPaxos, on the other hand, considers a value $v$ chosen on the fast path
of instance $I$ only after the leader of instance $I$ receives votes for $v$
from a fast quorum of other EPaxos nodes. That is, in EPaxos, an instance's
leader has the ultimate authority on whether a value is chosen on the fast
path.
%
Third, because of the previous two differences, EPaxos has smaller fast quorums
of size $f + \floor{\frac{f+1}{2}}$ compared to BPaxos' fast quorums of size
$\SuperQuorumSize$.
%
Fourth, Majority Commit BPaxos considers a value $v = (x, \deps{I})$ chosen on
the fast path only if there exists some fast quorum $\FastQuorum$ of acceptors
that voted for $v$ in round $0$ and if for every $I' \in \deps{I}$, there
exists a quorum $\Quorum' \subseteq \FastQuorum$ of acceptors that know $I'$ is
chosen.  EPaxos has a similar condition, but only requires that one acceptor in
$\FastQuorum$ know that $I'$ is chosen, rather than a quorum.

In summary, we can characterize EPaxos as one protocol in the BPaxos family. It
has smaller quorum sizes and more flexible conditions under which a value can
be chosen on the fast path, but it has suboptimal commit latency and is less
modular than the BPaxos protocols.

\paragraph{A Family of Leaderless Generalized Consensus Algorithms}
In~\cite{losa2016brief}, Losa et al.\ briefly announce a family of leaderless
generalized consensus algorithms. Losa et al.\ propose a generic generalized
consensus algorithm that is structured as the composition of a generic
dependency-set algorithm and a generic map-agreement algorithm. The invariants
of the dependency-set algorithm are essentially identical to
\invref{ConflictInvariant}, and the invariants of the map-agreement algorithm
are essentially identical to \invref{ConsensusInvariant}. Moreover, the example
implementations of these two algorithms given in~\cite{losa2016brief} form an
algorithm very similar to Simple BPaxos. Our paper builds on this body of work
by introducing Incorrect BPaxos, Unanimous BPaxos, Deadlock BPaxos, and
Majority Commit BPaxos.

\paragraph{Caesar}
Caesar~\cite{arun2017speeding} is a consensus algorithm that is very similar to
EPaxos and Majority Commit BPaxos. Caesar distinguishes itself by allowing a
command to be chosen on the fast path even if there does not exist a fast
quorum of nodes that agree on the command's dependencies. Instead, Caesar
assigns timestamps to commands and only requires that a fast quorum of nodes
agree on the timestamp. Doing so, Caesar implements generalized consensus
without maintaining \invref{ConsensusInvariant}. Caesar achieves optimal commit
latency and is completely leaderless, but it is complicated by the fact that it
does not maintain \invref{ConsensusInvariant} and is not modular. As an
interesting avenue of future work, we would like to better understand the
relationship between Caesar and the BPaxos protocols.

\paragraph{Other Paxos Variants}
TODO
