\section{Related Work}

\paragraph{Egalitarian Paxos}
Many of the core ideas behind BPaxos were taken directly from Egalitarian Paxos
(EPaxos)~\cite{moraru2013there, moraru2013proof}. For example, EPaxos nodes
construct directed graphs of commands and execute commands in reverse
topological order one strongly connected component at a time. BPaxos borrows
this execution model directly (with a bit of formalization). However, EPaxos
and the BPaxos protocols also have a number of fundamental differences.  For
example, EPaxos requires at least three message delays to commit a command.  In
short, we can think of EPaxos as one protocol in the BPaxos family. See
\secref{EPaxosAndBPaxos} for a more detailed comparison.

\paragraph{A Family of Leaderless Generalized Consensus Algorithms}
In~\cite{losa2016brief}, Losa et al.\ propose a generic generalized consensus
algorithm that is structured as the composition of a generic dependency-set
algorithm and a generic map-agreement algorithm. The invariants of the
dependency-set and map-agreement algorithm are essentially identical to
\invref{ConflictInvariant} and \invref{ConsensusInvariant}. Moreover, the
example implementations of these two algorithms given in~\cite{losa2016brief}
form an algorithm very similar to Simple BPaxos. Our paper builds on this body
of work by introducing Incorrect BPaxos, Unanimous BPaxos, Deadlock BPaxos, and
Majority Commit BPaxos.

\paragraph{Caesar}
Caesar~\cite{arun2017speeding} is a consensus algorithm that is very similar to
EPaxos and Majority Commit BPaxos. Caesar distinguishes itself by allowing a
command to be chosen on the fast path even if there does \emph{not} exist a
fast quorum of nodes that agree on the command's dependencies. Instead, Caesar
assigns timestamps to commands and only requires that a fast quorum of nodes
agree on a command's timestamp (as opposed to its dependencies). Doing so,
Caesar implements generalized consensus without maintaining
\invref{ConsensusInvariant}. In the future, we would like to better understand
the relationship between Caesar and the BPaxos protocols.

\paragraph{Generalized Paxos and Variants}
Generalized Paxos~\cite{lamport2005generalized} exploits the commutativity of
state machine commands to reduce the number of conflicts that arise in
non-generalized consensus algorithms. However, when a collision occurs, a
distinguished leader is responsible for arbitrating collision recovery.
%
GPaxos~\cite{sutra2011fast} is a variant of Generalized Paxos that resolves
collisions with lower latency. GPaxos' one-step recovery requires only one
message delay and distributes recovery among a set of acceptors. However,
GPaxos still increments the round number during recovery which affects all
commands, not just those that collided.

\paragraph{Complementary Paxos Variants}
S-Paxos~\cite{biely2012s} is a Paxos variant that decouples control flow from
data flow. Flexible Paxos~\cite{howard2016flexible} is a Paxos variant that
allows for flexible and dynamic quorum sizes. Speculative
Paxos~\cite{ports2015designing} is a Paxos variant in which replicas
speculatively execute state machine commands before they are known to be
chosen. We believe that the BPaxos protocols could benefit from ideas in all
three papers.
