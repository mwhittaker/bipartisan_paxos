\section{Deadlock BPaxos}
In this section, we present \defword{Deadlock BPaxos}, a slight tweak to
Unanimous BPaxos. Like Unanimous BPaxos, Deadlock BPaxos can choose a command
in one round trip, but unlike Unanimous BPaxos, Deadlock BPaxos only requires
fast quorums of size $\SuperQuorumSize$ (the same as Fast Paxos).
%
While Deadlock BPaxos is safe, it is not very live. There are failure-free
situations in which Deadlock Paxos can deadlock. Thus, Deadlock BPaxos, like
Unsafe BPaxos, is a purely pedagogical protocol. In the next section, we fix
Deadlock BPaxos's lack of liveness.

\subsection{Pruned Dependencies}
We discuss Deadlock BPaxos momentarily, but first we pause to understand one of
the key invariants that it maintains. Recall that in order to preserve
\invref{ConflictInvariant}, both Simple BPaxos and Unanimous BPaxos maintain
\invref{SimpleBpaxosInvariant}.
%
Deadlock BPaxos does not maintain this invariant. Instead, it maintains an
invariant that is slightly weaker but still sufficient to imply
\invref{ConflictInvariant}. The motivation for the weaker invariant is as
follows.

Imagine a Deadlock BPaxos node $b_i$ sends command $x$ to the dependency
service in instance $I$, and the dependency service responds with $(I, x,
\deps{I})$. Let $I' \in \deps{I}$ be one of $I$'s dependencies. Assume that
that $b_i$ knows that $I'$ has been chosen with command $y$ and dependencies
$\deps{I'}$. Further assume that $I \in \deps{I'}$. In this case, there is no
need for $b_i$ to include $I'$ in the dependencies of $I$!
\invref{ConflictInvariant} asserts that if two instances $I$ and $I'$ are
chosen with conflicting commands $x$ and $y$, then either $I \in \deps{I'}$ or
$I' \in \deps{I}$. Thus, if $I'$ has already been chosen with a dependency on
$I$, then there is no need to propose $I$ with a dependency on $I'$.
%
Similarly, if $I'$ has been chosen with $\noop$, then there is no need to
propose $I$ with a dependency to $I'$ because $x$ and $\noop$ do not conflict.

Let $(I, x, \deps{I})$ be a response from the dependency service. Let $P$ be a
set of instances $I'$ such that $I'$ has been chosen with $\noop$ or $I'$ has
been chosen with $I \in \deps{I'}$. We call $\deps{I} - P$ the \defword{pruned
dependencies} of $I$ with respect to $P$. Deadlock BPaxos maintains the
following invariant.

\begin{invariant}\invlabel{PrunedDependencies}
  For every instance $I$, a value $v$ is chosen in instance $I$ only if $v =
  (\noop, \emptyset)$ or if $(I, x, \deps{I})$ is a response from the
  dependency service and $v = (I, x, \deps{I} - P)$ where $\deps{I} - P$ are
  the pruned dependencies of $I$ with respect to some set $P$.
\end{invariant}

\invref{DependencyService} and \invref{PrunedDependencies} imply
\invref{ConflictInvariant}.
\begin{proof}
  Consider two conflicting commands $x$ and $y$ chosen in instances $I_x$ and
  $I_y$. $x$ and $y$ conflict, so neither is $\noop$. Thus, by
  \invref{PrunedDependencies}, $I_x$ is chosen with pruned dependencies
  $\deps{I_x} - P_x$ and $I_y$ is chosen with pruned dependencies $\deps{I_y} -
  P_y$ derived from dependencies $\deps{I_x}$ and $\deps{I_y}$ computed by the
  dependency service. We want to show that $I_x \in \deps{I_y} - P_y$ or $I_y
  \in \deps{I_x} - P_x$, or both.
  %
  By \invref{DependencyService}, either $I_y \in \deps{I_x}$ or $I_x \in
  \deps{I_y}$ or both. Without loss of generality, assume $I_y \in \deps{I_x}$.
  If $I_y$ is also in $\deps{I_x} - P_x$, then we're done. Otherwise, $I_y$ has
  been pruned from $\deps{I_x}$ (i.e.\ $I_y \in P_x$). This happens only if $y$
  is a $\noop$ or if $I_y$ has been chosen with $I_x \in \deps{I_y} - P_y$. $y$
  is not a $\noop$, so $I_x \in \deps{I_y} - P_y$.
\end{proof}

\subsection{The Protocol}
Deadlock BPaxos is identical to Unanimous BPaxos except for the
following modifications.
%
First, Deadlock BPaxos fast quorums are of size $\SuperQuorumSize$.
%
Second, every Fast Paxos acceptor maintains a partial BPaxos graph in exactly
the same way as the set of Deadlock BPaxos nodes. When an acceptor learns that
an instance $I$ has been chosen with value $(x, \deps{I})$, it adds $I$ to its
partial BPaxos graph labelled with $x$ and with edges to every instance in
$\deps{I}$.  We'll see momentarily that whenever a Deadlock BPaxos node $b_i$
sends a phase 2a message to acceptors with value $v = (x, \deps{I} - P)$ for
some pruned dependencies $\deps{I} - P$, $b_i$ includes $P$ and all of the
values chosen in $P$ in the phase 2a message. Thus, when an acceptor receives a
phase 2a message, it updates its partial BPaxos graph with the values chosen in
$P$.
%
Third, as discussed above, Deadlock BPaxos maintains
\invref{PrunedDependencies} instead of \invref{SimpleBpaxosInvariant}.
%
Fourth and most substantially, when a Deadlock BPaxos node $b_i$ executes
\lineref{FastPaxosTweakCase3} of \algoref{FastPaxosTweak} for instance $I$, it
performs a much more sophisticated procedure to determine if some value $v \in
V$ may have been chosen in round $0$. This procedure is shown in
\algoref{DeadlockBPaxos}.

In line 1, $b_i$ determines whether there exists some $v \in V$ that satisfies
$O4(v)$. $O4(v)$ is a necessary condition for $v$ to have been chosen in round
$0$, so if no such $v$ exists, then no value could have been chosen in round
$0$. Thus, in line 2, $b_i$ is free to propose any value that maintains
\invref{PrunedDependencies}.
%
Otherwise, there does exist a $v = (x, \deps{I}) \in V$ satisfying $O4(v)$. As
we saw with Unsafe BPaxos, if $v$ was maybe chosen in round $0$, then $b_i$
\emph{must} propose $v$ in order to maintain \invref{ConsensusInvariant}. But
simultaneously to maintain \invref{ConflictInvariant}, $b_i$ must \emph{not}
propose $v$ unless $\deps{I}$ was computed by the dependency service. Unanimous
BPaxos resolved this tension by increasing fast quorum sizes. Deadlock BPaxos
resolves the tension by performing a more sophisticated recovery procedure.  In
particular, $b_i$ does a bit of detective work to conclude either that $v$ was
definitely not chosen in round $0$ (in which case, $b_i$ can propose a
different value) or that $\deps{I}$ happens to be a pruned set of dependencies
(in which case, $b_i$ is safe to propose $v$).

\begin{wrapfigure}{r}{3.6in}
  \begin{minipage}{3.6in}
    \begin{algorithm}[H]
      \caption{Deadlock BPaxos recovery of instance $I$ by $b_i$}%
      \algolabel{DeadlockBPaxos}
      \begin{algorithmic}[1]
        \If{$\nexists v \in V$ such that $O4(v)$}
          \State send any value satisfying \invref{PrunedDependencies}
        \Else{}
          \State $(x, \deps{I}) \gets v$
        \EndIf{}
        \State send $(I, x)$ to quorum $\Quorum$ of dependency service nodes
        \State $(I, x, \deps{I}_\Quorum) \gets$ the dependency service response

        \State
        \State $P \gets \emptyset$
        \For{$I' \in \deps{I}_\Quorum - \deps{I}$}
          \If{$b_i$ doesn't know if $I'$ is chosen}
            \State recover $I'$, blocking until $I'$ is recovered
          \EndIf
          \If{$I'$ chosen with $\noop$ or with $I \in \deps{I'}$}
            \State $P \gets P \cup \set{I'}$
          \Else{}
            \State contact a quorum $\Quorum'$ of acceptors
            \If{an acceptor in $\Quorum'$ knows $I$ is chosen}
              \State abort recovery; $I$ has already been chosen
            \Else{}
              \State send any value satisfying \invref{PrunedDependencies}
            \EndIf{}
          \EndIf{}
        \EndFor{}
        \State send $(x, \deps{I})$ along with values chosen in $P$
      \end{algorithmic}
    \end{algorithm}
  \end{minipage}
\end{wrapfigure}

In lines $4$ and $5$, $b_i$ sends $(I, x)$ to the dependency service nodes
co-located with the acceptors in $\Quorum$ and receives the corresponding
dependency service reply $(I, x, \deps{I}_\Quorum)$%
\footnote{%
  Note that $(I, x)$ messages can be piggybacked on the phase 1a messages
  previously sent by $b_i$. This eliminates the extra round trip.
}.
$\deps{I}_\Quorum$ is the union of the dependencies computed by these
co-located dependency service nodes. Moreover, a majority of these co-located
dependency service nodes computed the dependencies of $I$ to be $\deps{I}$
(because $O4(v)$). Thus, $\deps{I} \subseteq \deps{I}_\Quorum$.

Next, $b_i$ enters a for loop in an attempt to prune $\deps{I}_\Quorum$ until
it is equal to $\deps{I}$. That is, $b_i$ attempts to construct a set of
instances $P$ such that $\deps{I} = \deps{I}_\Quorum - P$ is a set of pruned
dependencies. For every, $I' \in \deps{I}_\Quorum - \deps{I}$, $b_i$ first
recovers $I'$ if $b_i$ does not know if a value has been chosen in instance
$I'$. After recovering $I'$, assume $b_i$ learns that $I'$ is chosen with value
$(x', \deps{I'})$. If $x' = \noop$ or if $I \in \deps{I'}$, then $b_i$ can
safely prune $I'$ from $\deps{I}_\Quorum$, so it adds $I'$ to $P$.
%
Otherwise, $b_i$ contacts some quorum $\Quorum'$ of acceptors. If any acceptor
$a_j$ in $\Quorum'$ knows that instance $I$ has already been chosen, then $b_i$
can abort the recovery of $I$ and retrieve the chosen value directly from
$a_j$. Otherwise, in line 19, $b_i$ concludes that $v$ was not chosen in round
$0$ and is free to propose any value that maintains
\invref{PrunedDependencies}. We will explain momentarily why $b_i$ is able to
make such a conclusion. It is not obvious.
%
Finally, if $b_i$ exits the for loop, then it has successfully pruned
$\deps{I}_\Quorum$ into $\deps{I}_\Quorum - P = \deps{I}$ and can safely
propose it. As described above, when $b_i$ sends a phase 2a message with value
$v$, it also includes the values chosen in every instance in $P$.

We now return to line 19 and explain how $b_i$ is able to conclude that $v$ was
not chosen in round $0$. On line 19, $b_i$ has already concluded that $I'$ was
not chosen with $\noop$ and that $I \notin \deps{I'}$. By
\invref{PrunedDependencies}, $\deps{I'} = \deps{I'}_{\mathcal{R}} - P'$ is a
set of pruned dependencies where $\deps{I'}_{\mathcal{R}}$ is a set of
dependencies computed by a quorum $\mathcal{R}$ of dependency service nodes.
Because $I \notin \deps{I'}_{\mathcal{R}} - P'$, either $I \notin
\deps{I'}_{\mathcal{R}}$ or $I \in P'$.
%
$I$ cannot be in $P'$ because if $I'$ were chosen with dependencies
$\deps{I'}_{\mathcal{R}} - P'$, then some quorum of acceptors would have
received $P'$ and learned that $I$ was chosen. But, when $b_i$ contacted the
quorum $\Quorum'$ of acceptors, none knew that $I$ was chosen, and any two
quorums intersect.
%
Thus, $I \notin \deps{I'}_{\mathcal{R}}$. Thus, every dependency service node
in $\mathcal{R}$ processed instance $I'$ before instance $I$. If not, then a
dependency service node in $\mathcal{R}$ would have computed $I$ as a
dependency of $I'$. However, if every dependency service node in $\mathcal{R}$
processed $I'$ before $I$, then there cannot exist a fast quorum of dependency
service nodes that processed $I$ before $I'$. In this case, $v = (x, \deps{I})$
could not have been chosen in round $0$ because it necessitates a fast quorum
of dependency service nodes processing $I$ before $I'$ because $I' \notin
\deps{I}$.

\subsection{Safety}
Deadlock BPaxos maintains \invref{ConsensusInvariant} by implementing Fast
Paxos with the phase 2a tweak outlined in \algoref{FastPaxosTweak} and expanded
in \algoref{DeadlockBPaxos}. As described above, when a node $b_i$ executes
\algoref{DeadlockBPaxos}, it makes sure to propose a value $v$ if $v$ was maybe
chosen in round $0$. $b_i$ proposes a different value in phase 2a only if it
concludes that no value was chosen in round $0$. Thus, \algoref{DeadlockBPaxos}
faithfully implements \algoref{FastPaxosTweak}, and Deadlock BPaxos
successfully maintains \invref{ConsensusInvariant}.
%
As explained above, Deadlock BPaxos maintains \invref{ConflictInvariant} by
maintaining \invref{PrunedDependencies}. The proof that Deadlock BPaxos
maintains \invref{PrunedDependencies} is a straightforward extension of the
proof given in \secref{UnsafeBPaxos} with a case analysis on
\algoref{DeadlockBPaxos} using the arguments presented above.

Unfortunately, while Deadlock BPaxos is safe, it is not very live. There are
certain failure-free situations in which Deadlock BPaxos can permanently
deadlock (see \appendixref{DeadlockExample} for a concrete example). The reason
for this is line 11 of \algoref{DeadlockBPaxos} in which a Deadlock BPaxos node
defers the recovery of one instance for the recovery of another. There exists
executions of Deadlock BPaxos with a chain of instances $I_1, \ldots, I_n$
where the recovery of every instance $I_i$ depends on the recovery of instance
$I_{i+1 \bmod n}$.
