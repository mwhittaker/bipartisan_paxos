\section{Unanimous BPaxos}\seclabel{UnanimousBPaxos}
\defword{Unanimous BPaxos} is identical to Unsafe BPaxos except for the
following small modifications.  First, we increase the fast quorum size from
$\SuperQuorumSize$ to $2f + 1$. Thus, choosing a value in round $0$ requires a
unanimous vote. Second, we implement Unanimous BPaxos with the Fast Paxos
tweak shown in \algoref{FastPaxosTweak} in \appendixref{FastPaxos} where value
$v$ in \lineref{FastPaxosTweakCase3} may have been chosen in round $0$ only if
every acceptor in $\Quorum$ voted for $v$ in round $0$.
%
Like Unsafe BPaxos, Unanimous BPaxos can choose a command in one round trip (in
the best case), but unlike Unsafe BPaxos, Unanimous BPaxos is safe.

We now prove that Unanimous BPaxos maintains \invref{ConsensusInvariant} and
\invref{ConflictInvariant}. Unanimous BPaxos maintains
\invref{ConsensusInvariant} trivially by using Fast Paxos. Like Simple BPaxos,
Unanimous BPaxos maintains \invref{ConflictInvariant} by maintaining
\invref{SimpleBpaxosInvariant}. We now prove that Unanimous BPaxos maintains
\invref{SimpleBpaxosInvariant}.

\begin{proof}
  If a value $v = (x, \deps{I})$ is chosen in round $0$ of instance $I$, then
  every single acceptor voted for $v$ in round $0$. An acceptor $a_j$ only
  votes for a value $v$ in round $0$, if its co-located dependency service node
  $d_j$ proposed it. Thus, every single dependency service node proposed $(x,
  \deps{I})$, so $(I, x, \deps{I})$ is a valid response from the dependency
  service.

  Otherwise, $v = (x, \deps{I})$ is chosen in round $i > 0$. In order for $v$
  to be chosen in round $i$, a Unanimous BPaxos node $b_j$ must have proposed
  $v$ in round $i$. We perform a case analysis on \algoref{FastPaxosTweak}, the
  logic that $b_j$ uses to select the value $v$.
  %
  As described in \secref{UnsafeBPaxos}, if $b_j$ executes lines
  \lineref{FastPaxosTweakCase1Code} or \lineref{FastPaxosTweakCase4Code}, then
  $b_j$ makes sure to only propose $(\noop, \emptyset)$ or $(x, \deps{I})$
  after receiving $(I, x, \deps{I})$ from the dependency service.
  %
  If $b_j$ executes \lineref{FastPaxosTweakCase3Code}, then every acceptor in
  $\Quorum$ voted for $v$ in round $0$. Thus, every dependency service node
  co-located with an acceptor in $\Quorum$ proposed $(x, \deps{I})$, so $(I, x,
  \deps{I})$ is a valid response from the dependency service.
  %
  Finally, if $b_j$ executes \lineref{FastPaxosTweakCase2Code}, a simple
  inductive argument over $i$ shows that \invref{SimpleBpaxosInvariant} holds.
\end{proof}

This concludes the proof of Unanimous BPaxos's safety, but let's not miss the
forest through the proof. Taking a step back, we see that increasing the fast
quorum size from $\SuperQuorumSize$ to $2f + 1$ fixes Unsafe BPaxos' lack of
safety by resolving the tension between maintaining \invref{ConsensusInvariant}
and \invref{ConflictInvariant}. With Unsafe BPaxos, there were scenarios in
which a BPaxos node $b_i$ was simultaneously forced to propose a value $v$ to
maintain \invref{ConsensusInvariant} and forced \emph{not} to propose $v$ to
maintain \invref{ConflictInvariant}. By increasing the fast quorum size, we
eliminate these scenarios.  In particular, when a Unanimous BPaxos node $b_i$
is forced to propose a value $v = (x, \deps{x})$ on
\lineref{FastPaxosTweakCase3Code}, it is guaranteed that $\deps{I}$ was
computed by the dependency service.
