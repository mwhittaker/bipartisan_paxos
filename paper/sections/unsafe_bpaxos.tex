\section{Unsafe BPaxos}\seclabel{UnsafeBPaxos}
Simple BPaxos is easy to understand, but it has suboptimal commit latency.
With active clients, it takes at least two round trips before a command is
chosen: one round trip to the dependency service and one round trip to the
consensus service. In this section, we present a purely pedagogical BPaxos
protocol called \defword{Unsafe BPaxos}. Unsafe BPaxos attempts to choose
commands in one round trip, but as the name suggests, Unsafe BPaxos is unsafe.
It does not properly implement the safety properties of generalized consensus.
Still, understanding why Unsafe BPaxos is unsafe leads to some fundamental
insights into the BPaxos protocols still to come.

\subsection{The Protocol}
Unsafe BPaxos consists of the same three components as Simple BPaxos: a set of
Unsafe BPaxos nodes, a dependency service, and a consensus service. Unsafe
BPaxos is largely identical to Simple BPaxos except for the following
differences.
%
First, we implement the consensus service using Fast Paxos (see
\secref{FastPaxos}). The consensus service is implemented as a set $a_1,
\ldots, a_{2f + 1}$ of $2f + 1$ Fast Paxos acceptors, and the $f + 1$ Unsafe
BPaxos nodes play the role of Fast Paxos leaders. We employ classic quorums of
size $\QuorumSize$ and fast quorums of size $\SuperQuorumSize$. For every
instance $I$, we let round $0$ be a fast round and every other round be a
classic round. Doing so, we can skip phase 1a, phase 1b, and phase 2a of round
$0$ and have every Fast Paxos acceptor initialized in round $0$ as if it had
received a phase 2a message with distinguished value \emph{any}. Thus, a Fast
Paxos acceptor can vote for the first proposal that it receives for instance
$I$.
%
Second, we physically co-locate the $2f + 1$ dependency service nodes and the
$2f + 1$ Fast Paxos acceptors such that dependency service node $d_i$ and Fast
Paxos acceptor $a_i$ are on the same physical machine.

As with Simple BPaxos, when an Unsafe BPaxos node $b_i$ receives a state
machine command $x$ from a client, it selects a globally unique instance $I$
and sends the tuple $(I, x)$ to the dependency service. Upon receiving $(I,
x)$, dependency service node $d_j$ computes its reply $(I, x, \deps{I}_j)$.  An
Unsafe BPaxos dependency service node behaves identically to a Simple BPaxos
dependency service node, with the one exception that $d_j$ does not return its
reply $(I, x, \deps{I}_j)$ directly to $b_i$. Instead, it proposes the value
$(x, \deps{I}_j)$ in instance $I$ to $a_j$ (the co-located Fast Paxos
acceptor). As we described above, $a_j$ votes for the first proposal that it
receives for instance $I$, so $a_j$ votes for the value $(x, \deps{I}_j)$ in
instance $I$ and relays its phase 2b vote back to $b_i$.
%
If $b_i$ receives a fast quorum of phase 2b votes for the same value $v = (x,
\deps{I}_{j_1}) = \cdots = (x, \deps{I}_{j_m})$ in instance $I$ (where $m =
\SuperQuorumSize$), then $b_i$ learns that is $v$ is chosen. $b_i$ updates its
partial BPaxos graph and informs other Unsafe BPaxos nodes.

If $b_i$ does \emph{not} receive a fast quorum of phase 2b votes for some value
$v$ in round 0, then it is unsure whether or not a value was chosen and begins
recovery for instance $I$. Another Unsafe BPaxos node $b_j$ may also begin
recovery for instance $I$ if it detects that instance $I$ has been unchosen for
some time.
%
Unsafe BPaxos node $b_i$ recovers instance $I$ by attempting to get a value
chosen in instance $I$ in a higher Fast Paxos round. That is, $b_i$ chooses a
larger round number and executes the full two phases of Fast Paxos. If $b_i$
executes \lineref{FastPaxosCase1Code} or \lineref{FastPaxosCase4Code} of
\algoref{FastPaxos}, $b_i$ can propose the value $(\noop, \emptyset)$, or it
can can contact the dependency service to see if a command $x$ has already been
proposed in instance $I$ and then propose $(x, \deps{I})$ where $\deps{I}$ is
computed by the dependency service.

\subsection{Lack of Safety}
We now explain why Unsafe BPaxos is unsafe. Consider an Unsafe BPaxos
deployment with $f = 2$ and $n = 2f + 1 = 5$.
%
Assume Unsafe BPaxos node $b_1$ receives command $x$ from a client and
Unsafe BPaxos $b_2$ receives a conflicting command $y$ from a client. $b_1$
sends $(I_x, x)$ to the dependency service, and $b_2$ sends $(I_y, y)$ to the
dependency service. Dependency service nodes $d_1$ and $d_2$ receive $(I_x, x)$
and propose $(x, \emptyset)$ in instance $I_x$ to Fast Paxos acceptors $a_1$
and $a_2$. Similar, $d_4$ and $d_5$ receive $(I_y, y)$ and propose $(y,
\emptyset)$ in instance $I_y$ to acceptors $a_4$ and $a_5$. Then, $b_1$ and
$b_2$ crash and all other messages are dropped. Unsafe BPaxos node $b_3$
then attempts to recover $I_x$. In phase 1 of Fast Paxos, $b_3$ receives phase
1b messages from $a_1$, $a_2$, and $a_3$. Because $a_1$ and $a_2$ both voted
for the value $(x, \emptyset{})$ in round $0$, $b_3$ is forced to propose the
value $(x, \emptyset)$ in phase 2 (\lineref{FastPaxosCase2} of
\algoref{FastPaxos}). Assume this value gets chosen. Then, $b_3$ recovers
$I_y$. In phase 1 of Fast Paxos, $b_3$ receives phase 1b messages from $a_3$,
$a_4$, and $a_5$. $a_4$ and $a_5$ both voted for the value $(y, \emptyset{})$
in round $0$, so $b_3$ is forced to propose the value $(y, \emptyset)$ in phase
2 (\lineref{FastPaxosCase2} of \algoref{FastPaxos}). Assume this value gets
chosen. Now, instances $I_x$ and $I_y$ have both been chosen with conflicting
commands $x$ and $y$, but neither instance depends on the other. This violates
\invref{ConflictInvariant}, and can result in two replicas executing
conflicting commands in different orders.

Unsafe BPaxos illustrates a fundamental tension between preserving
\invref{ConsensusInvariant} and preserving \invref{ConflictInvariant}.
Maintaining \invref{ConsensusInvariant} in isolation is easy (e.g., use Paxos),
and maintaining \invref{ConflictInvariant} in isolation is also easy (e.g., use
the dependency service). But, maintaining both invariants simultaneously is
tricky. There are situations, like the one in our example above, where a BPaxos
node is forced to propose a particular value $(x, \deps{I_x})$ in order to
preserve \invref{ConsensusInvariant} (e.g., because the value may have been
chosen in an earlier Fast Paxos round) and simultaneously forced \emph{not} to
propose the value in order to preserve \invref{ConflictInvariant} (e.g.,
because $\deps{I_x}$ may not have been computed by the dependency service).
%
In the next couple of sections, we'll introduce a handful of BPaxos protocols.
Each of the BPaxos protocols runs into this fundamental tension between the two
invariants and resolves the tension in a unique way.
