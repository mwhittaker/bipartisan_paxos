\section{Incorrect BPaxos}\seclabel{IncorrectBPaxos}
Simple BPaxos is easy to understand, but it has high commit latency. After a
client sends a state machine command to a Simple BPaxos node, it takes at least
two round trips before the command is committed: one round trip to the
dependency service and one round trip to the consensus service. In this
section, we present a purely pedagogical BPaxos protocol called
\defword{Incorrect BPaxos}. Incorrect BPaxos attempts to commit commands in
one round trip (in the best case), but as the name suggests, Incorrect BPaxos
is incorrect. It does not properly implement generalized consensus. Still,
understanding why Incorrect BPaxos is incorrect leads to some fundamental
insights into the BPaxos protocols still to come.

\subsection{The Protocol}
Incorrect BPaxos consists of the same three logical components as Simple
BPaxos: a set of Incorrect BPaxos nodes, a dependency service, and a consensus
service. Incorrect BPaxos is largely identical to Simple BPaxos except for the
following differences.

First, we implement the consensus service using Fast
Paxos~\cite{lamport2006fast}. The consensus service is implemented as a set
$a_1, \ldots, a_{2f + 1}$ of $2f + 1$ Fast Paxos acceptors, and the $f + 1$
Incorrect BPaxos nodes play the role of Fast Paxos leaders. We employ classic
quorums of size $\QuorumSize$ and fast quorums of size $\SuperQuorumSize$. For
every instance $I$, we let round $0$ be a fast round and every other round be a
classic round. Doing so, we can skip phase 1a, phase 1b, and phase 2a of round
$0$ and have every Fast Paxos acceptor initialized in round $0$ as if it had
received a phase 2a message with distinguished value \emph{any}. Thus, a Fast
Paxos acceptor can vote for the first proposal that it receives for instance
$I$.
%
Second, we physically co-locate the $2f + 1$ dependency service nodes and the
$2f + 1$ Fast Paxos acceptors such that dependency service node $d_i$ and Fast
Paxos acceptor $a_i$ are on the same physical machine.

As with Simple BPaxos, when an Incorrect BPaxos node $b_i$ receives a state
machine command $x$ from a client, it selects a globally unique instance $I$
and sends the tuple $(I, x)$ to the dependency service. Upon receiving $(I,
x)$, dependency service node $d_j$ computes its reply $(I, x, \deps{I}_j)$.
The Incorrect BPaxos dependency service is identical to the Simple BPaxos
dependency service, with the one exception that $d_j$ does not return its reply
$(I, x, \deps{I}_j)$ directly to $b_i$. Instead, it proposes the value $(x,
\deps{I}_j)$ in instance $I$ to $a_j$ (the co-located Fast Paxos acceptor). As
we described above, $a_j$ votes for the first proposal that it receives for
instance $I$, so $a_j$ votes for the value $(x, \deps{I}_j)$ in instance $I$
and relays its phase 2b vote back to $b_i$.

If $b_i$ receives a fast quorum of phase 2b votes for the same value $v = (x,
\deps{I}_{j_1}) = \cdots = (x, \deps{I}_{j_m})$ in instance $I$ (where $m =
\SuperQuorumSize$), then the value $v$ is chosen. $b_i$ adds instance $I$ with
command $x$ and dependencies $\deps{I}_{j_1}$ to its partial instance graph and
informs other Incorrect BPaxos nodes. Thus, in the best case, Incorrect BPaxos
can commit a value in one round trip to a fast quorum of nodes.

If $b_i$ does \emph{not} receive a fast quorum of phase 2b votes for the same
value, then it is unsure whether or not a value was chosen in instance $I$ and
begins recovery for instance $I$. Another Incorrect BPaxos node $b_j$ may also
begin recovery for instance $I$ if it detects that instance $I$ has been
unchosen for some time.
%
Incorrect BPaxos recovery of instance $I$ by node $b_i$ is simply the process
of $b_i$ attempting to get a value chosen in instance $I$ in a higher Fast
Paxos round. That is, $b_i$ chooses a larger round number and executes the full
two phases of Fast Paxos. As with Simple BPaxos, $b_i$ can propose the value
$(\noop, \emptyset)$, or it can can contact the dependency service to see if a
command $x$ has already been proposed in instance $I$ and then propose $(x,
\deps{I})$ where $\deps{I}$ is computed by the dependency service.

\subsection{Incorrectness}
We now explain why Incorrect BPaxos is incorrect. Consider an Incorrect BPaxos
deployment with $f = 2$ and $n = 2f + 1 = 5$.
%
Assume Incorrect BPaxos node $b_1$ receives command $x$ from a client and
Incorrect BPaxos $b_2$ receives a conflicting command $y$ from a client. $b_1$
sends $(I_x, x)$ to the dependency service, and $b_2$ sends $(I_y, y)$ to the
dependency service. Dependency service nodes $d_1$ and $d_2$ receive $(I_x, x)$
and propose $(x, \emptyset)$ in instance $I_x$ to Fast Paxos acceptors $a_1$
and $a_2$. Similar, $d_4$ and $d_5$ receive $(I_y, y)$ and propose $(y,
\emptyset)$ in instance $I_y$ to acceptors $a_4$ and $a_5$. Then, $b_1$ and
$b_2$ crash and all other messages are dropped. Incorrect BPaxos node $b_3$
then attempts to recover $I_x$. In phase 1 of Fast Paxos, $b_3$ receives phase
1b messages from $a_1$, $a_2$, and $a_3$. Because $a_1$ and $a_2$ both voted
for the value $(x, \emptyset{})$ in round $0$, $b_3$ is forced to propose the
value $(x, \emptyset)$ in phase 2. Assume this value gets chosen.  Then, $b_3$
recovers $I_y$. In phase 1 of Fast Paxos, $b_3$ receives phase 1b messages from
$a_3$, $a_4$, and $a_5$. $a_4$ and $a_5$ both voted for the value $(y,
\emptyset{})$ in round $0$, so $b_3$ is forced to propose the value $(y,
\emptyset)$. Assume this value gets chosen. Now, instances $I_x$ and $I_y$ have
both been chosen with conflicting commands $x$ and $y$, but neither instance
depends on the other. This violates \invref{ConflictInvariant}, leads to an
ill-formed partial instance graph, and can result in two replicas executing
conflicting commands in different orders.

% Taking a step back, Incorrect BPaxos fails to maintain
% \invref{ConflictInvariant} because of a mismatch between the invariants that
% Incorrect BPaxos wants to maintain and the invariants that Fast Paxos provides.
% Incorrect BPaxos wants to get values of the form $(x, \deps{I_x})$ chosen, but
% only if the dependencies $\deps{I_x}$ were computed by the dependency service.
% This is required to ensure that conflicting commands depend on one another.
% However Fast Paxos wants to get \emph{any} value chosen. Fast Paxos has no
% understanding of the dependency service or any notion that some specific values
% should not be chosen. In the example above, when $b_3$ received two votes for
% $(x, \emptyset)$ in round $0$, Fast Paxos determined that this value may have
% been chosen in round $0$, so it forces $b_3$ to propose it. Fast Paxos doesn't
% understand the additional requirement that Incorrect BPaxos requires: that $(x,
% \emptyset)$ should not be proposed unless a majority of dependency service
% nodes determined that $I_x$'s dependencies should be the empty set.

Incorrect BPaxos illustrates a fundamental tension between preserving
\invref{ConsensusInvariant} and preserving \invref{ConflictInvariant}.
Maintaining \invref{ConsensusInvariant} in isolation is easy (e.g., use Paxos),
and maintaining \invref{ConflictInvariant} in isolation is also easy (e.g., use
the dependency service). But, maintaining both invariants simultaneously is
tricky. There are situations, like the one in our example above, where a BPaxos
node is forced to propose a particular value $(x, \deps{I_x})$ in order to
preserve \invref{ConsensusInvariant} (e.g., because the value may have been
chosen in an earlier Fast Paxos round) and simultaneously forced \emph{not} to
propose the value in order to preserve \invref{ConflictInvariant} (e.g.,
because $\deps{I_x}$ may not have been computed by the dependency service).
%
In the next couple of sections, we'll introduce a handful of BPaxos protocols.
Each of the BPaxos protocols runs into this fundamental tension between the two
invariants, and the distinguishing feature of each protocol is how it resolves
the tension.
