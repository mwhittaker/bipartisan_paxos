\section{Fast Paxos}\appendixlabel{FastPaxosAppendix}
\subsection{Standard Fast Paxos}
Fast Paxos~\cite{lamport2006fast} is a two-phase consensus algorithm structured
around a set of clients, leaders, acceptors, and learners. For a full
description of Fast Paxos, we refer the reader to~\cite{lamport2006fast}, but
we highlight the most relevant bits of Fast Paxos here. Fast Paxos proceeds in
a series of integer-valued rounds with $0$ being the smallest round and $-1$
being a null round. Every round is classified either as a fast round or a
classic round. In phase 2a of the algorithm, a leader has to choose a value to
send to the acceptors. The logic for choosing this value is shown in
\algoref{FastPaxos} where $O4(v)$ is true if there exists a fast quorum
$\FastQuorum$ of acceptors such that every acceptor in $\FastQuorum \cap
\Quorum$ voted for $v$ in round $k$. The sizes of fast and classic quorums
ensure that at most one value $v \in V$ satisfies $O4(v)$ in any given round.
For example, Fast Paxos is commonly deployed with $n = 2f + 1$ acceptors,
classic quorums of size $f + 1$, and fast quorums of size $\SuperQuorumSize$.
In this case, a value $v \in V$ satisfies $O4(v)$ only if a set $\mathcal{A}
\subseteq \Quorum$ of $\QuorumMajoritySize$ or more acceptors voted for $v$ in
round $k$. Because $\mathcal{A}$ constitutes a majority of $\Quorum$, at most
one $v \in V$ can satisfy $O4(v)$.

To prove the correctness of Fast Paxos, it suffices to prove the statement
$P(i)$ that says that if an acceptor votes for a value $v$ in round $i$, then
no value besides $v$ has been or will be chosen in any round $j$ less than $i$.
We prove this claim by induction. $P(0)$ is trivial because there are no rounds
less than $0$. For the general case, we perform a case analysis on
\algoref{FastPaxos}, noting that an acceptor will only vote for value $v$ in
round $i$ if some leader proposes it (or proposes distinguished value
\emph{any}). First, assume $k \neq -1$. We perform a case analysis on $j$.
\begin{itemize}
  \item \textbf{Case $k < j < i$.}
    If $k < j < i$, then no value has been or will be chosen in round $j$
    because the phase 1 quorum $\Quorum$ of acceptors did not vote in any round
    larger than $k$ and promised not to vote in any round less than $i$. In
    order for a value $w$ to get chosen in round $j$, it must get a quorum
    $\Quorum_w$ of acceptors to vote for it, but $\Quorum$ and $\Quorum_w$
    intersect, so no value can be chosen in round $j$.

  \item \textbf{Case $k = j$.}
    We perform a case analysis on whether a leader executes
    \lineref{FastPaxosCase2Code}, \lineref{FastPaxosCase3Code}, or
    \lineref{FastPaxosCase4Code}.
    \begin{itemize}
      \item \textbf{Case \lineref{FastPaxosCase2Code}.}
        If $V = \set{v}$, then the quorum $\Quorum$ of acceptors have either
        voted for $v$ in round $k$ (and hence will never vote again in round
        $k$) or promised to never vote in round $k$. Thus, no other value $w$
        can receive a quorum $\Quorum_w$ of votes in round $k$.
      \item \textbf{Case \lineref{FastPaxosCase3Code}.}
        If $k$ is a classic round, then $V = \set{v}$. Thus, by virtue of not
        executing \lineref{FastPaxosCase2Code}, we know that $k$ is a fast
        round in this case. In order for a value $v$ to be chosen in fast round
        $k$, $v$ needs to receive phase 2b votes from some fast quorum
        $\FastQuorum$ of acceptors. Thus, there needs to exist some
        $\FastQuorum$ such that every acceptor in $\FastQuorum \cap \Quorum$
        voted for $v$ in round $k$. By construction of fast quorums, there can
        exist at most one such value, so no value other than $v$ could have
        been chosen in round $k$.
      \item \textbf{Case \lineref{FastPaxosCase4Code}.}
        In this case, $k$ is a fast round. $O4(v)$ is a necessary condition for
        value $v$ to be chosen in round $k$, so if no such $v$ exists, then no
        value could have been chosen in round $k$. Thus, the leader is free to
        propose any value.
    \end{itemize}

  \item \textbf{Case $j < k$.}
    We again perform a case analysis on whether a leader executes
    \lineref{FastPaxosCase2Code}, \lineref{FastPaxosCase3Code}, or
    \lineref{FastPaxosCase4Code}.
    \begin{itemize}
      \item \textbf{Case \lineref{FastPaxosCase2Code}.}
        By $P(k)$, no value other than $v$ has been or will be chosen in round
        $j$.
      \item \textbf{Case \lineref{FastPaxosCase3Code} or
                         \lineref{FastPaxosCase4Code}.}
        Let $v_1, v_2 \in V$. By $P(k)$ applied to $v_1$ and $P(k)$ applied to
        $v_2$, no value has been or will be chosen in round $j$.
    \end{itemize}
\end{itemize}

Finally, if $k = -1$, then we know that no value has been or will be chosen in
any round less than $i$ by the same line of reasoning as the first case above,
so the leader is free to propose anything. This corresponds to
\lineref{FastPaxosCase1Code}.

\subsection{Fast Paxos Tweak}
If we assume that round $0$ is a fast round and every other round is a classic
round---as we do in the BPaxos protocols---then we can simplify the standard
phase 2a algorithm shown in \algoref{FastPaxos} to the variant shown in
\algoref{FastPaxosTweak}.
%
The proof of this tweak's correctness is a simplification of the Fast Paxos
proof given above. We again prove $P(i)$ by induction. $P(0)$ is still trivial.
For the general case, we again perform a case analysis on $j$. First, assume $k
\neq -1$.  \begin{itemize}
  \item \textbf{Case $k < j < i$ or $k = j$.}
    The proof for these two cases is identical to the Fast Paxos proof above.

  \item \textbf{Case $j < k$.}
    We perform a case analysis on whether a leader executes
    \lineref{FastPaxosTweakCase2Code}, \lineref{FastPaxosTweakCase3Code}, or
    \lineref{FastPaxosTweakCase4Code}.
    \begin{itemize}
      \item \textbf{Case \lineref{FastPaxosTweakCase2Code}.}
        By $P(k)$, no value other than $v$ has been or will be chosen in round
        $j$.
      \item \textbf{Case \lineref{FastPaxosTweakCase3Code} or
                         \lineref{FastPaxosTweakCase4Code}.}
        $k = 0$, and there are no rounds less than $k$, so this case
        holds trivially.
    \end{itemize}
\end{itemize}

Again, if $k = -1$, then we know that no value has been or will be chosen in
any round less than $i$ by the same line of reasoning as the first case above.
