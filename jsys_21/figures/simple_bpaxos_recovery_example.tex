% Processes.
\tikzstyle{proc}=[draw, circle, thick, inner sep=2pt]
\tikzstyle{boxproc}=[draw, thick, minimum width=1.5cm, minimum height=1.5cm]
\tikzstyle{client}=[proc, fill=clientcolor!25]
\tikzstyle{proposer}=[proc, fill=proposercolor!25]
\tikzstyle{depservice}=[boxproc, depservicecolor, fill=depservicecolor!10]
\tikzstyle{acceptor}=[proc, fill=acceptorcolor!25]
\tikzstyle{replica}=[boxproc, replicacolor, fill=replicacolor!10]
\tikzstyle{dead}=[fill=gray, text=white]
\tikzstyle{xcolor}=[red]
\tikzstyle{ycolor}=[blue]

% Graphs
\tikzstyle{vertex}=[draw, line width=1pt, align=center, inner sep=2pt, fill=white]
\tikzstyle{vertexlabel}=[label distance=-1pt, flatred]
\tikzstyle{highlight}=[red]
\tikzstyle{executed}=[fill=flatgreen, opacity=0.2, draw opacity=1,
                      text opacity=1]
\tikzstyle{arrow}=[line width=0.75pt, -latex]

% Process labels.
\tikzstyle{proclabel}=[inner sep=0pt, align=center]

% Components.
\tikzstyle{component}=[draw, thick, flatgray, rounded corners]

% Messages and communication.
\tikzstyle{comm}=[-latex, thick]
\tikzstyle{commnum}=[fill=white, inner sep=0pt]

\begin{figure*}[ht]
  \centering
  \begin{subfigure}[c]{0.45\textwidth}
    \centering
    \begin{tikzpicture}[]
      % Processes.
      \node[depservice, label={90:$d_1$}] (d1) at (0, 4) {};
      \node[depservice, label={90:$d_2$}] (d2) at (2, 4) {};
      \node[depservice, label={90:$d_3$}] (d3) at (4, 4) {};
      \node[proposer, dead] (p1) at (1.5, 2.25) {$p_1$};
      \node[proposer] (p2) at (2.5, 1.75) {$p_2$};

      \node[acceptor] (a1) at (5, 3) {$a_1$};
      \node[acceptor] (a2) at (5, 2) {$a_2$};
      \node[acceptor] (a3) at (5, 1) {$a_3$};
      \node[replica, label={-90:$r_1$}] (r1) at (1, 0) {};
      \node[replica, label={-90:$r_2$}] (r2) at (3, 0) {};

      % Communication.
      \draw[comm, xcolor] ($(p1) - (2, 0)$) to node[commnum]{$x$} (p1);
      \draw[comm, latex-latex] (p1) to (d1.south);
      \draw[comm, latex-latex] (p1) to (d2.south);
      \draw[comm, latex-latex] (p1) to (d3.south);

      % Graphs.
      \begin{scope}[shift={(-0.25, 3.75)}]
        \node[vertex, label={[vertexlabel]90:$v_x$}] (x) at (0, 0.5) {$x$};
      \end{scope}
      \begin{scope}[shift={(1.75, 3.75)}]
        \node[vertex, label={[vertexlabel]90:$v_x$}] (x) at (0, 0.5) {$x$};
      \end{scope}
      \begin{scope}[shift={(3.75, 3.75)}]
        \node[vertex, label={[vertexlabel]90:$v_x$}] (x) at (0, 0.5) {$x$};
      \end{scope}
    \end{tikzpicture}
    \caption{%
      $p_1$ receives $x$, talks to the dependency service, and fails.
    }
    \figlabel{SimpleBPaxosRecoveryExample1}
  \end{subfigure}
  \begin{subfigure}[c]{0.45\textwidth}
    \centering
    \begin{tikzpicture}[]
      % Processes.
      \node[depservice, label={90:$d_1$}] (d1) at (0, 4) {};
      \node[depservice, label={90:$d_2$}] (d2) at (2, 4) {};
      \node[depservice, label={90:$d_3$}] (d3) at (4, 4) {};
      \node[proposer, dead] (p1) at (1.5, 2.25) {$p_1$};
      \node[proposer] (p2) at (2.5, 1.75) {$p_2$};

      \node[acceptor] (a1) at (5, 3) {$a_1$};
      \node[acceptor] (a2) at (5, 2) {$a_2$};
      \node[acceptor] (a3) at (5, 1) {$a_3$};
      \node[replica, label={-90:$r_1$}] (r1) at (1, 0) {};
      \node[replica, label={-90:$r_2$}] (r2) at (3, 0) {};

      % Graphs.
      \begin{scope}[shift={(-0.25, 3.75)}]
        \node[vertex, label={[vertexlabel]90:$v_x$}] (x) at (0, 0.5) {$x$};
        \node[vertex, label={[vertexlabel]-90:$v_y$}] (y) at (0.5, 0) {$y$};
        \draw[arrow, bend left] (y) to (x);
      \end{scope}
      \begin{scope}[shift={(1.75, 3.75)}]
        \node[vertex, label={[vertexlabel]90:$v_x$}] (x) at (0, 0.5) {$x$};
        \node[vertex, label={[vertexlabel]-90:$v_y$}] (y) at (0.5, 0) {$y$};
        \draw[arrow, bend left] (y) to (x);
      \end{scope}
      \begin{scope}[shift={(3.75, 3.75)}]
        \node[vertex, label={[vertexlabel]90:$v_x$}] (x) at (0, 0.5) {$x$};
        \node[vertex, label={[vertexlabel]-90:$v_y$}] (y) at (0.5, 0) {$y$};
        \draw[arrow, bend left] (y) to (x);
      \end{scope}

      \begin{scope}[shift={(0.75, -0.25)}]
        \node[vertex, label={[vertexlabel]90:$v_x$}] (x) at (0, 0.5) {\phantom{$x$}};
        \node[vertex, label={[vertexlabel]-90:$v_y$}] (y) at (0.5, 0) {$y$};
        \draw[arrow, bend left] (y) to (x);
      \end{scope}
      \begin{scope}[shift={(2.75, -0.25)}]
        \node[vertex, label={[vertexlabel]90:$v_x$}] (x) at (0, 0.5) {\phantom{$x$}};
        \node[vertex, label={[vertexlabel]-90:$v_y$}] (y) at (0.5, 0) {$y$};
        \draw[arrow, bend left] (y) to (x);
      \end{scope}

      % Communication.
      \draw[comm, ycolor] ($(p2) + (2, -0.25)$) to node[commnum]{$y$} (p2);
      \draw[comm, latex-latex] (p2) to (d1.south);
      \draw[comm, latex-latex] (p2) to (d2.south);
      \draw[comm, latex-latex] (p2) to (d3.south);
      \draw[comm, latex-latex] (p2) to (a1);
      \draw[comm, latex-latex] (p2) to (a2);
      \draw[comm, latex-latex] (p2) to (a3);
      \draw[comm, latex-latex] (p2) to (r1.north);
      \draw[comm, latex-latex] (p2) to (r2.north);
    \end{tikzpicture}
    \caption{$p_2$ receives $y$ and gets it chosen with a dependency on $v_x$.}
    \figlabel{SimpleBPaxosRecoveryExample2}
  \end{subfigure}

  \vspace{12pt}

  \begin{subfigure}[c]{0.45\textwidth}
    \centering
    \begin{tikzpicture}[]
      % Processes.
      \node[depservice, label={90:$d_1$}] (d1) at (0, 4) {};
      \node[depservice, label={90:$d_2$}] (d2) at (2, 4) {};
      \node[depservice, label={90:$d_3$}] (d3) at (4, 4) {};
      \node[proposer, dead] (p1) at (1.5, 2.25) {$p_1$};
      \node[proposer] (p2) at (2.5, 1.75) {$p_2$};

      \node[acceptor] (a1) at (5, 3) {$a_1$};
      \node[acceptor] (a2) at (5, 2) {$a_2$};
      \node[acceptor] (a3) at (5, 1) {$a_3$};
      \node[replica, label={-90:$r_1$}] (r1) at (1, 0) {};
      \node[replica, label={-90:$r_2$}] (r2) at (3, 0) {};

      % Graphs.
      \begin{scope}[shift={(-0.25, 3.75)}]
        \node[vertex, label={[vertexlabel]90:$v_x$}] (x) at (0, 0.5) {$x$};
        \node[vertex, label={[vertexlabel]-90:$v_y$}] (y) at (0.5, 0) {$y$};
        \draw[arrow, bend left] (y) to (x);
      \end{scope}
      \begin{scope}[shift={(1.75, 3.75)}]
        \node[vertex, label={[vertexlabel]90:$v_x$}] (x) at (0, 0.5) {$x$};
        \node[vertex, label={[vertexlabel]-90:$v_y$}] (y) at (0.5, 0) {$y$};
        \draw[arrow, bend left] (y) to (x);
      \end{scope}
      \begin{scope}[shift={(3.75, 3.75)}]
        \node[vertex, label={[vertexlabel]90:$v_x$}] (x) at (0, 0.5) {$x$};
        \node[vertex, label={[vertexlabel]-90:$v_y$}] (y) at (0.5, 0) {$y$};
        \draw[arrow, bend left] (y) to (x);
      \end{scope}

      \begin{scope}[shift={(0.75, -0.25)}]
        \node[vertex, label={[vertexlabel]90:$v_x$}] (x) at (0, 0.5) {\phantom{$x$}};
        \node[vertex, label={[vertexlabel]-90:$v_y$}] (y) at (0.5, 0) {$y$};
        \draw[arrow, bend left] (y) to (x);
      \end{scope}
      \begin{scope}[shift={(2.75, -0.25)}]
        \node[vertex, label={[vertexlabel]90:$v_x$}] (x) at (0, 0.5) {\phantom{$x$}};
        \node[vertex, label={[vertexlabel]-90:$v_y$}] (y) at (0.5, 0) {$y$};
        \draw[arrow, bend left] (y) to (x);
      \end{scope}

      % Communication.
      \draw[comm] (r1) to (p2);
    \end{tikzpicture}
    \caption{A replica notifies $p_2$ that $v_x$ is unchosen.}
    \figlabel{SimpleBPaxosRecoveryExample3}
  \end{subfigure}
  \begin{subfigure}[c]{0.45\textwidth}
    \centering
    \begin{tikzpicture}[]
      % Processes.
      \node[depservice, label={90:$d_1$}] (d1) at (0, 4) {};
      \node[depservice, label={90:$d_2$}] (d2) at (2, 4) {};
      \node[depservice, label={90:$d_3$}] (d3) at (4, 4) {};
      \node[proposer, dead] (p1) at (1.5, 2.25) {$p_1$};
      \node[proposer] (p2) at (2.5, 1.75) {$p_2$};

      \node[acceptor] (a1) at (5, 3) {$a_1$};
      \node[acceptor] (a2) at (5, 2) {$a_2$};
      \node[acceptor] (a3) at (5, 1) {$a_3$};
      \node[replica, label={-90:$r_1$}] (r1) at (1, 0) {};
      \node[replica, label={-90:$r_2$}] (r2) at (3, 0) {};

      % Graphs.
      \begin{scope}[shift={(-0.25, 3.75)}]
        \node[vertex, label={[vertexlabel]90:$v_x$}] (x) at (0, 0.5) {$x$};
        \node[vertex, label={[vertexlabel]-90:$v_y$}] (y) at (0.5, 0) {$y$};
        \draw[arrow, bend left] (y) to (x);
      \end{scope}
      \begin{scope}[shift={(1.75, 3.75)}]
        \node[vertex, label={[vertexlabel]90:$v_x$}] (x) at (0, 0.5) {$x$};
        \node[vertex, label={[vertexlabel]-90:$v_y$}] (y) at (0.5, 0) {$y$};
        \draw[arrow, bend left] (y) to (x);
      \end{scope}
      \begin{scope}[shift={(3.75, 3.75)}]
        \node[vertex, label={[vertexlabel]90:$v_x$}] (x) at (0, 0.5) {$x$};
        \node[vertex, label={[vertexlabel]-90:$v_y$}] (y) at (0.5, 0) {$y$};
        \draw[arrow, bend left] (y) to (x);
      \end{scope}

      \begin{scope}[shift={(0.75, -0.25)}]
        \node[vertex, label={[vertexlabel]90:$v_x$}] (x) at (0, 0.5) {noop};
        \node[vertex, label={[vertexlabel]-90:$v_y$}] (y) at (0.5, 0) {$y$};
        \draw[arrow, bend left] (y) to (x);
      \end{scope}
      \begin{scope}[shift={(2.75, -0.25)}]
        \node[vertex, label={[vertexlabel]90:$v_x$}] (x) at (0, 0.5) {noop};
        \node[vertex, label={[vertexlabel]-90:$v_y$}] (y) at (0.5, 0) {$y$};
        \draw[arrow, bend left] (y) to (x);
      \end{scope}

      % Communication.
      \draw[comm, latex-latex] (p2) to (a1);
      \draw[comm, latex-latex] (p2) to (a2);
      \draw[comm, latex-latex] (p2) to (a3);
      \draw[comm, latex-latex] (p2) to (r1.north);
      \draw[comm, latex-latex] (p2) to (r2.north);
    \end{tikzpicture}
    \caption{$p_2$ gets a noop chosen in $v_x$.}
    \figlabel{SimpleBPaxosRecoveryExample4}
  \end{subfigure}

  \caption{An example execution of Simple \BPaxos{} recovery ($f=1$).}%
  \figlabel{SimpleBPaxosRecoveryExample}
\end{figure*}
