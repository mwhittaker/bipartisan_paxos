\begin{figure*}
  \centering

  \tikzstyle{logentry}=[draw, inner sep=2pt, line width=1pt,
                        minimum height=20pt, minimum width=20pt]
  \tikzstyle{logindex}=[flatred, font=\small]
  \tikzstyle{vertex}=[draw, line width=1pt, align=center, minimum width=0.3in,
                      minimum height=0.2in, inner sep=2pt]
  \tikzstyle{vertexlabel}=[label distance=-1pt, flatred]
  \tikzstyle{highlight}=[red]
  \tikzstyle{arrow}=[line width=0.75pt, -latex]

  \begin{subfigure}[c]{0.12\textwidth}
    \centering
    \begin{tikzpicture}[xscale=1.5, yscale=1.5]
      \node[vertex, highlight, label={[vertexlabel]180:$v_w$}] (w) at (0, 0) {$w$};
    \end{tikzpicture}
    \caption{}
  \end{subfigure}
  \begin{subfigure}[c]{0.17\textwidth}
    \centering
    \begin{tikzpicture}[xscale=1.5, yscale=1.5]
      \node[vertex, label={[vertexlabel]180:$v_w$}] (w) at (0, 0) {$w$};
      \node[vertex, highlight, label={[vertexlabel]90:$v_x$}] (x) at ($(w) + (30:1)$) {$x$};
      \draw[arrow, highlight] (x) to (w);
    \end{tikzpicture}
    \caption{}
  \end{subfigure}
  \begin{subfigure}[c]{0.17\textwidth}
    \centering
    \begin{tikzpicture}[xscale=1.5, yscale=1.5]
      \node[vertex, label={[vertexlabel]180:$v_w$}] (w) at (0, 0) {$w$};
      \node[vertex, label={[vertexlabel]90:$v_x$}] (x) at ($(w) + (30:1)$) {$x$};
      \node[vertex, highlight, label={[vertexlabel]-90:$v_y$}] (y) at ($(w) + (-30:1)$) {$y$};
      \draw[arrow] (x) to (w);
      \draw[arrow, highlight] (y) to (w);
      \draw[arrow, highlight] (y) to (x);
    \end{tikzpicture}
    \caption{}
  \end{subfigure}
  \begin{subfigure}[c]{0.26\textwidth}
    \centering
    \begin{tikzpicture}[xscale=1.5, yscale=1.5]
      \node[vertex, label={[vertexlabel]180:$v_w$}] (w) at (0, 0) {$w$};
      \node[vertex, label={[vertexlabel]90:$v_x$}] (x) at ($(w) + (30:1)$) {$x$};
      \node[vertex, label={[vertexlabel]-90:$v_y$}] (y) at ($(w) + (-30:1)$) {$y$};
      \node[vertex, highlight, label={[vertexlabel]0:$v_z$}] (z) at ($(x) + (-30:1)$) {$z$};
      \draw[arrow] (x) to (w);
      \draw[arrow] (y) to (w);
      \draw[arrow] (y) to (x);
      \draw[arrow, highlight] (z) to (w);
      \draw[arrow, highlight] (z) to (x);
    \end{tikzpicture}
    \caption{}
  \end{subfigure}
  \begin{subfigure}[c]{0.26\textwidth}
    \centering
    \begin{tikzpicture}[xscale=1.5, yscale=1.5]
      \node[vertex, label={[vertexlabel]180:$v_w$}] (w) at (0, 0) {$w$};
      \node[vertex, highlight, label={[vertexlabel]90:$v_x$}] (x) at ($(w) + (30:1)$) {$x$};
      \node[vertex, label={[vertexlabel]-90:$v_y$}] (y) at ($(w) + (-30:1)$) {$y$};
      \node[vertex, label={[vertexlabel]0:$v_z$}] (z) at ($(x) + (-30:1)$) {$z$};
      \draw[arrow, highlight] (x) to (w);
      \draw[arrow] (y) to (w);
      \draw[arrow] (y) to (x);
      \draw[arrow] (z) to (w);
      \draw[arrow] (z) to (x);
    \end{tikzpicture}
    \caption{}
  \end{subfigure}

  \caption{%
    In subfigures (a) -- (e), we see the execution of a dependency service node
    $d_i$.
    %
    (a) $d_i$ receives command $w$ in vertex $v_w$. $d_i$ adds this vertex to
    its conflict graph and because there are no other vertices, it returns the
    dependencies $\deps{v_w} = \emptyset{}$.
    %
    (b) $d_i$ receives command $x$ in vertex $v_x$. $d_i$ adds this vertex to
    its conflict graph. $x$ conflicts with $w$, so $d_i$ adds an edge from
    $v_x$ to $v_w$ and returns the dependencies $\deps{v_x} = \set{v_w}$.
    %
    (c) $d_i$ receives command $y$ in vertex $v_y$. $d_i$ adds this vertex to
    its conflict graph. $y$ conflicts with $w$ and $x$, so $d_i$ adds an edge
    from $v_y$ to $v_w$ and from $v_y$ to $v_x$. It returns the dependencies
    $\deps{v_y} = \set{v_w, v_x}$.
    %
    (d) $d_i$ receives command $z$ in vertex $v_z$. $d_i$ adds this vertex to
    its conflict graph. $z$ conflicts with $w$ and $x$, so $d_i$ adds an edge
    from $v_z$ to $v_w$ and from $v_z$ to $v_x$. It returns the dependencies
    $\deps{v_z} = \set{v_w, v_x}$.
    %
    (e) $d_i$ receives command $x$ in vertex $v_x$. $d_i$'s graph already
    contains vertex $v_x$, so $d_i$ returns the dependencies $\deps{v_x} =
    \set{v_w}$ and does not modify its graph.
  }\figlabel{DependencyService}
\end{figure*}
