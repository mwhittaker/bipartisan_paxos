\begin{figure*}
  \centering

  \tikzstyle{logentry}=[draw, inner sep=2pt, line width=1pt,
                        minimum height=20pt, minimum width=20pt]
  \tikzstyle{logindex}=[flatred, font=\small]
  \tikzstyle{vertex}=[draw, line width=1pt, align=center, minimum width=0.3in,
                      minimum height=0.2in, inner sep=2pt]
  \tikzstyle{vertexlabel}=[label distance=-1pt, flatred]
  \tikzstyle{executed}=[fill=flatgreen, opacity=0.2, draw opacity=1,
                        text opacity=1]
  \tikzstyle{highlight}=[red]
  \tikzstyle{arrow}=[line width=0.75pt, -latex]

  \begin{subfigure}[c]{0.05\textwidth}
    \centering
    \begin{tikzpicture}[xscale=1, yscale=1.5]
      \node[vertex, highlight, executed, label={[vertexlabel]-90:$v_0$}] (ab) at (0, 0) {\texttt{a=b}};
    \end{tikzpicture}
    \caption{}\figlabel{v0}
  \end{subfigure}
  \begin{subfigure}[c]{0.15\textwidth}
    \centering
    \begin{tikzpicture}[xscale=1, yscale=1.5]
      \node[vertex, executed, label={[vertexlabel]-90:$v_0$}] (ab) at (0, 0) {\texttt{a=b}};
      \node[vertex, highlight, executed, label={[vertexlabel]-90:$v_1$}] (a2) at (1, 0.5) {\texttt{a=2}};
      \draw[arrow, highlight] (a2) to (ab);
    \end{tikzpicture}
    \caption{}\figlabel{v1}
  \end{subfigure}
  \begin{subfigure}[c]{0.25\textwidth}
    \centering
    \begin{tikzpicture}[xscale=1, yscale=1.5]
      \node[vertex, executed, label={[vertexlabel]-90:$v_0$}] (ab) at (0, 0) {\texttt{a=b}};
      \node[vertex, executed, label={[vertexlabel]-90:$v_1$}] (a2) at (1, 0.5) {\texttt{a=2}};
      \node[vertex, label={[vertexlabel]-90:$v_2$}] (b1) at (1, -0.5) {\phantom{\texttt{b=1}}};
      \node[vertex, highlight, label={[vertexlabel]-90:$v_3$}] (ba) at (2, 0) {\texttt{b=a}};
      \node[vertex, label={[vertexlabel]-90:$v_4$}] (a22) at (3, 0) {\phantom{\texttt{a=2}}};

      \draw[arrow] (a2) to (ab);
      \draw[arrow, highlight] (ba) to (ab);
      \draw[arrow, highlight] (ba) to (a2);
      \draw[arrow, highlight] (ba) to (b1);
      \draw[arrow, highlight, bend right=60] (ba) to (a22);
    \end{tikzpicture}
    \caption{}\figlabel{v3}
  \end{subfigure}
  \begin{subfigure}[c]{0.25\textwidth}
    \centering
    \begin{tikzpicture}[xscale=1, yscale=1.5]
      \node[vertex, executed, label={[vertexlabel]-90:$v_0$}] (ab) at (0, 0) {\texttt{a=b}};
      \node[vertex, executed, label={[vertexlabel]-90:$v_1$}] (a2) at (1, 0.5) {\texttt{a=2}};
      \node[vertex, highlight, executed, label={[vertexlabel]-90:$v_2$}] (b1) at (1, -0.5) {\texttt{b=1}};
      \node[vertex, label={[vertexlabel]-90:$v_3$}] (ba) at (2, 0) {\texttt{b=a}};
      \node[vertex, label={[vertexlabel]-90:$v_4$}] (a22) at (3, 0) {\phantom{\texttt{a=2}}};

      \draw[arrow] (a2) to (ab);
      \draw[arrow, highlight] (b1) to (ab);
      \draw[arrow] (ba) to (ab);
      \draw[arrow] (ba) to (a2);
      \draw[arrow] (ba) to (b1);
      \draw[arrow, bend right=60] (ba) to (a22);
    \end{tikzpicture}
    \caption{}\figlabel{v2}
  \end{subfigure}
  \begin{subfigure}[c]{0.25\textwidth}
    \centering
    \begin{tikzpicture}[xscale=1, yscale=1.5]
      \node[vertex, executed, label={[vertexlabel]-90:$v_0$}] (ab) at (0, 0) {\texttt{a=b}};
      \node[vertex, executed, label={[vertexlabel]-90:$v_1$}] (a2) at (1, 0.5) {\texttt{a=2}};
      \node[vertex, executed, label={[vertexlabel]-90:$v_2$}] (b1) at (1, -0.5) {\texttt{b=1}};
      \node[vertex, executed, label={[vertexlabel]-90:$v_3$}] (ba) at (2, 0) {\texttt{b=a}};
      \node[vertex, highlight, executed, label={[vertexlabel]-90:$v_4$}] (a22) at (3, 0) {\texttt{a=2}};

      \draw[arrow] (a2) to (ab);
      \draw[arrow] (b1) to (ab);
      \draw[arrow] (ba) to (ab);
      \draw[arrow] (ba) to (a2);
      \draw[arrow] (ba) to (b1);
      \draw[arrow, highlight] (a22) to (ba);
      \draw[arrow, highlight, bend right=60] (a22) to (ab);
      \draw[arrow, bend right=60] (ba) to (a22);
    \end{tikzpicture}
    \caption{}\figlabel{v4}
  \end{subfigure}

  \begin{subfigure}[c]{0.05\textwidth}
    \centering
    \newcommand{\rightof}[1]{-1pt of #1}
    \begin{tikzpicture}[xscale=1, yscale=1.5]
      \node[logentry, highlight, executed, label={[logindex]90:0}] (0) {\texttt{a=b}};
    \end{tikzpicture}
    \caption{}\figlabel{l0}
  \end{subfigure}
  \begin{subfigure}[c]{0.15\textwidth}
    \centering
    \newcommand{\rightof}[1]{-1pt of #1}
    \begin{tikzpicture}[xscale=1, yscale=1.5]
      \node[logentry, executed, label={[logindex]90:0}] (0) {\texttt{a=b}};
      \node[logentry, highlight, executed, label={[logindex]90:1}, right=\rightof{0}] (1) {\texttt{a=2}};
    \end{tikzpicture}
    \caption{}\figlabel{l1}
  \end{subfigure}
  \begin{subfigure}[c]{0.25\textwidth}
    \centering
    \newcommand{\rightof}[1]{-1pt of #1}
    \begin{tikzpicture}[xscale=1, yscale=1.5]
      \node[logentry, executed, label={[logindex]90:0}] (0) {\texttt{a=b}};
      \node[logentry, executed, label={[logindex]90:1}, right=\rightof{0}] (1) {\texttt{a=2}};
      \node[logentry, label={[logindex]90:2}, right=\rightof{1}] (2) {\phantom{\texttt{b=1}}};
      \node[logentry, highlight, label={[logindex]90:3}, right=\rightof{2}] (3) {\texttt{b=a}};
    \end{tikzpicture}
    \caption{}\figlabel{l3}
  \end{subfigure}
  \begin{subfigure}[c]{0.25\textwidth}
    \centering
    \newcommand{\rightof}[1]{-1pt of #1}
    \begin{tikzpicture}[xscale=1, yscale=1.5]
      \node[logentry, executed, label={[logindex]90:0}] (0) {\texttt{a=b}};
      \node[logentry, executed, label={[logindex]90:1}, right=\rightof{0}] (1) {\texttt{a=2}};
      \node[logentry, highlight, executed, label={[logindex]90:2}, right=\rightof{1}] (2) {\texttt{b=1}};
      \node[logentry, executed, label={[logindex]90:3}, right=\rightof{2}] (3) {\texttt{b=a}};
      \node[logentry, highlight, executed, label={[logindex]90:2}, right=\rightof{1}] (2) {\texttt{b=1}};
    \end{tikzpicture}
    \caption{}\figlabel{l2}
  \end{subfigure}
  \begin{subfigure}[c]{0.25\textwidth}
    \centering
    \newcommand{\rightof}[1]{-1pt of #1}
    \begin{tikzpicture}[xscale=1, yscale=1.5]
      \node[logentry, executed, label={[logindex]90:0}] (0) {\texttt{a=b}};
      \node[logentry, executed, label={[logindex]90:1}, right=\rightof{0}] (1) {\texttt{a=2}};
      \node[logentry, executed, label={[logindex]90:2}, right=\rightof{1}] (2) {\texttt{b=1}};
      \node[logentry, executed, label={[logindex]90:3}, right=\rightof{2}] (3) {\texttt{b=a}};
      \node[logentry, highlight, executed, label={[logindex]90:4}, right=\rightof{3}] (4) {\texttt{a=2}};
    \end{tikzpicture}
    \caption{}\figlabel{l4}
  \end{subfigure}

  \caption{%
    In subfigures (a) - (e), we see a conflict graph constructed over time. The
    most recently chosen vertex is drawn in red. The executed commands are
    shaded green.
    %
    (a) The command \texttt{a=b} is chosen in vertex $v_0$ without any
    dependencies. The command is executed immediately.
    %
    (b) The command \texttt{a=2} is chosen in vertex $v_1$ with a dependency on
    $v_0$. The command is executed immediately.
    %
    (c) The command \texttt{b=a} is chosen in vertex $v_3$ with dependencies on
    $v_0$, $v_1$, $v_2$, and $v_4$. No commands have been chosen in $v_2$ and
    $v_4$ yet, so $v_3$ cannot be executed.
    %
    (d) The command \texttt{b=1} is chosen in vertex $v_2$ with a dependency on
    $v_0$. The command is executed immediately.
    %
    (e) The command \texttt{a=2} is chosen in vertex $v_2$ with dependencies on
    $v_0$ and $v_3$. Now $v_3$ and $v_4$ are executed.
    %
    In subfigures (f) - (j), we see an analogous execution for a log.
  }\figlabel{GraphExecutionExample}
\end{figure*}
