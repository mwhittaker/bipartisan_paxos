% Processes.
\tikzstyle{proc}=[draw, circle, thick, inner sep=2pt]
\tikzstyle{client}=[proc, fill=clientcolor!25]
\tikzstyle{proposer}=[proc, fill=proposercolor!25]
\tikzstyle{acceptor}=[proc, fill=acceptorcolor!25]

% Process labels.
\tikzstyle{proclabel}=[inner sep=0pt, align=center, font=\small]

% Components.
\tikzstyle{component}=[draw, thick, flatgray, rounded corners]

% Messages and communication.
\tikzstyle{comm}=[-latex, thick]
\tikzstyle{commnum}=[fill=white, inner sep=0pt]

\begin{figure}[ht]
  \centering
  \begin{subfigure}[b]{0.45\columnwidth}
    \centering
    \begin{tikzpicture}[xscale=1.5]
      % Processes.
      \node[client] (c1) at (0, 2) {$c_1$};
      \node[client] (c2) at (0, 1) {$c_2$};
      \node[client] (c3) at (0, 0) {$c_3$};
      \node[proposer] (p1) at (1, 1.5) {$p_1$};
      \node[proposer] (p2) at (1, 0.5) {$p_2$};
      \node[acceptor] (a1) at (2, 2) {$a_1$};
      \node[acceptor] (a2) at (2, 1) {$a_2$};
      \node[acceptor] (a3) at (2, 0) {$a_3$};

      % Labels.
      \node[proclabel] (clients) at (0, 3) {Clients};
      \node[proclabel] (proposers) at (1, 3) {$f+1$\\Proposers};
      \node[proclabel] (acceptors) at (2, 3) {$2f+1$\\Acceptors};
      \halffill{clients}{clientcolor!25}
      \quarterfill{proposers}{proposercolor!25}
      \quarterfill{acceptors}{acceptorcolor!25}

      % Communication.
      \draw[comm] (c1) to node[commnum]{1} (p1);
      \draw[comm, bend left=15] (p1) to node[commnum]{2} (a1);
      \draw[comm, bend left=15] (p1) to node[commnum]{2} (a2);
      \draw[comm, bend left=15] (a1) to node[commnum]{3} (p1);
      \draw[comm, bend left=15] (a2) to node[commnum]{3} (p1);
    \end{tikzpicture}%
    \caption{Phase 1}\figlabel{PaxosBackgroundDiagramPhase1}
  \end{subfigure}\hspace{0.09\columnwidth}
  \begin{subfigure}[b]{0.45\columnwidth}
    \centering
    \begin{tikzpicture}[xscale=1.25]
      % Processes.
      \node[client] (c1) at (0, 2) {$c_1$};
      \node[client] (c2) at (0, 1) {$c_2$};
      \node[client] (c3) at (0, 0) {$c_3$};
      \node[proposer] (p1) at (1, 1.5) {$p_1$};
      \node[proposer] (p2) at (1, 0.5) {$p_2$};
      \node[acceptor] (a1) at (2, 2) {$a_1$};
      \node[acceptor] (a2) at (2, 1) {$a_2$};
      \node[acceptor] (a3) at (2, 0) {$a_3$};

      % Labels.
      \node[proclabel] (clients) at (0, 3) {Clients};
      \node[proclabel] (proposers) at (1, 3) {$f+1$\\Proposers};
      \node[proclabel] (acceptors) at (2, 3) {$2f+1$\\Acceptors};
      \halffill{clients}{clientcolor!25}
      \quarterfill{proposers}{proposercolor!25}
      \quarterfill{acceptors}{acceptorcolor!25}

      % Communication.
      \draw[comm, bend left=15] (p1) to node[commnum]{4} (a1);
      \draw[comm, bend left=15] (p1) to node[commnum]{4} (a2);
      \draw[comm, bend left=15] (a1) to node[commnum]{5} (p1);
      \draw[comm, bend left=15] (a2) to node[commnum]{5} (p1);
      \draw[comm] (p1) to node[commnum]{6} (c1);
    \end{tikzpicture}
    \caption{Phase 2}\figlabel{PaxosBackgroundDiagramPhase2}
  \end{subfigure}
  \caption{An example execution of Paxos ($f=1$).}%
  \figlabel{PaxosBackgroundDiagram}
\end{figure}
