% Processes.
\tikzstyle{proc}=[draw, circle, thick, inner sep=2pt]
\tikzstyle{client}=[proc, fill=clientcolor!25]
\tikzstyle{proposer}=[proc, fill=proposercolor!25]
\tikzstyle{depservice}=[proc, fill=depservicecolor!25]
\tikzstyle{acceptor}=[proc, fill=acceptorcolor!25]
\tikzstyle{replica}=[proc, fill=replicacolor!25]

% Process labels.
\tikzstyle{proclabel}=[inner sep=0pt, align=center]

% Components.
\tikzstyle{component}=[draw, thick, flatgray, rounded corners]

% Messages and communication.
\tikzstyle{comm}=[-latex, thick]
\tikzstyle{commnum}=[fill=white, inner sep=0pt]

\begin{figure}[ht]
  \centering
  \begin{tikzpicture}[xscale=2]
    % Processes.
    \node[client] (c1) at (0, 2) {$c_1$};
    \node[client] (c2) at (0, 1) {$c_2$};
    \node[proposer] (p1) at (1.5, 1.5) {$p_1$};
    \node[proposer] (p2) at (2.5, 1.5) {$p_2$};
    \node[depservice] (d1) at (0.25, 3.25) {$d_1$};
    \node[depservice] (d2) at (0.75, 3.25) {$d_2$};
    \node[depservice] (d3) at (1.25, 3.25) {$d_3$};
    \node[acceptor] (a1) at (1.75, 3.25) {$a_1$};
    \node[acceptor] (a2) at (2.25, 3.25) {$a_2$};
    \node[acceptor] (a3) at (2.75, 3.25) {$a_3$};
    \node[replica] (r1) at (1, 0) {$r_1$};
    \node[replica] (r2) at (2, 0) {$r_2$};

    % % Labels.
    \crown{(p1.north)++(0,-0.15)}{0.25}{0.25}
    \crown{(p2.north)++(0,-0.15)}{0.25}{0.25}
    \node[proclabel] (clients) at (0, 0.5) {Clients};
    \node[proclabel] (proposers) at (3.25, 1.5) {$f+1$\\Proposers};
    \node[proclabel] (depservice) at (0.75, 4.25) {$2f+1$\\Dependency Service\\Nodes};
    \node[proclabel] (acceptors) at (2.25, 4) {$2f+1$\\Acceptors};
    \node[proclabel] (replicas) at (1.5, -0.5) {$f+1$\\Replicas};
    \halffill{clients}{clientcolor!25}
    \quarterfill{proposers}{proposercolor!25}
    \quarterfill{depservice}{depservicecolor!25}
    \quarterfill{acceptors}{acceptorcolor!25}
    \quarterfill{replicas}{replicacolor!25}

    % % Communication.
    \draw[comm] (c1) to node[commnum]{1} (p1);
    \draw[comm, bend left=20] (p1) to node[commnum]{2} (d1);
    \draw[comm, bend left=15] (p1) to node[commnum]{2} (d2);
    \draw[comm, bend left=10] (p1) to node[commnum]{2} (d3);
    \draw[comm, bend right=10] (d1) to node[commnum]{3} (p1);
    \draw[comm, bend right=5] (d2) to node[commnum]{3} (p1);
    \draw[comm, bend right=0] (d3) to node[commnum]{3} (p1);
    \draw[comm, bend right=10] (p1) to node[commnum]{4} (a1);
    \draw[comm, bend right=15] (p1) to node[commnum]{4} (a2);
    \draw[comm, bend right=20] (p1) to node[commnum]{4} (a3);
    \draw[comm, bend left=0] (a1) to node[commnum]{5} (p1);
    \draw[comm, bend left=5] (a2) to node[commnum]{5} (p1);
    \draw[comm, bend left=10] (a3) to node[commnum]{5} (p1);
    \draw[comm] (p1) to node[commnum]{6} (r1);
    \draw[comm] (p1) to node[commnum]{6} (r2);
    \draw[comm] (r1) to node[commnum]{7} (c1);
  \end{tikzpicture}
  \caption{An example execution of Simple \BPaxos{} ($f=1$).}%
  \figlabel{SimpleBPaxos}
\end{figure}
