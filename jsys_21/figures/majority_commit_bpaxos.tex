\newcommand{\nullbot}{\textsf{null}}

\begin{algorithm*}[ht]
  \caption{%
    Majority Commit \BPaxos{} Proposer. Pseudocode for initiating recovery and
    handling \msgfont{Phase2B} messages is ommitted because it is identical to
    the pseudocode in \algoref{FastPaxosProposer}.
  }%
  \algolabel{MajorityCommitBPaxos}
  \begin{algorithmic}[1]
    \GlobalState a value $v$, initially \nullbot{}
    \GlobalState a round $i$, initially $-1$

    \Upon{receiving \msg{Phase1B}{i, vr, vv} from $f+1$ acceptors $A$}
      \State $k \gets$ the largest $vr$ in any \msg{Phase1B}{i, vr, vv}
      \If{$k = -1$} \linelabel{McbKEqualsNegativeOne}
        \State $v \gets$ an arbitrary value satisfying the dependency invariant
        % Addresses Reviewer 3.
        %
        % > I am unsure why at line 21 of Algorithm 1, the propose is sending to at
        % > least $f+1$ acceptors instead of all acceptors. As $f$ of the acceptors
        % > may crash, so sending to all acceptors in important.
        \State send \msg{Phase2A}{i, v} to \markrevisions{the acceptors}
      \ElsIf{$k > 0$} \linelabel{McbKGreaterThanZero}
        \State $v \gets$ the corresponding $vv$ in round $k$
        % Addresses Reviewer 3.
        %
        % > I am unsure why at line 21 of Algorithm 1, the propose is sending to at
        % > least $f+1$ acceptors instead of all acceptors. As $f$ of the acceptors
        % > may crash, so sending to all acceptors in important.
        \State send \msg{Phase2A}{i, v} to \markrevisions{the acceptors}
      \ElsIf{%
        there are $\maj{f + 1}$ \msg{Phase1B}{i, 0, v'} messages for some value
        $v'$
      } \linelabel{McbMajority}
        \State $(x, \deps{v_x}) \gets v'$
               \linelabel{McbUnwrapVPrime}

        \State send $v_x$ and $x$ to the dependency service nodes co-located
               with the acceptors in $A$
               \linelabel{McbSendToDependencyService}
        \State $\deps{v_x}_A \gets$ the union of the dependencies returned by
               these dependency service nodes
               \linelabel{McbReceiveFromDependencyService}
        \State
        \State $P \gets \emptyset$
        \For{$v_y \in \deps{v_x}_A - \deps{v_x}$}
          \If{we don't know if $v_y$ is chosen}
            \State recover $v_y$, blocking until $v_y$ is recovered
                   \linelabel{McbRecoverVy}
          \EndIf
          \If{$v_y$ chosen with $\noop$ or with $v_x \in \deps{v_y}$}
            \State $P \gets P \cup \set{v_y}$
                   \linelabel{McbPruneVy}
          \Else{}
            \State contact a quorum $A'$ of acceptors
                   \linelabel{McbContactAPrime}
            \If{an acceptor in $A'$ knows $v_x$ is chosen}
              \State abort recovery; $v_x$ has already been chosen
                     \linelabel{McbAbortRecovery}
            \Else{}
              \State $v \gets$ an arbitrary value satisfying the dependency invariant
                     \linelabel{McbNothingChosen}
              % Addresses Reviewer 3.
              %
              % > I am unsure why at line 21 of Algorithm 1, the propose is sending to at
              % > least $f+1$ acceptors instead of all acceptors. As $f$ of the acceptors
              % > may crash, so sending to all acceptors in important.
              \State send \msg{Phase2A}{i, v} to \markrevisions{the acceptors}
            \EndIf{}
          \EndIf{}
        \EndFor{}
        \State $v \gets v'$
        \State send \msg{Phase2A}{i, v} and the values chosen in $P$ to at
               least $f + 1$ acceptors
               \linelabel{McbSuccessfulPruning}
      \Else{} \linelabel{McbNoMajority}
        \State $v \gets$ an arbitrary value satisfying the dependency invariant
        % Addresses Reviewer 3.
        %
        % > I am unsure why at line 21 of Algorithm 1, the propose is sending to at
        % > least $f+1$ acceptors instead of all acceptors. As $f$ of the acceptors
        % > may crash, so sending to all acceptors in important.
        \State send \msg{Phase2A}{i, v} to \markrevisions{the acceptors}
      \EndIf
    \EndUpon
  \end{algorithmic}
\end{algorithm*}
