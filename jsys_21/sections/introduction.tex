\section{Introduction}
State machine replication protocols are a critical component of many fault
tolerant distributed systems~\cite{burrows2006chubby, corbett2013spanner,
thomson2012calvin, baker2011megastore, taft2020cockroachdb}. Given an arbitrary
deterministic state machine, like a key-value store or a relational database, a
state machine replication protocol can be used to deploy multiple copies, or
replicas, of the state machine while guaranteeing that the states of the
replicas stay in sync and do not diverge.

The most popular and widely deployed state machine replication protocols are
Paxos~\cite{lamport1998part, burrows2006chubby, corbett2013spanner} and
Raft~\cite{ongaro2014search, taft2020cockroachdb, tidb2019website,
yugabyte2019website}. Despite their popularity, these protocols have two major
flaws. First, they are leader based. All communication is funneled through a
single leader, making the leader a throughput bottleneck. Second, these
protocols totally order state machine commands into a log and have state
machine replicas execute the commands in log order. This means that every
replica executes the exact same commands in the exact same order. While safe,
this is overly restrictive because commuting commands could instead be executed
in any order.

There is another family of \emph{generalized multi-leader} state machine
replication protocols---including EPaxos~\cite{moraru2013there},
Caesar~\cite{arun2017speeding}, and Atlas~\cite{enes2020state}---that do not
have these flaws. These protocols are \defword{multi-leader} and avoid being
throughput bottlenecked by a single leader. They are also
\defword{generalized}~\cite{lamport2005generalized, losa2016brief}. This means
that every replica executes non-commuting commands in the exact same order, but
the replicas are free to execute commuting commands in any order. As a result,
commuting commands do not interfere with one another.

Unfortunately, these generalized multi-leader protocols are extremely
complicated. Paxos has a well known reputation for being
complex~\cite{lamport2001paxos, van2015paxos, ongaro2014search}, and these
generalized multi-leader protocols are significantly more complex than that.
They require a strong understanding of more sophisticated Paxos variants like
Fast Paxos~\cite{lamport2006fast} and Generalized
Paxos~\cite{lamport2005generalized} and are overall less intuitive and more
nuanced. While it's hard to quantify complexity precisely, there are traces of
evidence. EPaxos, for example, had several bugs that went undiscovered for
years despite the popularity of the protocol~\cite{sutra2011fast}. Through
personal conversations, we have also found that even domain experts find these
protocols challenging to fully understand.

This complexity has negative consequences in industry and academia. Generalized
multi-leader protocols have little to no industry adoption despite their
performance advantages over protocols like MultiPaxos and Raft. We postulate
that this is largely due to their complexity. The engineers
in~\cite{chandra2007paxos} explain that implementing a state machine replication
protocol requires making many small changes to the protocol to match the
environment in which it is deployed. Making these changes without a strong
understanding of the protocol is infeasible. Academically, it is challenging to
compare and contrast the various protocols. They all seem very similar, yet
vaguely distinct. This also makes it difficult to extend the protocols with
further innovations. There are dozens of state machine replication protocols in
the literature, yet relatively few generalized multi-leader variants.

This paper is a tutorial on generalized multi-leader state machine replication
protocols. We answer questions such as: Why do these protocols work the way
they do? What do they have in common? Where do they differ? Which parts of
the protocols are straightforward? Which are more subtle than they appear? Are
there simpler variants out there? What trade-offs do the protocols make, and
which points in the design space are still unexplored?

The tutorial is a four part act, and in each act, we introduce a new protocol.
First, we present the simplest possible generalized multi-leader protocol, which
we called \textbf{Simple \BPaxos{}} (\secref{SimpleBPaxos}). Simple \BPaxos{}
sacrifices performance for simplicity and is designed with the sole goal of
being easy to understand.  Simple \BPaxos{} is the kernel from which all other
generalized multi-leader protocols can be constructed. It encapsulates all the
mechanisms and invariants that are common to the other protocols.

Second, we introduce a purely pedagogical protocol called \textbf{Fast
\BPaxos{}} (\secref{FastBPaxos}). Fast \BPaxos{} achieves higher performance
than Simple \BPaxos{}, but it is unsafe. The protocol does not properly
implement state machine replication. Why study a broken protocol? Because
understanding why Fast \BPaxos{} does \emph{not} work leads to a fundamental
insight on why other protocols do. Specifically, we discover that generalized
multi-leader protocols encounter a fundamental tension between agreeing on
commands and ordering commands. The way in which a protocol handles this
tension is its key distinguishing feature. We taxonomize the protocols into
protocols that avoid the tension and protocols that resolve the tension.

Third, we introduce \textbf{Unanimous \BPaxos{}}, a simple tension avoiding
protocol (\secref{TensionAvoidance}). We describe how tension avoiding
protocols carefully enlarge quorum sizes to sidestep the tension. We also
explain how Basic EPaxos~\cite{moraru2013there} and Atlas~\cite{enes2020state}
can be expressed as optimized variants of Unanimous \BPaxos{}.

Fourth, we introduce \textbf{Majority Commit \BPaxos{}}, a tension resolving
protocol (\secref{TensionResolution}). We describe how tension resolving
protocols perform detective work to resolve the tension without enlarging
quorum sizes.  We also discuss the relationship between Majority Commit
\BPaxos{} and other tension resolving protocols like
EPaxos~\cite{moraru2013proof} and Caesar~\cite{arun2017speeding}.

In summary, we make the following contributions.
\begin{itemize}
  \item
    We explain generalized multi-leader protocols carefully and thoroughly,
    bringing clarity to an otherwise dense area of popular research.

  \item
    We present four new generalized multi-leader state machine replication
    protocols: Simple \BPaxos{}, Fast \BPaxos{}, Unanimous \BPaxos{}, and
    Majority Commit \BPaxos{}.

  \item
    We identify a fundamental tension between agreeing on commands and ordering
    commands and use this insight to taxonomize generalized multi-leader
    protocols into those that avoid the tension and those that resolve it.
\end{itemize}
