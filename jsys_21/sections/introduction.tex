\section{Introduction}

SMR is super important

The most popular SMR protocolsa rePaxos and Raft. these protocols are leader
based and order everything into a totally ordered log. this means that they are
throughput bottlenecked and commands interfere.

there is another class of multileader genearlized protoco.s eg. atlas, caesar, epaxos. they are
multileader cause there is no leader. they are generalized because only
confilcting commands have to beexcuted in the same order.

unfortunately these protocols are very complex, with some bugs being disscovered many eyars leatar. this complexity negatively affects industry and acaemdia. in industry, these ptotocols do not see much adoption despite being haveing nice advatnages. in academia it can be hard to identify the novely of the protoocls and hard to build off of them

in this paper, we presenta tutorial on generalized multileader protocols. we aim to clairfy things such as why do they work, what distringuises them, which parts are hard, which parts are easy, which parts are sublte.

we introduce simple bpaxos to understand the basics of the stuff, not fast, but simple, common to all other protocols

then we introduce fast bpaxos that is simple, efficient, but unsafe. why look at borken? ebcause it reveleas a fundamental insight into the protocols there is a tension between two invariants and how the protocols  handle the tension is the distinguishing characterstic. we categroize protocols into tension avoiding and tension resolving

we introduce unan bpaxos as a prottpical tensoin avoiding and describe how basic epaxos and atlas are optimized variants
we then exaplin maj commit bpaxos as a prototypical tension resolving and describe the similarities with epaxos and caesar

ultimately our contributions are a better understanding of how the protocols work, where is subtle, blah blah

% \section{Introduction}
% Consensus and state machine replication are fundamental problems in distributed
% systems that are both well-studied in academia and widely implemented in
% industry.  Paxos~\cite{lamport1998part}, one of the earliest asynchronous
% consensus protocols, was developed roughly 30 years ago and has since become
% the de-facto standard in industry~\cite{burrows2006chubby, chandra2007paxos,
% baker2011megastore, corbett2013spanner}. Since that time, Paxos and Multi-Paxos
% (the state machine replication protocol built on Paxos) have been improved
% along three core dimensions: latency, throughput, and simplicity.
%
% First, the latency of Paxos---i.e.\ the minimum number of message delays
% between when a value is proposed by a client and when it is chosen by the
% protocol---is higher than necessary~\cite{lamport2006lower}. Fast
% Paxos~\cite{lamport2006fast} improves Paxos' latency to its theoretical minimum
% by allowing clients to propose commands directly to acceptors.
% %
% Second, the throughput of Multi-Paxos is bottlenecked by the throughput of a
% single leader. Fast Paxos partially resolves this problem by allowing clients
% to bypass the leader, but this leads to high conflict rates, lowering the
% throughput of the protocol.  Generalized Paxos~\cite{lamport2005generalized}
% and GPaxos~\cite{sutra2011fast} reduce the number of conflicts by taking
% advantage of the commutativity of state machine commands, but still rely on a
% single arbiter to resolve conflicts when they arise.
% EPaxos~\cite{moraru2013there, moraru2013proof} and
% Caesar~\cite{arun2017speeding} are both fully leaderless and improve on
% Generalized Paxos by not relying on a single process either during normal
% processing or conflict resolution.
% %
% Third, Paxos and Multi-Paxos have developed a reputation for being overly
% complicated, leading to a number of publications attempting to clarify the
% protocols~\cite{lamport2001paxos, lampson2001abcd, mazieres2007paxos,
% van2015paxos} and a number of protocols touted as simpler
% alternatives~\cite{ongaro2014search, rystsov2018caspaxos}.
%
% Despite the large body of work, no state machine replication protocol has
% claimed the trifecta of low latency, high throughput, and simplicity. Existing
% protocols typically sacrifice one of these features for the other two. In this
% paper, we present Bipartisan Paxos (or BPaxos, for short): a family of
% asynchronous state machine replication protocols that accomplish all three. The
% BPaxos protocols can commit a command in two message delays (the theoretical
% minimum). They are also fully leaderless and do not depend on a distinguished
% leader for normal processing or conflict resolution. Furthermore, we employ
% three techniques to make the BPaxos protocols as simple as possible.
% %
% First, the protocols are modular. Every protocol is composed of small
% subcomponents, each of which can be understood individually.
% %
% Second, we re-use existing algorithms to implement these subcomponents whenever
% possible, reducing the cognitive burden of understanding a new protocol that is
% written entirely from scratch.
% %
% Third, the three BPaxos protocols---Simple BPaxos, Unanimous BPaxos, and
% Majority Commit BPaxos---are all incremental refinements of one another. We
% begin with a very simple protocol, Simple BPaxos, and then slowly increase the
% complexity. This allows us to decouple the nuances of the protocols,
% understanding each in isolation.
