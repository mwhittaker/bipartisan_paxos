\section{Fast Paxos}\seclabel{FastPaxos}
{\newcommand{\nullbot}{\textsf{null}}

\begin{algorithm}[ht]
  \caption{Fast Paxos Proposer}%
  \algolabel{FastPaxosProposer}
  \begin{algorithmic}[1]
    \GlobalState a value $v$, initially \nullbot{}
    \GlobalState a round $i$, initially $-1$

    \Upon{%
      receiving \msg{Phase2B}{0, v'} from $f + \maj{f+1}$ acceptors as
      the proposer of round $0$ with $i=0$
    }\linelabel{Proposer1}
      \If{every value of $v'$ is the same}
        \State $v'$ is chosen, notify the learners \linelabel{Proposer2}
      \Else{} \linelabel{CoordinatedRecovery1}
        \State $i \gets 1$
        \State proceed to \lineref{computek} viewing every \msg{Phase2b}{0, v'}
               as a \msg{Phase1b}{0, 0, v'}
      \EndIf \linelabel{CoordinatedRecovery2}
    \EndUpon

    \Upon{recovery}\linelabel{ProposerRecovery1}
      \State $i \gets$ next largest round owned by this proposer
      \State send \msg{Phase1A}{i} to at least $f+1$ acceptors
    \EndUpon\linelabel{ProposerRecovery2}

    \Upon{receiving \msg{Phase1B}{i, vr, vv} from $f+1$ acceptors}
      \State $k \gets$ the largest $vr$ in any \msg{Phase1B}{i, vr, vv}\linelabel{computek}
      \If{$k = -1$}\linelabel{kn1}
        \State $v \gets$ an arbitrary value
      \ElsIf{$k > 0$}\linelabel{kg0}
        \State $v \gets$ the corresponding $vv$ in round $k$
      \ElsIf{$k = 0$}\linelabel{ke0}
      \If{there are $\maj{f + 1}$ \msg{Phase1B}{i, 0, v'} messages for some value $v'$}\linelabel{majority}
          \State $v \gets$ $v'$
        \Else{}\linelabel{nomajority}
          \State $v \gets$ an arbitrary value
        \EndIf
      \EndIf
      \State send \msg{Phase2a}{i, v} to at least $f + 1$ acceptors
    \EndUpon

    \Upon{receiving \msg{Phase2B}{i} from $f+1$ acceptors}
      \State $v$ is chosen, notify the learners
    \EndUpon
  \end{algorithmic}
\end{algorithm}
}
{\newcommand{\nullbot}{\textsf{null}}

\begin{algorithm}[ht]
  \caption{Fast Paxos Acceptor}%
  \algolabel{FastPaxosAcceptor}
  \begin{algorithmic}[1]
    \GlobalState the largest seen round $r$, initially $-1$
    \GlobalState the largest round $vr$ voted in, initially $-1$
    \GlobalState the value $vv$ voted for in round $vr$, initially \nullbot{}

    \Upon{receiving command $x$ from client}
      \If{$r = -1$}
        \State $r, vr, vv \gets 0, 0, x$
        \State send \msg{Phase2b}{0, x} to proposer of round $0$
      \EndIf
    \EndUpon{}

    \Upon{receiving \msg{Phase1A}{i} from $p$ with $i > r$}
      \State $r \gets i$
      \State send \msg{Phase1B}{i, vr, vv} to $p$
    \EndUpon

    \Upon{receiving \msg{Phase2A}{i, x} from $p$ with $i \geq r$}
      \State $r, vr, vv \gets i, i, x$
      \State send \msg{Phase2B}{i} to $p$
    \EndUpon
  \end{algorithmic}
\end{algorithm}
}

{% Processes.
\tikzstyle{proc}=[draw, circle, thick, inner sep=2pt]
\tikzstyle{client}=[proc, fill=clientcolor!25]
\tikzstyle{proposer}=[proc, fill=proposercolor!25]
\tikzstyle{acceptor}=[proc, fill=acceptorcolor!25]
\tikzstyle{replica}=[proc, fill=replicacolor!25]

% Process labels.
\tikzstyle{proclabel}=[inner sep=0pt, align=center]

% Components.
\tikzstyle{component}=[draw, thick, flatgray, rounded corners]

% Messages and communication.
\tikzstyle{comm}=[-latex, thick]
\tikzstyle{commnum}=[fill=white, inner sep=0pt]
\tikzstyle{request}=[red]
\tikzstyle{phase1}=[blue]
\tikzstyle{phase2}=[flatgreen]

\begin{figure*}[ht]
  \centering

  \begin{subfigure}[c]{0.3\textwidth}
    \centering
    \begin{tikzpicture}[xscale=2]
      % Processes.
      \node[proposer] (p1) at (0, 3) {$p_1$};
      \node[proposer] (p2) at (0, 2) {$p_2$};
      \node[proposer] (p3) at (0, 1) {$p_3$};
      \node[acceptor] (a1) at (1, 4) {$a_1$};
      \node[acceptor] (a2) at (1, 3) {$a_2$};
      \node[acceptor] (a3) at (1, 2) {$a_3$};
      \node[acceptor] (a4) at (1, 1) {$a_4$};
      \node[acceptor] (a5) at (1, 0) {$a_5$};
      \node[replica] (r1) at (2, 3) {$l_1$};
      \node[replica] (r2) at (2, 2) {$l_2$};
      \node[replica] (r3) at (2, 1) {$l_3$};

      % Labels.
      \crown{(p1.north)++(0,-0.15)}{0.25}{0.25}
      \node[proclabel] (proposers) at (0, 5) {$f+1$\\Proposers};
      \node[proclabel] (acceptors) at (1, 5) {$2f+1$\\Acceptors};
      \node[proclabel] (replicas) at (2, 5) {$f+1$\\Learners};
      \quarterfill{proposers}{proposercolor!25}
      \quarterfill{acceptors}{acceptorcolor!25}
      \quarterfill{replicas}{replicacolor!25}

      % Communication.
      \draw[comm, request] ($(a1) + (0.5, 0.5)$) to node[commnum]{1} (a1);
      \draw[comm, request] ($(a2) + (0.5, 0.5)$) to node[commnum]{1} (a2);
      \draw[comm, request] ($(a3) + (0.5, 0.5)$) to node[commnum]{1} (a3);
      \draw[comm, request] ($(a4) + (0.5, 0.5)$) to node[commnum]{1} (a4);
      \draw[comm, phase2] (a1) to node[commnum]{2} (p1);
      \draw[comm, phase2] (a2) to node[commnum]{2} (p1);
      \draw[comm, phase2] (a3) to node[commnum]{2} (p1);
      \draw[comm, phase2] (a4) to node[commnum]{2} (p1);
      \draw[comm, bend left=35] (p1) to node[commnum]{3} (r1);
      \draw[comm] (p1) to node[commnum]{3} (r2);
      \draw[comm, bend right=15] (p1) to node[commnum]{3} (r3);
    \end{tikzpicture}
    \caption{Normal case execution.}\figlabel{FastPaxosDiagram1}
  \end{subfigure}\hspace{0.01\textwidth}
  \begin{subfigure}[c]{0.3\textwidth}
    \centering
    \begin{tikzpicture}[xscale=2]
      % Processes.
      \node[proposer] (p1) at (0, 3) {$p_1$};
      \node[proposer] (p2) at (0, 2) {$p_2$};
      \node[proposer] (p3) at (0, 1) {$p_3$};
      \node[acceptor] (a1) at (1, 4) {$a_1$};
      \node[acceptor] (a2) at (1, 3) {$a_2$};
      \node[acceptor] (a3) at (1, 2) {$a_3$};
      \node[acceptor] (a4) at (1, 1) {$a_4$};
      \node[acceptor] (a5) at (1, 0) {$a_5$};
      \node[replica] (r1) at (2, 3) {$l_1$};
      \node[replica] (r2) at (2, 2) {$l_2$};
      \node[replica] (r3) at (2, 1) {$l_3$};

      % Labels.
      \crown{(p1.north)++(0,-0.15)}{0.25}{0.25}
      \node[proclabel] (proposers) at (0, 5) {$f+1$\\Proposers};
      \node[proclabel] (acceptors) at (1, 5) {$2f+1$\\Acceptors};
      \node[proclabel] (replicas) at (2, 5) {$f+1$\\Learners};
      \quarterfill{proposers}{proposercolor!25}
      \quarterfill{acceptors}{acceptorcolor!25}
      \quarterfill{replicas}{replicacolor!25}

      % Communication.
      \draw[comm, phase1, bend left=30] (p2) to node[commnum]{1} (a1);
      \draw[comm, phase1, bend left=35] (p2) to node[commnum]{1} (a2);
      \draw[comm, phase1, bend left=55] (p2) to node[commnum]{1} (a3);
      \draw[comm, phase1, bend right=20] (a1) to node[commnum]{2} (p2);
      \draw[comm, phase1, bend right=20] (a2) to node[commnum]{2} (p2);
      \draw[comm, phase1, bend right=20] (a3) to node[commnum]{2} (p2);
      \draw[comm, phase2, bend right=55] (p2) to node[commnum]{3} (a3);
      \draw[comm, phase2, bend right=35] (p2) to node[commnum]{3} (a4);
      \draw[comm, phase2, bend right=30] (p2) to node[commnum]{3} (a5);
      \draw[comm, phase2, bend left=20] (a3) to node[commnum]{4} (p2);
      \draw[comm, phase2, bend left=20] (a4) to node[commnum]{4} (p2);
      \draw[comm, phase2, bend left=20] (a5) to node[commnum]{4} (p2);
      \draw[comm, bend left=23] (p2) to node[commnum]{5} (r1);
      \draw[comm, bend left=25] (p2) to node[commnum]{5} (r2);
      \draw[comm, bend right=23] (p2) to node[commnum]{5} (r3);
    \end{tikzpicture}
    \caption{Recovery.}\figlabel{FastPaxosDiagram2}
  \end{subfigure}\hspace{0.01\textwidth}
  \begin{subfigure}[c]{0.3\textwidth}
    \centering
    \begin{tikzpicture}[xscale=2]
      % Processes.
      \node[proposer] (p1) at (0, 3) {$p_1$};
      \node[proposer] (p2) at (0, 2) {$p_2$};
      \node[proposer] (p3) at (0, 1) {$p_3$};
      \node[acceptor] (a1) at (1, 4) {$a_1$};
      \node[acceptor] (a2) at (1, 3) {$a_2$};
      \node[acceptor] (a3) at (1, 2) {$a_3$};
      \node[acceptor] (a4) at (1, 1) {$a_4$};
      \node[acceptor] (a5) at (1, 0) {$a_5$};
      \node[replica] (r1) at (2, 3) {$l_1$};
      \node[replica] (r2) at (2, 2) {$l_2$};
      \node[replica] (r3) at (2, 1) {$l_3$};

      % Labels.
      \crown{(p1.north)++(0,-0.15)}{0.25}{0.25}
      \node[proclabel] (proposers) at (0, 5) {$f+1$\\Proposers};
      \node[proclabel] (acceptors) at (1, 5) {$2f+1$\\Acceptors};
      \node[proclabel] (replicas) at (2, 5) {$f+1$\\Learners};
      \quarterfill{proposers}{proposercolor!25}
      \quarterfill{acceptors}{acceptorcolor!25}
      \quarterfill{replicas}{replicacolor!25}

      % Communication.
      \draw[comm, request] ($(a1) + (0.5, 0.5)$) to node[commnum]{1} (a1);
      \draw[comm, request] ($(a2) + (0.5, 0.5)$) to node[commnum]{1} (a2);
      \draw[comm, request] ($(a3) + (0.5, 0.5)$) to node[commnum]{1} (a3);
      \draw[comm, request] ($(a4) + (0.5, 0.5)$) to node[commnum]{1} (a4);
      \draw[comm, phase2] (a1) to node[commnum]{2} (p1);
      \draw[comm, phase2] (a2) to node[commnum]{2} (p1);
      \draw[comm, phase2] (a3) to node[commnum]{2} (p1);
      \draw[comm, phase2, bend right=10] (a4) to node[commnum]{2} (p1);
      \draw[comm, phase2, bend right=55] (p2) to node[commnum]{3} (a3);
      \draw[comm, phase2, bend right=35] (p2) to node[commnum]{3} (a4);
      \draw[comm, phase2, bend right=30] (p2) to node[commnum]{3} (a5);
      \draw[comm, phase2, bend left=20] (a3) to node[commnum]{4} (p2);
      \draw[comm, phase2, bend left=20] (a4) to node[commnum]{4} (p2);
      \draw[comm, phase2, bend left=20] (a5) to node[commnum]{4} (p2);
      \draw[comm, bend left=23] (p2) to node[commnum]{5} (r1);
      \draw[comm, bend left=25] (p2) to node[commnum]{5} (r2);
      \draw[comm, bend right=23] (p2) to node[commnum]{5} (r3);
    \end{tikzpicture}
    \caption{Coordinated recovery.}\figlabel{FastPaxosDiagram3}
  \end{subfigure}

  \caption{%
    An example execution of Fast MultiPaxos ($f=2$). The proposer of round $0$
    is adorned with a crown. Client requests are drawn in red. Phase 1 messages
    are drawn in blue. Phase 2 messages are drawn in green.
  }%
  \figlabel{FastPaxosDiagram}
\end{figure*}
}

EPaxos, Caesar, Atlas, and the remaining \BPaxos{} protocols all leverage Fast
Paxos~\cite{lamport2006fast}. Fast Paxos is a Paxos variant that allows clients
to propose commands directly to the acceptors without having to initially
contact a proposer. In this section, we review a variant of Fast Paxos.

Like Paxos, a Fast Paxos deployment consists of some number of clients, $f+1$
proposers, and $2f+1$ acceptors. We also include a set of $f+1$ learners. These
nodes are notified of the value chosen by Fast Paxos. A Fast Paxos deployment
is illustrated in \figref{FastPaxosDiagram}. Proposer and acceptor pseudocode
are given in \algoref{FastPaxosProposer} and \algoref{FastPaxosAcceptor}.

Like Paxos, Fast Paxos is divided into a number of integer valued rounds.  The
key difference is that round 0 of Fast Paxos is a special ``fast round.'' A
client can propose a value directly to an acceptor in round 0 without having to
contact a proposer first. The normal case execution of Fast Paxos is
illustrated in \figref{FastPaxosDiagram1}. The execution proceeds as follows:

\newcommand{\maj}[1]{\textsf{maj}(#1)}

\begin{itemize}
  \item \textbf{(1)}
    When a client wants to propose a command $x$, it sends $x$ to all of the
    acceptors.

  \item \textbf{(2)}
    When an acceptor receives a command $x$ from a client, the acceptor ignores
    $x$ if it has already received a message in some round $i \geq 0$.
    Otherwise, it votes for $x$ by updating its state and sending a
    \msg{Phase2b}{0, x} message to the proposer that owns round $0$, in this
    case $p_1$. This is shown in \algoref{FastPaxosAcceptor}
    \lineref{Acceptor1} -- \lineref{Acceptor2}.

  \item \textbf{(3)}
    Let $\maj{n}$ be a majority of $n$. $\maj{n} = \frac{n}{2} + 1$ if $n$ is
    even, and $\maj{n} = \ceil{n/2}$ if $n$ is odd. If the proposer that owns
    round $0$ receives \msg{Phase2b}{0, y} messages from $f + \maj{f+1}$
    acceptors for the same value $y$, then $y$ is chosen, and the proposer
    notifies the learners. This is shown in \algoref{FastPaxosProposer}
    \lineref{Proposer1} -- \lineref{Proposer2}. In this case, $p_0$ receives
    the same command from acceptors $a_1$, $a_2$, $a_3$ and $a_4$. We consider
    what happens when not every value of $y$ is the same later.
\end{itemize}

Note that in Paxos, a value is chosen when $f+1$ acceptors vote for it in some
round $i$. In round $0$ of Fast Paxos, a value is chosen when $f + \maj{f+1}$
acceptors vote for it. The larger number of required votes is needed to ensure
the safety of recovery, which we now describe.

Let $p$ be the proposer in charge of round $0$. Recovery is the process by
which a proposer other than $p$ gets a value chosen. For example, if $p$ fails,
some other proposer must take over and get a value chosen. Recovery is
illustrated in \figref{FastPaxosDiagram2}.

\begin{itemize}
  \item \textbf{(1) and (2)}
    $p_2$ performs Phase 1 of Paxos with at least $f+1$ acceptors in some round
    $i > 0$. This is shown in \algoref{FastPaxosProposer}
    \lineref{ProposerRecovery1} -- \lineref{ProposerRecovery2} and
    \algoref{FastPaxosAcceptor} \lineref{AcceptorPhase11} --
    \lineref{AcceptorPhase12}.

  \item \textbf{(3) and (4)}
    $p_2$ receives \msg{Phase1B}{i, vr, vv} messages from $f+1$ acceptors. Call
    this set of acceptors $Q$. $p_2$ computes $k$ as the largest received $vr$
    (\lineref{computek}). This is the largest round in which any acceptor in
    $Q$ has voted. If $k = -1$ (\lineref{kn1}), then none of the acceptors have
    voted in any round less than $i$, so the proposer is free to propose an
    arbitrary value. This is the same as in Paxos. If $k > 0$ (\lineref{kg0}),
    then the proposer must propose the value $vv$ proposed in round $k$. Again,
    this is the same as in Paxos. $vv$ may have been chosen in round $k$, so
    the proposer is forced to propose it as well. If $k = 0$ (\lineref{ke0}),
    then there are two cases to consider.

    First, if $\maj{f+1}$ of the acceptors in $Q$ have all voted for some value
    $y$ in round $0$, then it's possible that $y$ was chosen in round $0$.
    Specifically, if all $f$ of the acceptors not in $A$ voted for $y$ in round
    $0$, then along with the $\maj{f+1}$ of acceptors in $A$ who also voted for
    $y$ in round $0$, there is a quorum of $f + \maj{f+1}$ acceptors who voted
    for $y$ in round $0$. In this case, the proposer must propose $y$ as well
    since it might have been chosen. Second, if there does not exist
    $\maj{f+1}$ votes for any value $y$, then the proposer concludes that no
    value was chosen or every will be chosen in round $0$, so it is free to
    propose an arbitrary value $vv$.

    Note that a value must receive at least $f + \maj{f+1}$ votes in round $0$
    to be chosen. If this number were any smaller, it would be possible for a
    recovering proposer to find two distinct values $y$ and $y'$ that
    \emph{both} could have been chosen in round $0$. In this case, the proposer
    cannot make progress. It cannot propose $y$ because $y'$ might have been
    chosen, and it cannot propose $y'$ because $y$ might have been chosen

    Once the proposer, $p_2$ in this case, determines which value to propose,
    it gets the value chosen with the acceptors using the normal Phase 2 of
    Paxos.

  \item \textbf{(5)}
    Finally, $p_2$ notifies the learners of the chosen value.
\end{itemize}

Finally, we consider what happens when the proposer of round $0$ receives $f +
\maj{f+1}$ \msgfont{Phase1B} messages from the acceptors, but without all of
them containing the same command $y$. Naively, the proposer could simply
perform a recovery, executing both phases of Paxos is some round $i > 0$.
However, if we assign rounds to proposers in such a way that the proposer of
round $0$ is also the proposer of round $1$, then we can take advantage of an
optimization called \defword{coordinated recovery}. This is illustrated in
\figref{FastPaxosDiagram3} and proceeds as follows:

\begin{itemize}
  \item \textbf{(1)}
    Multiple clients send distinct commands directly to the acceptors.

  \item \textbf{(2)}
    The acceptors receive and vote for the commands and send \msgfont{Phase2B}
    messages to $p_1$. However, not every acceptor receives the same command
    first, so not all the acceptors vote for the same command.

  \item \textbf{(3) and (4)}
    $p_2$ receives \msgfont{Phase2B} messages from $f+1$ acceptors, but the
    acceptors have not all voted for the same command. At this point, the
    proposer could naively perform a recovery in round $1$ by executing Phase 1
    and then Phase 2 of Paxos. But, executing Phase 1 in round $1$ is
    redundant, since the \msgfont{Phase2b} messages in round $0$ contain
    exactly the same information as the \msgfont{Phase1b} messages in round
    $1$. Specifically, the proposer can view every \msg{Phase2b}{0, y} message
    as a proxy for a \msg{Phase1b}{0, 0, y} message. Thus, $p_2$ instead jumps
    immediately to Phase 2 in round $1$ to get a value chosen.

  \item \textbf{(5)}
    Finally, $p_2$ notifies the learners of the chosen value.
\end{itemize}
