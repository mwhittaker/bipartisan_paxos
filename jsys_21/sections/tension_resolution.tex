\section{Tension Resolution}\seclabel{TensionResolution}
The advantage of tension avoidance is that it is simple. The disadvantage is
that it requires large fast Phase 2 quorums. In this section, we explain how to
implement a generalized multi-leader state machine replication protocol using
\defword{tension resolution}. Tension resolution is significantly more
complicated than tension avoidance, but it does not require large fast Phase 2
quorums.

Instead of avoiding the tension between the consensus and dependency invariant,
tension resolution uses additional machinery to resolve it when it arrives.
Consider a scenario where a proposer $p$ is forced to propose a set of
$\deps{v_x}$ in round $i$ to maintain the consensus invariant because
$\deps{v_x}$ may have been chosen in a previous round. Simultaneously, $p$ is
forced not to propose $\deps{v_x}$ because it cannot guarantee that
$\deps{v_x}$ was computed by a majority of the dependency service nodes. This
is the moment of tension that tension avoidance avoids. Tension resolution, on
the other hand, allows this to happen. When it does, the proposer $p$ leverages
additional machinery built into the protocol to determine either that (a)
$\deps{v_x}$ was not chosen or (b) $\deps{v_x}$ was computed by a majority of
dependency service nodes.

We introduce Majority Commit \BPaxos{}, a protocol that implements tension
resolution. We then discuss the protocol's relationship with
EPaxos~\cite{moraru2013there} and Caesar~\cite{enes2020state}.

\subsection{Pruned Dependencies}
Before we discuss Majority Commit \BPaxos{}, we introduce the notion of pruned
dependencies. Imagine a proposer $p$ sends command $x$ to the dependency
service in vertex $v_x$, and the dependency service computes the dependencies
$\deps{v_x}$. Let $v_y \in \deps{v_x}$ be one of $v_x$'s dependencies. To
maintain the dependency invariant, all of the protocols that we have discussed
thus far would get $v_x$ chosen with a dependency on $v_y$, but this is not
always necessary.

Assume that that the proposer $p$ knows that $v_y$ has been chosen with command
$y$ and dependencies $\deps{v_y}$. Further assume that $v_x \in \deps{v_y}$.
That is, $v_y$ has already been chosen with a dependency on $v_x$. In this
case, there is no need for $v_x$ to depend on $v_y$. The dependency invariant
asserts that if two vertices $v_a$ and $v_b$ are chosen with conflicting
commands $a$ and $b$, then either $v_a \in \deps{v_b}$ or $v_b \in \deps{v_a}$.
Thus, in our example, if $v_y$ has already been chosen with a dependency on
$v_x$, then there is no need to propose $v_x$ with a dependency on $v_y$.
%
Similarly, if $v_y$ has been chosen with $\noop$ as part of a recovery, then
there is no need to propose $v_x$ with a dependency on $v_y$ because $x$ and
$\noop$ do not conflict.

Let $\deps{v_x}$ be a set of dependencies computed by the dependency service.
Let $P \subseteq \deps{v_x}$ be a set of vertices $v_y$ such that $v_y$ has
been chosen with $\noop$ or $v_y$ has been chosen with $v_x \in \deps{v_y}$. We
call $\deps{v_x} - P$ the \defword{pruned dependencies} of $v_x$ with respect
to $P$. Majority Commit \BPaxos{} maintains \invref{PrunedDependencies}, the
\defword{pruned dependency invariant}. The pruned dependency invariant is a
relaxation of the dependency invariant. If a protocol maintains the pruned
dependency invariant, it is guaranteed to maintain the dependency invariant.

\begin{invariant}[Pruned Dependency Invariant]\invlabel{PrunedDependencies}
  For every vertex $v_x$, either $(\noop, \emptyset{})$ is chosen in $v_x$ or
  $(x, \deps{v_x} - P)$ is chosen in $v_x$ where $\deps{v_x}$ are dependencies
  computed by the dependency service and where $\deps{v_x} - P$ are the pruned
  dependencies of $v_x$ with respect to some set $P$.
\end{invariant}

\subsection{Majority Commit \BPaxos{}}\seclabel{MajorityCommitBPaxos}
For clarity of exposition, we first introduce a version of Majority Commit
\BPaxos{} that can sometimes deadlock. Later, we modify the protocol to
eliminate the possibility of deadlock.

Majority Commit \BPaxos{} is identical to Fast \BPaxos{} except for the
following two modifications.
%
First, every Fast Paxos acceptor maintains a conflict graph in exactly the same
way as the replicas do. That is, when an acceptor learns that a vertex $v_x$
has been chosen with command $x$ and dependencies $\deps{v_x})$, it adds $v_x$
to its conflict graph with command $x$ and with edges to every vertex in
$\deps{v_x}$. We will see momentarily that whenever a Majority Commit \BPaxos{}
proposer sends a \msgfont{Phase2A} message to the acceptors with value $v = (x,
\deps{v_x} - P)$, the proposer also sends $P$ and all of the commands and
dependencies chosen in in the vertices in $P$. Thus, when an acceptor receives
a \msgfont{Phase2A} message, it updates its conflict graph with the values
chosen in $P$.
%
Second and more substantially, a proposer executes a significantly more complex
recovery procedure. This is shown in \algoref{MajorityCommitBPaxos}.

{\newcommand{\nullbot}{\textsf{null}}

\begin{algorithm*}[ht]
  \caption{%
    Majority Commit \BPaxos{} Proposer. Pseudocode for initiating recovery and
    handling \msgfont{Phase2B} messages is ommitted because it is identical to
    the pseudocode in \algoref{FastPaxosProposer}.
  }%
  \algolabel{MajorityCommitBPaxos}
  \begin{algorithmic}[1]
    \GlobalState a value $v$, initially \nullbot{}
    \GlobalState a round $i$, initially $-1$

    \Upon{receiving \msg{Phase1B}{i, vr, vv} from $f+1$ acceptors $A$}
      \State $k \gets$ the largest $vr$ in any \msg{Phase1B}{i, vr, vv}
      \If{$k = -1$} \linelabel{McbKEqualsNegativeOne}
        \State $v \gets$ an arbitrary value satisfying the dependency invariant
        % Addresses Reviewer 3.
        %
        % > I am unsure why at line 21 of Algorithm 1, the propose is sending to at
        % > least $f+1$ acceptors instead of all acceptors. As $f$ of the acceptors
        % > may crash, so sending to all acceptors in important.
        \State send \msg{Phase2A}{i, v} to \markrevisions{the acceptors}
      \ElsIf{$k > 0$} \linelabel{McbKGreaterThanZero}
        \State $v \gets$ the corresponding $vv$ in round $k$
        % Addresses Reviewer 3.
        %
        % > I am unsure why at line 21 of Algorithm 1, the propose is sending to at
        % > least $f+1$ acceptors instead of all acceptors. As $f$ of the acceptors
        % > may crash, so sending to all acceptors in important.
        \State send \msg{Phase2A}{i, v} to \markrevisions{the acceptors}
      \ElsIf{%
        there are $\maj{f + 1}$ \msg{Phase1B}{i, 0, v'} messages for some value
        $v'$
      } \linelabel{McbMajority}
        \State $(x, \deps{v_x}) \gets v'$
               \linelabel{McbUnwrapVPrime}

        \State send $v_x$ and $x$ to the dependency service nodes co-located
               with the acceptors in $A$
               \linelabel{McbSendToDependencyService}
        \State $\deps{v_x}_A \gets$ the union of the dependencies returned by
               these dependency service nodes
               \linelabel{McbReceiveFromDependencyService}
        \State
        \State $P \gets \emptyset$
        \For{$v_y \in \deps{v_x}_A - \deps{v_x}$}
          \If{we don't know if $v_y$ is chosen}
            \State recover $v_y$, blocking until $v_y$ is recovered
                   \linelabel{McbRecoverVy}
          \EndIf
          \If{$v_y$ chosen with $\noop$ or with $v_x \in \deps{v_y}$}
            \State $P \gets P \cup \set{v_y}$
                   \linelabel{McbPruneVy}
          \Else{}
            \State contact a quorum $A'$ of acceptors
                   \linelabel{McbContactAPrime}
            \If{an acceptor in $A'$ knows $v_x$ is chosen}
              \State abort recovery; $v_x$ has already been chosen
                     \linelabel{McbAbortRecovery}
            \Else{}
              \State $v \gets$ an arbitrary value satisfying the dependency invariant
                     \linelabel{McbNothingChosen}
              % Addresses Reviewer 3.
              %
              % > I am unsure why at line 21 of Algorithm 1, the propose is sending to at
              % > least $f+1$ acceptors instead of all acceptors. As $f$ of the acceptors
              % > may crash, so sending to all acceptors in important.
              \State send \msg{Phase2A}{i, v} to \markrevisions{the acceptors}
            \EndIf{}
          \EndIf{}
        \EndFor{}
        \State $v \gets v'$
        \State send \msg{Phase2A}{i, v} and the values chosen in $P$ to at
               least $f + 1$ acceptors
               \linelabel{McbSuccessfulPruning}
      \Else{} \linelabel{McbNoMajority}
        \State $v \gets$ an arbitrary value satisfying the dependency invariant
        % Addresses Reviewer 3.
        %
        % > I am unsure why at line 21 of Algorithm 1, the propose is sending to at
        % > least $f+1$ acceptors instead of all acceptors. As $f$ of the acceptors
        % > may crash, so sending to all acceptors in important.
        \State send \msg{Phase2A}{i, v} to \markrevisions{the acceptors}
      \EndIf
    \EndUpon
  \end{algorithmic}
\end{algorithm*}
}

As with Fast \BPaxos{}, if $k = -1$ (\lineref{McbKEqualsNegativeOne}), if $k
> 1$ (\lineref{McbKGreaterThanZero}), or if $k = 0$ and there does not exist
$\maj{f+1}$ matching values (\lineref{McbNoMajority}), recovery is
straightforward.

Otherwise, there does exist a $v' = (x, \deps{v_x})$ voted for by at least
$\maj{f+1}$ acceptors in round $0$ (\lineref{McbMajority}). As with Fast
\BPaxos{}, $v'$ may have been chosen in round $0$, so the proposer \emph{must}
propose $v'$ in order to maintain the consensus invariant. But $\deps{v_x}$ may
not be the union of dependencies computed by $f+1$ dependency service nodes, so
the proposer is simultaneously forced \emph{not} to propose $v'$ in order to
maintain the dependency invariant. Unanimous \BPaxos{} avoided this tension by
increasing the size of fast Phase 2 quorums.  Majority Commit \BPaxos{} instead
resolves the tension by performing a more sophisticated recovery procedure.  In
particular, the proposer does a bit of detective work to conclude either that
$v'$ was definitely not chosen in round $0$ (in which case, the proposer can
propose a different value) or that $\deps{v_x}$ happens to be a pruned set of
dependencies (in which case, proposer is safe to propose $v'$).

On \lineref{McbSendToDependencyService} and
\lineref{McbReceiveFromDependencyService}, the proposer sends $v_x$ and $x$ to
the dependency service nodes co-located with the acceptors in $A$ (i.e.\ the
$f+1$ acceptors from which the proposer received \msgfont{Phase1B} messages).
The proposer then computes the union of the returned dependencies, called
$\deps{v_x}_A$. Note that this communication can be piggybacked on the
\msgfont{Phase1A} messages that the proposer previously sent to avoid the extra
round trip of communication.
%
Also note that $\deps{v_x}$ was returned by $\maj{f+1}$ nodes in $A$, so
$\deps{v_x}$ is a subset of $\deps{v_x}_A$.

Next, the proposer enters a for loop in an attempt to prune $\deps{v_x}_A$
until it is equal to $\deps{v_x}$. That is, the proposer attempts to construct
a set of vertices $P$ such that $\deps{v_x} = \deps{v_x}_A - P$ is a set of
pruned dependencies. For every, $v_y \in \deps{v_x}_A - \deps{v_x}$, the
proposer first recovers $v_y$ if it does not know if a value has been chosen in
vertex $v_y$ (\lineref{McbRecoverVy}). After recovering $v_y$, assume the
proposer learns that $v_y$ is chosen with command $y$ and dependencies
$\deps{v_y}$. If $y = \noop$ or if $v_x \in \deps{v_y}$, then the proposer can
safely prune $v_y$ from $\deps{v_x}_A$, so it adds $v_y$ to $P$
(\lineref{McbPruneVy}).

Otherwise, the proposer contacts some quorum $A'$ of acceptors
(\lineref{McbContactAPrime}). If any acceptor $a_j$ in $A'$ knows that vertex
$v_x$ has already been chosen, then the proposer can abort the recovery of
$v_x$ and retrieve the chosen value directly from $a_j$
(\lineref{McbAbortRecovery}). Otherwise, the proposer concludes that no value
was chosen in $v_x$ in round $0$ and is free to propose any value that
maintains the dependency invariant (\lineref{McbNothingChosen}). We will
explain momentarily why the proposer is able to make such a conclusion. It is
not obvious. Note that the proposer can piggyback its communication with $A'$
on its \msgfont{Phase1A} messages.

Finally, if the proposer exits the for loop, then it has successfully pruned
$\deps{v_x}_A$ into $\deps{v_x}_A - P = \deps{v_x}$ and can safely propose it
without violating the consensus or pruned dependency invariant
(\lineref{McbSuccessfulPruning}).  As described above, when the proposer sends
a \msgfont{Phase2A} message with value $v'$, it also includes the values chosen
in every vertex in $P$.

We now return to \lineref{McbNothingChosen} and explain how the proposer is
able to conclude that $v'$ was not chosen in round $0$. On
\lineref{McbNothingChosen}, the proposer has already concluded that $v_y$ was
not chosen with $\noop$ and that $v_x \notin \deps{v_y}$. By the pruned
dependency invariant, $\deps{v_y} = \deps{v_y}_{D} - P'$ is a set of pruned
dependencies where $\deps{v_y}_{D}$ is a set of dependencies computed by a set
$D$ of $f+1$ dependency service nodes. Because $v_x \notin \deps{v_y}_{D} -
P'$, either $v_x \notin \deps{v_y}_{D}$ or $v_x \in P'$.

$v_x$ cannot be in $P'$ because if $v_y$ were chosen with dependencies
$\deps{v_y}_{D} - P'$, then some quorum of acceptors would have received $P'$
and learned that $v_x$ was chosen. But, when the proposer contacted the quorum
$A'$ of acceptors, none knew that $v_x$ was chosen, and any two quorums
intersect.

Thus, $v_x \notin \deps{v_y}_{D}$. Thus, every dependency service node in $D$
processed instance $v_y$ before instance $v_x$. If not, then a dependency
service node in $D$ would have computed $v_x$ as a dependency of $v_y$.
However, if every dependency service node in $D$ processed $v_y$ before $v_x$,
then there cannot exist a fast Phase 2 quorum of dependency service nodes that
processed $v_x$ before $v_y$. In this case, $v' = (x, \deps{v_x})$ could not
have been chosen in round $0$ because it necessitates a fast Phase 2 quorum of
dependency service nodes processing $v_x$ before $v_y$ because $v_y \notin
\deps{v_x}$.

\subsection{Ensuring Liveness}
Majority Commit \BPaxos{} is safe, but it is not very live. There are certain
failure-free situations in which Majority Commit \BPaxos{} can permanently
deadlock. The reason for this is \lineref{McbRecoverVy} in which a proposer
defers the recovery of one vertex for the recovery of another. There exist
executions of Majority Commit \BPaxos{} with a chain of vertices $v_1, \ldots,
v_m$ where the recovery of every vertex $v_i$ depends on the recovery of
vertex $v_{i+1 \bmod m}$.

We now modify Majority Commit \BPaxos{} to prevent deadlock.
%
First, we change the condition under which a value is considered chosen on the
fast path. A proposer considers a value $v = (x, \deps{v_x})$ chosen on the
fast path if a fast Phase 2 quorum $F$ of acceptors voted for $v$ in round $0$
\emph{and} for every vertex $v_y \in \deps{v_x}$, there exists a quorum  $A
\subseteq F$ of $f + 1$ acceptors that knew $v_y$ was chosen at the time of
voting for $v$.
%
Second, when an acceptor $a_i$ sends a \msgfont{Phase2B} vote in round $0$ for
value $v = (x, \deps{v_x})$, $a_i$ also includes the subset of vertices in
$\deps{v_x}$ that $a_i$ knows are chosen, as well as the values chosen in these
vertices.
%
Third, proposers execute \algoref{MajorityCommitBPaxos} but with the lines of
code shown in \algoref{MajorityCommitBPaxosTweak} inserted after
\lineref{McbUnwrapVPrime}.

{\begin{algorithm}[ht]
  \caption{Majority Commit \BPaxos{} proposer modification to prevent deadlock.}%
  \algolabel{MajorityCommitBPaxosTweak}
  \begin{algorithmic}[1]
    \makeatletter
    \setcounter{ALG@line}{10}
    \makeatother
    \State $M \gets$ the set of acceptors in $A$ that voted for $v'$ in round $0$\linelabel{McbmM}
    \If{$\exists v_y \in \deps{v_x}$ such that no acceptor in $M$ knows

        \hspace{-0.13in}that $v_y$ is chosen}
        \linelabel{McbmObliviousM}
      \State send any value satisfying the dependency invariant
    \EndIf{}
    \If{the proposer was recovering $v_z$ and deferred to the

        \hspace{-0.13in}recovery of $v_x$}
      \If{$v_z \in \deps{v_x}$}
        \State abort recovery of $v_z$; $v_z$ has already been chosen
        \linelabel{McbmVzInVx}
      \Else{}
        \State in vertex $v_z$, send any value satisfying the

               \hspace{0.15in}dependency invariant
               \linelabel{McbmVzNotInVx}
      \EndIf{}
    \EndIf{}
  \end{algorithmic}
\end{algorithm}
}

We now explain \algoref{MajorityCommitBPaxosTweak}. On \lineref{McbmM}, the
proposer computes the subset $M \subseteq A$ of acceptors that voted for $v'$
in round $0$. On \lineref{McbmObliviousM}, the proposer determines whether
there exists some instance $v_y \in \deps{v_x}$ such that no acceptor in $M$
knows that $v_y$ is chosen. If such an $v_y$ exists, then $v'$ was not chosen
in round $0$. To see why, assume for contradiction that $v'$ was chosen in
round $0$. Then, there exists some fast Phase 2 quorum $F$ of acceptors that
voted for $v'$ in round $0$, and there exists some quorum $A' \subseteq F$ of
acceptors that know $v_y$ has been chosen. However, $A$ and $A'$ intersect, but
no acceptor in $A$ both voted for $v'$ in round $0$ and knows that $v_y$ was
chosen.  This is a contradiction. Thus, the proposer is free to propose any
value satisfying the dependency invariant.

Next, it's possible that the proposer was previously recovering instance $v_z$
with value $(z, \deps{v_z})$ and executed \lineref{McbRecoverVy} of
\algoref{MajorityCommitBPaxos}, deferring the recovery of instance $v_z$ until
after the recovery of instance $v_x$.
%
If so and if $v_z \in \deps{v_x}$, then some acceptor $a_j \in M$ knows that
$v_z$ is chosen. Thus, the proposer can abort the recovery of instance $v_z$
and retrieve the chosen value directly from $a_j$ (\lineref{McbmVzInVx}).
%
Otherwise, $v_z \notin \deps{v_x}$. In this case, no value was chosen in round
$0$ of instance $v_z$, so the proposer is free to propose any value satisfying
the pruned dependency invariant in instance $v_z$. Here's why. $v_z \notin
\deps{v_x}$, so every dependency service node co-located with an acceptor in
$M$ processed $v_x$ before $v_z$. $|M| \geq \maj{f+1}$, so there strictly fewer
than $f + \maj{f+1}$ remaining dependency service nodes that could have
processed $v_z$ before $v_x$.  If the proposer was recovering instance $v_z$
but deferred to the recovery of instance $v_x$, then $v_x \notin \deps{v_z}$.
In order for $v_z$ to have been chosen in round $0$ with $v_x \notin
\deps{v_y}$, it requires that at least $f + \maj{f}$ dependency service nodes
processed $v_z$ before $v_x$, which we just concluded is impossible. Thus,
$v_z$ was not chosen in round $0$.

Majority Commit \BPaxos{} is deadlock free for the following reason. If a
proposer is recovering instance $v_z$ and defers to the recovery of instance
$v_x$, then either the proposer will recover $v_x$ using
\lineref{McbmObliviousM} of \algoref{MajorityCommitBPaxosTweak}
or the proposer will recover $v_z$ using \lineref{McbmVzInVx} or
\lineref{McbmVzNotInVx} of \algoref{MajorityCommitBPaxosTweak}.
In either case, any potential deadlock is avoided.

\subsection{EPaxos and Caesar}
EPaxos~\cite{moraru2013there} and Caesar~\cite{arun2017speeding} are two
generalized multi-leader protocols that implement tension resolution. EPaxos is
very similar Majority Commit \BPaxos{} with the Basic EPaxos optimization from
\secref{BasicEPaxosOptimization} used to reduce fast Phase 2 quorum sizes by 1.
Majority Commit \BPaxos{} and EPaxos both prune dependencies and perform a
recursive recovery procedure with extra machinery to avoid deadlocks.
%
Caesar improves on EPaxos in two dimensions. First, much like Atlas, a Caesar
proposer does not require that a fast Phase 2 quorum of acceptors vote for the
exact same value in order to take the fast path. Second, Caesar avoids a
recursive recovery procedure. Caesar accomplishes this using a combination of
logical timestamps and carefully placed barriers in the protocol.
