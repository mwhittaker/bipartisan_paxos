\section{Tension Avoidance}\seclabel{TensionAvoidance}
In this section, we explain how to implement a generalized multi-leader state
machine replication protocol using \defword{tension avoidance}. The key idea
behind tension avoidance is to avoid the tension between the consensus and
dependency invariants entirely. By manipulating quorum sizes in clever ways, we
can ensure that whenever a proposer is forced to propose a set of dependencies
$\deps{v_x}$, this set of dependencies is guaranteed to satisfy the dependency
invariant.

We first introduce Unanimous \BPaxos{}, a simple protocol that implements
tension avoidance. We then explain how Basic EPaxos and Atlas can be expressed
as two optimized variants of Unanimous \BPaxos{}.

\subsection{Unanimous \BPaxos{}}\seclabel{UnanimousBPaxos}
A Fast \BPaxos{} deployment consists of $2f+1$ servers. A proposer communicates
with $f+1$ acceptors in Phase 1 called a \defword{Phase 1 quorum}, $f +
\maj{f+1}$ acceptors in Phase 2 of round $0$ called a \defword{fast Phase 2
quorum}, and $f + 1$ acceptors in Phase 2 of rounds greater than $0$ called a
\defword{classic Phase 2 quorum}. If we adjust the sizes of these quorums, we
can avoid the tension between implementing consensus and computing
dependencies.

In~\cite{howard2021fast}, Howard et.\ al prove that Fast Paxos is safe so long
as the following two conditions are met.
\begin{enumerate}
  \item
    Every Phase 1 quorum and every classic Phase 2 quorum intersect. That is,
    for every Phase 1 quorum $Q$ and for every classic Phase 2 quorum $Q'$, $Q
    \cap Q' \neq \emptyset$.

  \item
    Every Phase 1 quorum and every pair of fast Phase 2 quorums intersect. That
    is, for every Phase 1 quorum $Q$ and for every pair of fast Phase 2 quorum
    $Q', Q''$, $Q \cap Q' \cap Q'' \neq \emptyset$.
\end{enumerate}

\defword{Unanimous \BPaxos{}} takes advantage of this result and increases the
sizes of fast Phase 2 quorums. Specifically, Unanimous \BPaxos{} is identical
to Fast \BPaxos{} except with fast Phase 2 quorums of size $2f+1$. Unanimous
\BPaxos{} proposer pseudocode is given in \algoref{UnanimousBPaxosProposer}. It
is identical to the pseudocode of a Fast Paxos proposer
(\algoref{FastPaxosProposer}) except for a couple small changes highlighted in
red.

{\newcommand{\nullbot}{\textsf{null}}
\newcommand{\newstuff}[1]{\textcolor{red}{\underline{#1}}}

\begin{algorithm}[ht]
  \caption{%
    Unanimous \BPaxos{} Proposer. Changes to \algoref{FastPaxosProposer} are
    underlined and highlighted in red.
  }%
  \algolabel{UnanimousBPaxosProposer}
  \begin{algorithmic}[1]
    \GlobalState a value $x$, initially \nullbot{}
    \GlobalState a round $i$, initially $-1$

    \Upon{%
      receiving \msg{Phase2B}{0, y} from \newstuff{all $2f + 1$} acceptors as
      the proposer of round $0$
    }
      \If{every value of $y$ is the same}
        \State $y$ is chosen, notify the learners
      \Else{}
        \State $i \gets 1$
        \State \newstuff{$x \gets$ an arbitrary value}
        \State \newstuff{send \msg{Phase2a}{i, x} to at least $f + 1$ acceptors}
      \EndIf
    \EndUpon

    \Upon{receiving \msg{Phase2B}{i} from $f+1$ acceptors}
      \State $x$ is chosen, notify the learners
    \EndUpon

    \Upon{recovery}
      \State $i \gets$ next largest round owned by this proposer
      \State send \msg{Phase1A}{i} to at least $f+1$ acceptors
    \EndUpon

    \Upon{receiving \msg{Phase1B}{i, vr, vv} from $f+1$ acceptors}
      \State $k \gets$ the largest $vr$ in any \msg{Phase1B}{i, vr, vv}
      \If{$k = -1$}
        \State $x \gets$ an arbitrary value
      \ElsIf{$k > 0$}
        \State $x \gets$ the corresponding $vv$ in round $k$
      \ElsIf{$k = 0$}
      \If{\newstuff{all $f + 1$ messages are of the form} \msg{Phase1B}{i, 0, y} for some value $y$}\linelabel{UnanimousBPaxos}
        \State $x \gets$ $y$ \linelabel{Unanimousk0Start}
        \Else{}
          \State $x \gets$ an arbitrary value
        \EndIf
      \EndIf
      \State send \msg{Phase2a}{i, x} to at least $f + 1$ acceptors \linelabel{Unanimousk0End}
    \EndUpon
  \end{algorithmic}
\end{algorithm}
}

Unlike Fast \BPaxos{}, Unanimous \BPaxos{} is safe. The critical change is on
\lineref{UnanimousBPaxos}. With fast Phase 2 quorums of size $2f+1$, a
recovering proposer knows that a value $v'$ may have been chosen in round $0$
only if all $f+1$ acceptors that it communicates with in Phase 1 voted for $v'$
in round $0$. If even a single acceptor did not vote for $v'$ in round $0$,
then $v'$ could not have received a unanimous vote and therefore was not
chosen in round 0.

With Fast \BPaxos{}, a proposer executing \lineref{majority} of
\algoref{FastPaxosProposer} is forced to propose a value $(x, \deps{v_x})$ if
$\maj{f+1}$ acceptors voted for it in round $0$, but the dependencies
$\deps{v_x}$ may have only been computed by $\maj{f+1}$ dependency service
nodes, violating the dependency invariant. This is exactly what happened in
\figref{FastBPaxosBug}. Unanimous \BPaxos{} avoids the tension entirely because
a proposer is only forced to propose a value $(x, \deps{v_x})$ if $f+1$
acceptors voted for it in round $0$. Now, we are guaranteed that $\deps{v_x}$
was computed by a majority (i.e. $f+1$) of the dependency service nodes. Thus,
Unanimous \BPaxos{} safely maintains the consensus and dependency service
invariants.

We now present two independent optimizations that improve the performance of
Unanimous \BPaxos{}. These optimizations were introduced in
EPaxos~\cite{moraru2013there} and Atlas~\cite{enes2020state}.

\subsection{Basic EPaxos Optimization}\seclabel{BasicEPaxosOptimization}
Unanimous \BPaxos{} has a lower commit time than Simple \BPaxos{} (4 network
delays instead of 7), but has larger fast Phase 2 quorums ($2f+1$ acceptors
instead of $f+1$). We now discuss an optimization, used by Basic
EPaxos~\cite{moraru2013there}, to reduce the fast Phase 2 quorum size to $2f$.

{% Processes.
\tikzstyle{proc}=[draw, circle, thick, inner sep=2pt]
\tikzstyle{client}=[proc, fill=clientcolor!25]
\tikzstyle{proposer}=[proc, fill=proposercolor!25]
\tikzstyle{depservice}=[proc, fill=depservicecolor!25]
\tikzstyle{acceptor}=[proc, fill=acceptorcolor!25]
\tikzstyle{replica}=[proc, fill=replicacolor!25]
\tikzstyle{machine}=[draw=gray, thick]

% Process labels.
\tikzstyle{proclabel}=[inner sep=0pt, align=center]

% Messages and communication.
\tikzstyle{comm}=[-latex, thick]
\tikzstyle{commnum}=[fill=white, inner sep=0pt]
\tikzstyle{newcomm}=[comm, red]

\begin{figure}[ht]
  \centering
  \begin{tikzpicture}[yscale=1.25, xscale=1.5]
    % Processes.
    \foreach \i in {1, 2, 3, 4, 5} {%
      \node[depservice] (d\i) at (\i, 3) {$d_\i$};
      \node[acceptor] (a\i) at (\i, 2) {$a_\i$};
      \node[proposer] (p\i) at (\i, 1) {$p_\i$};
      \node[replica] (r\i) at (\i, 0) {$r_\i$};
    }
    \node[client] (c1) at (0, 2) {$c_1$};
    \node[client] (c2) at (0, 1) {$c_2$};

    % Labels.
    \foreach \i in {1, 2, 3, 4, 5} {
      \crown{(p\i.north)++(0,-0.15)}{0.25}{0.25}
      \draw[machine] ($(d\i.north west) + (-0.15, 0.25)$) rectangle
                     ($(r\i.south east) + (0.15, -0.25)$);
      \node at (\i, -0.75) {Server \i};
    }

    % Communication.
    \draw[comm] (c1) to node[commnum]{1} (p1);
    \draw[comm, bend left] (p1) to node[commnum]{2} (d1);
    \foreach \i in {2, 3, 4, 5} {
      \draw[newcomm, bend left, near end] (d1) to node[commnum]{3} (d\i);
    }
    \foreach \i in {1, 2, 3, 4, 5} {
      \draw[comm, near start] (d\i) to node[commnum]{4} (a\i);
    }
    \draw[comm, bend left] (a1) to node[commnum]{5} (p1);
    \foreach \i in {2, 3, 4, 5} {
      \draw[comm, near start] (a\i) to node[commnum]{5} (p1);
    }
    \draw[newcomm, bend left] (p1) to node[commnum]{6} (a1);
    \draw[newcomm] (a1) to node[commnum]{7} (p1);
    \foreach \i in {1, 2, 3, 4, 5} {
      \draw[comm, near start] (p1) to node[commnum]{8} (r\i);
    }
    \draw[comm] (r1) to node[commnum]{9} (c1);
  \end{tikzpicture}
  \caption{%
    An example execution of Unanimous \BPaxos{} ($f=2$) with the Basic EPaxos
    optimization. The messages introduced by the optimization are drawn in red.
  }\figlabel{BasicEPaxos}
\end{figure}
}

\newcommand{\change}{\textbf{\textcolor{red}{Change:}}}

An execution of Unanimous \BPaxos{} with the Basic EPaxos optimization is shown
in \figref{BasicEPaxos}. We walk through the execution, highlighting the
optimization's key changes. We assume $f > 1$ for now. Later, we discuss the
case when $f = 1$.
\begin{itemize}
  \item \textbf{(1)}
    When a client wants to propose a state machine command $x$, it sends $x$ to
    \emph{any} of the proposers.

  \item \textbf{(2)}
    When a proposer $p_i$ receives a command $x$, from a client, it places $x$
    in a vertex with globally unique vertex id $v_x = (p_i, m)$.
    \change{} $p_i$ then sends $v_x$ and $x$ to the co-located dependency
    service node $d_i$. It does not yet communicate with the other dependency
    service nodes.

  \item \textbf{(3)}
    \change{} When $d_i$ receives $v_x$ and $x$, it computes the
    dependencies $\deps{v_x}_i$ as usual using its acyclic conflict graph.
    $d_i$ then sends $x$, $v_x$, and $\deps{v_x}_i$ to all the other dependency
    service nodes.

  \item \textbf{(4)}
    When a dependency service node $d_j$ receives $v_x$, $x$, and
    $\deps{v_x}_i$, it computes the dependencies $\deps{v_x}_j$ as usual using
    its acyclic conflict graph. \change{} Then, $d_j$ proposes to its
    co-located acceptor $a_j$ that the value $(x, \deps{v_x}_i \cup
    \deps{v_x}_j)$ be chosen in vertex $v_x$ in round $0$. $d_j$ combines the
    dependencies that it computed with the dependencies computed by $d_i$.

  \item \textbf{(5)}
    The acceptors are unchanged. In the normal case, when an acceptor $a_j$
    receives value $(x, \deps{v_x})$ in vertex $v_x = (p_i, m)$, it votes for
    the value and sends the vote to $p_i$.

  \item \textbf{(6) and (7)}
    \change{} In Unanimous \BPaxos{}, a value $v = (x, \deps{v_x})$ is
    considered chosen in round $0$ if all $2f+1$ acceptors vote for $v$ in
    round $0$. With the Basic EPaxos optimization, we only require $2f$ votes,
    and the act of choosing a value in round $0$ is made more explicit.
    Specifically, if $p_i$ receives $2f$ votes for value $v = (x, \deps{v_x})$
    in round $0$, including a vote from $d_i$, then it sends $v$ to the
    co-located acceptor $a_i$. When $a_i$ receives $v$ and is still in round
    $0$ (i.e.\ $r = 0$ on \algoref{FastPaxosAcceptor} \lineref{AcceptorRound}),
    then it records $v$ as chosen and responds to $p_i$. The value $v$ is
    considered chosen precisely when it is recorded by the acceptor. In the
    future $a_i$ responds to every \msgfont{Phase1A} and \msgfont{Phase2A}
    message with a notification that $v$ is chosen. If $a_i$ receives $v$ but
    is already in a round larger than $0$ (i.e.\ $r > 0$), then it ignores $v$
    and sends a negative acknowledgement back to $p_i$.

  \item \textbf{(8)}
    In the normal case, $p_i$ learns that $v$ was successfully chosen by $a_i$
    and it broadcasts $v$ to all the acceptors. If $p_i$ receives a negative
    acknowledgement, it performs coordinated recovery as in Unanimous
    \BPaxos{}.

  \item \textbf{(9)}
    The replicas are unchanged. They maintain and execute conflict graphs,
    returning the results of executing commands to the clients.
\end{itemize}

In addition to these changes made to the normal path of execution, the Basic
EPaxos optimization also introduces a key change to the recovery procedure.
Specifically, we replace \lineref{Unanimousk0Start} -- \lineref{Unanimousk0End}
in \algoref{UnanimousBPaxosProposer} with the following procedure.

Assume that proposer $p$ is recovering vertex $v_x = (p_j, m)$ in round $i >
0$. Either $p$ received a message from $a_j$ or it did not. We consider each
case separately.
%
First, assume that $p$ does receive a message from $a_j$. If $p$ receives a
message indicating that some value $v'$ has already been chosen in round $0$,
then $p$ can terminate the recovery immediately. Otherwise, $p$ receives a
\msgfont{Phase1B} message from $a_j$. From this, $p$ can conclude that $a_j$ is
in a round at least as large as $i$ and therefore did not and will not record
any value $v'$ chosen in round $0$. Because of this, $p$ is safe to propose any
value that satisfies the dependency invariant (e.g., $(\noop, \emptyset{})$).

Otherwise, $p$ does not receive a message from $a_j$. If $p$ receives $f$
\msg{Phase1B}{i, 0, v'} messages for the same value $v' = (x, \deps{v_x})$,
then $v'$ may have been chosen in round $0$, so $p$ must propose $v'$ in order
to maintain the consensus invariant. Note that $\deps{v_x}$ also satisfies the
dependency invariant despite the fact that $p$ only received $\deps{v_x}$ from
$f$, as opposed to $f+1$, dependency service nodes. This is because the
dependency service nodes that are not co-located with $d_j$ all propose
dependencies that include the dependencies computed by $d_j$. Therefore, $p$
determines that $\deps{v_x}$ is the union of $f+1$ dependencies and maintains
the dependency invariant.
%
If $p$ does not receive $f$ \msg{Phase1B}{i, 0, v'} for the same value $v'$,
then $p$ concludes no value was chosen or will be chosen in round $0$,  so $p$
is safe to propose any value that satisfies the dependency invariant.

Note that when $f = 1$ and $n = 3$, Phase 1 quorums, classic Phase 2 quorums,
and fast Phase 2 quorums are all of size $2$. This does \emph{not} satisfy the
conditions outlined by Howard et.\ al~\cite{howard2021fast}. As a result, our
protocol as stated is not safe for $f=1$. The reason is that a recovering
proposer may receive two different values in two separate \msgfont{Phase1B}
messages from the two non-leader acceptors with values $v'$ and $v''$. In this
situation, the proposer is unable to determine which value to propose.
Thankfully, we can avoid this situation by having the leader send only to $2f$
acceptors rather than to all $2f+1$ acceptors.

Ignoring some minor cosmetic differences, Unanimous \BPaxos{} with the Basic
EPaxos optimization is roughly equivalent to Basic
EPaxos~\cite{moraru2013there}.

\subsection{Atlas Optimization}
In the best case, also called the \defword{fast path}, Unanimous \BPaxos{}
achieves a commit time of four network delays. As shown in
\lineref{UnanimousUnanimous} of \algoref{UnanimousBPaxosProposer}, a proposer
executes the fast path only when every single acceptor votes for the exact same
set of dependencies. As we saw in \figref{FastBPaxosBug}, if any two dependency
service nodes receive conflicting commands in different orders, their computed
dependencies will not be the same. If a proposer does not receive a unanimous
vote, it executes coordinated recovery, adding two more network delays to the
commit time.

\newcommand{\popular}[1]{\textsf{popular}(#1)}
Atlas~\cite{enes2020state} introduces the following optimization to relax the
requirement of a unanimous vote and increase the probability of a proposer
executing the fast path. Let $X_1, \ldots, X_{2f+1}$ be $2f+1$ sets. Let
$\popular{X_1, \ldots, X_{2f+1}} = \setst{x}{\text{$x$ appears in at least
$f+1$ of the sets}}$.

We change \lineref{UnanimousUnanimous} as follows. When a proposer receives
dependencies $\deps{v_x}_1, \ldots, \deps{v_x}_{2f+1}$ from the $2f+1$
acceptors, it executes the fast path with dependencies $\deps{v_x} =
\deps{v_x}_1 \cup \cdots \cup \deps{v_x}_{2f+1}$ if $\deps{v_x} =
\popular{\deps{v_x}_1, \ldots, \deps{v_x}_{2f+1}}$. That is, the proposer takes
the fast path only if every dependency $v_y$ computed by any of the dependency
service nodes was computed by a majority of the dependency service nodes.

We also simplify \lineref{Unanimousk0Start} -- \lineref{Unanimousk0End}. If a
recovering proposer receives $f+1$ sets of dependencies, it proposes their
union. Otherwise, it proposes an arbitrary value. This is safe because a set of
dependencies $\deps{v_x}$ can be chosen in round $0$, only if every dependency
in $\deps{v_x}$ was computed by a majority of the dependency service nodes.
Thus, every such element will appear in at least one of the $f+1$ dependency
sets. Thus, the recovering proposer is sure to propose a dependency set if it
was previously chosen (maintaining the consensus invariant), and it also
proposes the union of $f+1$ dependency sets (maintaining the dependency
invariant).

Atlas~\cite{enes2020state} is roughly equivalent to Unanimous \BPaxos{} with
the Basic EPaxos optimization, the Atlas optimization, and the flexible
constraints on quorum sizes outlined in~\cite{howard2021fast}.
