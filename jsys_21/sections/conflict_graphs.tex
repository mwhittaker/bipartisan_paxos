\section{Conflict Graphs}\seclabel{BipartisanPaxos}
\subsection{Defining Conflict Graphs}
By totally ordering state machine commands into a log, state machine
replication protocols like MultiPaxos ensure that \emph{every} replica executes
\emph{every} command in \emph{exactly the same order}. This is a simple way to
ensure that replicas are always in sync, but it is sometimes
unnecessary~\cite{lamport2005generalized}. For example, consider the log shown
at the top of \figref{SwapCommands}. The command \texttt{a=2} (i.e.\ set the
value of variable \texttt{a} to 2) is chosen in log entry 1, and the command
\texttt{b=1} is chosen in log entry 2. With MultiPaxos, every replica would
execute these two commands in exactly the same order, but this is not necessary
because the commands commute. It is safe for some replicas to execute
\texttt{a=2} before \texttt{b=1} while other replicas execute \texttt{b=1}
before \texttt{a=2}. The execution order of the two commands has no effect on
the final state of the state machine, so they can be safely reordered, as shown
in \figref{SwapCommands}.

{\begin{figure}[ht]
  \centering
  \tikzstyle{logentry}=[draw, inner sep=4pt, line width=1pt,
                        minimum height=22pt, minimum width=22pt]
  \tikzstyle{logindex}=[flatred, font=\small]
  \tikzstyle{ignore}=[draw=gray, color=gray]
  \tikzstyle{swap}=[thick, -latex]
  \newcommand{\rightof}[1]{-1pt of #1}
  \begin{tikzpicture}[yscale=1.5]
    % Note that we draw log entry 2 twice so that it's border is not gray.
    \node[logentry, ignore, label={[logindex]90:0}] (top0) at (0, 1) {$x$};
    \node[logentry,         label={[logindex]90:1}, right=\rightof{top0}] (top1) {\texttt{a=2}};
    \node[logentry,         label={[logindex]90:2}, right=\rightof{top1}] (top2) {\texttt{b=1}};
    \node[logentry, ignore, label={[logindex]90:3}, right=\rightof{top2}] (top3) {$y$};
    \node[logentry,         label={[logindex]90:2}, right=\rightof{top1}] (top2) {\texttt{b=1}};
    \node[logentry, ignore, label={[logindex]90:4}, right=\rightof{top3}] (top4) {$z$};
    \node[right=0pt of top4] {$\cdots$};

    % Note that we draw log entry 2 twice so that it's border is not gray.
    \node[logentry, ignore, label={[logindex]-90:0}] (bot0) at (0, 0) {$x$};
    \node[logentry,         label={[logindex]-90:1}, right=\rightof{bot0}] (bot1) {\texttt{b=1}};
    \node[logentry,         label={[logindex]-90:2}, right=\rightof{bot1}] (bot2) {\texttt{a=2}};
    \node[logentry, ignore, label={[logindex]-90:3}, right=\rightof{bot2}] (bot3) {$y$};
    \node[logentry,         label={[logindex]-90:2}, right=\rightof{bot1}] (bot2) {\texttt{a=2}};
    \node[logentry, ignore, label={[logindex]-90:4}, right=\rightof{bot3}] (bot4) {$z$};
    \node[right=0pt of bot4] {$\cdots$};

    % Draw the swapping arrows.
    \draw[swap] (top1.south) to (bot2.north);
    \draw[swap] (top2.south) to (bot1.north);
  \end{tikzpicture}

  \caption{%
    If two commands commute, replicas can safely execute them in either order.
  }\figlabel{SwapCommands}
\end{figure}
}

More formally, we say two commands $x$ and $y$ \defword{conflict} if there
exists a state in which executing $x$ and then $y$ does not produce the same
responses or final state as executing $y$ and then $x$. We say two commands
\defword{commute} if they do not conflict. If two commands conflict (e.g.,
\texttt{a=1} and \texttt{a=2}), then they need to be executed by every state
machine replica in the same order. But, if two commands commute (e.g.,
\texttt{a=2} and \texttt{b=1}), then they do \emph{not} need to be totally
ordered. State machine replicas can execute them in either order.

Generalized Multi-leader state machine replication protocols like EPaxos,
Caesar, Atlas, and all the \BPaxos{} variants presented in this paper take
advantage of command commutativity. Rather than totally ordering commands into
a log, these protocols \emph{partially} order commands into a directed graph
such that every pair of conflicting commands has an edge between them. We call
these graphs \defword{conflict graphs}. An example log and corresponding
conflict graph is illustrated in \figref{ConflictGraph}. A log consists of a
number log entries, and every log entry has a unique log index (e.g., 4). A
conflict graph consists of a number of \defword{vertices}, and every vertex has
a unique \defword{vertex id} (e.g., $v_4$).

% Addresses Reviewer 1.
%
% > Need an explanation on directed edges. There is no introduction of the
% > meaning of direction. The definition of dependency explained so far
% > doesn't pose any order.
Moreover, a vertex $v$ can have \markrevisions{directed} edges to other
vertices. These are called the \defword{dependencies} of $v$, denoted
$\deps{v}$. \markrevisions{For example, if vertex $v_i$ depends on vertex
$v_j$, then there is an edge from $v_i$ to $v_j$.}
Note that if a pair of commands conflict, then they must have an edge between
them. \markrevisions{This ensures that every replica executes the two commands
in the same order.} For example in \figref{ConflictGraph}, the commands
\texttt{a=b} ($v_0$) and \texttt{a=2} ($v_1$) conflict, so they have an edge
between them. If two commands commute, then they do not have an edge between
them. \markrevisions{This allows replicas to execute the commands in either
oder.} For example, the commands \texttt{a=2} ($v_1$) and \texttt{b=1} ($v_2$)
commute, so there is no edge between them. Finally note that some conflicting
commands (e.g., \texttt{b=a} ($v_3)$ and \texttt{a=3} $(v_4)$) have edges in
both directions, forming a cycle. Ideally, conflict graphs would be acyclic,
but cycles are sometimes unavoidable. The reason for this will become clear
soon.


{\begin{figure}
  \centering

  \begin{subfigure}[c]{0.2\textwidth}
    \centering
    \tikzstyle{logentry}=[draw, inner sep=2pt, line width=1pt,
                          minimum height=20pt, minimum width=20pt]
    \tikzstyle{logindex}=[flatred, font=\small]
    \newcommand{\rightof}[1]{-1pt of #1}
    \begin{tikzpicture}[xscale=1, yscale=1.5]
      \node[logentry, label={[logindex]90:0}] (0) {\texttt{a=b}};
      \node[logentry, label={[logindex]90:1}, right=\rightof{0}] (1) {\texttt{a=2}};
      \node[logentry, label={[logindex]90:2}, right=\rightof{1}] (2) {\texttt{b=1}};
      \node[logentry, label={[logindex]90:3}, right=\rightof{2}] (3) {\texttt{b=a}};
      \node[logentry, label={[logindex]90:4}, right=\rightof{3}] (4) {\texttt{a=2}};
    \end{tikzpicture}
    \caption{}\figlabel{Log}
  \end{subfigure}\hspace{12pt}%
  \begin{subfigure}[c]{0.25\textwidth}
    \centering
    \tikzstyle{vertex}=[draw, line width=1pt, align=center, minimum width=0.3in,
                        minimum height=0.2in, inner sep=2pt]
    \tikzstyle{vertexlabel}=[label distance=-1pt, flatred]
    \tikzstyle{arrow}=[line width=0.75pt, -latex]
    \begin{tikzpicture}[xscale=1.2, yscale=1.5]
      \node[vertex, draw=flatblue, fill=flatblue!10, label={[vertexlabel]-90:$v_0$}] (ab) at (0, 0) {\texttt{a=b}};
      \node[vertex, draw=flatgreen, fill=flatgreen!10, label={[vertexlabel]-90:$v_1$}] (x2) at (1, 0.5) {\texttt{a=2}};
      \node[vertex, draw=flatorange, fill=flatorange!10, label={[vertexlabel]-90:$v_2$}] (y1) at (1, -0.5) {\texttt{b=1}};
      \node[vertex, draw=flatpurple, fill=flatpurple!10, label={[vertexlabel]-90:$v_3$}] (ba) at (2, 0) {\texttt{b=a}};
      \node[vertex, draw=flatpurple, fill=flatpurple!10, label={[vertexlabel]-90:$v_4$}] (x22) at (3, 0) {\texttt{a=2}};

      \draw[arrow] (x2) to (ab);
      \draw[arrow] (y1) to (ab);
      \draw[arrow] (ba) to (ab);
      \draw[arrow] (ba) to (x2);
      \draw[arrow] (ba) to (y1);
      \draw[arrow] (x22) to (ba);
      \draw[arrow, bend right=60] (x22) to (ab);
      \draw[arrow, bend right=60] (ba) to (x22);
    \end{tikzpicture}
    \caption{}\figlabel{Graph}
  \end{subfigure}

  \caption{A log and corresponding conflict graph.}\figlabel{ConflictGraph}
\end{figure}
}

\subsection{Executing Conflict Graphs}
We now explain how to execute a static conflict graph. In the next subsection,
we explain how to execute a dynamic conflict graph that grows over time.
%
Replicas execute logs in prefix order. Replicas execute conflict graphs in
reverse topological order, one strongly connected component at a time.  The
order of executing commands within a strongly connected component is not
important, but every replica must choose the same order. For example, replicas
can execute commands within a component sorted by their vertex id.
%
The conflict graph in \figref{ConflictGraph} has four strongly connected
components, each shaded a different color. Vertices $v_0$, $v_1$, and $v_2$ are
each in their own components, and commands $v_3$ and $v_4$ are in their own
component. Replicas execute these four strongly connected components in reverse
topological order as follows:
\begin{itemize}
  \item
    First, replicas execute \texttt{a=b} ($v_0$).

  \item
    Next, replicas either execute \texttt{a=2} ($v_1$) then \texttt{b=1}
    ($v_2$) or \texttt{b=1} ($v_2$) then \texttt{a=2} ($v_1$). There are no
    edges between vertex $v_1$ and vertex $v_2$, so every replica can execute
    the two vertices in either order.

  \item
    Finally, replicas execute \texttt{b=a} ($v_3$) and \texttt{a=3} ($v_4$) in
    some arbitrary but fixed order. For example, if replicas execute commands
    sorted by their vertex ids, then the replicas would all execute $v_3$ and
    then $v_4$.
\end{itemize}
Executing commands in this way, state machine replicas are guaranteed to remain
in sync. Every replica executes conflicting commands in the same order, but are
free to execute commuting commands in any order.

\subsection{Constructing Conflict Graphs}
In the previous subsection, we explained how to execute a static conflict
graph. In reality, graphs are dynamic and grow over time. MultiPaxos constructs
% Addresses Reviewer 1.
%
% > 3.3 ``a log one entry'' $\Rightarrow$ typo?
\markrevisions{one} log entry at a time. It uses one instance of consensus for
every log entry $i$ to choose which command should be placed in log entry $i$.
Analogously, generalized multi-leader protocols construct a conflict graph one
vertex at a time. They use one instance of consensus for every vertex $v$ to
choose which command should be placed in vertex $v$ and what dependencies, or
outbound edges, $v$ should have.

In \figref{GraphExecution}, we illustrate an example execution of how
the conflict graph from \figref{ConflictGraph} could be constructed over time.
\figref{GraphExecution} also shows an analogous execution in which a log
is constructed over time. Note that a vertex $v$ can be chosen with
dependencies $\deps{v}$ before every vertex in $\deps{v}$ has itself been
chosen. For example in \figref{v3}, $v_3$ is chosen with $\deps{v_3} =
\set{v_0, v_1, v_2, v_4}$ before vertices $v_2$ and $v_4$ are chosen. This is
analogous to how a command is chosen in log entry 3 in \figref{l3} before a
command is chosen in entry 2.

A summary of the differences between logs and graphs is given in
\tabref{LogVsGraph}.

{\begin{figure*}
  \centering

  \tikzstyle{logentry}=[draw, inner sep=2pt, line width=1pt,
                        minimum height=20pt, minimum width=20pt]
  \tikzstyle{logindex}=[flatred, font=\small]
  \tikzstyle{vertex}=[draw, line width=1pt, align=center, minimum width=0.3in,
                      minimum height=0.2in, inner sep=2pt]
  \tikzstyle{vertexlabel}=[label distance=-1pt, flatred]
  \tikzstyle{executed}=[fill=flatgreen, opacity=0.2, draw opacity=1,
                        text opacity=1]
  \tikzstyle{highlight}=[red]
  \tikzstyle{arrow}=[line width=0.75pt, -latex]

  \begin{subfigure}[c]{0.05\textwidth}
    \centering
    \begin{tikzpicture}[xscale=1, yscale=1.5]
      \node[vertex, highlight, executed, label={[vertexlabel]-90:$v_0$}] (ab) at (0, 0) {\texttt{a=b}};
    \end{tikzpicture}
    \caption{}\figlabel{v0}
  \end{subfigure}
  \begin{subfigure}[c]{0.15\textwidth}
    \centering
    \begin{tikzpicture}[xscale=1, yscale=1.5]
      \node[vertex, executed, label={[vertexlabel]-90:$v_0$}] (ab) at (0, 0) {\texttt{a=b}};
      \node[vertex, highlight, executed, label={[vertexlabel]-90:$v_1$}] (a2) at (1, 0.5) {\texttt{a=2}};
      \draw[arrow, highlight] (a2) to (ab);
    \end{tikzpicture}
    \caption{}\figlabel{v1}
  \end{subfigure}
  \begin{subfigure}[c]{0.25\textwidth}
    \centering
    \begin{tikzpicture}[xscale=1, yscale=1.5]
      \node[vertex, executed, label={[vertexlabel]-90:$v_0$}] (ab) at (0, 0) {\texttt{a=b}};
      \node[vertex, executed, label={[vertexlabel]-90:$v_1$}] (a2) at (1, 0.5) {\texttt{a=2}};
      \node[vertex, label={[vertexlabel]-90:$v_2$}] (b1) at (1, -0.5) {\phantom{\texttt{b=1}}};
      \node[vertex, highlight, label={[vertexlabel]-90:$v_3$}] (ba) at (2, 0) {\texttt{b=a}};
      \node[vertex, label={[vertexlabel]-90:$v_4$}] (a22) at (3, 0) {\phantom{\texttt{a=2}}};

      \draw[arrow] (a2) to (ab);
      \draw[arrow, highlight] (ba) to (ab);
      \draw[arrow, highlight] (ba) to (a2);
      \draw[arrow, highlight] (ba) to (b1);
      \draw[arrow, highlight, bend right=60] (ba) to (a22);
    \end{tikzpicture}
    \caption{}\figlabel{v3}
  \end{subfigure}
  \begin{subfigure}[c]{0.25\textwidth}
    \centering
    \begin{tikzpicture}[xscale=1, yscale=1.5]
      \node[vertex, executed, label={[vertexlabel]-90:$v_0$}] (ab) at (0, 0) {\texttt{a=b}};
      \node[vertex, executed, label={[vertexlabel]-90:$v_1$}] (a2) at (1, 0.5) {\texttt{a=2}};
      \node[vertex, highlight, executed, label={[vertexlabel]-90:$v_2$}] (b1) at (1, -0.5) {\texttt{b=1}};
      \node[vertex, label={[vertexlabel]-90:$v_3$}] (ba) at (2, 0) {\texttt{b=a}};
      \node[vertex, label={[vertexlabel]-90:$v_4$}] (a22) at (3, 0) {\phantom{\texttt{a=2}}};

      \draw[arrow] (a2) to (ab);
      \draw[arrow, highlight] (b1) to (ab);
      \draw[arrow] (ba) to (ab);
      \draw[arrow] (ba) to (a2);
      \draw[arrow] (ba) to (b1);
      \draw[arrow, bend right=60] (ba) to (a22);
    \end{tikzpicture}
    \caption{}\figlabel{v2}
  \end{subfigure}
  \begin{subfigure}[c]{0.25\textwidth}
    \centering
    \begin{tikzpicture}[xscale=1, yscale=1.5]
      \node[vertex, executed, label={[vertexlabel]-90:$v_0$}] (ab) at (0, 0) {\texttt{a=b}};
      \node[vertex, executed, label={[vertexlabel]-90:$v_1$}] (a2) at (1, 0.5) {\texttt{a=2}};
      \node[vertex, executed, label={[vertexlabel]-90:$v_2$}] (b1) at (1, -0.5) {\texttt{b=1}};
      \node[vertex, executed, label={[vertexlabel]-90:$v_3$}] (ba) at (2, 0) {\texttt{b=a}};
      \node[vertex, highlight, executed, label={[vertexlabel]-90:$v_4$}] (a22) at (3, 0) {\texttt{a=2}};

      \draw[arrow] (a2) to (ab);
      \draw[arrow] (b1) to (ab);
      \draw[arrow] (ba) to (ab);
      \draw[arrow] (ba) to (a2);
      \draw[arrow] (ba) to (b1);
      \draw[arrow, highlight] (a22) to (ba);
      \draw[arrow, highlight, bend right=60] (a22) to (ab);
      \draw[arrow, bend right=60] (ba) to (a22);
    \end{tikzpicture}
    \caption{}\figlabel{v4}
  \end{subfigure}

  \begin{subfigure}[c]{0.05\textwidth}
    \centering
    \newcommand{\rightof}[1]{-1pt of #1}
    \begin{tikzpicture}[xscale=1, yscale=1.5]
      \node[logentry, highlight, executed, label={[logindex]90:0}] (0) {\texttt{a=b}};
    \end{tikzpicture}
    \caption{}\figlabel{l0}
  \end{subfigure}
  \begin{subfigure}[c]{0.15\textwidth}
    \centering
    \newcommand{\rightof}[1]{-1pt of #1}
    \begin{tikzpicture}[xscale=1, yscale=1.5]
      \node[logentry, executed, label={[logindex]90:0}] (0) {\texttt{a=b}};
      \node[logentry, highlight, executed, label={[logindex]90:1}, right=\rightof{0}] (1) {\texttt{a=2}};
    \end{tikzpicture}
    \caption{}\figlabel{l1}
  \end{subfigure}
  \begin{subfigure}[c]{0.25\textwidth}
    \centering
    \newcommand{\rightof}[1]{-1pt of #1}
    \begin{tikzpicture}[xscale=1, yscale=1.5]
      \node[logentry, executed, label={[logindex]90:0}] (0) {\texttt{a=b}};
      \node[logentry, executed, label={[logindex]90:1}, right=\rightof{0}] (1) {\texttt{a=2}};
      \node[logentry, label={[logindex]90:2}, right=\rightof{1}] (2) {\phantom{\texttt{b=1}}};
      \node[logentry, highlight, label={[logindex]90:3}, right=\rightof{2}] (3) {\texttt{b=a}};
    \end{tikzpicture}
    \caption{}\figlabel{l3}
  \end{subfigure}
  \begin{subfigure}[c]{0.25\textwidth}
    \centering
    \newcommand{\rightof}[1]{-1pt of #1}
    \begin{tikzpicture}[xscale=1, yscale=1.5]
      \node[logentry, executed, label={[logindex]90:0}] (0) {\texttt{a=b}};
      \node[logentry, executed, label={[logindex]90:1}, right=\rightof{0}] (1) {\texttt{a=2}};
      \node[logentry, highlight, executed, label={[logindex]90:2}, right=\rightof{1}] (2) {\texttt{b=1}};
      \node[logentry, executed, label={[logindex]90:3}, right=\rightof{2}] (3) {\texttt{b=a}};
      \node[logentry, highlight, executed, label={[logindex]90:2}, right=\rightof{1}] (2) {\texttt{b=1}};
    \end{tikzpicture}
    \caption{}\figlabel{l2}
  \end{subfigure}
  \begin{subfigure}[c]{0.25\textwidth}
    \centering
    \newcommand{\rightof}[1]{-1pt of #1}
    \begin{tikzpicture}[xscale=1, yscale=1.5]
      \node[logentry, executed, label={[logindex]90:0}] (0) {\texttt{a=b}};
      \node[logentry, executed, label={[logindex]90:1}, right=\rightof{0}] (1) {\texttt{a=2}};
      \node[logentry, executed, label={[logindex]90:2}, right=\rightof{1}] (2) {\texttt{b=1}};
      \node[logentry, executed, label={[logindex]90:3}, right=\rightof{2}] (3) {\texttt{b=a}};
      \node[logentry, highlight, executed, label={[logindex]90:4}, right=\rightof{3}] (4) {\texttt{a=2}};
    \end{tikzpicture}
    \caption{}\figlabel{l4}
  \end{subfigure}

  \caption{%
    In subfigures (a) - (e), we see a conflict graph constructed over time. The
    most recently chosen vertex is drawn in red. The executed commands are
    shaded green.
    %
    (a) The command \texttt{a=b} is chosen in vertex $v_0$ without any
    dependencies. The command is executed immediately.
    %
    (b) The command \texttt{a=2} is chosen in vertex $v_1$ with a dependency on
    $v_0$. The command is executed immediately.
    %
    (c) The command \texttt{b=a} is chosen in vertex $v_3$ with dependencies on
    $v_0$, $v_1$, $v_2$, and $v_4$. No commands have been chosen in $v_2$ and
    $v_4$ yet, so $v_3$ cannot be executed.
    %
    (d) The command \texttt{b=1} is chosen in vertex $v_2$ with a dependency on
    $v_0$. The command is executed immediately.
    %
    (e) The command \texttt{a=2} is chosen in vertex $v_2$ with dependencies on
    $v_0$ and $v_3$. Now $v_3$ and $v_4$ are executed.
    %
    In subfigures (f) - (j), we see an analogous execution for a log.
  }\figlabel{GraphExecution}
\end{figure*}
}

\begin{table}
  \caption{%
    The differences between protocols like MultiPaxos and Raft that organize
    commands in logs and protocols like EPaxos, Caesar, and Atlas that organize
    commands in graphs.
  }
  \tablabel{LogVsGraph}
  \begin{tabular}{lll}
    \toprule
                    & Logs                & Graphs \\\midrule
    data structure  & log                 & conflict graph \\
                    & log entry           & vertex \\
                    & log index (e.g., 4) & vertex id (e.g., $v_4$) \\
                    & total order         & partial order \\
    execution order & log order           & reverse topological order \\
    what's chosen?  & commands            & commands \& dependencies \\
    \bottomrule
  \end{tabular}
\end{table}

\subsection{Two Key Invariants}
Protocols like EPaxos, Caesar, Atlas, and the \BPaxos{} protocols in this paper
all differ in how they assign commands to vertices, how they compute
dependencies, how they implement consensus, and so on. Despite the differences,
all the protocols construct conflict graphs one vertex at a time, choosing a
command and a set of dependencies for every vertex. The protocols all rely on
the following two key invariants for correctness. We call these the
\defword{consensus invariant} (\invref{ConsensusInvariant}) and the
\defword{dependency invariant} (\invref{DependencyInvariant}).

\begin{invariant}[Consensus Invariant]\invlabel{ConsensusInvariant}
  Consensus is implemented for every vertex $v$. That is, at most one value
  $(x, \deps{v})$ is chosen for every vertex $v$.
\end{invariant}%

\begin{invariant}[Dependency Invariant]\invlabel{DependencyInvariant}
  If $(x, \deps{v_x})$ is chosen in vertex $v_x$ and $(y, \deps{v_y})$ is
  chosen in instance $v_y$, and if $x$ and $y$ conflict, then either $v_x \in
  \deps{v_y}$ or $v_y \in \deps{v_x}$ or both. That is, if two chosen commands
  conflict, there is an edge between them.
\end{invariant}

The consensus invariant ensures that replicas always agree on the state of the
conflict graph. It makes it impossible, for example, for two replicas to
disagree on which command is in a vertex or disagree on what dependencies a
vertex has. The dependency invariant ensures that replicas execute conflicting
commands in the same order but does not require that replicas execute commuting
commands in the same order.
% Addresses Reviewer 1.
%
% > 4.5 doesn't prove that the two invariants guarantee linearizability or
% > something similar. I would like to see the correctness argument for SMR,
% > not just to the two invariants.
\markrevisions{%
  These two invariants are sufficient to ensure linearizable execution.
  Intuitively, the history of command execution is equivalent to a serial history
  following any reverse topological ordering of the conflict graph. In fact,
  replicas literally do execute commands serially according to one of the
  reverse topological orderings. For a more formal proof, refer to
  \cite{lamport2005generalized} and \cite{moraru2013proof}.
}
