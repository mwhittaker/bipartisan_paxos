\section{Background}\seclabel{Background}

\subsection{System Model}
Throughout the paper, we assume an asynchronous network model in which messages
can be arbitrarily dropped, delayed, and reordered. We assume machines can fail
by crashing but do not act maliciously; i.e., we do not consider Byzantine
failures. We assume that machines operate at arbitrary speeds, and we do not
assume clock synchronization. Every protocol discussed in this paper assumes
that at most $f$ machines will fail for some configurable $f$.

\subsection{Paxos}
A \defword{consensus protocol} is a protocol that selects a single value from a
set of proposed values. \defword{Paxos}~\cite{lamport1998part,
lamport2001paxos} is one of the oldest and most popular consensus protocols. A
Paxos deployment that tolerates $f$ faults consists of an arbitrary number of
clients, $f+1$ nodes called \defword{proposers}, and $2f+1$ nodes called
\defword{acceptors}, as illustrated in \figref{PaxosBackgroundDiagram}.
%
To reach consensus on a value, an execution of Paxos is divided into a number
of integer valued rounds, each round having two phases: Phase 1 and Phase 2.
Every round is orchestrated by a single pre-determined proposer. For example,
with two proposers $p_1$ and $p_2$, we can predetermine that $p_1$ is
responsible for executing even rounds (i.e.\ $0$, $2$, $4$, $\ldots$), while
$p_2$ is responsible for executing odd rounds (i.e.\ $1$, $3$, $5$, $\ldots$).

{\input{figures/paxos_background_diagram.tex}}

When a proposer executes a round, say round $i$, it attempts to get some value
$x$ chosen in that round. Paxos is a consensus protocol, so it must only choose
a single value. Thus, Paxos must ensure that if a value $x$ is chosen in round
$i$, then no other value besides $x$ can ever be chosen in any round less than
$i$. This is the purpose of Paxos' two phases. In Phase 1 of round $i$, the
proposer contacts the acceptors to (a) learn of any value that may have already
been chosen in any round less than $i$ and (b) prevent any new values from
being chosen in any round less than $i$. In Phase 2, the proposer proposes a
value to the acceptors, and the acceptors vote on whether or not to choose it.
In Phase 2, the proposer is careful to only propose a value $x$ if it learned
through Phase 1 that no other value has been or will be chosen in a previous
round.

More concretely, Paxos executes as follows. When a client wants to propose a
value $x$, it sends $x$ to a proposer $p$. Upon receiving $x$, $p$ begins
executing one round of Paxos, say round $i$. First, it executes Phase 1. It
sends $\PhaseIA{i}$ messages to at least a majority of the acceptors. An
acceptor ignores a $\PhaseIA{i}$ message if it has already received a message
in a larger round. Otherwise, it replies with a $\PhaseIB{i}{vr}{vv}$ message
containing the largest round $vr$ in which the acceptor voted and the value it
voted for, $vv$. If the acceptor hasn't voted yet, then $vr = -1$ and $vv = $
\textsf{null}. When the proposer receives \textsc{Phase1b} messages from a
majority of the acceptors, Phase 1 ends and Phase 2 begins.

% Phase 1 serves two purposes.
% \begin{enumerate}[1.]
%   \item
%     The proposer $p$ learns of any value $v$ that may have already been chosen
%     in in some round less than $i$. If any such value exists, then the proposer
%     must propose it in Phase 2 (if it proposed some other value, then two
%     different values could end up chosen).
%
%   \item
%     Phase 1 prevents any new values from being chosen in any round less than
%     $i$.
% \end{enumerate}
% These two properties are \textbf{critical} to ensure that Paxos chooses at most
% one value.

At the start of Phase 2, the proposer uses the \textsc{Phase1B} messages that
it received in Phase 1 to select a value $x$ such that no value other than $x$
has been or will be chosen in any round less than $i$. Specifically $x$ is the
vote value associated with the largest received vote round, or any value if no
acceptor had voted (see~\cite{lamport2001paxos} for details). Then, the
proposer sends $\PhaseIIA{i}{x}$ messages to at least a majority of the
acceptors. An acceptor ignores a $\PhaseIIA{i}{x}$ message if it has already
received a message in a larger round. Otherwise, it votes for $x$ and sends
back a $\PhaseIIB{i}$ message to the proposer. If a majority of acceptors vote
for the value (i.e.\ if the proposer receives $\PhaseIIB{i}$ messages from at
least a majority of the acceptors), then the value is chosen, and the proposer
informs the client. This execution is illustrated in
\figref{PaxosBackgroundDiagram}.

\TODO[mwhittaker]{Find a citation for this optimization. Heidi's thesis?}
Note that it is safe for the leader of round $0$ (the smallest round) to skip
Phase 1 and proceed directly to Phase 2. Recall that the leader of round $i$
executes Phase 1 to learn of any value that may have already been chosen in any
round less than $i$ and to prevent any new values from being chosen in any
round less than $i$. There are no rounds less than $0$, so these properties are
satisfied trivially. This is a common optimization.

% \subsection{Flexible Paxos}
% Paxos deploys a set of $2f+1$ acceptors, and proposers communicate with at
% least a \emph{majority} of the acceptors in Phase 1 and in Phase 2.
% \defword{Flexible Paxos}~\cite{howard2017flexible} is a Paxos variant that
% generalizes the notion of a \emph{majority} to that of a \emph{quorum}.
%
% Specifically, Flexible Paxos introduces the notion of a \defword{read-write
% quorum system} $(P1; P2)$~\cite{vukolic2013origin}. Given a set $A$ of
% acceptors, $P1$ and $P2$ are sets of \defword{quorums}, where each quorum is a
% subset of $A$. A read-write quorum system satisfies the property that every
% quorum in $P1$ (known as a \defword{Phase 1 quorum}) intersects every quorum in
% $P2$ (known as a \defword{Phase 2 quorum}).
%
% Flexible Paxos is identical to Paxos with the exception that proposers now
% communicate with an arbitrary Phase 1 quorum in Phase 1 and an arbitrary Phase
% 2 quorum in Phase 2. In the remainder of this paper, we assume that all
% protocols operate using quorums from an arbitrary read-write quorum system
% rather than majorities from a fixed set of $2f+1$ acceptors. So when we say
% Paxos, for example, we really mean Flexible Paxos.

%
% For example,
% \[
%   C = (
%     \underbrace{\set{a_1, a_2}}_A;
%     \underbrace{\set{\set{a_1, a_2}}}_{P1};
%     \underbrace{\set{\set{a_1}, \set{a_2}}}_{P2}
%   )
% \]
% is a configuration with two acceptors $a_1$ and $a_2$. There is a single Phase
% 1 quorum that includes both acceptors, and there are two singleton Phase 2
% quorums. On the other hand
% \[
%   C = (\set{a_1, a_2}; \set{\set{a_1}}; \set{\set{a_1, a_2}, \set{a_2}})
% \]
% is \emph{not} a configuration since the Phase 1 quorum $\set{a_1}$ does not
% intersect the Phase 2 quorum $\set{a_2}$.

% \NOTE[heidi]{I might be simpler to define a configuration as just the set of
% phase-2 quorums. So for example, $C=\{\{a_1\},\{a_2\}\}$. Then you can union
% configurations together to calculate the set of quorum to intersect with in
% phase-1.}

% \subsection{Reconfiguration}
% \defword{Reconfiguring} a protocol is the act of changing the set of nodes that
% are participating in the protocol, while the protocol executes. For example,
% we may want to reconfigure a MultiPaxos deployment from acceptors $a_1$, $a_2$,
% $a_3$ to acceptors $a_1$, $a_2$, $a_4$ if acceptor $a_3$ has crashed. There are
% a number of reasons to perform a reconfiguration.  Reconfiguration can be used
% to replace failed nodes. Reconfiguration can be used to increase the fault
% tolerance of a protocol like MultiPaxos by increasing $f$ (and hence the number
% of nodes). Reconfiguration can be used to keep nodes geographically close to
% clients, thereby reducing latency. If a protocol is managed by a job scheduler
% like Borg~\cite{verma2015large} or is deployed in a serverless environment,
% reconfiguration allows the protocol to tolerate frequent preemptions.

\subsection{MultiPaxos}
While consensus is the act of choosing a single value, \defword{state machine
replication} is the act of choosing a sequence (a.k.a.\ log) of values. A state
machine replication protocol manages a number of copies, or \defword{replicas},
of a deterministic state machine. Over time, the protocol constructs a growing
log of state machine commands, and replicas execute the commands in log
order.  By beginning in the same initial state, and by executing the same
commands in the same order, all state machine replicas are kept in sync. This
is illustrated in \figref{StateMachineReplicationExample}.

{\newlength{\logentryinnersep}
\setlength{\logentryinnersep}{4pt}
\newlength{\logentrylinewidth}
\setlength{\logentrylinewidth}{1pt}
\newlength{\logentrywidth}
\setlength{\logentrywidth}{\widthof{$X$}+2\logentryinnersep}
\newcommand{\cmdi}{$x$}
\newcommand{\cmdii}{$y$}
\newcommand{\cmdiii}{$z$}

\tikzstyle{logentry}=[%
  draw,
  inner sep=\logentryinnersep,
  line width=\logentrylinewidth,
  minimum height=\logentrywidth,
  minimum width=\logentrywidth]
\tikzstyle{executed}=[fill=flatgreen, opacity=0.2, draw opacity=1, text opacity=1]
\tikzstyle{logindex}=[flatred, font=\small]

\newcommand{\rightof}[1]{-\logentrylinewidth of #1}

\newcommand{\multipaxoslog}[6]{%
  \node[logentry, label={[logindex]90:0}, #2] (0) {#1};
  \node[logentry, label={[logindex]90:1}, right=\rightof{0}, #4] (1) {#3};
  \node[logentry, label={[logindex]90:2}, right=\rightof{1}, #6] (2) {#5};
}

\begin{figure}[ht]
  \centering
  \begin{subfigure}[b]{0.2\columnwidth}
    \begin{tikzpicture}
      \multipaxoslog{}{}%
                    {}{}%
                    {}{}
    \end{tikzpicture}
    \caption{$t=0$}\figlabel{StateMachineReplicationExampleA}
  \end{subfigure}\hspace{12pt}
  \begin{subfigure}[b]{0.2\columnwidth}
    \begin{tikzpicture}
      \multipaxoslog{\cmdi}{executed}%
                    {}{}%
                    {}{}
    \end{tikzpicture}
    \caption{$t=1$}\figlabel{StateMachineReplicationExampleB}
  \end{subfigure}\hspace{12pt}
  \begin{subfigure}[b]{0.2\columnwidth}
    \begin{tikzpicture}
      \multipaxoslog{\cmdi}{executed}%
                    {}{}%
                    {\cmdiii}{}
    \end{tikzpicture}
    \caption{$t=2$}\figlabel{StateMachineReplicationExampleC}
  \end{subfigure}\hspace{12pt}
  \begin{subfigure}[b]{0.2\columnwidth}
    \begin{tikzpicture}
      \multipaxoslog{\cmdi}{executed}%
                    {\cmdii}{executed}%
                    {\cmdiii}{executed}
    \end{tikzpicture}
    \caption{$t=3$}\figlabel{StateMachineReplicationExampleD}
  \end{subfigure}
  \caption{%
    At time $t=0$, no state machine commands are chosen. At time $t=1$ command
    $x$ is chosen in slot $0$. At times $t=2$ and $t=3$, commands $z$ and $y$
    are chosen in slots $2$ and $1$. Executed commands are shaded green. Note
    that all state machines execute the commands $x$, $y$, $z$ in log order.
  }\figlabel{StateMachineReplicationExample}
\end{figure}
}

\defword{MultiPaxos} is one of the most widely used state machine replication
protocols. Again, we assume the reader is familiar with MultiPaxos, but we
review the most salient bits.
%
MultiPaxos uses one instance of Paxos for every log entry, choosing the command
in the $i$th log entry using the $i$th instance of Paxos.
%
A MultiPaxos deployment that tolerates $f$ faults consists of an arbitrary
number of clients, at least $f+1$ proposers, and $2f+1$ acceptors (like Paxos),
as well as at least $f+1$ replicas, as illustrated in
\figref{MultiPaxosBackgroundDiagram}.

{% Processes.
\tikzstyle{proc}=[draw, circle, thick, inner sep=2pt]
\tikzstyle{client}=[proc, fill=clientcolor!25]
\tikzstyle{proposer}=[proc, fill=proposercolor!25]
\tikzstyle{acceptor}=[proc, fill=acceptorcolor!25]
\tikzstyle{replica}=[proc, fill=replicacolor!25]

% Process labels.
\tikzstyle{proclabel}=[inner sep=0pt, align=center]

% Components.
\tikzstyle{component}=[draw, thick, flatgray, rounded corners]

% Messages and communication.
\tikzstyle{comm}=[-latex, thick]
\tikzstyle{commnum}=[fill=white, inner sep=0pt]

\begin{figure}[h]
  \centering
  \begin{tikzpicture}[xscale=2]
    % Processes.
    \node[client] (c1) at (0, 2) {$c_1$};
    \node[client] (c2) at (0, 1) {$c_2$};
    \node[client] (c3) at (0, 0) {$c_3$};
    \node[proposer] (p1) at (1, 1.5) {$p_1$};
    \node[proposer] (p2) at (1, 0.5) {$p_2$};
    \node[acceptor] (a1) at (2, 2) {$a_1$};
    \node[acceptor] (a2) at (2, 1) {$a_2$};
    \node[acceptor] (a3) at (2, 0) {$a_3$};
    \node[replica] (r1) at (3, 1.5) {$r_1$};
    \node[replica] (r2) at (3, 0.5) {$r_2$};

    % Labels.
    \crown{(p1.north)++(0,-0.15)}{0.25}{0.25}
    \node[proclabel] (clients) at (0, 3) {Clients};
    \node[proclabel] (proposers) at (1, 3) {$f+1$\\Proposers};
    \node[proclabel] (acceptors) at (2, 3) {$2f+1$\\Acceptors};
    \node[proclabel] (replicas) at (3, 3) {$f+1$\\Replicas};
    \halffill{clients}{clientcolor!25}
    \quarterfill{proposers}{proposercolor!25}
    \quarterfill{acceptors}{acceptorcolor!25}
    \quarterfill{replicas}{replicacolor!25}

    % Communication.
    \draw[comm] (c1) to node[commnum]{1} (p1);
    \draw[comm, bend left=15] (p1) to node[commnum]{2} (a1);
    \draw[comm, bend left=15] (p1) to node[commnum]{2} (a2);
    \draw[comm, bend left=15] (a1) to node[commnum]{3} (p1);
    \draw[comm, bend left=15] (a2) to node[commnum]{3} (p1);
    \draw[comm] (p1) to node[commnum]{4} (r1);
    \draw[comm, bend right=40] (p1) to node[commnum]{4} (r2);
    \draw[comm, bend right=45] (r1) to node[commnum]{5} (c1);
  \end{tikzpicture}
  \caption{%
    An example execution of MultiPaxos ($f=1$). The leader is adorned with a
    crown.
  }%
  \figlabel{MultiPaxosBackgroundDiagram}
\end{figure}
}

Initially, one of the proposers is elected leader and runs Phase 1 of Paxos for
every log entry. When a client wants to propose a state machine command
$x$, it sends the command to the leader (1). The leader assigns the command a
log entry $i$ and then runs Phase 2 of the $i$th Paxos instance to get the
value $x$ chosen in entry $i$. That is, the leader sends \msg{Phase2a}{i, x}
messages to the acceptors to vote for value $x$ in slot $i$ (2). In the normal
case, the acceptors all vote for $x$ in slot $i$ and respond with
\msg{Phase2b}{i, x} messages (3). Once the leader learns that a command has
been chosen in a given log entry (i.e.\ once the leader receives
\msg{Phase2b}{i, x} messages from a majority of the acceptors), it informs the
replicas (4). Replicas insert commands into their logs and execute the logs in
prefix order.
