\section{Background}

\subsection{System Model}
Throughout the paper, we assume an asynchronous network model in which messages
can be arbitrarily dropped, delayed, and reordered. We assume machines can fail
by crashing but do not act maliciously; i.e., we do not consider Byzantine
failures. We assume that machines operate at arbitrary speeds, and we do not
assume clock synchronization. Every protocol discussed in this paper assumes
that at most $f$ machines will fail for some configurable $f$.

\subsection{Paxos}
\defword{Consensus} is the act of choosing a single value among a set of
proposed values, and \defword{Paxos}~\cite{lamport1998part} is the de facto
standard consensus protocol. We assume the reader is familiar with Paxos, but
we pause to review the parts of the protocol that are most important to
understand for the rest of this paper.

A Paxos deployment that tolerates $f$ faults consists of an arbitrary number of
clients, at least $f+1$ \defword{proposers}, and $2f+1$ \defword{acceptors}, as
illustrated in \figref{PaxosBackgroundDiagram}. When a client wants to propose
a value, it sends the value to a proposer $p$. The proposer then initiates a
two-phase protocol. In Phase 1, the proposer contacts the acceptors and learns
of any values that may have already been chosen. In Phase 2, the proposer
proposes a value to the acceptors, and the acceptors vote on whether or not to
choose the value. If a value receives votes from a majority of the acceptors,
the value is considered chosen.

More concretely, in Phase 1, $p$ sends \msgfont{Phase1a} messages to at least a
majority of the $2f+1$ acceptors. When an acceptor receives a \msgfont{Phase1a}
message, it replies with a \msgfont{Phase1b} message. When the leader receives
\msgfont{Phase1b} messages from a majority of the acceptors, it begins Phase 2.
%
In Phase 2, the proposer sends \msg{Phase2a}{x} messages to the acceptors with
some value $x$. Upon receiving a \msg{Phase2a}{x} message, an acceptor can
either ignore the message, or vote for the value $x$ and return a
\msg{Phase2b}{x} message to the proposer. Upon receiving \msg{Phase2b}{x}
messages from a majority of the acceptors, the proposed value $x$ is considered
chosen.

{\input{figures/paxos_background_diagram.tex}}

% For simplicity of exposition, we have skipped over quite a few details.  For a
% complete description of Paxos, refer to~\cite{lamport2001paxos}.
% We have not explained how a proposer chooses a value to propose in Phase 2, we
% have not explained how an acceptor decides to vote for a value or ignore it, we
% did not discuss rounds, and so on. For the purposes of this paper though, our
% simplified discussion should suffice.

\subsection{MultiPaxos}
While consensus is the act of choosing a single value, \defword{state machine
replication} is the act of choosing a sequence (a.k.a.\ log) of values. A state
machine replication protocol manages a number of copies, or \defword{replicas},
of a deterministic state machine. Over time, the protocol constructs a growing
log of state machine commands, and replicas execute the commands in log
order.  By beginning in the same initial state, and by executing the same
commands in the same order, all state machine replicas are kept in sync. This
is illustrated in \figref{StateMachineReplicationExample}.

{\newlength{\logentryinnersep}
\setlength{\logentryinnersep}{4pt}
\newlength{\logentrylinewidth}
\setlength{\logentrylinewidth}{1pt}
\newlength{\logentrywidth}
\setlength{\logentrywidth}{\widthof{$X$}+2\logentryinnersep}
\newcommand{\cmdi}{$x$}
\newcommand{\cmdii}{$y$}
\newcommand{\cmdiii}{$z$}

\tikzstyle{logentry}=[%
  draw,
  inner sep=\logentryinnersep,
  line width=\logentrylinewidth,
  minimum height=\logentrywidth,
  minimum width=\logentrywidth]
\tikzstyle{executed}=[fill=flatgreen, opacity=0.2, draw opacity=1, text opacity=1]
\tikzstyle{logindex}=[flatred, font=\small]

\newcommand{\rightof}[1]{-\logentrylinewidth of #1}

\newcommand{\multipaxoslog}[6]{%
  \node[logentry, label={[logindex]90:0}, #2] (0) {#1};
  \node[logentry, label={[logindex]90:1}, right=\rightof{0}, #4] (1) {#3};
  \node[logentry, label={[logindex]90:2}, right=\rightof{1}, #6] (2) {#5};
}

\begin{figure}[ht]
  \centering
  \begin{subfigure}[b]{0.2\columnwidth}
    \begin{tikzpicture}
      \multipaxoslog{}{}%
                    {}{}%
                    {}{}
    \end{tikzpicture}
    \caption{$t=0$}\figlabel{StateMachineReplicationExampleA}
  \end{subfigure}\hspace{12pt}
  \begin{subfigure}[b]{0.2\columnwidth}
    \begin{tikzpicture}
      \multipaxoslog{\cmdi}{executed}%
                    {}{}%
                    {}{}
    \end{tikzpicture}
    \caption{$t=1$}\figlabel{StateMachineReplicationExampleB}
  \end{subfigure}\hspace{12pt}
  \begin{subfigure}[b]{0.2\columnwidth}
    \begin{tikzpicture}
      \multipaxoslog{\cmdi}{executed}%
                    {}{}%
                    {\cmdiii}{}
    \end{tikzpicture}
    \caption{$t=2$}\figlabel{StateMachineReplicationExampleC}
  \end{subfigure}\hspace{12pt}
  \begin{subfigure}[b]{0.2\columnwidth}
    \begin{tikzpicture}
      \multipaxoslog{\cmdi}{executed}%
                    {\cmdii}{executed}%
                    {\cmdiii}{executed}
    \end{tikzpicture}
    \caption{$t=3$}\figlabel{StateMachineReplicationExampleD}
  \end{subfigure}
  \caption{%
    At time $t=0$, no state machine commands are chosen. At time $t=1$ command
    $x$ is chosen in slot $0$. At times $t=2$ and $t=3$, commands $z$ and $y$
    are chosen in slots $2$ and $1$. Executed commands are shaded green. Note
    that all state machines execute the commands $x$, $y$, $z$ in log order.
  }\figlabel{StateMachineReplicationExample}
\end{figure}
}

\defword{MultiPaxos} is one of the most widely used state machine replication
protocols. Again, we assume the reader is familiar with MultiPaxos, but we
review the most salient bits.
%
MultiPaxos uses one instance of Paxos for every log entry, choosing the command
in the $i$th log entry using the $i$th instance of Paxos.
%
A MultiPaxos deployment that tolerates $f$ faults consists of an arbitrary
number of clients, at least $f+1$ proposers, and $2f+1$ acceptors (like Paxos),
as well as at least $f+1$ replicas, as illustrated in
\figref{MultiPaxosBackgroundDiagram}.

{% Processes.
\tikzstyle{proc}=[draw, circle, thick, inner sep=2pt]
\tikzstyle{client}=[proc, fill=clientcolor!25]
\tikzstyle{proposer}=[proc, fill=proposercolor!25]
\tikzstyle{acceptor}=[proc, fill=acceptorcolor!25]
\tikzstyle{replica}=[proc, fill=replicacolor!25]

% Process labels.
\tikzstyle{proclabel}=[inner sep=0pt, align=center]

% Components.
\tikzstyle{component}=[draw, thick, flatgray, rounded corners]

% Messages and communication.
\tikzstyle{comm}=[-latex, thick]
\tikzstyle{commnum}=[fill=white, inner sep=0pt]

\begin{figure}[h]
  \centering
  \begin{tikzpicture}[xscale=2]
    % Processes.
    \node[client] (c1) at (0, 2) {$c_1$};
    \node[client] (c2) at (0, 1) {$c_2$};
    \node[client] (c3) at (0, 0) {$c_3$};
    \node[proposer] (p1) at (1, 1.5) {$p_1$};
    \node[proposer] (p2) at (1, 0.5) {$p_2$};
    \node[acceptor] (a1) at (2, 2) {$a_1$};
    \node[acceptor] (a2) at (2, 1) {$a_2$};
    \node[acceptor] (a3) at (2, 0) {$a_3$};
    \node[replica] (r1) at (3, 1.5) {$r_1$};
    \node[replica] (r2) at (3, 0.5) {$r_2$};

    % Labels.
    \crown{(p1.north)++(0,-0.15)}{0.25}{0.25}
    \node[proclabel] (clients) at (0, 3) {Clients};
    \node[proclabel] (proposers) at (1, 3) {$f+1$\\Proposers};
    \node[proclabel] (acceptors) at (2, 3) {$2f+1$\\Acceptors};
    \node[proclabel] (replicas) at (3, 3) {$f+1$\\Replicas};
    \halffill{clients}{clientcolor!25}
    \quarterfill{proposers}{proposercolor!25}
    \quarterfill{acceptors}{acceptorcolor!25}
    \quarterfill{replicas}{replicacolor!25}

    % Communication.
    \draw[comm] (c1) to node[commnum]{1} (p1);
    \draw[comm, bend left=15] (p1) to node[commnum]{2} (a1);
    \draw[comm, bend left=15] (p1) to node[commnum]{2} (a2);
    \draw[comm, bend left=15] (a1) to node[commnum]{3} (p1);
    \draw[comm, bend left=15] (a2) to node[commnum]{3} (p1);
    \draw[comm] (p1) to node[commnum]{4} (r1);
    \draw[comm, bend right=40] (p1) to node[commnum]{4} (r2);
    \draw[comm, bend right=45] (r1) to node[commnum]{5} (c1);
  \end{tikzpicture}
  \caption{%
    An example execution of MultiPaxos ($f=1$). The leader is adorned with a
    crown.
  }%
  \figlabel{MultiPaxosBackgroundDiagram}
\end{figure}
}

Initially, one of the proposers is elected leader and runs Phase 1 of Paxos for
every log entry. When a client wants to propose a state machine command
$x$, it sends the command to the leader (1). The leader assigns the command a
log entry $i$ and then runs Phase 2 of the $i$th Paxos instance to get the
value $x$ chosen in entry $i$. That is, the leader sends \msg{Phase2a}{i, x}
messages to the acceptors to vote for value $x$ in slot $i$ (2). In the normal
case, the acceptors all vote for $x$ in slot $i$ and respond with
\msg{Phase2b}{i, x} messages (3). Once the leader learns that a command has
been chosen in a given log entry (i.e.\ once the leader receives
\msg{Phase2b}{i, x} messages from a majority of the acceptors), it informs the
replicas (4). Replicas insert commands into their logs and execute the logs in
prefix order.

Note that the leader assigns log entries to commands in increasing order. The
first received command is put in entry $0$, the next command in entry $1$, the
next command in entry $2$, and so on.
%
Also note that even though every replica executes every command, for any given
state machine command $x$, only one replica needs to send the result of
executing $x$ back to the client (5). For example, log entries can be
round-robin partitioned across the replicas.
%
% With $n$ replicas for example, replica $r_i$ returns results for log entries
% $j$ where $j \bmod n \equiv i$.

\subsection{MultiPaxos Doesn't Scale?}\seclabel{MultiPaxosDoesntScale}
It is widely believed that MultiPaxos does not scale. Throughout the paper, we
will explain that this is not true. We can scale MultiPaxos, but first it helps
to understand why trying to scale MultiPaxos in the straightforward and obvious
way does not work. MultiPaxos consists of proposers, acceptors, and replicas.
We discuss each.

First, increasing the number of proposers \emph{does not improve performance}
because every client must send its requests to the leader regardless of the
number proposers. The non-leader replicas are idle and do not contribute to the
protocol during normal operation.

Second, increasing the number of acceptors \emph{hurts performance}. To get a
value chosen, the leader must contact a majority of the acceptors. When we
increase the number of acceptors, we increase the number of acceptors that the
leader has to contact. This decreases throughput because the leader---which is
the throughput bottleneck---has to send and receive more messages per command.
Moreover, every acceptor processes at least half of all commands regardless of
the number of acceptors.

Third, increasing the number of replicas \emph{hurts performance}. The leader
broadcasts chosen commands to all of the replicas, so when we increase the
number of replicas, we increase the load on the leader and decrease MultiPaxos'
throughput. Moreover, every replica must execute every state machine command,
so increasing the number of replicas does not decrease the replicas' load.
