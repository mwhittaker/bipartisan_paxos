\section{Simple \BPaxos{}}\seclabel{SimpleBPaxos}
In this section, we introduce \defword{Simple \BPaxos{}}, an inefficient
protocol that is designed to be easy to understand. By understanding Simple
\BPaxos{}, we will understand of the core mechanisms and invariants that are
common to all generalized multi-leader protocols.

\subsection{Overview}
{\begin{figure}[ht]
  \centering

  \begin{tikzpicture}
    \tikzstyle{machine}=[draw, circle, inner sep=2pt]

    % Ordering Service
    \node[machine] (o1) at (0, 2) {$o_1$};
    \node[machine] (o2) at (1, 2) {$o_2$};
    \node[machine] (o3) at (2, 2) {$o_3$};
    \node[machine] (o4) at (3, 2) {$o_4$};
    \node[machine] (o5) at (4, 2) {$o_5$};
    \node (os) at (2, 3) {Ordering Service};
    \draw[rounded corners]
      ($(o1.south west) + (-0.2, -0.2)$) rectangle
      ($(o5.north east) + (0.2, 0.2)$);

    % Consensus
    \node[machine] (p1) at (6, 2) {$p_1$};
    \node[machine] (p2) at (7, 2) {$p_2$};
    \node[machine] (p3) at (8, 2) {$p_3$};
    \node[machine] (p4) at (9, 2) {$p_4$};
    \node[machine] (p5) at (10, 2) {$p_5$};
    \node (os) at (8, 3) {Consensus Service};
    \draw[rounded corners]
      ($(p1.south west) + (-0.2, -0.2)$) rectangle
      ($(p5.north east) + (0.2, 0.2)$);

    % BPaxos Nodes
    \node[machine] (b1) at (3, 0) {$R_1$};
    \node[machine] (b2) at (4, 0) {$R_2$};
    \node[machine] (b3) at (5, 0) {$R_3$};
    \node[machine] (b4) at (6, 0) {$R_4$};
    \node[machine] (b5) at (7, 0) {$R_5$};
    \draw[rounded corners]
      ($(b1.south west) + (-0.2, -0.2)$) rectangle
      ($(b5.north east) + (0.2, 0.2)$);
    \node (bpaxos) at (5, -1) {BPaxos Nodes};
  \end{tikzpicture}

  \caption{Simple BPaxos}\figlabel{SimpleBPaxos}
\end{figure}
}

As illustrated in \figref{SimpleBPaxos}, a Simple \BPaxos{} deployment consists
of a number of clients, a set of at least $f+1$ Paxos proposers, a set of
$2f+1$ \defword{dependency service nodes}, a set of $2f+1$ Paxos acceptors, and
a set of at least $f+1$ replicas. These nodes have the following responsibilities.
\begin{itemize}
  \item
    The dependency service nodes, collectively called the \defword{dependency
    service}, compute dependencies and maintain the dependency invariant
    (\invref{DependencyInvariant}).

  \item
    The proposers and acceptors implement one instance of Paxos for every
    vertex and maintain the consensus invariant (\invref{ConsensusInvariant}).

  \item
    The replicas construct and execute conflict graphs and send the results of
    executing commands back to the clients.
\end{itemize}

More concretely, Simple \BPaxos{} executes as follows. The numbers here
correspond to the numbered arrows in \figref{SimpleBPaxos}.
\begin{itemize}
  \item \textbf{(1)}
    When a client wants to propose a state machine command $x$, it sends $x$ to
    \emph{any} of the proposers. Note that with MultiPaxos, only one proposer
    is elected leader, but in Simple \BPaxos{}, every proposer is a leader.

  \item \textbf{(2) and (3)}
    When a proposer $p_i$ receives a command $x$, from a client, it places $x$
    in a vertex with globally unique vertex id $v_x = (p_i, m)$ where $m$ is a
    monotonically increasing integer local to $p_i$. For example, proposer
    $p_i$ places the first command that it receives in vertex $(p_i, 0)$, the
    next command in vertex $(p_i, 1)$, the next in $(p_i, 2)$, and so on. The
    proposer then performs a round trip of communication with the dependency
    service. It sends $v_x$ and $x$ to the dependency service, and the
    dependency service replies with the dependencies $\deps{v_x}$. For now, we
    leave this process abstract. We'll explain how the dependency service
    computes dependencies in \secref{DependencyService}.

  \item \textbf{(4) and (5)}
    The proposer $p_i$ then executes Phase 2 of Paxos with the acceptors,
    proposing that the value $(x, \deps{v_x})$ be chosen in the instance of
    Paxos associated with vertex $v_x = (p_i, m)$. This is analogous to a
    MultiPaxos leader running Phase 2, proposing the command $x$ be chosen in
    the instance of Paxos associated with log entry $m$.

    Recall from \secref{Background} that the Paxos proposer executing round 0
    can safely bypass Phase 1. By design, we predetermine that the proposer
    $p_i$ leads round $0$ for vertices of the form $(p_i, m)$. This is why
    $p_i$ can safely bypass Phase 1 and immediately execute Phase 2.

    In the normal case, $p_i$ gets the value $(x, \deps{v_x})$ chosen in vertex
    $v_x$. It is also possible that some other proposer erroneously concluded
    that $p_i$ had failed and proposed some other value in vertex $v_x$, but we
    discuss this scenario later.

  \item \textbf{(6)}
    The proposer $p_i$ broadcasts $v_x$, $x$, and $\deps{v_x}$ to all of the
    replicas. The replicas add vertex $v_x$ to their conflict graph with
    command $x$ and with edges to the vertices in $\deps{v_x}$. The replicas
    execute their conflict graphs as described in \secref{BipartisanPaxos}.

  \item \textbf{(7)}
    Once a replica executes command $x$, it sends the result of executing
    command $x$ back to the client.
\end{itemize}

\subsection{Dependency Service}\seclabel{DependencyService}
The dependency service consists of $2f+1$ dependency service nodes $d_1,
\ldots, d_{2f+1}$. Every dependency service node maintains an acyclic conflict
graph.  These conflict graphs are similar but not equal to the conflict graph
that Simple \BPaxos{} ultimately executes.

\newcommand{\out}[1]{\text{out}(#1)}
When a proposer sends a vertex $v_x$ with command $x$ to the dependency
service, it sends $v_x$ and $x$ to every dependency service node. When a
dependency service node $d_i$ receives $v_x$ and $x$, it performs one of the
following two actions depending on whether $d_i$'s graph already contains
vertex $v_x$.
\begin{itemize}
  \item
    If $d_i$'s conflict graph does not contain vertex $v_x$, then $d_i$ adds
    vertex $v_x$ to its graph with command $x$. $d_i$ adds an edge from $v_x$
    to every other vertex $v_y$ with command $y$ if $x$ and $y$ conflict.
    Letting $\out{v_x}$ be the set of vertices to which $v_x$ has an edge,
    $d_i$ then returns $\out{v_x}$ to the proposer.

  \item
    Otherwise, if $d_i$'s conflict graph already contains vertex $v_x$, then
    $d_i$ does not modify its conflict graph. It immediately returns
    $\out{v_x}$ to the proposer.
\end{itemize}
An example execution of a dependency service node is given in
\figref{DependencyService}.

When a proposer receives replies from $f+1$ dependency service nodes, it
takes the union of these responses as the value of $\deps{v_x}$. For example,
imagine $f= 1$ and a proposer receives dependencies $\set{v_w, v_y}$ from $d_1$
and dependencies $\set{v_w, v_z}$ from $d_2$. The proposer computes $\deps{v_x}
= \set{v_w, v_y, v_z}$. The dependency service maintains
\invref{DependencyService}.

{\begin{figure*}
  \centering

  \tikzstyle{logentry}=[draw, inner sep=2pt, line width=1pt,
                        minimum height=20pt, minimum width=20pt]
  \tikzstyle{logindex}=[flatred, font=\small]
  \tikzstyle{vertex}=[draw, line width=1pt, align=center, minimum width=0.3in,
                      minimum height=0.2in, inner sep=2pt]
  \tikzstyle{vertexlabel}=[label distance=-1pt, flatred]
  \tikzstyle{highlight}=[red]
  \tikzstyle{arrow}=[line width=0.75pt, -latex]

  \begin{subfigure}[c]{0.12\textwidth}
    \centering
    \begin{tikzpicture}[xscale=1.5, yscale=1.5]
      \node[vertex, highlight, label={[vertexlabel]180:$v_w$}] (w) at (0, 0) {$w$};
    \end{tikzpicture}
    \caption{}
  \end{subfigure}
  \begin{subfigure}[c]{0.17\textwidth}
    \centering
    \begin{tikzpicture}[xscale=1.5, yscale=1.5]
      \node[vertex, label={[vertexlabel]180:$v_w$}] (w) at (0, 0) {$w$};
      \node[vertex, highlight, label={[vertexlabel]90:$v_x$}] (x) at ($(w) + (30:1)$) {$x$};
      \draw[arrow, highlight] (x) to (w);
    \end{tikzpicture}
    \caption{}
  \end{subfigure}
  \begin{subfigure}[c]{0.17\textwidth}
    \centering
    \begin{tikzpicture}[xscale=1.5, yscale=1.5]
      \node[vertex, label={[vertexlabel]180:$v_w$}] (w) at (0, 0) {$w$};
      \node[vertex, label={[vertexlabel]90:$v_x$}] (x) at ($(w) + (30:1)$) {$x$};
      \node[vertex, highlight, label={[vertexlabel]-90:$v_y$}] (y) at ($(w) + (-30:1)$) {$y$};
      \draw[arrow] (x) to (w);
      \draw[arrow, highlight] (y) to (w);
      \draw[arrow, highlight] (y) to (x);
    \end{tikzpicture}
    \caption{}
  \end{subfigure}
  \begin{subfigure}[c]{0.26\textwidth}
    \centering
    \begin{tikzpicture}[xscale=1.5, yscale=1.5]
      \node[vertex, label={[vertexlabel]180:$v_w$}] (w) at (0, 0) {$w$};
      \node[vertex, label={[vertexlabel]90:$v_x$}] (x) at ($(w) + (30:1)$) {$x$};
      \node[vertex, label={[vertexlabel]-90:$v_y$}] (y) at ($(w) + (-30:1)$) {$y$};
      \node[vertex, highlight, label={[vertexlabel]0:$v_z$}] (z) at ($(x) + (-30:1)$) {$z$};
      \draw[arrow] (x) to (w);
      \draw[arrow] (y) to (w);
      \draw[arrow] (y) to (x);
      \draw[arrow, highlight] (z) to (w);
      \draw[arrow, highlight] (z) to (x);
    \end{tikzpicture}
    \caption{}
  \end{subfigure}
  \begin{subfigure}[c]{0.26\textwidth}
    \centering
    \begin{tikzpicture}[xscale=1.5, yscale=1.5]
      \node[vertex, label={[vertexlabel]180:$v_w$}] (w) at (0, 0) {$w$};
      \node[vertex, highlight, label={[vertexlabel]90:$v_x$}] (x) at ($(w) + (30:1)$) {$x$};
      \node[vertex, label={[vertexlabel]-90:$v_y$}] (y) at ($(w) + (-30:1)$) {$y$};
      \node[vertex, label={[vertexlabel]0:$v_z$}] (z) at ($(x) + (-30:1)$) {$z$};
      \draw[arrow, highlight] (x) to (w);
      \draw[arrow] (y) to (w);
      \draw[arrow] (y) to (x);
      \draw[arrow] (z) to (w);
      \draw[arrow] (z) to (x);
    \end{tikzpicture}
    \caption{}
  \end{subfigure}

  \caption{%
    In subfigures (a) -- (e), we see the execution of a dependency service node
    $d_i$.
    %
    (a) $d_i$ receives command $w$ in vertex $v_w$. $d_i$ adds this vertex to
    its conflict graph and because there are no other vertices, it returns the
    dependencies $\deps{v_w} = \emptyset{}$.
    %
    (b) $d_i$ receives command $x$ in vertex $v_x$. $d_i$ adds this vertex to
    its conflict graph. $x$ conflicts with $w$, so $d_i$ adds an edge from
    $v_x$ to $v_w$ and returns the dependencies $\deps{v_x} = \set{v_w}$.
    %
    (c) $d_i$ receives command $y$ in vertex $v_y$. $d_i$ adds this vertex to
    its conflict graph. $y$ conflicts with $w$ and $x$, so $d_i$ adds an edge
    from $v_y$ to $v_w$ and from $v_y$ to $v_x$. It returns the dependencies
    $\deps{v_y} = \set{v_w, v_x}$.
    %
    (d) $d_i$ receives command $z$ in vertex $v_z$. $d_i$ adds this vertex to
    its conflict graph. $z$ conflicts with $w$ and $x$, so $d_i$ adds an edge
    from $v_z$ to $v_w$ and from $v_z$ to $v_x$. It returns the dependencies
    $\deps{v_z} = \set{v_w, v_x}$.
    %
    (e) $d_i$ receives command $x$ in vertex $v_x$. $d_i$'s graph already
    contains vertex $v_x$, so $d_i$ returns the dependencies $\deps{v_x} =
    \set{v_w}$ and does not modify its graph.
  }\figlabel{DependencyService}
\end{figure*}
}

\begin{invariant}\invlabel{DependencyService}
If two conflicting commands $x$ and $y$ in vertices $v_x$ and $v_y$ yield
dependencies $\deps{v_x}$ and $\deps{v_y}$ from the dependency service, then
either $v_x \in \deps{v_y}$ or $v_y \in \deps{v_x}$ or both.
\end{invariant}

\begin{proof}
  Consider conflicting commands $x$ and $y$ in vertices $v_x$ and $v_y$ with
  dependencies $\deps{v_x}$ and $\deps{v_y}$ computed by the dependency
  service. $\deps{v_x}$ is the union of dependencies computed by $f+1$
  dependency service nodes $D_x$. Similarly, $\deps{v_y}$ is the union of
  dependencies computed by $f+1$ dependency service nodes $D_y$. Because $f+1$
  is a majority of $2f+1$, $D_x$ and $D_y$ necessarily intersect. That is,
  there is some dependency service node $d_i$ that is in $D_x$ and $D_y$.
  $d_i$ either received $v_x$ or $v_y$ first.  If it received $v_x$ first, then
  it returns $v_x$ as a dependency of $v_y$, so $v_x \in \deps{v_y}$.  If it
  received $v_y$ first, then it returns $v_y$ as a dependency of $v_x$, so $v_y
  \in \deps{v_x}$.
\end{proof}

% Note that the dependency service may process a vertex more than once, yielding
% different dependencies each time. For example, a proposer may sent $v_x$ and
% $x$ to the dependency service and get back dependencies $\set{v_w, v_y}$.  The
% proposer might resend $v_x$ and $x$ to the dependency service and get a
% different set of dependencies $\set{v_y, v_z}$. Even though the dependency
% service may compute different dependencies for the same vertex, the dependency
% service still maintains \invref{DependencyService} for every possible pair of
% computed dependencies.

\subsection{An Example}
{% Processes.
\tikzstyle{proc}=[draw, circle, thick, inner sep=2pt]
\tikzstyle{boxproc}=[draw, thick, minimum width=1.5cm, minimum height=1.5cm]
\tikzstyle{client}=[proc, fill=clientcolor!25]
\tikzstyle{proposer}=[proc, fill=proposercolor!25]
\tikzstyle{depservice}=[boxproc, depservicecolor, fill=depservicecolor!10]
\tikzstyle{acceptor}=[proc, fill=acceptorcolor!25]
\tikzstyle{replica}=[boxproc, replicacolor, fill=replicacolor!10]
\tikzstyle{xcolor}=[red]
\tikzstyle{ycolor}=[blue]

% Graphs
\tikzstyle{vertex}=[draw, line width=1pt, align=center, inner sep=2pt, fill=white]
\tikzstyle{vertexlabel}=[label distance=-1pt, flatred]
\tikzstyle{highlight}=[red]
\tikzstyle{executed}=[fill=flatgreen, opacity=0.2, draw opacity=1,
                      text opacity=1]
\tikzstyle{arrow}=[line width=0.75pt, -latex]

% Process labels.
\tikzstyle{proclabel}=[inner sep=0pt, align=center]

% Components.
\tikzstyle{component}=[draw, thick, flatgray, rounded corners]

% Messages and communication.
\tikzstyle{comm}=[-latex, thick]
\tikzstyle{commnum}=[fill=white, inner sep=0pt]

\begin{figure*}[ht]
  \centering
  \begin{subfigure}[c]{0.45\textwidth}
    \centering
    \begin{tikzpicture}[]
      % Processes.
      \node[depservice, label={90:$d_1$}] (d1) at (0, 4) {};
      \node[depservice, label={90:$d_2$}] (d2) at (2, 4) {};
      \node[depservice, label={90:$d_3$}] (d3) at (4, 4) {};
      \node[proposer] (p1) at (1.5, 2.25) {$p_1$};
      \node[proposer] (p2) at (2.5, 1.75) {$p_2$};

      \node[acceptor] (a1) at (5, 3) {$a_1$};
      \node[acceptor] (a2) at (5, 2) {$a_2$};
      \node[acceptor] (a3) at (5, 1) {$a_3$};
      \node[replica, label={-90:$r_1$}] (r1) at (1, 0) {};
      \node[replica, label={-90:$r_2$}] (r2) at (3, 0) {};

      % Communication.
      \draw[comm, xcolor] ($(p1) - (2, 0)$) to node[commnum]{$x$} (p1);
      \draw[comm, ycolor] ($(p2) + (2, 0)$) to node[commnum]{$y$} (p2);
    \end{tikzpicture}
    \caption{%
      \normalsize
      $p_1$ receives command $x$; $p_2$ receives command $y$.
    }
    \figlabel{SimpleBPaxosExample1}
  \end{subfigure}
  \begin{subfigure}[c]{0.45\textwidth}
    \centering
    \begin{tikzpicture}[]
      % Processes.
      \node[depservice, label={90:$d_1$}] (d1) at (0, 4) {};
      \node[depservice, label={90:$d_2$}] (d2) at (2, 4) {};
      \node[depservice, label={90:$d_3$}] (d3) at (4, 4) {};
      \node[proposer] (p1) at (1.5, 2.25) {$p_1$};
      \node[proposer] (p2) at (2.5, 1.75) {$p_2$};

      \node[acceptor] (a1) at (5, 3) {$a_1$};
      \node[acceptor] (a2) at (5, 2) {$a_2$};
      \node[acceptor] (a3) at (5, 1) {$a_3$};
      \node[replica, label={-90:$r_1$}] (r1) at (1, 0) {};
      \node[replica, label={-90:$r_2$}] (r2) at (3, 0) {};

      % Graphs.
      \begin{scope}[shift={(-0.25, 3.75)}]
        \node[vertex, label={[vertexlabel]90:$v_x$}] (x) at (0, 0.5) {$x$};
        \node[vertex, label={[vertexlabel]-90:$v_y$}] (y) at (0.5, 0) {$y$};
        \draw[arrow, bend left] (y) to (x);
      \end{scope}
      \begin{scope}[shift={(1.75, 3.75)}]
        \node[vertex, label={[vertexlabel]90:$v_x$}] (x) at (0, 0.5) {$x$};
        \node[vertex, label={[vertexlabel]-90:$v_y$}] (y) at (0.5, 0) {$y$};
        \draw[arrow, bend left] (y) to (x);
      \end{scope}
      \begin{scope}[shift={(3.75, 3.75)}]
        \node[vertex, label={[vertexlabel]90:$v_x$}] (x) at (0, 0.5) {$x$};
        \node[vertex, label={[vertexlabel]-90:$v_y$}] (y) at (0.5, 0) {$y$};
        \draw[arrow, bend left] (x) to (y);
      \end{scope}

      % Communication.
      \draw[comm, latex-latex, xcolor] (p1) to (d1.south);
      \draw[comm, latex-latex, xcolor] (p1) to (d2.south);
      \draw[comm, latex-latex, xcolor] (p1) to (d3.south);
      \draw[comm, latex-latex, ycolor, bend left=40] (p2) to (d1.south);
      \draw[comm, latex-latex, ycolor] (p2) to (d2.south);
      \draw[comm, latex-latex, ycolor] (p2) to (d3.south);
    \end{tikzpicture}
    \caption{%
      \normalsize
      The proposers contact the dependency service.
    }
    \figlabel{SimpleBPaxosExample2}
  \end{subfigure}

  \vspace{12pt}

  \begin{subfigure}[c]{0.45\textwidth}
    \centering
    \begin{tikzpicture}[]
      % Processes.
      \node[depservice, label={90:$d_1$}] (d1) at (0, 4) {};
      \node[depservice, label={90:$d_2$}] (d2) at (2, 4) {};
      \node[depservice, label={90:$d_3$}] (d3) at (4, 4) {};
      \node[proposer] (p1) at (1.5, 2.25) {$p_1$};
      \node[proposer] (p2) at (2.5, 1.75) {$p_2$};

      \node[acceptor] (a1) at (5, 3) {$a_1$};
      \node[acceptor] (a2) at (5, 2) {$a_2$};
      \node[acceptor] (a3) at (5, 1) {$a_3$};
      \node[replica, label={-90:$r_1$}] (r1) at (1, 0) {};
      \node[replica, label={-90:$r_2$}] (r2) at (3, 0) {};

      % Graphs.
      \begin{scope}[shift={(-0.25, 3.75)}]
        \node[vertex, label={[vertexlabel]90:$v_x$}] (x) at (0, 0.5) {$x$};
        \node[vertex, label={[vertexlabel]-90:$v_y$}] (y) at (0.5, 0) {$y$};
        \draw[arrow, bend left] (y) to (x);
      \end{scope}
      \begin{scope}[shift={(1.75, 3.75)}]
        \node[vertex, label={[vertexlabel]90:$v_x$}] (x) at (0, 0.5) {$x$};
        \node[vertex, label={[vertexlabel]-90:$v_y$}] (y) at (0.5, 0) {$y$};
        \draw[arrow, bend left] (y) to (x);
      \end{scope}
      \begin{scope}[shift={(3.75, 3.75)}]
        \node[vertex, label={[vertexlabel]90:$v_x$}] (x) at (0, 0.5) {$x$};
        \node[vertex, label={[vertexlabel]-90:$v_y$}] (y) at (0.5, 0) {$y$};
        \draw[arrow, bend left] (x) to (y);
      \end{scope}

      % Communication.
      \draw[comm, latex-latex, xcolor] (p1) to (a1);
      \draw[comm, latex-latex, xcolor] (p1) to (a2);
      \draw[comm, latex-latex, xcolor, bend right] (p1) to (a3);
      \draw[comm, latex-latex, ycolor] (p2) to (a1);
      \draw[comm, latex-latex, ycolor] (p2) to (a2);
      \draw[comm, latex-latex, ycolor] (p2) to (a3);
    \end{tikzpicture}
    \caption{%
      \normalsize
      The proposers contact the acceptors.
    }
    \figlabel{SimpleBPaxosExample3}
  \end{subfigure}
  \begin{subfigure}[c]{0.45\textwidth}
    \centering
    \begin{tikzpicture}[]
      % Processes.
      \node[depservice, label={90:$d_1$}] (d1) at (0, 4) {};
      \node[depservice, label={90:$d_2$}] (d2) at (2, 4) {};
      \node[depservice, label={90:$d_3$}] (d3) at (4, 4) {};
      \node[proposer] (p1) at (1.5, 2.25) {$p_1$};
      \node[proposer] (p2) at (2.5, 1.75) {$p_2$};

      \node[acceptor] (a1) at (5, 3) {$a_1$};
      \node[acceptor] (a2) at (5, 2) {$a_2$};
      \node[acceptor] (a3) at (5, 1) {$a_3$};
      \node[replica, label={-90:$r_1$}] (r1) at (1, 0) {};
      \node[replica, label={-90:$r_2$}] (r2) at (3, 0) {};

      % Graphs.
      \begin{scope}[shift={(-0.25, 3.75)}]
        \node[vertex, label={[vertexlabel]90:$v_x$}] (x) at (0, 0.5) {$x$};
        \node[vertex, label={[vertexlabel]-90:$v_y$}] (y) at (0.5, 0) {$y$};
        \draw[arrow, bend left] (y) to (x);
      \end{scope}
      \begin{scope}[shift={(1.75, 3.75)}]
        \node[vertex, label={[vertexlabel]90:$v_x$}] (x) at (0, 0.5) {$x$};
        \node[vertex, label={[vertexlabel]-90:$v_y$}] (y) at (0.5, 0) {$y$};
        \draw[arrow, bend left] (y) to (x);
      \end{scope}
      \begin{scope}[shift={(3.75, 3.75)}]
        \node[vertex, label={[vertexlabel]90:$v_x$}] (x) at (0, 0.5) {$x$};
        \node[vertex, label={[vertexlabel]-90:$v_y$}] (y) at (0.5, 0) {$y$};
        \draw[arrow, bend left] (x) to (y);
      \end{scope}

      \begin{scope}[shift={(0.75, -0.25)}]
        \node[vertex, label={[vertexlabel]90:$v_x$}] (x) at (0, 0.5) {$x$};
        \node[vertex, label={[vertexlabel]-90:$v_y$}] (y) at (0.5, 0) {$y$};
        \draw[arrow, bend left] (x) to (y);
        \draw[arrow, bend left] (y) to (x);
      \end{scope}
      \begin{scope}[shift={(2.75, -0.25)}]
        \node[vertex, label={[vertexlabel]90:$v_x$}] (x) at (0, 0.5) {$x$};
        \node[vertex, label={[vertexlabel]-90:$v_y$}] (y) at (0.5, 0) {$y$};
        \draw[arrow, bend left] (x) to (y);
        \draw[arrow, bend left] (y) to (x);
      \end{scope}

      % Communication.
      \draw[comm, xcolor] (p1) to (r1.north);
      \draw[comm, xcolor] (p1) to (r2.north);
      \draw[comm, ycolor] (p2) to (r1.north);
      \draw[comm, ycolor] (p2) to (r2.north);
    \end{tikzpicture}
    \caption{%
      \normalsize
      The proposers notify the replicas.
    }
    \figlabel{SimpleBPaxosExample4}
  \end{subfigure}

  \caption{An example execution of Simple \BPaxos{} ($f=1$).}%
  \figlabel{SimpleBPaxosExample}
\end{figure*}
}

An example execution of Simple \BPaxos{} with $f=1$ is illustrated in
\figref{SimpleBPaxosExample}.
\begin{itemize}
  \item
    In \figref{SimpleBPaxosExample1}, proposer $p_1$ receives command $x$ from
    a client, while proposer $p_2$ receives command $y$ from a client. The
    commands are placed in vertices $v_x$ and $v_y$ respectively.

  \item
    In \figref{SimpleBPaxosExample2}, $p_1$ sends $x$ in $v_x$ to the dependency
    service, while $p_2$ concurrently sends $y$ in $v_y$. Dependency service
    nodes $d_1$ and $d_2$ receive $x$ and then $y$, so they compute $\deps{v_x}
    = \emptyset$ and $\deps{v_y} = \set{v_x}$. $d_3$, on the other hand,
    receives $y$ and then $x$ and computes $\deps{v_x} = \set{v_y}$ and
    $\deps{v_y} = \emptyset$

    $p_1$ receives $\emptyset$ from $d_2$ and $\set{v_y}$ from $d_3$. Two
    dependency service nodes form a majority, so $p_1$ computes $\deps{v_x} =
    \set{v_y} \cup \emptyset = \set{v_y}$. Similarly, $p_2$ receives
    $\set{v_x}$ from $d_2$ and $\emptyset$ from $d_3$, so $p_2$ computes
    $\deps{v_y} = \set{v_x} \cup \emptyset = \set{v_x}$. Note that $p_1$ and
    $p_2$ also receive responses from $d_1$, but proposers form dependencies
    from the first set of $f+1$ dependency service nodes they hear from.

  \item
    In \figref{SimpleBPaxosExample3}, $p_1$ executes Phase 2 of Paxos to get
    the value $(x, \set{v_y})$ chosen in vertex $v_x$. $p_2$ likewise gets the
    value $(y, \set{v_x})$ chosen in vertex $v_y$.

  \item
    In \figref{SimpleBPaxosExample4}, the proposers broadcast their commands to
    the replicas. The replicas add $v_x$ and $v_y$ to their conflict graphs and
    execute the commands once they have received both. One or more of the
    replicas also sends the results of executing $x$ and $y$ back to the
    clients.
\end{itemize}

Note that the replicas' conflict graphs contain a cycle. This is because the
dependency service nodes do not receive every command in the same order. In
\figref{SimpleBPaxosExample}, dependency service nodes $d_2$ and $d_3$ receive
$x$ and $y$ in opposite orders, leading to the two commands depending on each
other. It is tempting to enforce that every dependency service node receive
every command in exactly the same order, but unfortunately, this would be
tantamount to solving consensus~\cite{chandra2007paxos}.

\subsection{Recovery}
{% Processes.
\tikzstyle{proc}=[draw, circle, thick, inner sep=2pt]
\tikzstyle{boxproc}=[draw, thick, minimum width=1.5cm, minimum height=1.5cm]
\tikzstyle{client}=[proc, fill=clientcolor!25]
\tikzstyle{proposer}=[proc, fill=proposercolor!25]
\tikzstyle{depservice}=[boxproc, depservicecolor, fill=depservicecolor!10]
\tikzstyle{acceptor}=[proc, fill=acceptorcolor!25]
\tikzstyle{replica}=[boxproc, replicacolor, fill=replicacolor!10]
\tikzstyle{dead}=[fill=gray, text=white]
\tikzstyle{xcolor}=[red]
\tikzstyle{ycolor}=[blue]

% Graphs
\tikzstyle{vertex}=[draw, line width=1pt, align=center, inner sep=2pt, fill=white]
\tikzstyle{vertexlabel}=[label distance=-1pt, flatred]
\tikzstyle{highlight}=[red]
\tikzstyle{executed}=[fill=flatgreen, opacity=0.2, draw opacity=1,
                      text opacity=1]
\tikzstyle{arrow}=[line width=0.75pt, -latex]

% Process labels.
\tikzstyle{proclabel}=[inner sep=0pt, align=center]

% Components.
\tikzstyle{component}=[draw, thick, flatgray, rounded corners]

% Messages and communication.
\tikzstyle{comm}=[-latex, thick]
\tikzstyle{commnum}=[fill=white, inner sep=0pt]

\begin{figure*}[ht]
  \centering
  \begin{subfigure}[c]{0.45\textwidth}
    \centering
    \begin{tikzpicture}[]
      % Processes.
      \node[depservice, label={90:$d_1$}] (d1) at (0, 4) {};
      \node[depservice, label={90:$d_2$}] (d2) at (2, 4) {};
      \node[depservice, label={90:$d_3$}] (d3) at (4, 4) {};
      \node[proposer, dead] (p1) at (1.5, 2.25) {$p_1$};
      \node[proposer] (p2) at (2.5, 1.75) {$p_2$};

      \node[acceptor] (a1) at (5, 3) {$a_1$};
      \node[acceptor] (a2) at (5, 2) {$a_2$};
      \node[acceptor] (a3) at (5, 1) {$a_3$};
      \node[replica, label={-90:$r_1$}] (r1) at (1, 0) {};
      \node[replica, label={-90:$r_2$}] (r2) at (3, 0) {};

      % Communication.
      \draw[comm, xcolor] ($(p1) - (2, 0)$) to node[commnum]{$x$} (p1);
      \draw[comm, latex-latex] (p1) to (d1.south);
      \draw[comm, latex-latex] (p1) to (d2.south);
      \draw[comm, latex-latex] (p1) to (d3.south);

      % Graphs.
      \begin{scope}[shift={(-0.25, 3.75)}]
        \node[vertex, label={[vertexlabel]90:$v_x$}] (x) at (0, 0.5) {$x$};
      \end{scope}
      \begin{scope}[shift={(1.75, 3.75)}]
        \node[vertex, label={[vertexlabel]90:$v_x$}] (x) at (0, 0.5) {$x$};
      \end{scope}
      \begin{scope}[shift={(3.75, 3.75)}]
        \node[vertex, label={[vertexlabel]90:$v_x$}] (x) at (0, 0.5) {$x$};
      \end{scope}
    \end{tikzpicture}
    \caption{%
      $p_1$ receives $x$, talks to the dependency service, and fails.
    }
    \figlabel{SimpleBPaxosRecovery1}
  \end{subfigure}
  \begin{subfigure}[c]{0.45\textwidth}
    \centering
    \begin{tikzpicture}[]
      % Processes.
      \node[depservice, label={90:$d_1$}] (d1) at (0, 4) {};
      \node[depservice, label={90:$d_2$}] (d2) at (2, 4) {};
      \node[depservice, label={90:$d_3$}] (d3) at (4, 4) {};
      \node[proposer, dead] (p1) at (1.5, 2.25) {$p_1$};
      \node[proposer] (p2) at (2.5, 1.75) {$p_2$};

      \node[acceptor] (a1) at (5, 3) {$a_1$};
      \node[acceptor] (a2) at (5, 2) {$a_2$};
      \node[acceptor] (a3) at (5, 1) {$a_3$};
      \node[replica, label={-90:$r_1$}] (r1) at (1, 0) {};
      \node[replica, label={-90:$r_2$}] (r2) at (3, 0) {};

      % Graphs.
      \begin{scope}[shift={(-0.25, 3.75)}]
        \node[vertex, label={[vertexlabel]90:$v_x$}] (x) at (0, 0.5) {$x$};
        \node[vertex, label={[vertexlabel]-90:$v_y$}] (y) at (0.5, 0) {$y$};
        \draw[arrow, bend left] (y) to (x);
      \end{scope}
      \begin{scope}[shift={(1.75, 3.75)}]
        \node[vertex, label={[vertexlabel]90:$v_x$}] (x) at (0, 0.5) {$x$};
        \node[vertex, label={[vertexlabel]-90:$v_y$}] (y) at (0.5, 0) {$y$};
        \draw[arrow, bend left] (y) to (x);
      \end{scope}
      \begin{scope}[shift={(3.75, 3.75)}]
        \node[vertex, label={[vertexlabel]90:$v_x$}] (x) at (0, 0.5) {$x$};
        \node[vertex, label={[vertexlabel]-90:$v_y$}] (y) at (0.5, 0) {$y$};
        \draw[arrow, bend left] (y) to (x);
      \end{scope}

      \begin{scope}[shift={(0.75, -0.25)}]
        \node[vertex, label={[vertexlabel]90:$v_x$}] (x) at (0, 0.5) {\phantom{$x$}};
        \node[vertex, label={[vertexlabel]-90:$v_y$}] (y) at (0.5, 0) {$y$};
        \draw[arrow, bend left] (y) to (x);
      \end{scope}
      \begin{scope}[shift={(2.75, -0.25)}]
        \node[vertex, label={[vertexlabel]90:$v_x$}] (x) at (0, 0.5) {\phantom{$x$}};
        \node[vertex, label={[vertexlabel]-90:$v_y$}] (y) at (0.5, 0) {$y$};
        \draw[arrow, bend left] (y) to (x);
      \end{scope}

      % Communication.
      \draw[comm, ycolor] ($(p2) + (2, -0.25)$) to node[commnum]{$y$} (p2);
      \draw[comm, latex-latex] (p2) to (d1.south);
      \draw[comm, latex-latex] (p2) to (d2.south);
      \draw[comm, latex-latex] (p2) to (d3.south);
      \draw[comm, latex-latex] (p2) to (a1);
      \draw[comm, latex-latex] (p2) to (a2);
      \draw[comm, latex-latex] (p2) to (a3);
      \draw[comm, latex-latex] (p2) to (r1.north);
      \draw[comm, latex-latex] (p2) to (r2.north);
    \end{tikzpicture}
    \caption{$p_2$ receives $y$ and gets it chosen with a dependency on $v_x$.}
    \figlabel{SimpleBPaxosRecovery2}
  \end{subfigure}

  \vspace{12pt}

  \begin{subfigure}[c]{0.45\textwidth}
    \centering
    \begin{tikzpicture}[]
      % Processes.
      \node[depservice, label={90:$d_1$}] (d1) at (0, 4) {};
      \node[depservice, label={90:$d_2$}] (d2) at (2, 4) {};
      \node[depservice, label={90:$d_3$}] (d3) at (4, 4) {};
      \node[proposer, dead] (p1) at (1.5, 2.25) {$p_1$};
      \node[proposer] (p2) at (2.5, 1.75) {$p_2$};

      \node[acceptor] (a1) at (5, 3) {$a_1$};
      \node[acceptor] (a2) at (5, 2) {$a_2$};
      \node[acceptor] (a3) at (5, 1) {$a_3$};
      \node[replica, label={-90:$r_1$}] (r1) at (1, 0) {};
      \node[replica, label={-90:$r_2$}] (r2) at (3, 0) {};

      % Graphs.
      \begin{scope}[shift={(-0.25, 3.75)}]
        \node[vertex, label={[vertexlabel]90:$v_x$}] (x) at (0, 0.5) {$x$};
        \node[vertex, label={[vertexlabel]-90:$v_y$}] (y) at (0.5, 0) {$y$};
        \draw[arrow, bend left] (y) to (x);
      \end{scope}
      \begin{scope}[shift={(1.75, 3.75)}]
        \node[vertex, label={[vertexlabel]90:$v_x$}] (x) at (0, 0.5) {$x$};
        \node[vertex, label={[vertexlabel]-90:$v_y$}] (y) at (0.5, 0) {$y$};
        \draw[arrow, bend left] (y) to (x);
      \end{scope}
      \begin{scope}[shift={(3.75, 3.75)}]
        \node[vertex, label={[vertexlabel]90:$v_x$}] (x) at (0, 0.5) {$x$};
        \node[vertex, label={[vertexlabel]-90:$v_y$}] (y) at (0.5, 0) {$y$};
        \draw[arrow, bend left] (y) to (x);
      \end{scope}

      \begin{scope}[shift={(0.75, -0.25)}]
        \node[vertex, label={[vertexlabel]90:$v_x$}] (x) at (0, 0.5) {\phantom{$x$}};
        \node[vertex, label={[vertexlabel]-90:$v_y$}] (y) at (0.5, 0) {$y$};
        \draw[arrow, bend left] (y) to (x);
      \end{scope}
      \begin{scope}[shift={(2.75, -0.25)}]
        \node[vertex, label={[vertexlabel]90:$v_x$}] (x) at (0, 0.5) {\phantom{$x$}};
        \node[vertex, label={[vertexlabel]-90:$v_y$}] (y) at (0.5, 0) {$y$};
        \draw[arrow, bend left] (y) to (x);
      \end{scope}

      % Communication.
      \draw[comm] (r1) to (p2);
    \end{tikzpicture}
    \caption{A replica notifies $p_2$ that $v_x$ is unchosen.}
    \figlabel{SimpleBPaxosRecovery3}
  \end{subfigure}
  \begin{subfigure}[c]{0.45\textwidth}
    \centering
    \begin{tikzpicture}[]
      % Processes.
      \node[depservice, label={90:$d_1$}] (d1) at (0, 4) {};
      \node[depservice, label={90:$d_2$}] (d2) at (2, 4) {};
      \node[depservice, label={90:$d_3$}] (d3) at (4, 4) {};
      \node[proposer, dead] (p1) at (1.5, 2.25) {$p_1$};
      \node[proposer] (p2) at (2.5, 1.75) {$p_2$};

      \node[acceptor] (a1) at (5, 3) {$a_1$};
      \node[acceptor] (a2) at (5, 2) {$a_2$};
      \node[acceptor] (a3) at (5, 1) {$a_3$};
      \node[replica, label={-90:$r_1$}] (r1) at (1, 0) {};
      \node[replica, label={-90:$r_2$}] (r2) at (3, 0) {};

      % Graphs.
      \begin{scope}[shift={(-0.25, 3.75)}]
        \node[vertex, label={[vertexlabel]90:$v_x$}] (x) at (0, 0.5) {$x$};
        \node[vertex, label={[vertexlabel]-90:$v_y$}] (y) at (0.5, 0) {$y$};
        \draw[arrow, bend left] (y) to (x);
      \end{scope}
      \begin{scope}[shift={(1.75, 3.75)}]
        \node[vertex, label={[vertexlabel]90:$v_x$}] (x) at (0, 0.5) {$x$};
        \node[vertex, label={[vertexlabel]-90:$v_y$}] (y) at (0.5, 0) {$y$};
        \draw[arrow, bend left] (y) to (x);
      \end{scope}
      \begin{scope}[shift={(3.75, 3.75)}]
        \node[vertex, label={[vertexlabel]90:$v_x$}] (x) at (0, 0.5) {$x$};
        \node[vertex, label={[vertexlabel]-90:$v_y$}] (y) at (0.5, 0) {$y$};
        \draw[arrow, bend left] (y) to (x);
      \end{scope}

      \begin{scope}[shift={(0.75, -0.25)}]
        \node[vertex, label={[vertexlabel]90:$v_x$}] (x) at (0, 0.5) {noop};
        \node[vertex, label={[vertexlabel]-90:$v_y$}] (y) at (0.5, 0) {$y$};
        \draw[arrow, bend left] (y) to (x);
      \end{scope}
      \begin{scope}[shift={(2.75, -0.25)}]
        \node[vertex, label={[vertexlabel]90:$v_x$}] (x) at (0, 0.5) {noop};
        \node[vertex, label={[vertexlabel]-90:$v_y$}] (y) at (0.5, 0) {$y$};
        \draw[arrow, bend left] (y) to (x);
      \end{scope}

      % Communication.
      \draw[comm, latex-latex] (p2) to (a1);
      \draw[comm, latex-latex] (p2) to (a2);
      \draw[comm, latex-latex] (p2) to (a3);
      \draw[comm, latex-latex] (p2) to (r1.north);
      \draw[comm, latex-latex] (p2) to (r2.north);
    \end{tikzpicture}
    \caption{$p_2$ gets a noop chosen in $v_x$.}
    \figlabel{SimpleBPaxosRecovery4}
  \end{subfigure}

  \caption{An example execution of Simple \BPaxos{} recovery ($f=1$).}%
  \figlabel{SimpleBPaxosRecovery}
\end{figure*}
}

It is possible that a command $y$ chosen in vertex $v_y$ depends on an unchosen
vertex $v_x$. If vertex $v_x$ remains forever unchosen, then the command $y$
will never be executed. To avoid this liveness violation, if any replica
notices that vertex $v_x$ has been unchosen for some time, it notifies a
proposer. The proposer then executes Phase 1 and Phase 2 of Paxos with the
acceptors to get a \defword{$\noop$} chosen in vertex $v_x$ without any
dependencies. $\noop$ is a distinguished command that does not affect the state
machine and does not conflict with any other command. An example of this
execution is given in \figref{SimpleBPaxosRecovery}.

\begin{itemize}
  \item
    In \figref{SimpleBPaxosRecovery1}, proposer $p_1$ receives command $x$ from
    a client. It places $x$ in vertex $v_x$ and sends $v_x$ and $x$ to the
    dependency service. Shortly after, it fails.

  \item
    In \figref{SimpleBPaxosRecovery2}, proposer $p_2$ receives command
    $y$ from a client. It places $y$ in $v_y$ and contacts the dependency
    service. The dependency service nodes have already received $x$ in $v_x$,
    so they compute $\deps{v_y} = \set{v_x}$. $p_2$ then gets $y$ chosen in
    vertex $v_y$ with a dependency on $v_x$ and broadcasts it to the replicas.

  \item
    In \figref{SimpleBPaxosRecovery3}, the replicas cannot execute
    vertex $v_y$ because it depends on the unchosen vertex $v_x$. After a
    timeout expires, replica $r_1$ notifies $p_2$ that the vertex has been
    unchosen for some time.

  \item
    In \figref{SimpleBPaxosRecovery4}, $p_2$ executes Phase 1 and Phase
    2 of Paxos in some round $r > 0$ with the acceptors to get the command
    $\noop$ chosen in vertex $v_x$ without any dependencies. $p_2$ notifies the
    replicas, and the replicas place the $\noop$ in vertex $v_x$. The replicas
    execute their conflict graphs in reverse topological order. They execute
    the $\noop$ first (which has no effect) and then execute $y$.

    Note that $p_2$ must execute both phases of Paxos because it is not in
    round $0$. This is necessary to ensure that no other value could have been
    chosen in $v_x$.
\end{itemize}

% Alternatively, $b_i$ can contact the dependency service and check if any
% dependency service node has recorded a command $y$ in instance $I'$. If such a
% command exists, $b_i$ can send the tuple $(I', y)$ to the dependency service,
% and propose $y$ with the resulting dependencies to the consensus service. If no
% such $y$ exists, $b_i$ can propose a $\noop$.

\subsection{Safety}
To ensure that Simple \BPaxos{} is safe, we must ensure that it maintains the
consensus invariant and the dependency invariant. Simple \BPaxos{} maintains
the consensus invariant because it implements Paxos.  The dependency invariant
follows immediately from \invref{DependencyService} and the fact that $\noop$s
don't conflict with any other command.
