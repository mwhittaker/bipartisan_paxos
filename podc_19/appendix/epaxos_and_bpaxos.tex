\section{Comparing EPaxos and BPaxos}\appendixlabel{EPaxosAndBPaxos}
\paragraph{Similarities}
The BPaxos protocols and EPaxos share a lot in common. This is no accident. The
BPaxos protocols are heavily inspired by EPaxos. Here, we list some of the
similarities.
\begin{itemize}
  \item
    EPaxos nodes construct directed graphs of commands and execute commands in
    reverse topological order one strongly connected component at a time.
    BPaxos borrows this execution model directly, formalizing it using notions
    of generalized consensus, dependency graphs, eligible suffixes, and
    condensations.
  \item
    EPaxos also maintains \invref{ConsensusInvariant} and
    \invref{ConflictInvariant}, the two invariants core to the BPaxos
    protocols.
  \item
    Moraru et al.\ present two EPaxos variants in~\cite{moraru2013there}.  The
    basic EPaxos protocol is a simpler version of EPaxos with large fast quorum
    sizes, while the full EPaxos protocol is a much more complex version of
    EPaxos with smaller quorum sizes and sophisticated recovery. This parallels
    our presentation of a simple protocol with large quorum sizes (Unanimous
    BPaxos) and a more complex protocol with smaller quorum sizes (Majority
    Commit BPaxos).
\end{itemize}

\paragraph{Differences}
Here, we list some of the fundamental differences between EPaxos and the BPaxos
protocols.
\begin{itemize}
  \item
    EPaxos requires at least three message delays to commit a command, whereas
    Unanimous BPaxos and Majority Commit BPaxos only require two (the
    theoretical minimum).

  \item
    The BPaxos protocols considers a value $v$ chosen on the fast path if there
    exists a fast quorum of acceptors that voted for $v$ in round $0$.  EPaxos,
    on the other hand, considers a value $v$ chosen on the fast path in
    instance $I$ only if the leader of instance $I$ receives a fast quorum of
    votes for $v$ in round $0$.  That is, in EPaxos, an instance's leader has
    the ultimate authority on whether a value is chosen on the fast path.

  \item
    Because of the previous two differences, EPaxos allows for smaller fast
    quorums of size $f + \floor{\frac{f+1}{2}}$ compared to Majority Commit
    BPaxos' fast quorums of size $\SuperQuorumSize$. These smaller fast quorums
    come at the cost of increased commit latency.

  \item
    Majority Commit BPaxos considers a value $v = (x, \deps{I})$ chosen on the
    fast path only if there exists some fast quorum $\FastQuorum$ of acceptors
    that voted for $v$ in round $0$ and if for every $I' \in \deps{I}$, there
    exists a quorum $\Quorum' \subseteq \FastQuorum$ of acceptors that know
    $I'$ is chosen. EPaxos has a similar condition, but only requires that one
    acceptor in $\FastQuorum$ know that $I'$ is chosen, rather than a quorum.

  \item
    The BPaxos protocols are more modular than EPaxos, making them easier to
    understand. All three of the BPaxos protocols are composed of a set of
    BPaxos nodes, a dependency service, and a consensus service. This allows us
    to understand the BPaxos protocols piece by piece. EPaxos also has notions
    of consensus and dependency computation, but they are tightly coupled.
    Moreover, the BPaxos protocols heavily leverage Fast Paxos as a consensus
    subroutine. EPaxos does leverage some elements of Paxos, but it does not
    rely on an existing consensus protocol for correctness. Thus, proving the
    correctness of EPaxos requires proving that EPaxos successfully implements
    consensus.
\end{itemize}
