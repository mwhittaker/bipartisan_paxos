\section{Caesar Dependency Service}
If BPaxos' dependency service produces replies $(x, \deps{x})$ and $(y,
\deps{y})$ in instances $I_x$ and $I_y$ for conflicting commands $x$ and $y$,
then either $I_x \in \deps{y}$ or $I_y \in \deps{x}$ or both.
Caesar~\cite{arun2017speeding} provides a stronger guarantee. Informally, all
instances are tagged with timestamps, and if two commands conflict the one with
the larger timestamp is guaranteed to have edge to the one with the lower
timestamp. In this section, we describe a dependency service that provides this
property.

\subsection{Not Live Dependency Service}
First, we present a dependency service that is not live. The dependency service
receives requests of the form $Req(I, x, t)$ where $I$ is an instance, $x$ is a
command, and $t$ is a timestamp. Timestamps are assumed to be globally unique.
The dependency service can respond with one of two messages:
\begin{itemize}
  \item
    $Ok(I, x, t, \deps{x})$, which indicates that the request was successfully
    processed and has dependencies $\deps{x}$, or
  \item
    $Nack(I, x, t')$ which indicates that something went wrong and that the
    user should re-send the $Req(I, x, t)$ request with timestamp at least as
    large as $t'$.
\end{itemize}

The dependency service guarantees that for any two responses $Ok(I_x, x, t_x,
\deps{x})$ and $Ok(I_y, y, t_y, \deps{y})$ if $x$ and $y$ conflict, then $I_x
\in \deps{y}$ if $t_x < t_y$ or $I_y \in \deps{x}$ if $t_y < t_x$. In other
words, conflicting commands are guaranteed to have an edge from the command
with a higher timestamp to the command with a lower timestamp.

We implement the dependency service with at least $2f + 1$ dependency service
nodes. Each dependency service node maintains a log indexed by timestamps, as
illustrated below.

\tikzstyle{ibox}=[thick, inner sep=2pt, minimum height=24pt, minimum width=24pt]
\tikzstyle{box}=[draw, ibox]
\begin{center}
  \begin{tikzpicture}
    \node[box, label={270:$0$}] (0) at (0, 0) {};
    \node[box, label={270:$1$}, right=-1pt of 0] (1) {};
    \node[box, label={270:$2$}, right=-1pt of 1] (2) {};
    \node[box, label={270:$3$}, right=-1pt of 2] (3) {};
    \node[box, label={270:$4$}, right=-1pt of 3] (4) {};
    \node[ibox, right=-1pt of 4] (5) {$\cdots$};
  \end{tikzpicture}
\end{center}

When an $Req(I, x, t)$ message is sent to the dependency service, it is sent to
every dependency service node. When dependency service node receives an $Req(I,
x, t)$ message, it does one of two things.
\begin{itemize}
  \item
    If the dependency service node does not have a log entry for any timestamp
    larger than $t$, then it inserts $(I,x)$ in log entry $t$ and responds with
    $Ok(I,x,t,\deps{x})$ where $\deps{x}$ is the set of all instances in the
    log with a command that conflict with $x$. For example, if a dependency
    service node receives $Req(I_x, x, 1)$, it updates its log to look like
    this:

    \begin{center}
      \begin{tikzpicture}
        \node[box, label={270:$0$}] (0) at (0, 0) {};
        \node[box, label={270:$1$}, right=-1pt of 0] (1) {$I_x,x$};
        \node[box, label={270:$2$}, right=-1pt of 1] (2) {};
        \node[box, label={270:$3$}, right=-1pt of 2] (3) {};
        \node[box, label={270:$4$}, right=-1pt of 3] (4) {};
        \node[ibox, right=-1pt of 4] (5) {$\cdots$};
      \end{tikzpicture}
    \end{center}

    and sends back $Ok(I_x, x, 1, \emptyset{})$. If it then receives $Req(I_y,
    y, 3)$ where $x$ conflicts with $y$, then it updates its log to look like
    this:

    \begin{center}
      \begin{tikzpicture}
        \node[box, label={270:$0$}] (0) at (0, 0) {};
        \node[box, label={270:$1$}, right=-1pt of 0] (1) {$I_x,x$};
        \node[box, label={270:$2$}, right=-1pt of 1] (2) {};
        \node[box, label={270:$3$}, right=-1pt of 2] (3) {$I_y,y$};
        \node[box, label={270:$4$}, right=-1pt of 3] (4) {};
        \node[ibox, right=-1pt of 4] (5) {$\cdots$};
      \end{tikzpicture}
    \end{center}

    and sends back $Ok(I_y, y, 3, \set{I_x})$.

  \item
    If the dependency service already has a log entry for a timestamp larger
    than $t$, then it does not modify its log. It replies with $Nack(I_x, x,
    t')$ where $t'$ is the largest log entry it has. For example, if the
    dependency service node above receives $Req(I_z, z, 0)$, it would leave its
    log unchanged and reply with $Nack(I_z, z, 3)$.
\end{itemize}

If a majority of the dependency service nodes respond with an $Ok$ reply, the
union of their dependencies is taken as the response from the dependency
service. If a $Nack(I_x, x, t')$ is received, $Req(I_x, x, t)$ must be re-sent
with a timestamp $t$ larger than $t'$.

Why does this dependency service provide its guarantee. Well consider two
responses $Ok(I_x, x, t_x, \deps{x})$ and $Ok(I_y, y, t_y, \deps{y})$ for
conflicting commands $x$ and $y$. Without loss of generality, assume $t_x <
t_y$. $x$'s $Ok$ response was taken from some quorum $Q_x$ and $y$'s response
from was taken from some quorum $Q_y$. The quorums intersect, so some node saw
both $x$ and $y$. The node must have processed $Req(I_x, x, t_x)$ before
$Req(I_y, y, t_y)$. Otherwise, it would have ignored $x$'s request. Thus, the
node includes $I_x$ in the dependencies of $I_y$.
