\section{Majority Commit BPaxos}
\paragraph{Overview}
Majority Commit BPaxos is almost identical to Deadlock BPaxos except for a key
change to prevent deadlock. The key idea is that we'll change the condition
under which a gadget is considered chosen on the fast path. Up until now, a
value $(a, \deps{a}, \emptyset{}, \emptyset{})$ was considered chosen on the
fast path (i.e., in round 0) if it was voted for by a superquorum
$\superquorum$ of $f + \QuorumMajoritySize$ Fast Paxos acceptors. With Majority
Commit BPaxos, we only consider a value $(a, \deps{a}, \emptyset{},
\emptyset{})$ chosen on the fast path if it was voted for by a superquorum
$\superquorum$ of $\QuorumMajoritySize$ Fast Paxos acceptors \emph{and} for
every instance $I_b \in \deps{a}$, there exists a quorum $\quorum_b \subseteq
\superquorum$ of $f + 1$ of these acceptors that have recorded $(b,
\pruned(\deps{b}), \Pnoop{b}, \Pnotnoop{b})$ as committed in instance $I_b$.

For example, imagine a Majority Commit BPaxos deployment with $5$ nodes (i.e.,
$f = 2$). Imagine the value $(a, \set{I_b, I_c}, \emptyset, \emptyset)$ was
voted for in instance $I_a$ in round $0$ by Fast Paxos acceptors $\superquorum
= \set{p_1, p_2, p_3, p_4}$. For simplicity, assume that $p_5$ voted for
something else. Consider the following scenarios:
\begin{itemize}
  \item
    If none of the acceptors in $\superquorum$ know that $I_b$ has been chosen,
    then $I_a$ is \emph{not} chosen.
  \item
    If $p_1$ knows that $I_b$ has been chosen, but no other node does, then
    $I_a$ is \emph{not} chosen.
  \item
    If $p_1$, $p_2$, and $p_3$ know that $I_b$ has been chosen, but no node in
    $\superquorum$ knows that $I_c$ has been chosen, then $I_a$ is \emph{not}
    chosen.
  \item
    If $p_1$, $p_2$, and $p_3$ know that $I_b$ has been chosen, and $p_2$,
    $p_3$, and $p_4$ know that $I_c$ has been chosen, then $I_a$ is chosen.
\end{itemize}

In addition to this key change, we also make a couple of small modifications to
\algoref{DeadlockBPaxos} that use this key change to avoid deadlock.

\paragraph{Ordering Service}
Majority Commit BPaxos uses the same ordering service as BPaxos, Unanimous
BPaxos, and Deadlock BPaxos.

\paragraph{Consensus Service}
Majority Commit BPaxos uses the same consensus service as Deadlock BPaxos, with
one small modification. With Majority Commit BPaxos, if a Fast Paxos acceptor
$p_j$ receives a phase 2b message in round 0 and votes for value $(a, \deps{a},
\emptyset, \emptyset)$, then it includes in its vote, the values $(b,
\pruned(\deps{b}), \Pnoop{b}, \Pnotnoop{b})$ for every instance $I_b \in
\deps{a}$ that $p_j$ knows has been chosen.

\paragraph{Majority Commit BPaxos Nodes}
% See [1] for information on line coloring.
%
% [1]: https://tex.stackexchange.com/a/65952
\begin{algorithm}[t!]
  \caption{Majority Commit BPaxos recovery for instance $R.i$ (Case 2 and Case 3)}%
  \algolabel{MajorityCommitBPaxos}
  \begin{algorithmic}[1]
    \If{%
      $\nexists v \in V$ such that at least $\QuorumMajoritySize$ acceptors in
      $\quorum$ voted for $v$
    }
    \State propose anything that satisfies \invref{PrunedDependencies}
    \EndIf{}

    \State
    \algrenewcommand{\alglinenumber}[1]{\color{flatredalt}\footnotesize#1:}
    \State %
      $(a, \deps{a}, \emptyset, \emptyset) \gets$ the value voted by at least
      $\QuorumMajoritySize$ acceptors in $\quorum' \subseteq \quorum$
    \If{%
      $\exists I_b \in \deps{a}$ such that no node $o_j \in \quorum'$ knows that
      $I_b$ is committed
    }
    \State propose anything that satisfies \invref{PrunedDependencies}
    \EndIf

    \If{%
      a thread $t$ recovering $I_b$ is waiting for the recovery of $R.i$ and
      $I_b \notin \deps{a}$
    }\linelabel{DeadlockFreeIf1}
      \State %
        notify $t$ to propose anything that satisfies
        \invref{PrunedDependencies}
    \EndIf

    \If{%
      a thread $t$ recovering $I_b$ is waiting for the recovery of $R.i$
    }\linelabel{DeadlockFreeIf2}
      \State notify $t$ to abort, $I_b$ has been chosen
    \EndIf
    \algrenewcommand{\alglinenumber}[1]{\footnotesize#1:}

    \State
    \State %
      Send $(R.i, a)$ to the quorum $\quorum$ of ordering service nodes
    \State $\deps{a}' \gets$ the union of ordering service responses

    \State
    \State $\Pnoop{a} \gets \emptyset$
    \State $\Pnotnoop{a} \gets \emptyset$
    \For{$I_b \in \deps{a}' - \deps{a}$}
      \If{$I_b$ not committed}
        \algrenewcommand{\alglinenumber}[1]{\color{flatredalt}\footnotesize#1:}
        \State spawn thread $t$ to recover $I_b$
        \State if $t$ notifies us to propose anything, propose anything
        \State if $t$ notifies us to abort, abort
        \State otherwise, wait for $t$ to recover $I_b$
        \algrenewcommand{\alglinenumber}[1]{\footnotesize#1:}
      \EndIf
      \If{$I_b$ committed with $\noop$}
        \State $\Pnoop{a} \gets \Pnoop{a} \cup \set{I_b}$
        \State continue
      \EndIf{}
      \If{$I_b$ committed with $R.i \in \pruned(\deps{b})$}
        \Comment{$b \to a$ unpruned}
        \State $\Pnotnoop{a} \gets \Pnotnoop{a} \cup \set{I_b}$
        \State continue
      \ElsIf{$I_b$ committed with $R.i \notin \Pnotnoop{b}$}
        \Comment{$b \not\to a$}
        \State propose anything that satisfies \invref{PrunedDependencies}
      \Else{}
        \Comment{$b \not\to a$ pruned}
        \State $I_b$ committed with $R.i \in \Pnotnoop{b}$
        \State abort recovery; $R.i$ has already been chosen
      \EndIf{}
    \EndFor{}
    \State propose $(a, \deps{a}, \Pnoop{a}, \Pnotnoop{a})$
  \end{algorithmic}
\end{algorithm}

Majority Commit BPaxos nodes behave identically to Deadlock BPaxos nodes except
that they use \algoref{MajorityCommitBPaxos} instead of
\algoref{DeadlockBPaxos}. \algoref{MajorityCommitBPaxos} is very similar to
\algoref{DeadlockBPaxos}. If you look at \algoref{MajorityCommitBPaxos}, you'll
see that some lines have red line numbers. These lines are new to
\algoref{MajorityCommitBPaxos} and not present in \algoref{DeadlockBPaxos}. All
other lines are unchanged from \algoref{DeadlockBPaxos}. We now explain
\algoref{MajorityCommitBPaxos}.

\algoref{MajorityCommitBPaxos} begins the same as \algoref{DeadlockBPaxos}. If
there is no value $v \in V$ with at least $\QuorumMajoritySize$ votes from
acceptors in $\quorum$, then no value could have been chosen in round $0$, so
$S$ is free to propose anything, so long as it satisfies
\invref{PrunedDependencies}.

Otherwise, there is some value $v = (a, \deps{a}, \emptyset, \emptyset)$ that
received votes from some set $\quorum' \subseteq \quorum$ of acceptors where
$|\quorum'| \geq \QuorumMajoritySize$.
%
If $v$ was chosen in round 0, then there exists some superquorum $\superquorum$
of acceptors that voted for $v$, and for every $I_b \in \deps{a}$, there exists
some quorum $\quorum_b \subseteq \superquorum$ that has recorded $I_b$ as
committed. Because any two quorums intersect, this means that for every quorum
$\quorum$, there exists at least one node $o_j \in \quorum$ such that $o_j$
voted for $v$ in round $0$ and recorded $I_b$ as committed.
%
Thus, by the converse, if there exists some $I_b \in \deps{a}$ such that no
node $o_j \in \quorum'$ knows that $I_b$ is committed, then $S$ know that $v$
was \emph{not} chosen in round $0$, so $S$ are free to propose anything that
satisfies \invref{PrunedDependencies}.

Next, $S$ checks to see if there is a thread $t$ recovering $I_b$ waiting for
the recovery of $R.i$ such that $I_b \notin \deps{a}$. In this case, $S$
concludes that $I_b$ was not chosen in round $0$, and informs $t$ to propose
anything that satisfies \invref{PrunedDependencies}. Here's how $S$ can make
such a conclusion.
%
If $t$ is recovering $I_b$ and waiting for the recovery of $R.i$, then $R.i
\notin \deps{b}$. If $I_b \notin \deps{a}$, then every ordering service node in
$\quorum'$ processed $a$ before $b$. $|\quorum'| \geq \QuorumMajoritySize$, so
there strictly fewer than $f + \QuorumMajoritySize$ remaining ordering service
nodes that could have processed $b$ before $a$. In order for $I_b$ to have been
chosen in round $0$ with $R.i \notin \deps{b}$, it requires that at least $f +
\QuorumMajoritySize$ ordering service nodes processed $b$ before $a$, so $I_b$
was not chosen in round $0$.

Next, if there is still some thread $t$ recovering $I_b$ waiting for the
recovery of $R.i$, then $I_b \in \deps{a}$. If $I_b \in \deps{a}$, then there
is some node $o_j \in \quorum'$ that knows $I_b$ is committed. Thus, $S$
notifies $t$ that $I_b$ has already been committed.

\algoref{MajorityCommitBPaxos} then proceeds the same as
\algoref{DeadlockBPaxos} until the first if inside the for loop. Here, $S$
spawns a thread $t$ to recover $I_b$ and waits for $t$ to either send a
notification or finish, whichever happens first. If $S$ receives a notification
from $t$, it simply does what the notification tells it to do. Otherwise, $t$
successfuly recovered $I_b$, and $S$ continues on as normal.

\paragraph{Correctness}
Majority Commit BPaxos' proof of correctness is pretty much the same as for
Deadlock BPaxos. Majority Commit BPaoxs satisfies \invref{GadgetsChosen} by
using Fast Paxos. It satisfies \invref{ConflictingGadgets} by satisfies
\invref{PrunedDependencies}.

\paragraph{Deadlock Freeness}
If a Majority Commit BPaxos node $S$ is recovering $I_b$ and waiting for the
recovery of $I_a$, then one of the two if-statements on
\lineref{DeadlockFreeIf1} and \lineref{DeadlockFreeIf2} will be executed, and
$S$ will stop waiting. Thus, deadlock is avoided.

\TODO[mwhittaker]{%
  Check if EPaxos is susceptible to the deadlock issues that Deadlock BPaxos
  has. The EPaxos proof of deadlock freedom seems like it might have a bug in
  it.
}

\TODO[mwhittaker]{%
  Explain that Unanimous BPaxos, Deadlock BPaxos, and Majority Commit BPaxos
  can be "fast" in the sense of Fast Paxos. Clients can initiate the protocol.
  They don't have to send to a command leader first. The one wrinkle is that a
  client will have to randomly pick a leader for the instance, say $R$, and
  send the command in some fresh instance owned by $R$. These fresh instances
  have to be default initialized in the any state, but this doesn't affect the
  correctness of anything.
}

\TODO[mwhittaker]{%
  Explain how EPaxos gets quorum sizes one smaller. Explain why EPaxos cannot
  be "fast" and why that affects the protocol in non-trivial ways.
}

\TODO[mwhittaker]{%
  Think about a Caesar variant of BPaxos. BPaxos may not satisfy the two main
  invariants. It seems like they don't run consensus on edges, only on
  commands?
}
