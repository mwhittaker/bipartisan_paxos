\section{Bipartisan Paxos}\seclabel{BipartisanPaxos}
In this section, we introduce \defword{Bipartisan Paxos}, or \defword{BPaxos}
for short. BPaxos is designed to be as simple to understand as possible, even
at the cost of performance. Other variants of BPaxos that we'll see later
(e.g.\ Unanimous BPaxos and Majority Commit BPaxos) improve on BPaxos'
performance.

\paragraph{Overview}
MultiPaxos allows a set of nodes to agree on a totally ordered sequence of
state machine commands. This is illustrated in \figref{MultiPaxosCartoon}.

\begin{figure}[h]
  \centering
  \begin{tikzpicture}
    \node[square] (a) at (0, 0) {\texttt{x=1}};
    \node[square, right=-1pt of a] (b) {\texttt{y=1}};
    \node[square, right=-1pt of b] (c) {\texttt{y+=x}};
    \node[square, right=-1pt of c] (d) {\texttt{y+=1}};
    \node[square, right=-1pt of d] (e) {\texttt{x=2\\y=2}};
    \node[square, right=-1pt of e] (f) {};
    \node[square, right=-1pt of f] (g) {};
    \node[right=0in of g] (h) {\ldots};

    \foreach \label/\i in {a/0, b/1, c/2, d/3, e/4, f/5, g/6} {%
      \node[below=0in of \label] {\i};
    }
  \end{tikzpicture}
  \caption{MultiPaxos}\figlabel{MultiPaxosCartoon}
\end{figure}

While simple, agreeing on a totally ordered sequence of state machine commands
can be overly prescriptive. If two commands conflict (e.g., \texttt{x = 1} and
\texttt{x = 2}), then they \emph{do} need to be executed by every state machine
replica in the same order. But, if two commands do \emph{not} conflict (e.g.,
\texttt{x = 1} and \texttt{y = 1}), then they do \emph{not} need to be totally
ordered.  State machine replicas can execute them in either order.

EPaxos takes advantage of this fact and attempts to order commands only if they
conflict. To do so, it ditches the totally ordered sequence and instead agrees
on a directed graph of commands such that every pair of conflicting commands
have an edge between them. An example is illustrated in \figref{EPaxosCartoon}%
\footnote{%
  In reality, the graph would be the transitive closure of the one in
  \figref{EPaxosCartoon}, but we do not draw all edges to keep things simple
}.
EPaxos then executes this graph in reverse topological order one strongly
connected component at a time, executing commands within a component in an
arbitrary but deterministic order. For example, in \figref{EPaxosCartoon}, we
could execute commands in the order $R.1, R.2, S.1, Q.2, Q.1$.

\begin{figure}[h]
  \centering
  \begin{tikzpicture}[scale=2.5]
    \node[square] (a) at (0, 1) {\texttt{x=1}};
    \node[square] (b) at (0, 0) {\texttt{y=1}};
    \node[square] (c) at (1, 1) {\texttt{y+=x}};
    \node[square] (d) at (1, 0) {\texttt{y+=1}};
    \node[square] (e) at (2, 0.5) {\texttt{x=2\\y=2}};
    \draw[ultra thick, -latex] (c) to (a);
    \draw[ultra thick, -latex] (c) to (b);
    \draw[ultra thick, -latex, bend right] (c) to (e);
    \draw[ultra thick, -latex] (d) to (b);
    \draw[ultra thick, -latex, bend right] (e) to (c);
    \draw[ultra thick, -latex] (e) to (d);

    \foreach \label/\i in {a/$R.1$, b/$R.2$, c/$Q.1$, d/$S.1$, e/$Q.2$} {%
      \node[above=0in of \label] {\i};
    }
  \end{tikzpicture}
  \caption{EPaxos}\figlabel{EPaxosCartoon}
\end{figure}

Every vertex in the graph has a unique name like $Q.1$ or $R.1$. EPaxos calls
these \defword{instances}. We can view a named vertex, the command inside the
vertex, and the vertex's outbound edges as a little gadget. For example,
\figref{EPaxosGadgets} shows gadgets for instances $R.2$, $Q.1$, and $Q.2$.
%
More formally, we can think of these gadgets as tuples like $(Q.1,
\texttt{y+=x}, \set{R.1, R.2, Q.2})$ where $Q.1$ is the instance, \texttt{y+=x}
is the command in the instance, and the set $\set{R.1, R.2, Q.2}$ is the set of
\defword{dependencies} of $Q.1$ (or of \texttt{y+=x} if $Q.1$ is clear from
context).

\begin{figure}[h]
  \centering

  \begin{subfigure}[b]{0.19\textwidth}
    \begin{tikzpicture}[scale=2.5]
      \node[square] (b) at (0, 0) {\texttt{y=1}};
      \foreach \label/\i in {b/$R.2$} {%
        \node[above=0in of \label] {\i};
      }
    \end{tikzpicture}
  \end{subfigure}
  %
  \begin{subfigure}[b]{0.49\textwidth}
    \begin{tikzpicture}[scale=2.5]
      \node[square] (a) at (0, 1) {};
      \node[square] (b) at (0, 0) {};
      \node[square] (c) at (1, 1) {\texttt{y+=x}};
      \node[square] (e) at (2, 0.5) {};
      \draw[ultra thick, -latex] (c) to (a);
      \draw[ultra thick, -latex] (c) to (b);
      \draw[ultra thick, -latex, bend right] (c) to (e);
      \foreach \label/\i in {a/$R.1$, b/$R.2$, c/$Q.1$, e/$Q.2$} {%
        \node[above=0in of \label] {\i};
      }
    \end{tikzpicture}
  \end{subfigure}
  %
  \begin{subfigure}[b]{0.29\textwidth}
    \begin{tikzpicture}[scale=2.5]
      \node[square] (c) at (1, 1) {};
      \node[square] (d) at (1, 0) {};
      \node[square] (e) at (2, 0.5) {\texttt{x=2\\y=2}};
      \draw[ultra thick, -latex, bend right] (e) to (c);
      \draw[ultra thick, -latex] (e) to (d);
      \foreach \label/\i in {c/$Q.1$, d/$S.1$, e/$Q.2$} {%
        \node[above=0in of \label] {\i};
      }
    \end{tikzpicture}
  \end{subfigure}

  \caption{EPaxos Gadgets}\figlabel{EPaxosGadgets}
\end{figure}

EPaxos nodes collectively construct the directed graph of commands by stitching
together a bunch of gadgets. More precisely, an EPaxos node $R$ proposes a
gadget for instance $R.i$ and attempts to get the gadget chosen. Once a gadget
is chosen, it is considered part of the directed graph and is eligible for
execution. EPaxos' correctness hinges on the following two key invariants
(later we'll see why these invariants ensure EPaxos' correctness).

\begin{boxedinvariant}\invlabel{GadgetsChosen}
  Once a gadget $(R.i, a, \deps{a})$ has been chosen, no other gadget can be
  chosen for instance $R.i$.
\end{boxedinvariant}

\begin{boxedinvariant}\invlabel{ConflictingGadgets}
  If two gadgets $(I_a, a, \deps{a})$ and $(I_b, b, \deps{b})$ are chosen and
  commands $a$ and $b$ conflict, then either $I_a \in \deps{b}$ or $I_b \in
  \deps{a}$ or both.
\end{boxedinvariant}

BPaxos, like EPaxos, also constructs a directed graph of commands and executes
them in reverse topological order one strongly connected component at a time.
In fact, BPaxos and EPaxos execute commands in 100\% the same way. BPaxos also
maintains the same two key invariants as EPaxos. BPaxos and EPaxos differ only
in how they construct the directed graph of commands.
%
BPaxos is illustrated in \figref{BPaxos}. A BPaxos instance has three main
components: an ordering service, a consensus service, and a set of BPaxos
nodes. The ordering service helps provide \invref{ConflictingGadgets}, the
consensus service helps provide \invref{GadgetsChosen}, and the BPaxos nodes
glue the two together. We explain each of these three components in turn.

{\begin{figure}[ht]
  \centering

  \begin{tikzpicture}
    \tikzstyle{machine}=[draw, circle, inner sep=2pt]

    % Ordering Service
    \node[machine] (o1) at (0, 2) {$o_1$};
    \node[machine] (o2) at (1, 2) {$o_2$};
    \node[machine] (o3) at (2, 2) {$o_3$};
    \node[machine] (o4) at (3, 2) {$o_4$};
    \node[machine] (o5) at (4, 2) {$o_5$};
    \node (os) at (2, 3) {Ordering Service};
    \draw[rounded corners]
      ($(o1.south west) + (-0.2, -0.2)$) rectangle
      ($(o5.north east) + (0.2, 0.2)$);

    % Consensus
    \node[machine] (p1) at (6, 2) {$p_1$};
    \node[machine] (p2) at (7, 2) {$p_2$};
    \node[machine] (p3) at (8, 2) {$p_3$};
    \node[machine] (p4) at (9, 2) {$p_4$};
    \node[machine] (p5) at (10, 2) {$p_5$};
    \node (os) at (8, 3) {Consensus Service};
    \draw[rounded corners]
      ($(p1.south west) + (-0.2, -0.2)$) rectangle
      ($(p5.north east) + (0.2, 0.2)$);

    % BPaxos Nodes
    \node[machine] (b1) at (3, 0) {$R_1$};
    \node[machine] (b2) at (4, 0) {$R_2$};
    \node[machine] (b3) at (5, 0) {$R_3$};
    \node[machine] (b4) at (6, 0) {$R_4$};
    \node[machine] (b5) at (7, 0) {$R_5$};
    \draw[rounded corners]
      ($(b1.south west) + (-0.2, -0.2)$) rectangle
      ($(b5.north east) + (0.2, 0.2)$);
    \node (bpaxos) at (5, -1) {BPaxos Nodes};
  \end{tikzpicture}

  \caption{BPaxos}\figlabel{BPaxos}
\end{figure}
}

\paragraph{Ordering Service}
The ordering service is responsible for computing the dependencies between
conflicting state machine commands. When a BPaxos node $R$ receives a state
machine command $a$ from a client, it chooses a unique instance $R.i$ for $a$.
Then, $R$ sends $(R.i, a)$ to the ordering service. When the ordering service
receives $(R.i, a)$, it replies with a tuple $(R.i, a, \deps{a})$ where
$\deps{a} = \set{I_1, \ldots, I_n}$ is a set of instances called $a$'s
dependencies. The ordering service maintains the following invariant.

\begin{boxedinvariant}\invlabel{OrderingService}
If two conflicting commands $a$ and $b$ in instances $I_a$ and $I_b$ yield
responses $(I_a, a, \deps{a})$ and $(I_b, b, \deps{b})$ from the ordering
service, then either $I_a \in \deps{b}$ or $I_b \in \deps{a}$ (or both). That
is, if two conflicting commands produce responses from the ordering service, at
least one is guaranteed to be a dependency of the other.
\end{boxedinvariant}

There are a couple things to note about the ordering service:
\begin{itemize}
  \item
    The ordering service has a precondition that at most one command can be
    sent to the ordering service in any given instance. That is, if a BPaxos
    node sends command $(R.i, a)$ to the ordering service, then no other BPaxos
    node can send $(R.i, b)$ to the ordering service for some other command
    $b$.

  \item
    It's possible that a BPaxos node (or multiple BPaxos nodes) sends the tuple
    $(R.i, a)$ to the ordering service more than once. The ordering service
    does not guarantee that all of the responses to these requests are the
    same.  For example, BPaxos node $R$ may send $(R.1, a)$ to the ordering
    service and get a response $(R.1, a, \set{Q.1, R.2})$. Later, $Q$ might
    send $(R.1, a)$ to the ordering service and get a different response of
    $(R.1, a, \set{Q.2, R.1})$. Note that even though the ordering service may
    produce different responses for the same request, the ordering service
    maintains \invref{OrderingService} for every possible pair of ordering
    service responses.
\end{itemize}

\newcommand{\out}{\text{out}}
Implementing a fault tolerant ordering service is straightforward. We employ
$2f + 1$ ordering service nodes $o_{1}, \ldots, o_{2f + 1}$. When a BPaxos node
$R$ sends a command $a$ in instance $R.i$ to the ordering service, it sends the
tuple $(R.i, a)$ to all $2f + 1$ of the ordering service nodes. Every ordering
service node $o_i$ maintains a directed acyclic graph $G_i$. Every vertex of
the graph is labelled with a BPaxos instance and contains a state machine
command. When $o_i$ receives a command $a$ for instance $R.i$ from a Fast
BPaxos node, it performs the following actions:
\begin{enumerate}
  \item
    Let $\out(R.i)$ be the set of outbound edges emanating from the vertex
    labelled $R.i$. If there is already a vertex in $G_i$ labelled $R.i$, then
    $o_i$ returns $(R.i, a, \out(R.i))$ and does nothing else.  Note that the
    ordering service's precondition that most one command can be sent in any
    given instance guarantees that command stored in vertex $R.i$ is $a$.
  \item
    If there does not exist a vertex labelled $R.i$ in $G_i$, then $o_i$
    inserts a vertex into $G_i$ with label $R.i$ and with command $a$. An edge
    is added from instance $R.i$ to instance $Q.j$ for every other instance
    $Q.j$ in the graph that contains a command $b$ that conflicts with $a$.
    $o_i$ then returns the tuple $(R.i, a, \out(R.i))$.
\end{enumerate}

An example of such a graph is given in
\figref{FastPaxosOrderingServiceCartoonBefore}. The same graph is shown
\figref{FastPaxosOrderingServiceCartoonAfter}, except that the command $e$ has
arrived in instance $Q.2$ and conflicts with commands $c$ and $d$.

{\begin{figure}[h]
  \centering

  \begin{subfigure}[b]{0.35\textwidth}
    \begin{tikzpicture}[scale=2.5]
      \node[square] (a) at (0, 1) {$a$};
      \node[square] (b) at (0, 0) {$b$};
      \node[square] (c) at (1, 1) {$c$};
      \node[square] (d) at (1, 0) {$d$};
      \draw[ultra thick, -latex] (c) to (a);
      \draw[ultra thick, -latex] (c) to (b);
      \draw[ultra thick, -latex] (d) to (b);
      \foreach \label/\i in {a/$R.1$, b/$R.2$, c/$Q.1$, d/$S.1$} {%
        \node[above=0in of \label] {\i};
      }
    \end{tikzpicture}
    \caption{}\figlabel{FastPaxosOrderingServiceCartoonBefore}
  \end{subfigure}
  %
  \begin{subfigure}[b]{0.35\textwidth}
    \begin{tikzpicture}[scale=2.5]
      \node[square] (a) at (0, 1) {$a$};
      \node[square] (b) at (0, 0) {$b$};
      \node[square] (c) at (1, 1) {$c$};
      \node[square] (d) at (1, 0) {$d$};
      \node[square] (e) at (2, 0.5) {$e$};
      \draw[ultra thick, -latex] (c) to (a);
      \draw[ultra thick, -latex] (c) to (b);
      \draw[ultra thick, -latex] (d) to (b);
      \draw[ultra thick, -latex] (e) to (c);
      \draw[ultra thick, -latex] (e) to (d);
      \foreach \label/\i in {a/$R.1$, b/$R.2$, c/$Q.1$, d/$S.1$, e/$Q.2$} {%
        \node[above=0in of \label] {\i};
      }
    \end{tikzpicture}
    \caption{}\figlabel{FastPaxosOrderingServiceCartoonAfter}
  \end{subfigure}

  \caption{%
    (left) A directed acyclic graph stored at a Fast BPaxos ordering service
    node. (right) The same graph after the command $e$ arrives in instance
    $Q.2$.
  }\figlabel{FastPaxosOrderingServiceCartoon}
\end{figure}

}

There are a couple of things worth noting about an ordering service node $o_i$
and the corresponding graph $G_i$.
\begin{itemize}
  \item
    $G_i$ is always acyclic.
  \item
    One a gadget $(I_a, a, \out(I_a))$ is inserted into $G_i$, the gadget is
    never modified.
  \item
    The edges in $G_i$ reflect the order in which conflicting commands were
    received by $o_i$. If there is an edge from instance $I_a$ to instance
    $I_b$, then $o_i$ must have received command $b$ in $I_b$ before receiving
    command $a$ in $I_a$.
  \item
    EPaxos and BPaxos both construct a directed cyclic graph of commands that
    are then executed in reverse topological order by strongly component. This
    graph is not the same graph as $G_i$.  $G_i$ is a completely separate graph
    that serves a completely different purpose.
\end{itemize}

When a BPaxos node receives replies $(R.i, a, \deps{a}_{i_1}), \ldots, (R.i, a,
\deps{a}_{i_{f+1}})$ from a quorum $\quorum_a$ of $f + 1$ ordering service
nodes, it takes $(R.i, a, \deps{a}_{i_1} \cup \ldots \cup \deps{a}_{i_{f+1}})$
to be the response from the ordering service. That is, it computes $\deps{a}$
by taking the union of $\deps{a}_i$ from a majority of the ordering service
nodes.

To understand why this ordering service implementation maintains
\invref{OrderingService}, consider conflicting commands $a$ and $b$ in
instances $I_a$ and $I_b$. Assume $a$'s reply $(I_a, a, \deps{a})$ was formed
from a quorum $\quorum_a$ and $b$'s reply $(I_b, b, \deps{b})$ was formed from
a quorum $\quorum_b$. Any two quorums intersect, so $\quorum_a \cap \quorum_b$
is nonempty. Let $o_i$ be an ordering service node in this intersection. $o_i$
either received $a$ or $b$ first. If it received $a$ first, then $G_i$ has a
vertex $I_a$ that contained $a$ when $o_i$ processed command $b$, so $o_i$
includes $I_a$ in $\deps{b}_i$, so $I_a$ is in $\deps{b}$. Symmetrically, if
$o_i$ received $b$ first, then it includes $I_b$ in $\deps{a}$.

\paragraph{Consensus Service}
We assume we have some set $p_1, \ldots, p_n$ of nodes that implement Plain
Jane consensus. A BPaxos node can propose to the consensus service that some
value $v$ be chosen in some instance $I$. The consensus service replies with
the value that has been chosen in instance $I$, which may or may not be $v$
depending on if there are concurrent proposers proposing to instance $i$. The
consensus service guarantees that for every instance $I$, at most one value is
ever chosen in $I$.

We can implement the consensus service with one Paxos instance for every BPaxos
instance, with consensus service nodes playing the role of Paxos acceptors and
the BPaxos nodes playing the role of Paxos proposers. Similar to MultiPaxos, we
can have each BPaxos node $R$ serve as the leader for every Paxos instance
$R.i$. Initially, $R$ runs phase 1 of Paxos for every instance $R.i$, so that
later, $R$ can get a command committed in instance $R.i$ in one round trip to
the consensus service (in the best case).

\paragraph{BPaxos Nodes}
We assume a fixed set $R_1, \ldots, R_{2f+1}$ of $2f + 1$ BPaxos nodes.
%
Clients sends state machine commands to BPaxos nodes to be executed by the
replicated state machine. When a BPaxos node $R$ receives a command $a$, it
sends the command to the ordering service in a previously unused instance $R.i$
and receives a reply $(R.i, a, \deps{a})$. $R$ then proposes the value $(a,
\deps{a})$ to the consensus service in instance $R.i$. The consensus service
then replies with some chosen value $(a', \deps{a'})$ (which is equal to $(a,
\deps{a})$ in the failure-free case). At this point, the gadget $(R.i, a',
\deps{a'})$ is considered chosen in the directed graph of state machine
commands and is eligible for execution. Node $R$ also informs the other BPaxos
nodes that the value $(a', \deps{a'})$ has been chosen.

As noted earlier, the execution of the commands in the directed graph is
identical to that of EPaxos. Committed commands are executed in reverse
topological order, one strongly connected component at a time. Within a
strongly connected component, BPaxos executes commands in an arbitrary but
deterministic order. Unlike with EPaxos, BPaxos does not annotate instances
with sequence numbers. To provide linearizability, when a BPaxos node $R$
receives command $a$ from a client, $R$ does not respond to the client until
the command $a$ has been executed by the replicated state machine. If $R$
returns to the client immediately after $a$ is chosen (as with EPaxos), then
BPaxos provides serializability instead of linearizability.

% Mention that we don't HAVE to no-op.
Note that it's possible that (1) a committed command $a$ depends on an
uncommitted instance $Q.j$, and (2) the BPaxos node $Q$ that manages the
instance $Q.j$ has crashed. If the instance $Q.j$ remains forever uncommitted,
then the command $a$ will never be executed. To avoid this liveness violation,
if any BPaxos node $R$ notices that instance $Q.j$ has been uncommitted for
some time, it can perform one of the following actions:
\begin{enumerate}
  \item
    $R$ can propose to the consensus service that the value $(\noop,
    \emptyset)$ be chosen in instance $Q.j$ where $\noop$ is a command that
    doesn't affect the state machine and doesn't conflict with any other
    command. Note that $R$ does not contact the ordering service here. $R$
    contacts the consensus service directly.

  \item
    $R$ can contact the ordering service and check if any ordering service node
    has recorded a command $b$ in instance $Q.j$. If such a command exists, $R$
    can send the tuple $(Q.j, b)$ to the ordering service, and propose the
    resulting dependencies. If no such command exists, $R$ can propose a
    $\noop$.
\end{enumerate}

\paragraph{Correctness}
BPaxos maintains \invref{GadgetsChosen} trivially because it chooses commands
using consensus. The ordering service maintains \invref{OrderingService}, which
says that the dependencies computed for conflicting commands $a$ and $b$ will
have one in the other (or both). \invref{ConflictingGadgets} says that the
dependencies of conflicting \emph{chosen} commands $a$ and $b$ will have one in
the other (or both). Thus, BPaxos nodes maintain \invref{ConflictingGadgets} by
only proposing dependencies computed by the ordering service. BPaxos nodes also
propose $\noop$'s with empty dependencies, but since $\noop$s do not conflict
with any command, \invref{ConflictingGadgets} is satisfied trivially.

Now, we discuss why \invref{GadgetsChosen} and \invref{ConflictingGadgets}
ensure BPaxos' correctness. Fortunate for us, EPaxos maintains the same two
invariants, so we can leverage EPaxos' proof of correctness. Open up
\cite{moraru2013proof} and head to section 5.6, which contains proofs of
EPaxos' correctness.
\begin{itemize}
  \item
    \textbf{Theorem 1} says that EPaxos satisfies nontriviality. Clearly, so
    does BPaxos.

  \item
    \textbf{Lemma 1} is \invref{GadgetsChosen}. As we saw, BPaxos maintains
    this invariant.

  \item
    \textbf{Theorem 2} follows from Lemma 1.

  \item
    \textbf{Theorem 3} is trivial.

  \item
    \textbf{Theorem 4} states that if two conflicting commands are both
    committed, then they will be executed in the same order by every BPaxos
    node. This follows from \invref{ConflictingGadgets}. If two commands
    conflict, then one will depend on the other or both. If both commands end
    up in the same strongly connected component, they will be executed in the
    same deterministic order. And, if the commands end up in different strongly
    connected components, then one component is guaranteed to be ordered before
    the other, so the two commands are executed in reverse topological order.
\end{itemize}

