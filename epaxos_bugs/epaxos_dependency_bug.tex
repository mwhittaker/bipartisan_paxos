\documentclass{mwhittaker}

\usepackage{pervasives}
\usepackage{subcaption}
\usepackage{tikz}
\usepackage{url}
\usetikzlibrary{calc}
\usetikzlibrary{positioning}
\usetikzlibrary{shapes.misc}

\newcommand{\TLAplus}{TLA+}

\begin{document}
\begin{center}
  \Huge EPaxos Dependency Set Compaction Bug
\end{center}
In this document, we outline a minor bug in an optimization that EPaxos
performs to keep dependency sets small~\cite{moraru2013there, moraru2013proof}.

\section{EPaxos Dependency Set Compaction}
When an EPaxos replica $R$ receives a command $x$ from a client, it first
assigns the command to an instance $R.i$. It then computes the command's
dependencies: the set of instances $Q.j$ that contain a command that conflict
with $x$. $R$ forwards these dependencies to other replicas, which also compute
the command's dependencies.

Over time, as more and more commands are executed, these sets of dependencies
grow larger and larger. Every command starts to depend on more and more
commands. The EPaxos paper proposes the following mechanism to keep these
dependency lists small (see Section 4.5):

\begin{quote}
  Instead of including all interfering instances, we include only $N$
  dependencies in each list: the instance number $R.i$ with the highest $i$ for
  which the current replica has seen an interfering command (not necessarily
  committed). If interference is transitive (usually the case in practice) the
  most recent interfering command suffices, because its dependency graph will
  contain all interfering instances $R.j$, with $j < i$.
\end{quote}

There are some details omitted from this description. When a replica receives a
pre-accept request from a leader, how does it merge its locally computed
dependencies with the dependencies from the leader? And when a leader takes the
slow path, how does it merge dependencies? The TLA+ specification does not
implement dependency set compaction, so we can't look there for clarification.

The Go implementation seems to perform all merging using an element wise
maximum. More specifically, we represent a dependency set for an EPaxos
deployment with $n$ replicas as a list $d = i_1, \ldots, i_n$ where entry $i_m$
is the highest $i$ for which replica $m$ has seen an interfering command. To
merge two dependency sets $i_1, \ldots, i_n$ and $j_1, \ldots, j_n$, we compute
an element wise maximum: $\max(i_1, j_1), \ldots, \max(i_n, j_n)$.  Throughout
the rest of the paper, we assume that EPaxos uses this mechanism to merge
dependency sets.

\section{The Bug}
\newcommand{\preaccepted}{\textbf{pre}}
\newcommand{\accepted}{\textbf{acc}}
\newcommand{\committed}{\textbf{comm}}
\newcommand{\noop}{\bot}

\tikzstyle{instance}=[draw]
\tikzstyle{dep}=[-latex, ultra thick]

We now present an execution of EPaxos in which two replicas execute conflicting
commands in different orders. The gist of the bug is that we create a sequence
of commands $x$, $y$, $z$ that all conflict with one another. We get $y$ to
depend on $x$ and then $z$ to depend on $y$ (and not $x$). We get $x$ and $z$
chosen but then recover $y$'s instance, changing it to a noop. $z$ had and
transitive dependence on $x$, but this transitive dependence is erased when we
recover $y$'s instance with a noop. This allows replicas to execute $x$ and $z$
in either order. The bug is summarized in \figref{bug_cartoon}.

\begin{figure}[h]
  \centering

  \tikzstyle{vert}=[draw, thick, minimum height=0.75cm, minimum width=0.75cm]
  \tikzstyle{dep}=[-latex, thick]

  \begin{subfigure}[t]{0.3\textwidth}
    \centering
    \begin{tikzpicture}[xscale=1.5]
      \node[vert, dashed] (x) at (0, 0) {$x$};
      \node[vert, dashed] (y) at (1, 0) {$y$};
      \node[vert, dashed] (z) at (2, 0) {$z$};
      \draw[dep, dashed] (y) to (x);
      \draw[dep, dashed] (z) to (y);
    \end{tikzpicture}
    \caption{%
      Initially, commands $x$, $y$, and $z$ are all uncommitted. $y$ depends on
      $x$ and $z$ depends on $y$ but not $x$. $z$ conflicts with $x$, so
      without any optimizations, it would have an edge to $x$. However, EPaxos'
      dependency compaction optimization allows to omit the edge from $z$ to
      $x$ because of the transitive dependence through $y$.
    }
    \figlabel{bug_cartoon_a}
  \end{subfigure}%
  \hspace{0.5cm}%
  \begin{subfigure}[t]{0.3\textwidth}
    \centering
    \begin{tikzpicture}[xscale=1.5]
      \node[vert] (x) at (0, 0) {$x$};
      \node[vert, dashed] (y) at (1, 0) {$y$};
      \node[vert] (z) at (2, 0) {$z$};
      \draw[dep, dashed] (y) to (x);
      \draw[dep] (z) to (y);
    \end{tikzpicture}
    \caption{%
      Next, $x$ and $z$ are both committed, but $y$ remains uncommitted for
      some time.
    }
    \figlabel{bug_cartoon_b}
  \end{subfigure}%
  \hspace{0.5cm}%
  \begin{subfigure}[t]{0.3\textwidth}
    \centering
    \begin{tikzpicture}[xscale=1.5]
      \node[vert] (x) at (0, 0) {$x$};
      \node[vert] (y) at (1, 0) {$\bot$};
      \node[vert] (z) at (2, 0) {$z$};
      \draw[dep] (z) to (y);
    \end{tikzpicture}
    \caption{%
      Eventually, a replica recovers $y$'s instance and gets a noop, $\bot$,
      chosen. $\bot$ doesn't conflict with any other commands so is committed
      without any dependencies. $x$ and $z$ conflict but do not have any path
      between them. This allows some replicas to execute $x$ before $z$ and
      others to execute $z$ before $x$.
    }
    \figlabel{bug_cartoon_c}
  \end{subfigure}

  \caption{An cartoon overview of the bug.}\figlabel{bug_cartoon}
\end{figure}

Now, we outline the execution in detail. \preaccepted{} is short for
pre-accepted, \accepted{} is short for accepted, and \committed{} is short for
committed. We do not show sequence numbers or ballot numbers because they
aren't relevant to the bug. Commands $x$, $y$, and $z$ are all assumed to
conflict. We use an EPaxos deployment with 5 replicas: $A$, $B$, $C$, $D$, and
$E$. At each step of the execution, we draw the command log at each replica.

Initially, replica $E$ receives command $x$. $E$ chooses $x$ on the fast path
and sends out commit messages to the other replicas. These commit messages are
delayed.
\begin{center}
  \begin{tikzpicture}
    \begin{scope}
      \node (A) at (0, 0) {$A$};
      \node[instance, label={180:$E.0$}] (x) at (0, 1) {$x$; \preaccepted};
    \end{scope}
    \begin{scope}[shift={(3,0)}]
      \node (B) at (0, 0) {$B$};
      \node[instance, label={180:$E.0$}] (x) at (0, 1) {$x$; \preaccepted};
    \end{scope}
    \begin{scope}[shift={(6,0)}]
      \node (C) at (0, 0) {$C$};
      \node[instance, label={180:$E.0$}] (x) at (0, 1) {$x$; \preaccepted};
    \end{scope}
    \begin{scope}[shift={(9,0)}]
      \node (D) at (0, 0) {$D$};
      \node[instance, label={180:$E.0$}] (x) at (0, 1) {$x$; \preaccepted};
    \end{scope}
    \begin{scope}[shift={(12,0)}]
      \node (E) at (0, 0) {$E$};
      \node[instance, label={180:$E.0$}] (x) at (0, 1) {$x$; \committed};
    \end{scope}
  \end{tikzpicture}
\end{center}

Then, replica $E$ receives command $y$ and pre-accepts it locally. It sends out
pre-accept messages to the other replicas, but they are lost.
\begin{center}
  \begin{tikzpicture}
    \begin{scope}
      \node (A) at (0, 0) {$A$};
      \node[instance, label={180:$E.0$}] (x) at (0, 1) {$x$; \preaccepted};
    \end{scope}
    \begin{scope}[shift={(3,0)}]
      \node (B) at (0, 0) {$B$};
      \node[instance, label={180:$E.0$}] (x) at (0, 1) {$x$; \preaccepted};
    \end{scope}
    \begin{scope}[shift={(6,0)}]
      \node (C) at (0, 0) {$C$};
      \node[instance, label={180:$E.0$}] (x) at (0, 1) {$x$; \preaccepted};
    \end{scope}
    \begin{scope}[shift={(9,0)}]
      \node (D) at (0, 0) {$D$};
      \node[instance, label={180:$E.0$}] (x) at (0, 1) {$x$; \preaccepted};
    \end{scope}
    \begin{scope}[shift={(12,0)}]
      \node (E) at (0, 0) {$E$};
      \node[instance, label={180:$E.0$}] (x) at (0, 1) {$x$; \committed};
      \node[instance, label={180:$E.1$}] (y) at (0, 2) {$y$; \preaccepted};
      \draw[dep] (y) to (x);
    \end{scope}
  \end{tikzpicture}
\end{center}

Then, replica $A$ receives command $z$ and pre-accepts it locally. It sends out
pre-accept messages to the other replicas. Replicas $D$ and $E$ receive the
pre-accept messages and process them. Note that replica $E$ computes $z$'s
dependencies to be $\set{E.1}$. $E$ does not include $E.0$ in the dependencies
of $z$.
\begin{center}
  \begin{tikzpicture}
    \begin{scope}
      \node (A) at (0, 0) {$A$};
      \node[instance, label={180:$E.0$}] (x) at (0, 1) {$x$; \preaccepted};
      \node[instance, label={180:$A.0$}] (z) at (0, 3) {$z$; \preaccepted};
      \draw[dep] (z) to (x);
    \end{scope}
    \begin{scope}[shift={(3,0)}]
      \node (B) at (0, 0) {$B$};
      \node[instance, label={180:$E.0$}] (x) at (0, 1) {$x$; \preaccepted};
    \end{scope}
    \begin{scope}[shift={(6,0)}]
      \node (C) at (0, 0) {$C$};
      \node[instance, label={180:$E.0$}] (x) at (0, 1) {$x$; \preaccepted};
    \end{scope}
    \begin{scope}[shift={(9,0)}]
      \node (D) at (0, 0) {$D$};
      \node[instance, label={180:$E.0$}] (x) at (0, 1) {$x$; \preaccepted};
      \node[instance, label={180:$A.0$}] (z) at (0, 3) {$z$; \preaccepted};
      \draw[dep] (z) to (x);
    \end{scope}
    \begin{scope}[shift={(12,0)}]
      \node (E) at (0, 0) {$E$};
      \node[instance, label={180:$E.0$}] (x) at (0, 1) {$x$; \committed};
      \node[instance, label={180:$E.1$}] (y) at (0, 2) {$y$; \preaccepted};
      \node[instance, label={180:$A.0$}] (z) at (0, 3) {$z$; \preaccepted};
      \draw[dep] (y) to (x);
      \draw[dep] (z) to (y);
    \end{scope}
  \end{tikzpicture}
\end{center}

Replica $A$ receives pre-accept replies from replica $D$ and $E$. After a
timeout, it takes the slow path. It computes $z$'s dependencies to be
$\set{E.1}$. Again note that $z$ does not include $E.0$ in its dependencies.
$A$ accepts $z$ locally and sends out accept messages to all other replicas.
All other replicas receive the accept message and send back a reply.
\begin{center}
  \begin{tikzpicture}
    \begin{scope}
      \node (A) at (0, 0) {$A$};
      \node[instance, label={180:$E.0$}] (x) at (0, 1) {$x$; \preaccepted};
      \node[instance, label={180:$E.1$}] (y) at (0, 2) {\phantom{y}};
      \node[instance, label={180:$A.0$}] (z) at (0, 3) {$z$; \accepted};
      \draw[dep] (z) to (y);
    \end{scope}
    \begin{scope}[shift={(3,0)}]
      \node (B) at (0, 0) {$B$};
      \node[instance, label={180:$E.0$}] (x) at (0, 1) {$x$; \preaccepted};
      \node[instance, label={180:$E.1$}] (y) at (0, 2) {\phantom{y}};
      \node[instance, label={180:$A.0$}] (z) at (0, 3) {$z$; \accepted};
      \draw[dep] (z) to (y);
    \end{scope}
    \begin{scope}[shift={(6,0)}]
      \node (C) at (0, 0) {$C$};
      \node[instance, label={180:$E.0$}] (x) at (0, 1) {$x$; \preaccepted};
      \node[instance, label={180:$E.1$}] (y) at (0, 2) {\phantom{y}};
      \node[instance, label={180:$A.0$}] (z) at (0, 3) {$z$; \accepted};
      \draw[dep] (z) to (y);
    \end{scope}
    \begin{scope}[shift={(9,0)}]
      \node (D) at (0, 0) {$D$};
      \node[instance, label={180:$E.0$}] (x) at (0, 1) {$x$; \preaccepted};
      \node[instance, label={180:$E.1$}] (y) at (0, 2) {\phantom{y}};
      \node[instance, label={180:$A.0$}] (z) at (0, 3) {$z$; \accepted};
      \draw[dep] (z) to (y);
    \end{scope}
    \begin{scope}[shift={(12,0)}]
      \node (E) at (0, 0) {$E$};
      \node[instance, label={180:$E.0$}] (x) at (0, 1) {$x$; \committed};
      \node[instance, label={180:$E.1$}] (y) at (0, 2) {$y$; \preaccepted};
      \node[instance, label={180:$A.0$}] (z) at (0, 3) {$z$; \accepted};
      \draw[dep] (y) to (x);
      \draw[dep] (z) to (y);
    \end{scope}
  \end{tikzpicture}
\end{center}

Replica $A$ receives accept replies from all other replicas and commits $z$
locally. It sends commit messages to all other replicas, but the messages are
delayed.
\begin{center}
  \begin{tikzpicture}
    \begin{scope}
      \node (A) at (0, 0) {$A$};
      \node[instance, label={180:$E.0$}] (x) at (0, 1) {$x$; \preaccepted};
      \node[instance, label={180:$E.1$}] (y) at (0, 2) {\phantom{y}};
      \node[instance, label={180:$A.0$}] (z) at (0, 3) {$z$; \committed};
      \draw[dep] (z) to (y);
    \end{scope}
    \begin{scope}[shift={(3,0)}]
      \node (B) at (0, 0) {$B$};
      \node[instance, label={180:$E.0$}] (x) at (0, 1) {$x$; \preaccepted};
      \node[instance, label={180:$E.1$}] (y) at (0, 2) {\phantom{y}};
      \node[instance, label={180:$A.0$}] (z) at (0, 3) {$z$; \accepted};
      \draw[dep] (z) to (y);
    \end{scope}
    \begin{scope}[shift={(6,0)}]
      \node (C) at (0, 0) {$C$};
      \node[instance, label={180:$E.0$}] (x) at (0, 1) {$x$; \preaccepted};
      \node[instance, label={180:$E.1$}] (y) at (0, 2) {\phantom{y}};
      \node[instance, label={180:$A.0$}] (z) at (0, 3) {$z$; \accepted};
      \draw[dep] (z) to (y);
    \end{scope}
    \begin{scope}[shift={(9,0)}]
      \node (D) at (0, 0) {$D$};
      \node[instance, label={180:$E.0$}] (x) at (0, 1) {$x$; \preaccepted};
      \node[instance, label={180:$E.1$}] (y) at (0, 2) {\phantom{y}};
      \node[instance, label={180:$A.0$}] (z) at (0, 3) {$z$; \accepted};
      \draw[dep] (z) to (y);
    \end{scope}
    \begin{scope}[shift={(12,0)}]
      \node (E) at (0, 0) {$E$};
      \node[instance, label={180:$E.0$}] (x) at (0, 1) {$x$; \committed};
      \node[instance, label={180:$E.1$}] (y) at (0, 2) {$y$; \preaccepted};
      \node[instance, label={180:$A.0$}] (z) at (0, 3) {$z$; \accepted};
      \draw[dep] (y) to (x);
      \draw[dep] (z) to (y);
    \end{scope}
  \end{tikzpicture}
\end{center}

After a timeout, replica $A$ notices that $E.1$ has not been chosen. This
prevents $z$ from being executed, so replica $A$ begins to recover instance
$E.1$. It sends prepare requests to replicas $A$ (itself), $B$, and $C$. All
three replicas reply without having seen $E.1$, so $A$ begins the slow path to
get a noop chosen (see line 37 of Figure 3 in \cite{moraru2013there}). We
denote a noop with $\noop{}$. Talking exclusively to replicas $A$, $B$, and
$C$, replica $A$ eventually gets the $\noop{}$ chosen and broadcasts a commit
message to all replicas. All replicas receive the commit message. Note that a
$\noop$ has no dependencies.
\begin{center}
  \begin{tikzpicture}
    \begin{scope}
      \node (A) at (0, 0) {$A$};
      \node[instance, label={180:$E.0$}] (x) at (0, 1) {$x$; \preaccepted};
      \node[instance, label={180:$E.1$}] (y) at (0, 2) {$\noop$; \committed};
      \node[instance, label={180:$A.0$}] (z) at (0, 3) {$z$; \committed};
      \draw[dep] (z) to (y);
    \end{scope}
    \begin{scope}[shift={(3,0)}]
      \node (B) at (0, 0) {$B$};
      \node[instance, label={180:$E.0$}] (x) at (0, 1) {$x$; \preaccepted};
      \node[instance, label={180:$E.1$}] (y) at (0, 2) {$\noop$; \committed};
      \node[instance, label={180:$A.0$}] (z) at (0, 3) {$z$; \accepted};
      \draw[dep] (z) to (y);
    \end{scope}
    \begin{scope}[shift={(6,0)}]
      \node (C) at (0, 0) {$C$};
      \node[instance, label={180:$E.0$}] (x) at (0, 1) {$x$; \preaccepted};
      \node[instance, label={180:$E.1$}] (y) at (0, 2) {$\noop$; \committed};
      \node[instance, label={180:$A.0$}] (z) at (0, 3) {$z$; \accepted};
      \draw[dep] (z) to (y);
    \end{scope}
    \begin{scope}[shift={(9,0)}]
      \node (D) at (0, 0) {$D$};
      \node[instance, label={180:$E.0$}] (x) at (0, 1) {$x$; \preaccepted};
      \node[instance, label={180:$E.1$}] (y) at (0, 2) {$\noop$; \committed};
      \node[instance, label={180:$A.0$}] (z) at (0, 3) {$z$; \accepted};
      \draw[dep] (z) to (y);
    \end{scope}
    \begin{scope}[shift={(12,0)}]
      \node (E) at (0, 0) {$E$};
      \node[instance, label={180:$E.0$}] (x) at (0, 1) {$x$; \committed};
      \node[instance, label={180:$E.1$}] (y) at (0, 2) {$\noop$; \committed};
      \node[instance, label={180:$A.0$}] (z) at (0, 3) {$z$; \accepted};
      \draw[dep] (z) to (y);
    \end{scope}
  \end{tikzpicture}
\end{center}

Replica $B$ receives the commit message for $E.0$ from $E$ and executes $x$. It
then receives the commit message for $A.0$ from $A$ and executes $z$. Replica
$C$ receives the messages in the opposite order and executes $z$ and then $x$.
The two replicas execute conflicting commands in different orders, which is a
bug.

\bibliographystyle{plain}
\bibliography{references}
\end{document}
