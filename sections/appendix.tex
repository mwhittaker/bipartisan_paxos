\section{An Aside on Fast Paxos}\seclabel{FastPaxos}
In this section, we briefly summarize the bits of Fast
Paxos~\cite{lamport2006fast} that are relevant to BPaxos.
%
Consider a Fast Paxos deployment with $2f + 1$ acceptors, classic quorums of
size $f + 1$, and superquorums of size $f + \QuorumMajoritySize$. Recall that
after a Fast Paxos proposer receives phase 1b messages from a quorum $\quorum$
of acceptors, it performs the logic shown in \algoref{FastPaxos} to select a
value to propose in 2a.

\begin{algorithm}[ht]
  \caption{Fast Paxos Phase 2a}%
  \algolabel{FastPaxos}
  \begin{algorithmic}[1]
    \State{} $M \gets$ phase 1b messages from a quorum $\quorum$ of acceptors
    \State{} $k \gets$ the largest vote round in $M$
    \State{} $V \gets$ the vote values in $M$ for round $k$
    \If{$k = -1$}
      \State{} propose anything
    \ElsIf{$V = \set{v}$}
      \Comment{(Case 1)}
      \State{} propose $v$
    \ElsIf{$V$ contains a value $v$ with $\QuorumMajoritySize$ or more votes}
      \Comment{(Case 2)}
      \State{} propose $v$
    \Else{}
      \Comment{(Case 3)}
      \State{} propose anything
    \EndIf{}
  \end{algorithmic}
\end{algorithm}

To prove the correctness of Fast Paxos, it suffices to prove the statement
$P(i)$ that says that if an acceptor votes for a value $v$ in round $i$, then
no other value besides $v$ has been or will be chosen in any round $j$ less
than $i$. We prove this claim by induction. $P(0)$ is trivial because there are
no rounds less than $0$. For the general case, we perform a case analysis on
$j$. First, assume $k \neq -1$.
\begin{itemize}
  \item
    If $k < j < i$, then no value has been or will be chosen in round $j$
    because a phase 1 quorum $\quorum$ of acceptors had not voted in any round
    larger than $k$ and promised not to vote in any round less than $i$.

  \item
    If $k = j$, then we perform a case analysis on the proposer's logic.
    \begin{itemize}
      \item
        (Case 1) If $V = \set{v}$, then a quorum of acceptors have either voted
        for $v$ in round $k$ or promised not to vote in round $k$. Thus, no
        other value besides $v$ can be chosen in round $k$.
      \item
        (Case 2) If $k$ is a classic round, then $V = \set{v}$. Thus, in Case
        2, we know that $k$ is a fast round. In order for a value $v$ to be
        chosen in fast round $k$, it needs to receive phase 2b votes from $f +
        \QuorumMajoritySize$ acceptors. This is only possible if it has at
        least $\QuorumMajoritySize$ phase 2b votes from acceptors in $\quorum$.
        Two different values cannot both have $\QuorumMajoritySize$ phase 2b
        votes from acceptors in $\quorum$, so there is at most one such value
        $v$, and this is the only value that could have been chosen in round
        $k$.
      \item
        (Case 3) No value could have been chosen in round $k$.
    \end{itemize}

  \item
    If $j < k$, we again perform a case analysis on the proposer's logic.
    \begin{itemize}
      \item
        (Case 1) By $P(k)$, no value other than $v$ has been or will be chosen
        in round $j$.
      \item
        (Case 2) Let $v_1, v_2 \in V$. By $P(v_1), P(v_2)$, no value has been
        or will be chosen in round $j$.
      \item
        (Case 3) Same as Case 2.
    \end{itemize}
\end{itemize}
If $k = -1$, then we know that no value has been or will be chosen in any round
less than $i$ by the same line of reasoning as above, so we're free to propose
anything.

\section{Fast Paxos Tweak}\seclabel{FastPaxosTweak}
In this section, we make a small tweak to Fast Paxos that makes it more
suitable for BPaxos. Our tweaked Fast Paxos is identical to Fast Paxos, except
for the following two differences. First, we require that round $0$ is a fast
round and all other rounds are classic rounds. Second, our tweaked Fast Paxos
proposers perform a slightly different, slightly more generic protocol for
deciding what value to propose in phase 2a. The protocol is shown in
\algoref{FastPaxosTweak}. Case 2 of the algorithm is left intentionally vague.
A proposer has the freedom to use any means necessary to determine whether a
value was chosen in round $k$ of Case 2.

\begin{algorithm}[ht]
  \caption{Fast Paxos Phase 2a Tweak}%
  \algolabel{FastPaxosTweak}
  \begin{algorithmic}[1]
    \Require{} $0$ is a fast round, and all other rounds are classic rounds
    \State{} $M \gets$ phase 1b messages from a quorum $\quorum$ of acceptors
    \State{} $k \gets$ the largest vote round in $M$
    \State{} $V \gets$ the vote values in $M$ for round $k$

    \If{$k = -1$}
      \State{} propose anything
    \EndIf{}

    \If{$k \neq 0$}
      \Comment{(Case 1)}
      \State{} propose unique $v \in V$
    \EndIf{}

    \If{$V$ contains a value $v$ that may have been chosen in round $k$}
      \Comment{(Case 2)}
      \State{} propose $v$
    \Else{}
      \Comment{(Case 3)}
      \State{} propose anything
    \EndIf{}
  \end{algorithmic}
\end{algorithm}

The proof of this tweak's correctness is more or less the same as the proof of
Fast Paxos' correctness. We again prove $P(i)$ by induction. $P(0)$ is still
trivial. For the general case, we again perform a case analysis on $j$. First,
assume $k \neq -1$.
\begin{itemize}
  \item
    If $k < j < i$, then no value has been or will be chosen in round $j$
    because a phase 1 quorum $\quorum$ of acceptors had not voted in any round
    larger than $k$ and promised not to vote in any round less than $i$.

  \item
    If $k = j$, then we perform a case analysis on the proposer's logic.
    \begin{itemize}
      \item
        (Case 1) If $k \neq 0$, then $k$ is a classic round. At most one value
        can be proposed in a classic round, so $V = \set{v}$ is a singleton
        set, and no value other than $v$ can be or will be proposed in round
        $k$.
      \item
        (Case 2) If $k = 0$, then $k$ is fast round. A necessary condition for
        a value $v$ to be chosen in this fast round is that at least
        $\QuorumMajoritySize$ acceptors in $\quorum$ must have voted for $v$.
        There can be at most one such value $v$, so no value other than this
        $v$ can be chosen in round $k$.
      \item
        (Case 3) No value could have been chosen in round $k$.
    \end{itemize}

  \item
    If $j < k$, we again perform a case analysis on the proposer's logic.
    \begin{itemize}
      \item
        (Case 1) By $P(k)$, no value other than $v$ has been or will be chosen
        in round $j$.
      \item
        (Case 2) $k = 0$, and there are no rounds less than $k$, so this case
        holds trivially.
      \item
        (Case 3) Same as Case 2.
    \end{itemize}
\end{itemize}
Again, if $k = -1$, then we know that no value has been or will be chosen in
any round less than $i$ by the same line of reasoning as above, so we're free
to propose anything.
