\section{An Aside on Fast Paxos}\seclabel{FastPaxos}
Before we move on, we summarize the relevant bits of Fast
Paxos~\cite{lamport2006fast} that are critical to our remaining BPaxos
variants. In Fast Paxos, after a proposer receives phase 1b messages from a
quorum of acceptors, it performs the following logic to select a value to
propose:
\begin{itemize}
  \item
    Let $k$ be the largest vote round in the quorum of phase 1b messages.
    Let $V$ be the corresponding set of vote values for round $k$.
  \item
    (Case 1) If $V = \set{v}$, then propose $v$.
  \item
    (Case 2) If $V$ contains a value $v$ that may have been chosen in round
    $k$, propose $v$. Note that quorum sizes are selected in such a way that at
    most one such $v$ can exist.
  \item
    (Case 3) Otherwise, propose anything.
\end{itemize}

Proving the correctness of Fast Paxos involves proving the statement $P(i)$
that says that if an acceptor votes for a value $v$ in round $i$, then no
other value besides $v$ has been or will be chosen in any round $j$ less than
$i$. We prove this claim by induction. $P(0)$ is trivial because there are no
rounds less than $0$. For the general case, we perform a case analysis on $j$.
First, assume $k \neq -1$.
\begin{itemize}
  \item
    If $k < j < i$, then no value has been or will be chosen in round $j$
    because a phase 1 quorum of acceptors had not voted in any round larger
    than $k$ and promised not to vote in any round less than $i$.

  \item
    If $k = j$, then we perform a case analysis on the proposer's logic.
    \begin{itemize}
      \item
        (Case 1) If $V = \set{v}$, then a quorum of acceptors have either voted
        for $v$ in round $k$ or promised not to vote in round $k$. Thus, no
        other value besides $v$ can be chosen in round $k$.
      \item
        (Case 2) If $V$ contains a value that may have been chosen in round
        $k$, we propose it. Quorum sizes are set up such that no other value
        could have been chosen in round $k$.
      \item
        (Case 3) No value could have been chosen in round $k$.
    \end{itemize}

  \item
    If $j < k$, we again perform a case analysis on the proposer's logic.
    \begin{itemize}
      \item
        (Case 1) By $P(k)$, no value other than $v$ has been or will be chosen
        in round $j$.
      \item
        (Case 2) Let $v_1, v_2 \in V$. By $P(v_1), P(v_2)$, no value has been
        or will be chosen in round $j$.
      \item
        (Case 3) Same as Case 2.
    \end{itemize}
\end{itemize}
If $k = -1$, then we know that no value has been or will be chosen in any round
less than $i$ by the same line of reasoning as above, so we're free to propose
anything.

Note that we can make the following small tweak to Fast Paxos without
compromising its correctness. We can change Case 1 of the proposer's logic from
``If $V = \set{v}$, then propose $v$'' to ``If $V = \set{v}$ and $k \neq 0$,
then propose $v$''. That is, a proposer will only perform Case 1 if $k \neq 0$.

\section{Reducing $\noop$s}
TODO: Mention how we can avoid sending too many noops by having BPaxos nodes
send a union of votes instead of noops.

TODO: Explain that Unanimous BPaxos and Fast BPaxos can be "fast" in the sense
of fast paxos. Clients can initiate the protocol. They don't have to send to a
command leader first. The one wrinkle is that a client will have to randomly
pick a leader for the instance, say $R$, and send the command in some fresh
instance owned by $R$. These fresh instances have to be default initialized in
the any state, but this doesn't affect the correctness of anything.
