\section{Unanimous Bipartisan Paxos}\seclabel{UnanimousBPaxos}
\paragraph{Overview}
Take the incorrect BPaxos variant from the previous section and increase the
Fast Paxos superquorum sizes from $f + \floor{\frac{f+1}{2}} + 1$ to $2f + 1$.
That is, choosing a value in a fast round requires a unanimous vote. We call
this variant \defword{Unanimous BPaxos}. This variant, like the incorrect
variant from the previous section, can commit a command in one round trip (in
the best case), but unlike the previous variant, this variant is correct.

\paragraph{Correctness}
Unanimous BPaxos maintains \invref{GadgetsChosen} trivially. To prove Unanimous
BPaxos maintains \invref{ConflictingGadgets}, we prove the claim $P(i)$ that
says if a Fast Paxos acceptor votes for value $(a, \deps{a})$ in round $i$,
then either
\begin{itemize}
  \item
    $i = 0$, or
  \item
    $\deps{a}$ is a response from the ordering service, or
  \item
    $(a, \deps{a}) = (\noop, \emptyset)$.
\end{itemize}

Note that if a value $(a, \deps{a})$ is chosen in round $0$, then every
ordering service node voted for the same dependencies $\deps{a}$. Thus,
$\deps{a}$ is a response from the ordering service. Thus, $P(i)$ implies that
if any value $(a, \deps{a})$ is chosen, then either $\deps{a}$ is a response
from the ordering service or $(a, \deps{a}) = (\noop, \emptyset)$.

We prove $P(i)$ by induction. $P(0)$ is trivial; the first disjunct is
satisfied. For the general case, we perform a case analysis on the proposer's
logic. Call this proposer $Q$.
\begin{itemize}
  \item (Case 1)
    $V = \set{(a, \deps{a})}$ and $k \neq 0$. $P(i)$ holds directly from
    $P(k)$.

  \item (Case 2)
    If $k \neq 0$, then $P(i)$ holds directly from $P(k)$. Otherwise, $k = 0$.
    It's only possible that value $(a, \deps{a})$ was chosen in round $0$ if
    some proposer received phase 1b messages from every acceptor such that the
    acceptors unanimously voted for $(a, \deps{a})$.

    In this case, the quorum of phase 1b messages that $Q$ received is also a
    unanimous vote for $(a, \deps{a})$.  Thus, $\deps{a}$ is the union of
    responses from a majority of ordering service nodes (the majority that $Q$
    contacted), so the second disjunct of $P(i)$ holds.

  \item (Case 3)
    In this case, a proposer either chooses to propose a $\noop$ or propose a
    response from the ordering service, so $P(i)$ holds trivially.
\end{itemize}

It follows that Unanimous BPaxos maintains \invref{ConflictingGadgets} and is
therefore correct. But, let's not miss the forest through the proof. Let's take
a step back and think about why increasing the superquorum size from $f +
\floor{\frac{f+1}{2}} + 1$ to $2f+1$ turns our incorrect BPaxos variant into a
correct one. Referring to \figref{BPaxosLogic}, we see that the top right
corner is now impossible. If a Unanimous BPaxos node concludes that a value
$(a, \deps{a})$ was maybe previously chosen, then it knows for sure that
$\deps{a}$ is a response from the ordering service.
