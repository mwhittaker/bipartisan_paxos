\section{An Incorrect BPaxos Variant}\seclabel{IncorrectBPaxos}
BPaxos is easy to understand, but it's inefficient. After a client sends a
state machine command to a BPaxos node, it takes two round trips (in the best
case) before the command is committed: one round trip to the ordering service
nodes and one round trip to the consensus service. In this section, we present
a simple BPaxos variant that can commit commands in one round trip (in the best
case). Unfortunately, this variant is incorrect. Still, understanding why the
variant is incorrect and how to fix it is helpful to understand the correct
BPaxos variants described below.

\paragraph{The Protocol}
First, we colocate the ordering service nodes and Paxos acceptors. This is
illustrated in \figref{ColocatedBPaxos}.
%
Second, we implement our consensus service using the Fast Paxos tweak outlined
in \algoref{FastPaxosTweak} instead of regular Paxos. As required by
\algoref{FastPaxosTweak}, we let round $0$ be a fast round and every other
round be a classic round. As before, every BPaxos node $R$ runs phase 1 of Fast
Paxos for every instance $R.i$. In the normal case, $R$ finishes executing
phase 1 and sends phase 2a ``any'' messages to the Fast Paxos acceptors. At
this point, the acceptors are free to vote for the first proposal that they
hear from anyone (not just from $R$).

{\begin{figure}[ht]
  \centering

  \begin{tikzpicture}
    \tikzstyle{machine}=[draw, circle, inner sep=2pt]

    % Ordering Service / Paxos Acceptors
    \node[machine] (o1) at (0, 2) {$o_1, p_1$};
    \node[machine, right=0.1cm of o1] (o2) {$o_2, p_2$};
    \node[machine, right=0.1cm of o2] (o3) {$o_3, p_3$};
    \node[machine, right=0.1cm of o3] (o4) {$o_4, p_4$};
    \node[machine, right=0.1cm of o4] (o5) {$o_5, p_5$};
    \node (os) at ($(o3.north) + (0, 0.5)$) {Ordering Service and Paxos Acceptors};
    \draw[rounded corners]
      ($(o1.south west) + (-0.2, -0.2)$) rectangle
      ($(o5.north east) + (0.2, 0.2)$);

    % BPaxos Nodes
    \node[machine, below=1cm of o3] (b3) {$R_3$};
    \node[machine, left=0.1cm of b3] (b2) {$R_2$};
    \node[machine, left=0.1cm of b2] (b1) {$R_1$};
    \node[machine, right=0.1cm of b3] (b4) {$R_4$};
    \node[machine, right=0.1cm of b4] (b5) {$R_5$};
    \draw[rounded corners]
      ($(b1.south west) + (-0.2, -0.2)$) rectangle
      ($(b5.north east) + (0.2, 0.2)$);
      \node (bpaxos) at ($(b3.south) + (0, -0.5)$) {BPaxos Nodes};
  \end{tikzpicture}

  \caption{Colocated BPaxos}\figlabel{ColocatedBPaxos}
\end{figure}
}

As before, when a BPaxos node $R$ receives a command $a$ from a client, it
sends the command to the ordering service nodes in some instance $R.i$. Upon
receiving $a$, an ordering service node $o_j$ computes its reply $(R.i, a,
\deps{a}_j)$. Unlike with BPaxos, $o_j$ does not return the reply $(R.i, a,
\deps{a}_j)$ directly to $R$. Instead, it proposes the value $(a, \deps{a}_j)$
in instance $R.i$ to $p_j$ (the colocated Fast Paxos acceptor). As we just
described, $p_j$ votes for the first proposal that it hears from anyone, so
$p_j$ votes for the value $(a, \deps{a}_j)$ in instance $R.i$ and relays its
phase 2b vote back to $R$.

If $R$ receives a superquorum (i.e.\ $f + \floor{\frac{f+1}{2}} + 1$) of phase
2b votes for the same value $(a, \deps{a}_{j_1}) = \cdots = (a,
\deps{a}_{j_m})$ in instance $R.i$, then Fast Paxos (and hence BPaxos)
considers the value chosen. Thus, in the best case, BPaxos can commit a value
in one round trip to a superquorum (just like EPaxos).

If $R$ does \emph{not} receive a superquorum of phase 2b votes for the same
value, then it is unsure whether or not a value was chosen and begins
\defword{recovery} for instance $R.i$. We'll explain recovery in one moment. At
any point in time, any other BPaxos node $Q$ may also begin recovery for
instance $R.i$. In practice, a BPaxos node will only begin recovery if it
believes $R$ has failed, but any node may attempt to recover any instance at
any time.

In this incorrect BPaxos variant, the recovery of instance $R.i$ by node $Q$ is
simply the process of $Q$ attempting to get a value chosen in $R.i$ with a
higher ballot. As with BPaxos, $Q$ can propose the value $(\noop, \emptyset)$,
or it can can contact the ordering service to see if a command $a$ has already
been proposed in instance $R.i$ and then propose $(a, \deps{a})$ where
$\deps{a}$ comes from a response from the ordering service.

\paragraph{Incorrectness}
Now, we explore why this variant of BPaxos is incorrect. Assume we are running
this BPaxos variant with $f = 2$ (i.e.\ $5$ total nodes), and imagine EPaxos
node $Q$ is attempting to recovery instance $R.i$. $Q$ begins Fast Paxos with
ballot 1 and sends phase 1a messages to some majority of consensus nodes. $Q$
receives the following phase 1b messages from a majority of acceptors:
\begin{itemize}
  \item
    $p_3$ voted for $(a, \set{I_b})$ in round $0$.
  \item
    $p_4$ voted for $(a, \set{I_b})$ in round $0$.
  \item
    $p_4$ voted for $(a, \set{I_c})$ in round $0$.
\end{itemize}
where $I_b$ is an instance with command $b$ and $I_c$ is an instance with
command $c$.
%
$Q$ then begins phase 2 of Fast Paxos. $p_3$ and $p_4$ voted for the value $(a,
\set{I_b})$ in round $0$. $p_1$ and $p_2$ maybe also voted for $(a, \set{I_b})$
in round $0$, which means that $(a, \set{I_b})$ may have been chosen in round
$0$. Thus, $Q$ executes Case 2 of \algoref{FastPaxosTweak} and proposes that
the value $(a, \set{I_b})$ be chosen.

This is incorrect! It's possible that $p_1$ and $p_2$ did \emph{not} vote for
$(a, \set{I_b})$ in round $0$. For example, they could have voted for $(a,
\set{I_c})$. In this case, $\set{I_b}$, the dependencies proposed by $Q$, is
not a response from the ordering service. Because of this, this incorrect
BPaxos variant does not maintain \invref{ConflictingGadgets}.

More concretely, we can construct the following execution of this incorrect
BPaxos variant in which two conflicting commands are both chosen and neither
depends on the other. $R$ receives command $a$ and $Q$ receives a conflicting
command $b$.  $o_1$ and $o_2$ receive $a$ and propose $(a, \emptyset)$ in
instance $R.i$ to acceptors $p_1$ and $p_2$. Similar, $o_4$ and $o_5$ receive
$b$ and propose $(b, \emptyset)$ in instance $Q.j$ to acceptors $p_4$ and
$p_5$. Then, $R$ and $Q$ crash and all other messages are dropped. Another
BPaxos node $S$ attempts to recover $R.i$. In phase 1 of Fast Paxos, it hears
from $p_1$, $p_2$, and $p_3$, and then gets the value $(a, \emptyset)$ chosen.
Then, another node $T$ recovers $Q.j$. In phase 1 of Fast Paxos, it hears from
$p_3$, $p_4$, and $p_5$ and then gets the value $(b, \emptyset)$ chosen. $S$
executes $a$ and $T$ executes $b$.

Taking a step back, we see that this BPaxos variant is incorrect because it
fails to maintain \invref{ConflictingGadgets}. It fails to maintain this
invariant because of a mismatch between the invariants that BPaxos wants to
maintain and the invariants that Fast Paxos provides. BPaxos wants to get
values of the form $(a, \deps{a})$ chosen, but only if $\deps{a}$ is a response
from the ordering service. This is required to ensure that conflicting commands
will depend on one another. However Fast Paxos wants to get \emph{any} value
chosen. Fast Paxos has no understanding of the ordering service nodes, or any
notion that some specific values should not be chosen. Fast Paxos tries to get
any proposed value chosen. In our example above, when node $S$ received 2 votes
for $(a, \emptyset)$ in round 0, Fast Paxos determined that this value may have
been chosen in round 0, so it happily proposes it. Fast Paxos doesn't
understand the additional requirement that BPaxos requires: that $(a,
\emptyset)$ cannot be chosen unless a majority of ordering service nodes
determined that $a$'s dependencies should be the empty set.

Here's another way to interpret the incorrectness of this BPaxos variant. When
a node $Q$ is recovering an instance $R.i$ and wants to propose a value $v =
(a, \deps{a})$, it has to perform the logic illustrated in
\figref{BPaxosLogic}.
%
If it's possible that a value $v$ was previously chosen, then to avoid choosing
two different values, we have no choice but to propose $v$.
%
Similarly, if $v$ is a union of responses from a majority of ordering service
nodes, then we're free to propose it, but if $v$ is not the union of responses
from a majority of ordering service nodes, then we can't propose it.
%
If a recovering node $Q$ has a value $v$ that may have been chosen \emph{and}
is maybe not a union of responses from a majority of ordering service nodes,
then $Q$ is stuck. It simultaneously has to propose $v$ and cannot propose $v$.
This is exactly the situation in which our BPaxos variant is incorrect. When
faced with such a value $v$, it erroneously proposes $v$.

\begin{figure}[h]
  \centering
  \begin{tabular}{rccc}
    %
    &
    &
    \multicolumn{2}{p{3in}}{%
      Is $v$ a response from the ordering service?
    } \\
    %
    &
    &
    yes &
    maybe not \\\cline{3-4}
    %
    \multirow{2}{1.8in}{Was $v$ previously chosen?} &
    maybe &
    \multicolumn{1}{|c|}{Propose it} &
    \multicolumn{1}{|c|}{We're stuck!} \\\cline{3-4}
    %
    &
    definitely not &
    \multicolumn{1}{|c|}{Propose it, or propose noop} &
    \multicolumn{1}{|c|}{Propose noop} \\\cline{3-4}
  \end{tabular}
  \caption{The logic that any BPaxos variant should follow}%
  \figlabel{BPaxosLogic}
\end{figure}

Thus, in order for a BPaxos variant to be correct, it has to eliminate the
possibility that a value $v$ may have been chosen and simultaneously may not be
a response from the ordering service. In the next couple of sections, we'll
introduce a couple of BPaxos variants. We'll see that each variant uses
different mechanisms to handle this situation.
