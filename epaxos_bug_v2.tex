\documentclass{mwhittaker}
\title{EPaxos Bug}

\usepackage{epaxos}
\usepackage{pervasives}
\usepackage{tikz}
\usetikzlibrary{calc}
\usetikzlibrary{positioning}
\usetikzlibrary{shapes.misc}

% Epaxos node style.
\tikzstyle{epaxosnode}=[draw, line width=1pt, shape=circle]

% Node x-coordinates.
\newcommand{\Qx}{0}
\newcommand{\Rx}{2}
\newcommand{\Sx}{4}
\newcommand{\Tx}{6}
\newcommand{\Ux}{8}

\begin{document}
\maketitle

In this document, we outline a bug in EPaxos~\cite{moraru2013there,
moraru2013proof}. We consider an EPaxos deployment with $n = 5$ EPaxos nodes
(i.e., $f = 2$) and show that EPaxos does not preserve execution consistency
(i.e.\ Theorem $7$ in \cite{moraru2013proof}).

Consider the five EPaxos nodes $Q, R, S, T, U$ with EPaxos instances $Q.1$ and
$R.1$. Initially, $Q$ receives a $Request(\gamma)$ message from a client. It
marks the tuple $(\gamma, seq_\gamma, deps_\gamma) = (\gamma, 1, \emptyset)$ as
pre-accepted and sends the tuple to a fast quorum of other acceptors.
%
Simultaneously, $R$ receives a $Request(\delta)$ message from a client, marks
the tuple $(\delta, seq_\delta, deps_\delta) = (\delta, 1, \emptyset)$ as
pre-accepted, and sends the tuple to a fast quorum of other acceptors.
%
This is illustrated in \figref{BeforePreAccept}.

\begin{figure}[h]
  \centering

  \begin{tikzpicture}
    \node[epaxosnode] at (\Qx, 4) {$Q$};
    \instance[%
      tikz id=QQ1, instance id=Q.1, command=\gamma, seq no=1,
      status=pre-accepted, coordinate={\Qx, 2}, label position=top]{}
    \instance[%
      tikz id=QR1, instance id=R.1, command=, seq no=,
      status=, coordinate={\Qx, 0}, label position=bottom]{}

    \node[epaxosnode] at (\Rx, 4) {$R$};
    \instance[%
      tikz id=RQ1, instance id=Q.1, command=, seq no=,
      status=, coordinate={\Rx, 2}, label position=top]{}
    \instance[%
      tikz id=RR1, instance id=R.1, command=\delta, seq no=1,
      status=pre-accepted, coordinate={\Rx, 0}, label position=bottom]{}

    \node[epaxosnode] at (\Sx, 4) {$S$};
    \instance[%
      tikz id=SQ1, instance id=Q.1, command=, seq no=,
      status=, coordinate={\Sx, 2}, label position=top]{}
    \instance[%
      tikz id=SR1, instance id=R.1, command=, seq no=,
      status=, coordinate={\Sx, 0}, label position=bottom]{}

    \node[epaxosnode] at (\Tx, 4) {$T$};
    \instance[%
      tikz id=TQ1, instance id=Q.1, command=, seq no=,
      status=, coordinate={\Tx, 2}, label position=top]{}
    \instance[%
      tikz id=TR1, instance id=R.1, command=, seq no=,
      status=, coordinate={\Tx, 0}, label position=bottom]{}

    \node[epaxosnode] at (\Ux, 4) {$U$};
    \instance[%
      tikz id=UQ1, instance id=Q.1, command=, seq no=,
      status=, coordinate={\Ux, 2}, label position=top]{}
    \instance[%
      tikz id=UR1, instance id=R.1, command=, seq no=,
      status=, coordinate={\Ux, 0}, label position=bottom]{}
  \end{tikzpicture}

  \caption{}\figlabel{BeforePreAccept}
\end{figure}

Node $S$ receives the $PreAccept(\gamma, 1, \emptyset, Q.1)$ message from $Q$
and pre-accepts it unmodified. Similarly, nodes $T$ and $U$ receive the
$PreAccept(\delta, 1, \emptyset, R.1)$ message from $R$ and pre-accept it
unmodified.
%
Then, $S$ receives $PreAccept(\delta, 1, \emptyset, R.1)$ and pre-accepts
$(\delta, 2, \set{Q.1})$. Similarly, $T$ and $U$ receive $PreAccept(\gamma, 1,
\emptyset, Q.1)$ and pre-accept $(\gamma, 2, \set{R.1})$.
%
This is illustrated in \figref{AfterPreAccept}.

\begin{figure}[h]
  \centering

  \begin{tikzpicture}
    \node[epaxosnode] at (\Qx, 4) {$Q$};
    \instance[%
      tikz id=QQ1, instance id=Q.1, command=\gamma, seq no=1,
      status=pre-accepted, coordinate={\Qx, 2}, label position=top]{}
    \instance[%
      tikz id=QR1, instance id=R.1, command=, seq no=,
      status=, coordinate={\Qx, 0}, label position=bottom]{}

    \node[epaxosnode] at (\Rx, 4) {$R$};
    \instance[%
      tikz id=RQ1, instance id=Q.1, command=, seq no=,
      status=, coordinate={\Rx, 2}, label position=top]{}
    \instance[%
      tikz id=RR1, instance id=R.1, command=\delta, seq no=1,
      status=pre-accepted, coordinate={\Rx, 0}, label position=bottom]{}

    \node[epaxosnode] at (\Sx, 4) {$S$};
    \instance[%
      tikz id=SQ1, instance id=Q.1, command=\gamma, seq no=1,
      status=pre-accepted, coordinate={\Sx, 2}, label position=top]{}
    \instance[%
      tikz id=SR1, instance id=R.1, command=\delta, seq no=2,
      status=pre-accepted, coordinate={\Sx, 0}, label position=bottom]{}
    \draw[->, thick] (SR1) to (SQ1);

    \node[epaxosnode] at (\Tx, 4) {$T$};
    \instance[%
      tikz id=TQ1, instance id=Q.1, command=\gamma, seq no=2,
      status=pre-accepted, coordinate={\Tx, 2}, label position=top]{}
    \instance[%
      tikz id=TR1, instance id=R.1, command=\delta, seq no=1,
      status=pre-accepted, coordinate={\Tx, 0}, label position=bottom]{}
    \draw[->, thick] (TQ1) to (TR1);

    \node[epaxosnode] at (\Ux, 4) {$U$};
    \instance[%
      tikz id=UQ1, instance id=Q.1, command=\gamma, seq no=2,
      status=pre-accepted, coordinate={\Ux, 2}, label position=top]{}
    \instance[%
      tikz id=UR1, instance id=R.1, command=\delta, seq no=1,
      status=pre-accepted, coordinate={\Ux, 0}, label position=bottom]{}
    \draw[->, thick] (UQ1) to (UR1);
  \end{tikzpicture}

  \caption{}\figlabel{AfterPreAccept}
\end{figure}

Next, $Q$ and $R$ crash. After some time, $S$ attempts to recover $\gamma$ and
$U$ attempts to recover $\delta$. We assume we are running optimized EPaxos
with FP-quorum and FP-deps-committed, but not Accept-Deps (see page 15 of
\cite{moraru2013proof}). The following occurs:
\begin{itemize}
  \item
    $S$ sends $Prepare$ messages to all other replicas. $S$ receives $(\gamma,
    1, \emptyset, \text{pre-accepted})$ from itself and $(\gamma, 2, \set{Q.1},
    \text{pre-accepted})$ from $T$, and $U$.
  \item
    Steps 2, 3, and 4 of the recovery protocol (page 16 of
    \cite{moraru2013proof}) are skipped. At least $\floor{\frac{f+1}{2}} = 1$
    replica has pre-accepted $(\gamma, 1, \emptyset, \text{pre-accepted})$ in
    the default ballot.
  \item
    $S$ sends $TentativePreAccept(Q.1, \gamma, 1, \emptyset)$ to nodes $T$ and
    $U$. $T$ and $U$ have recored a conflicting command $\delta$, but
\end{itemize}

\bibliographystyle{plain}
\bibliography{references}
\end{document}
