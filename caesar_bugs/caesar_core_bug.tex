\documentclass{mwhittaker}
\title{Caesar Bug}

\usepackage{pervasives}
\usepackage{tikz}
\usepackage{url}
\usetikzlibrary{calc}
\usetikzlibrary{positioning}
\usetikzlibrary{shapes.misc}

\newcommand{\Time}{\mathcal{T}}
\newcommand{\Ballot}{\mathcal{B}}
\newcommand{\Log}{\mathcal{H}}
\newcommand{\Ballots}{\mathcal{B}allots}
\newcommand{\Pred}{\mathcal{P}red}
\newcommand{\Nack}{\mathcal{NACK}}

\begin{document}
\begin{center}
  \Large Caesar Bug
\end{center}

We present an execution of Caesar in which replicas execute conflicting
commands in opposite orders. We consider a Caesar deployment with $f = 2$ and
five replicas: $A$, $B$, $C$, $D$, and $E$. The execution involves conflicting
commands $x$ and $y$. At every step of the execution, we show the state of all
the replicas. We begin in the initial state:

{\begin{center}
  \fbox{
    \begin{minipage}{\textwidth}
      \begin{center}
        Replica $A$
      \end{center}
      $\Time_A = (0, A)$ \\
      $\Log_A$ =
      \begin{tabular}{|c|c|c|c|c|c|}
        \hline
        $c$ & $\Time$ & Pred & status & $\Ballot$ & forced \\\hline
      \end{tabular} \\
      $\Ballots_A$ =
      \begin{tabular}{|c|c|}
      \end{tabular}
    \end{minipage}
  }

  \fbox{
    \begin{minipage}{\textwidth}
      \begin{center}
        Replica $B$
      \end{center}
      $\Time_B = (0, B)$ \\
      $\Log_B$ =
      \begin{tabular}{|c|c|c|c|c|c|}
        \hline
        $c$ & $\Time$ & Pred & status & $\Ballot$ & forced \\\hline
      \end{tabular} \\
      $\Ballots_B$ =
      \begin{tabular}{|c|c|}
      \end{tabular}
    \end{minipage}
  }

  \fbox{
    \begin{minipage}{\textwidth}
      \begin{center}
        Replica $C$
      \end{center}
      $\Time_C = (0, C)$ \\
      $\Log_C$ =
      \begin{tabular}{|c|c|c|c|c|c|}
        \hline
        $c$ & $\Time$ & Pred & status & $\Ballot$ & forced \\\hline
      \end{tabular} \\
      $\Ballots_C$ =
      \begin{tabular}{|c|c|}
      \end{tabular}
    \end{minipage}
  }

  \fbox{
    \begin{minipage}{\textwidth}
      \begin{center}
        Replica $D$
      \end{center}
      $\Time_D = (0, D)$ \\
      $\Log_D$ =
      \begin{tabular}{|c|c|c|c|c|c|}
        \hline
        $c$ & $\Time$ & Pred & status & $\Ballot$ & forced \\\hline
      \end{tabular} \\
      $\Ballots_D$ =
      \begin{tabular}{|c|c|}
      \end{tabular}
    \end{minipage}
  }

  \fbox{
    \begin{minipage}{\textwidth}
      \begin{center}
        Replica $E$
      \end{center}
      $\Time_E = (0, E)$ \\
      $\Log_E$ =
      \begin{tabular}{|c|c|c|c|c|c|}
        \hline
        $c$ & $\Time$ & Pred & status & $\Ballot$ & forced \\\hline
      \end{tabular} \\
      $\Ballots_E$ =
      \begin{tabular}{|c|c|}
      \end{tabular}
    \end{minipage}
  }
\end{center}
}

First, $A$ receives command $x$ from a client. It increments $\Time_A$ and
enters the fast proposal phase for $x$ with $\Time_A$, ballot $0$, and an empty
whitelist. $A$ sends FastPropose messages to all replicas. Replicas $A$, $B$,
$C$, and $D$ receive and process the FastPropose request without issue. The
FastPropose sent to $E$ is delayed. $A$ receives responses from replicas $A$,
$B$, $C$, and $D$ and computes $\Time{}ime$ and $\Pred$ to $(1, A)$ and
$\emptyset$. $A$ didn't receive any $\Nack$s, so it enters the stable phase.
$A$ sends Stable messages to all replicas, but they are all dropped except the
one sent to $A$. $A$ receives the Stable message and executes $x$.

{\begin{center}
  \fbox{
    \begin{minipage}{\textwidth}
      \begin{center}
        Replica $A$
      \end{center}
      $\Time_A = (2, A)$ \\
      $\Log_A$ =
      \begin{tabular}{|c|c|c|c|c|c|}
        \hline
        $c$ & $\Time$  & Pred        & status                & $\Ballot$ & forced \\\hline
        $x$ & $(1, A)$ & $\emptyset$ & \textit{stable} & 0         & $\bot$ \\\hline
      \end{tabular} \\
      $\Ballots_A$ =
      \begin{tabular}{|c|c|}
        \hline
        $x$ & 0 \\\hline
      \end{tabular}
    \end{minipage}
  }

  \fbox{
    \begin{minipage}{\textwidth}
      \begin{center}
        Replica $B$
      \end{center}
      $\Time_B = (1, B)$ \\
      $\Log_B$ =
      \begin{tabular}{|c|c|c|c|c|c|}
        \hline
        $c$ & $\Time$  & Pred        & status                & $\Ballot$ & forced \\\hline
        $x$ & $(1, A)$ & $\emptyset$ & \textit{fast-pending} & 0         & $\bot$ \\\hline
      \end{tabular} \\
      $\Ballots_B$ =
      \begin{tabular}{|c|c|}
        \hline
        $x$ & 0 \\\hline
      \end{tabular}
    \end{minipage}
  }

  \fbox{
    \begin{minipage}{\textwidth}
      \begin{center}
        Replica $C$
      \end{center}
      $\Time_C = (1, C)$ \\
      $\Log_C$ =
      \begin{tabular}{|c|c|c|c|c|c|}
        \hline
        $c$ & $\Time$  & Pred        & status                & $\Ballot$ & forced \\\hline
        $x$ & $(1, A)$ & $\emptyset$ & \textit{fast-pending} & 0         & $\bot$ \\\hline
      \end{tabular} \\
      $\Ballots_C$ =
      \begin{tabular}{|c|c|}
        \hline
        $x$ & 0 \\\hline
      \end{tabular}
    \end{minipage}
  }

  \fbox{
    \begin{minipage}{\textwidth}
      \begin{center}
        Replica $D$
      \end{center}
      $\Time_D = (1, D)$ \\
      $\Log_D$ =
      \begin{tabular}{|c|c|c|c|c|c|}
        \hline
        $c$ & $\Time$  & Pred        & status                & $\Ballot$ & forced \\\hline
        $x$ & $(1, A)$ & $\emptyset$ & \textit{fast-pending} & 0         & $\bot$ \\\hline
      \end{tabular} \\
      $\Ballots_D$ =
      \begin{tabular}{|c|c|}
        \hline
        $x$ & 0 \\\hline
      \end{tabular}
    \end{minipage}
  }

  \fbox{
    \begin{minipage}{\textwidth}
      \begin{center}
        Replica $E$
      \end{center}
      $\Time_E = (0, E)$ \\
      $\Log_E$ =
      \begin{tabular}{|c|c|c|c|c|c|}
        \hline
        $c$ & $\Time$ & Pred & status & $\Ballot$ & forced \\\hline
      \end{tabular} \\
      $\Ballots_E$ =
      \begin{tabular}{|c|c|}
      \end{tabular}
    \end{minipage}
  }
\end{center}
}

Next, $A$ receives command $y$ and enters the fast proposal phase with $(3,
a)$, ballot $0$, and an empty whitelist. It sends FastPropose messages to all
replicas. All the replicas receive the FastPropose message and reply.

{\begin{center}
  \fbox{
    \begin{minipage}{\textwidth}
      \begin{center}
        Replica $A$
      \end{center}
      $\Time_A = (3, A)$ \\
      $\Log_A$ =
      \begin{tabular}{|c|c|c|c|c|c|}
        \hline
        $c$ & $\Time$  & Pred        & status                & $\Ballot$ & forced \\\hline
        $x$ & $(1, A)$ & $\emptyset$ & \textit{stable} & 0         & $\bot$ \\\hline
        $y$ & $(3, A)$ & $\set{x}$ & \textit{stable} & 0         & $\bot$ \\\hline
      \end{tabular} \\
      $\Ballots_A$ =
      \begin{tabular}{|c|c|}
        \hline
        $x$ & 0 \\\hline
      \end{tabular}
    \end{minipage}
  }

  \fbox{
    \begin{minipage}{\textwidth}
      \begin{center}
        Replica $B$
      \end{center}
      $\Time_B = (1, B)$ \\
      $\Log_B$ =
      \begin{tabular}{|c|c|c|c|c|c|}
        \hline
        $c$ & $\Time$  & Pred        & status                & $\Ballot$ & forced \\\hline
        $x$ & $(1, A)$ & $\emptyset$ & \textit{fast-pending} & 0         & $\bot$ \\\hline
        $y$ & $(3, A)$ & $\set{A}$   & \textit{fast-pending} & 0         & $\bot$ \\\hline
      \end{tabular} \\
      $\Ballots_B$ =
      \begin{tabular}{|c|c|}
        \hline
        $x$ & 0 \\\hline
      \end{tabular}
    \end{minipage}
  }

  \fbox{
    \begin{minipage}{\textwidth}
      \begin{center}
        Replica $C$
      \end{center}
      $\Time_C = (1, C)$ \\
      $\Log_C$ =
      \begin{tabular}{|c|c|c|c|c|c|}
        \hline
        $c$ & $\Time$  & Pred        & status                & $\Ballot$ & forced \\\hline
        $x$ & $(1, A)$ & $\emptyset$ & \textit{fast-pending} & 0         & $\bot$ \\\hline
        $y$ & $(3, A)$ & $\set{A}$   & \textit{fast-pending} & 0         & $\bot$ \\\hline
      \end{tabular} \\
      $\Ballots_C$ =
      \begin{tabular}{|c|c|}
        \hline
        $x$ & 0 \\\hline
      \end{tabular}
    \end{minipage}
  }

  \fbox{
    \begin{minipage}{\textwidth}
      \begin{center}
        Replica $D$
      \end{center}
      $\Time_D = (1, D)$ \\
      $\Log_D$ =
      \begin{tabular}{|c|c|c|c|c|c|}
        \hline
        $c$ & $\Time$  & Pred        & status                & $\Ballot$ & forced \\\hline
        $x$ & $(1, A)$ & $\emptyset$ & \textit{fast-pending} & 0         & $\bot$ \\\hline
        $y$ & $(3, A)$ & $\set{A}$   & \textit{fast-pending} & 0         & $\bot$ \\\hline
      \end{tabular} \\
      $\Ballots_D$ =
      \begin{tabular}{|c|c|}
        \hline
        $x$ & 0 \\\hline
      \end{tabular}
    \end{minipage}
  }

  \fbox{
    \begin{minipage}{\textwidth}
      \begin{center}
        Replica $E$
      \end{center}
      $\Time_E = (0, E)$ \\
      $\Log_E$ =
      \begin{tabular}{|c|c|c|c|c|c|}
        \hline
        $c$ & $\Time$  & Pred        & status                & $\Ballot$ & forced \\\hline
        $y$ & $(3, A)$ & $\emptyset$ & \textit{fast-pending} & 0         & $\bot$ \\\hline
      \end{tabular} \\
      $\Ballots_E$ =
      \begin{tabular}{|c|c|}
      \end{tabular}
    \end{minipage}
  }
\end{center}
}

\bibliographystyle{plain}
\bibliography{references}
\end{document}
